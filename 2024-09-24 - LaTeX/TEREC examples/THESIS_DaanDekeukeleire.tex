%%%%%%%%%%%%%%%% PRESETTINGS %%%%%%%%%%%%%%%%%%%%%%%


\documentclass[10pt, twoside]{book} %

\usepackage[paperheight=24cm,paperwidth=17cm,margin=2.3cm]{geometry}
%\usepackage[margin=2.3cm,b5paper]{geometry} %lmargin=2.5cm,rmargin=4cm,tmargin=3cm,bmargin=3cm
%\pdfpagewidth 17cm
%\pdfpageheight 24cm




\usepackage[table]{xcolor}

\usepackage{amsmath,amssymb,mathtools,graphicx,textcomp,booktabs,url,setspace,xcolor,soul,eurosym}
\usepackage{multirow,listings,setspace,gnuplottex,latexsym,keyval,ifthen,moreverb,lscape,forest}
\usepackage[pagewise]{lineno}
\usepackage{wallpaper}
\graphicspath{{fig/}}

\usepackage[figuresright]{rotating}

\usepackage[]{microtype} %activate={true,nocompatibility},final,kerning=true,spacing=true,factor=1100,stretch=10,shrink=10
\usepackage{booktabs}

\usepackage{lmodern}
\usepackage{tabularx}

\usepackage{tikz,pgfplots}

\usetikzlibrary{calc,shapes,arrows,intersections,shadows}
\usepackage{textcomp}

\usepackage[most]{tcolorbox}
\usepackage{longtable}

\usepackage[]{threeparttable}

\usepackage[indention=0.5cm,labelsep=colon,font={sf,small},labelfont={bf,sf}]{caption}
\usepackage[indention=0.5cm,font={sf,small},labelfont={bf,sf}]{subcaption}

\definecolor{lightgray}{gray}{0.90} %not always visible on a dell screen!!!
\definecolor{darkgray}{gray}{0.50}







\usepackage{fancyhdr}  
\pagestyle{fancy} 
\renewcommand{\chaptermark}[1]{\markboth{#1}{}}
\fancypagestyle{frontmatter}{%    
	\fancyhf{} 	
	\fancyfoot[RO,LE]{\textsf{\thepage}} %Page 
	\fancypagestyle{plain}{
		\renewcommand{\headrulewidth}{0pt}
		\fancyhead{}
		\fancyfoot[RO,LE]{\textsf{\thepage}} %Page
	} 
}
\fancypagestyle{mainmatter}{%    
	\fancyhf{} 	
	\fancyhead[RO,LE]{\nouppercase{\small \textsf{\leftmark}}}
	\fancyfoot[RO,LE]{\textsf{\thepage}} %Page 
}
\setlength{\headheight}{10pt}

\usepackage{pdfpages}

\newcounter{letternum}
\newcounter{lettersum}
\setcounter{lettersum}{13}
\newlength{\thumbtopmargin}
\setlength{\thumbtopmargin}{1cm}
\newlength{\thumbbottommargin}
\setlength{\thumbbottommargin}{3cm}
\newlength{\thumbheight}
\pgfmathsetlength{\thumbheight}{%
(\paperheight-\thumbtopmargin-\thumbbottommargin)/\value{lettersum}}

\newlength{\thumbwidth}
\setlength{\thumbwidth}{1.2cm}
\setlength{\thumbheight}{1cm}


\tikzset{
   thumb/.style={
%   draw=black,
  fill=light-gray,
   text=black,
   minimum height=\thumbheight, %\thumbheight,
   text width=\thumbwidth,
	outer sep=0pt,%   outer sep=10pt,
   font=\sffamily\Large,
}
}
\newcommand{\oddthumb}[1]{%
      \begin{tikzpicture}[remember picture, overlay]
        \node [thumb,text centered,anchor=north east,] at ($%
            (current page.north east)-%
            (0,\thumbtopmargin+\value{letternum}*\thumbheight)%
        $) {#1};
   \end{tikzpicture}
}
\newcommand{\eventhumb}[1]{%
      \begin{tikzpicture}[remember picture, overlay]
        \node [thumb,text centered,anchor=north west,] at ($%
            (current page.north west)-%
            (0,\thumbtopmargin+\value{letternum}*\thumbheight)%
        $) {#1};
   \end{tikzpicture}
}
% create a new command to set a new lettergroup
\newcommand{\lettergroup}[1]{%
\fancyhead[LO]{\oddthumb{#1}}%
\fancyhead[RE]{\eventhumb{#1}}%
\fancypagestyle{chapterstart}{%
    \renewcommand{\headrulewidth}{0pt}
    \renewcommand{\footrulewidth}{0pt}
    \fancyhf{}
    \chead{\oddthumb{#1}}% chapters start only on odd pages
    \fancyfoot[RO,LE]{\textsf{\thepage}} %\fancyfoot[RO,LE]{\textsf{Page \thepage}}
  }
    \thispagestyle{chapterstart}
\stepcounter{letternum}%
}

%%%% CREATE BOX CHAPTER

\usepackage[english]{babel}
\usepackage[newparttoc,explicit, clearempty]{titlesec}%
\usepackage[titles]{tocloft}
\renewcommand{\cftpartpresnum}{BOOK\enspace}
\renewcommand{\cftchapaftersnum}{.}
\renewcommand\cftchapdotsep{\cftdotsep}

\titleformat{\part}[display]{\bfseries\filcenter \def\partname{Book}}{\Huge\MakeUppercase{\partname}\enspace\thepart}{10pt}{\Huge #1}[\thispagestyle{empty}]%

\titleformat{\chapter}[display]{\filcenter\bfseries}{\LARGE\MakeUppercase{\chaptername}~\thechapter}%
{1\baselineskip}
{\huge#1}%
\titleformat{name=\chapter, numberless}[block]{\filcenter\bfseries}{}%
{0pt}{\huge#1\ifstrequal{#1}{\contentsname}{}{\addcontentsline{toc}{chapter}{#1}}}%


\usepackage[titletoc]{appendix} %
\AtBeginEnvironment{appendices}{\def\chaptername\appendixname}
\AtEndEnvironment{appendices}{\def\chaptername\oldchaptername}
\newenvironment{newchapterbox}{%
\def\chaptername{BOX}\def\appendixname{BOX}\appendices}%
{\endappendices}



%\makeatletter\renewcommand\tableofcontents{%
%\chapter*{\contentsname}%
%\@starttoc{toc}%
%}
%\makeatother
%%% END


\usepackage[nottoc]{tocbibind}	%numbib


%\usepackage[final]{pdfpages}

\usepackage[]{natbib} %numbers,sort&compress

\usepackage[Sonny]{fncychap} %Sonny

\usepackage[colorlinks,linkcolor=black,urlcolor=black,citecolor=black]{hyperref} %load hyperref after fncychap
\hypersetup{%
    pdftitle = {Effects of forest fragmentation and resource availability on mobility and reproduction of two avian insectivores, and consequences for ecosystem functioning},
    pdfsubject = {PhD thesis},
    pdfkeywords = {Forest fragmentation, Parus major, Cyanistes caeruleus, Quercus robur, Quercus rubra, Fagus sylvatica, forest composition},
    pdfauthor = {Daan Dekeukeleire},
    pdfcreator = {\LaTeX\ with package \flqq hyperref\frqq},
}
\usepackage[]{cleveref} % load cleveref after hyperref


\setlength{\parindent}{2em} 
\renewcommand{\contentsname}{Table of Contents}
\renewcommand{\listfigurename}{List of Figures}
\renewcommand{\listtablename}{List of Tables}
\renewcommand{\appendixname}{}

\makeatletter
\newenvironment{chapquote}[2][2em]
  {\setlength{\@tempdima}{#1}%
   \def\chapquote@author{#2}%
   \parshape 1 \@tempdima \dimexpr\textwidth-2\@tempdima\relax%
   \itshape}
  {\par\normalfont\hfill--\ \chapquote@author\hspace*{\@tempdima}\par\bigskip}
\makeatother

\makeatletter
\def\mainmatter{%
	\cleardoublepage
	\@mainmattertrue
	\pagenumbering{arabic}
	\def\mainmatter{\cleardoublepage\@mainmattertrue}
}
\makeatother



\definecolor{light-gray}{gray}{0.70} %not always visible on a dell screen!!!
\definecolor{mygreen}{HTML}{23A48B}
\definecolor{myyellow}{HTML}{F49F1F}
\definecolor{myred}{HTML}{C24133}
\definecolor{mygray}{gray}{0.90}
\setlength{\parindent}{2em}

\usepackage{pdflscape}
\usepackage{afterpage}


%\renewcommand{\arraystretch}{1.7}



%%%%%%%%%%%%%%%% BEGIN DOCUMENT %%%%%%%%%%%%%%%%%%%%%%%

\begin{document}

\includepdf[pages=-]{cover.pdf}
	
\clearpage
	
	\frontmatter
	\pagestyle{frontmatter}
	\lstset{language=Perl}
	%%%%%%%%%%%%%%%%  BEGIN TITLEPAGE  %%%%%%%%%%%%%%%%%%
	\begin{titlepage}
		
		\begin{center}	
			
			\thispagestyle{empty}
			
			\vspace*{3.00cm}
			
			{\huge \textbf{Effects of forest fragmentation and resource availability on mobility and reproduction of two avian insectivores, and consequences for ecosystem functioning}}\\
			
			\vspace{7.0 cm}
			
		\end{center}
		

		
	\end{titlepage}

\newpage
		
	\color{black}
	\newpage 
	\thispagestyle{empty}

	\vspace*{\fill}

	\begin{small}

	\noindent \textcopyright 2021 Daan Dekeukeleire

	\vspace{0.5cm}	

	\noindent Dekeukeleire D. (2021). \textit{Effects of forest fragmentation and resource availability on mobility and reproduction of two avian insectivores, and consequences for ecosystem functioning}. Ph.D. thesis, Ghent University, Ghent, Belgium.

	\vspace{0.5cm}	
\begin{tabbing}
 Printed by: \= University Press, Wachtebeke, Belgium  \\
Lay-out: \> Batsleer Femke, inpsired by Irene Lantman \\
Photos: \> Irene Lantman, Stephanie Schelfhout, Daan Dekeukeleire, Gert Arijs,\\
\> Pieter Vantieghem, Pieter Dierickx\\
Cover: \> Irene Lantman\\
\end{tabbing}

	\vspace*{0.5cm}
	
	\noindent The research presented in this study was financially supported by Concerted Research Actions -- Special Research Fund -- Ghent university as part of UGent GOA (Concentrated Research Actions) project ``Scaling up Functional Biodiversity Research: from Individuals to Landscapes and Back (TREEWEB)''.
	
	\vspace{1cm}
\end{small}	

	
	\newpage{\thispagestyle{empty}\cleardoublepage}
	\color{black}
	\newpage 
	\thispagestyle{empty}
\begin{center}
			\thispagestyle{empty}
			
			\vspace*{3.00cm}
			
			{\Large Effects of forest fragmentation and resource availability on mobility and reproduction of two avian insectivores, and consequences for ecosystem functioning}

			
			\vspace{7.5 cm}
			
			{\normalsize Daan Dekeukeleire} 
			
			\vspace{1.0 cm}
			
			{\normalsize 2021}	
			
			\vspace{2.0 cm}
			
			{\footnotesize Ghent University, Faculty of Sciences, Department of Biology, Terrestrial Ecology Unit}
			
			\vspace{0.5cm}
			
			{\footnotesize Thesis submitted in fulfillment of the requirements for the degree of\\
 			Doctor (Ph.D.) in Science: Biology}

\end{center}
\newpage
		
	\color{black}
	\newpage 
	\thispagestyle{empty}

		
	{\small \noindent \textbf{Supervisor:} \\
			\hspace{10mm}Prof. Dr. Luc Lens\\
			\hspace{10mm}Dr. Diederik Strubbe}\\

	\vspace*{1.0cm}
	
	{\small \noindent \textbf{Examination committee:}\\
		\hspace{10mm}Prof. Dr. Dries Bonte (chairman) \\
		\hspace{10mm}Prof. Dr. Lander Baeten (secretary)\\
		\hspace{10mm}Prof. Dr. Kirsty Park\\
		\hspace{10mm}Prof. Dr. Erik Matthysen\\
		\hspace{10mm}Dr. Lieze Rouffaer} \\
	
%	\newpage
%	\begin{center}	
%		\thispagestyle{empty}		
%		\vspace*{8.00cm}	
%		
%	\end{center}
		
	
	%%%%%%%%%%%%%%%%  BEGIN LISTS   %%%%%%%%%%%%%%%%%%%%%%%
	\newpage{\thispagestyle{empty}\cleardoublepage}
	{\setstretch{0.98}\tableofcontents}
	%\tableofcontents
	

	\clearpage
	\thispagestyle{empty} % empty 


	% TEMPORARY
	%\linenumbers

%%%%%%%%%%%%%%%%  SUMMARY   %%%%%%%%%%%%%%%%%%%%%%%
%\csname @openrightfalse\endcsname
	\mainmatter
\cleardoublepage
\thispagestyle{plain} % empty
\chapter*{Summary}
\addcontentsline{toc}{chapter}{Summary}

Due to anthropogenic land use changes, natural habitats such as forests are being subdivided in small and isolated fragments. This process of forest fragmentation has a large impact on biodiversity. First, population sizes decrease in smaller patches, exacerbating extinction risks. Second, fragmentation increases the amount of edges with other land uses, where abiotic (e.g. pollution) and biotic (e.g. predation) conditions differ from the core of the forest. Third, spatial isolation decreases dispersal. Reduced immigration can increase extinction risk of small populations, and can also decrease (re)colonization possibilities. Fourth, fragmentation can disrupt complex ecological interactions, such as those between predators and prey or pollinators and plants, which can have cascading effects on other species. European forests are not only highly fragmented, the tree species composition and structure is also highly impacted by intensive management, raising concerns for biodiversity conservation and forest resilience in the face of global changes. For instance, almost one third of the forest cover in Europe consists of a single tree species.\\

While conservation of the remaining natural forest habitats should be a priority, reforestation and adequate management of secondary forests will be essential to safeguard biodiversity and for the provision of ecosystem services (such as water and soil protection, recreation and carbon sequestration). Recent studies have demonstrated the importance of tree species richness and composition in this context, yet selecting the most beneficial species or species mixtures is no trivial matter. Thus, there is a need for empirical studies in real-world forest landscapes in order to inform policy makers and practitioners of where reforestation should take place, and of how current and future forest habitats should be managed.
Most studies addressing the effects of forest fragmentation on biodiversity use occurrence or count data. Such measures constitute time-lagged responses that may obscure the actual status of populations in fragmented landscapes. Contrarily, studies focusing on how fragmentation impacts individual behaviour and physiological condition can provide early warnings of population decline of forest-dependent species, and inform management actions. In this context, insectivorous birds form ideal sentinel species, i.e. species that respond to environmental change in a timely and measurable way, and can indicate an otherwise unobserved change in the ecosystem. Due to their high trophic position in the forest food web, these species are expected to be impacted more by forest fragmentation. Forest fragmentation strongly impacts the distribution of critical resources for forest-dependent birds. Resource availability does not only depend on the landscape context, but also on the local conditions and habitat quality in the remaining forest fragments, such as tree species composition and structural diversity, and such effects can be especially severe in seasons with low-resource availability, such as winter.\\

The overarching aim of this dissertation is to assess how forest fragmentation and resource availability, shaped by tree species composition and diversity, synergistically affect the fitness of avian insectivores across their full annual cycle, and how this feeds back to lower trophic levels. Common insectivorous birds such as great tits (\textit{Parus major}) and blue tits (\textit{Cyanistes caeruleus}), are ideal species for such research, as they occur in a wide range of forest types, from large old-growth deciduous forests to small fragments and plantations. Moreover, individuals readily visit feeders and accept nest boxes to breed, allowing detailed year-round monitoring.\\

First, I investigated how forest fragmentation and tree species composition affect reproductive success of great and blue tits, through resource availability during the spring breeding season. In order to do so, I collected data on the biomass of caterpillars, the main food source of these two species, in 53 research plots across independent gradients of forest fragmentation and tree species composition, and monitored the reproductive performance in 200 nests of great tits and 112 nests of blue tits in these plots. I show that both resource availability, shaped by tree species composition, and forest fragmentation jointly shape breeding performance. Effects of tree species composition were mainly driven by tree species identity, rather than by tree species diversity, and the highest breeding success for both species was obtained in monocultures of pedunculate oaks. Structural Equation Modelling revealed diverse and species-specific pathways: for great tits, tree composition effects on breeding performance were driven by availability of caterpillar prey, while for blue tits, these effects were driven by variation in clutch size. Fragmentation effects were only observed in resource-poor beech monocultures where breeding performance declined inversely with forest area.\\

Second, I investigated how conditions during winter and stress hormone levels affect winter habitat use and movement behaviour, and how these effects in winter are linked to performance in spring through carry-over effects. I used a combined observational and experimental design, whereby I first radio tracked great tits in 20 forest fragments, varying in size, across two winters to investigate winter habitat use. One winter, food was abundantly available in the form of beech mast, the other winter, food was scarce. Second, I experimentally increased the brood size of individuals differing in winter habitat use in one of these forest fragments, to investigate if variation in winter movement leads to reproductive costs when conditions to raise young are challenging. I show that resource availability in winter and forest fragment size influence winter habitat use and movement behaviour of great tits. When food resources were abundantly present in forests in the form of beech mast, individuals preferred to forage in forests, and avoided other habitat types. Individuals residing in smaller forests flew larger distances to forage in neighbouring forest fragments compared to those residing in large fragments. The brood size increase experiment shows that foraging outside the forest in winter had a clear reproductive cost in spring, i.e. a lower fledging success, but was not related to phenotypic stress markers, i.e. stress hormone levels or body condition.\\

Third, I zoomed in on the effect of stress, measured by stress hormone levels, on social behaviour. I inferred the winter social network of great and blue tits visiting artificial feeders at one of the studied forest fragments. On a subset of great tits, I quantified the levels of corticosterone, the main avian stress-hormone, in feathers as an archive of stress experienced during feather growth. I relate corticosterone in original feathers (grown in late summer) to social network position in winter, and in turn, relate social network position in winter to corticosterone in induced feathers (grown during winter). I show that feather corticosterone correlate with social behaviour in winter. Individuals with higher levels of corticosterone in their feathers associated with more individuals, and thus occupied more central positions in the winter social network. This could in turn lead to reduced stress levels, through increased access to social information on food resources, with possible carry-over effects to the subsequent breeding season.\\

Fourth, I investigated how forest fragmentation and tree species composition jointly influence the functional role of insectivorous birds, specifically arthropod control. I focus on top-down control of arthropod herbivores on saplings (i.e. young trees). In a field experiment, I planted saplings of three tree species (beech, red oak and pedunculate oak) in research plots varying in tree species composition and fragmentation context, and excluded avian predators from a subset of these. I investigate how avian top-down control, overstorey tree species composition and edge effects jointly shape herbivory levels and growth of these saplings. I show that avian top-down control decreased herbivory on all investigated tree species and --in the case of two out of the three investigated tree species-- increased growth. On pedunculate oak saplings, the tree species with the highest number of specialized herbivores, avian top-down control was stronger closer to the forest edge. Pedunculate oak was also the only species where effects of overstorey tree species composition could be found (with herbivory increasing with the number of pedunculate oaks), but these effects were only present in the absence of birds.\\

To conclude, I demonstrate that landscape level factors --forest fragmentation-- and local factors --resource availability, shaped by tree species composition-- jointly affect the fitness of two common forest-dependent insectivorous birds in a highly human-modified landscape. My data shows that tree species-linked food resource availability can affect breeding performance of tits both directly through arthropod availability during spring and indirectly through carry-over effects from winter habitat use, linked to food availability in winter. Fragmentation effects on breeding success were only observed in resource poor forests fragments. Moreover, both forest fragmentation and tree species composition affected avian functioning, as top down control of herbivorous arthropods and its cascading effects on plant growth, were influenced by forest edges and by tree species composition. My study indicates that diversifying forest stands, especially in small forest fragments, represents a management strategy that promotes both avian fitness and avian ecosystem functioning in fragmented landscapes.\\


\newpage
\thispagestyle{empty}
\chapter*{Samenvatting}
\addcontentsline{toc}{chapter}{Samenvatting}

Door intensiever landgebruik door de mens worden natuurgebieden zoals bossen versnipperd in kleine en ge\"{i}soleerde fragmenten. Dit proces van bosversnippering of -fragmentatie heeft een grote impact op biodiversiteit. Ten eerste zijn populaties van dier- en plantensoorten kleiner in kleine bosgebieden, waardoor ze een groter risico lopen om uit te sterven. Ten tweede zorgt bosversnippering voor meer, en vaak scherpe, bosranden, waar de omstandigheden anders zijn dan in de kern van een bos (bv. meer predatie of meer vervuiling). Ten derde zorgt een toegenomen afstand tussen bosfragmenten onderling voor minder uitwisseling, wat het risico op uitsterven vergroot en de kans op (re)kolonisatie verkleint. Tenslotte kan versnippering er ook voor zorgen dat soorten elkaar meer of minder tegenkomen, wat complexe ecologische netwerken, zoals tussen predatoren en hun prooisoorten, kan wijzigen. Bossen in Europa zijn niet alleen sterk gefragmenteerd, maar ook de boomsoortsamenstelling en de bosstructuur is sterk be\"{i}nvloed door mensen. Zo bestaat bijna \'{e}\'{e}n derde van de Europese bosgebieden uit \'{e}\'{e}n enkele boomsoort. Dit zorgt er voor dat het behoud van typische bossoorten in gevaar komt, en dat bos-ecosystemen minder veerkrachtig zijn bij veranderende milieuomstandigheden.\\

Het beschermen en behouden van bestaande natuurlijke bossen is een prioriteit, maar voor het behoud van bosbiodiversiteit en bos-ecosysteemdiensten (zoals onder andere grondwaterbescherming, recreatie en koolstofopslag) zijn ook nieuwe bossen en een goed beheer van secundaire bossen essentieel. Recente studies tonen aan dat een diverse boomsoortsamenstelling belangrijk is in deze context, maar het selecteren van de ideale boomsoorten is niet altijd gemakkelijk, zeker in sterk versnipperde landschappen. Er is daarom dus nood aan meer studies in versnipperde landschappen om beheerders te informeren waar bosuitbreiding best plaatsvindt, en hoe nieuwe en bestaande bossen best beheerd worden.\\

De meeste studies die de effecten van bosversnippering onderzoeken, kijken naar de aan- of afwezigheid van soorten, maar dit geeft echter niet altijd het beste beeld. Veel soorten reageren namelijk traag op veranderingen in het landschap, waardoor ze lang aanwezig kunnen blijven in kleine bossen, ook al zijn de omstandigheden daar niet meer geschikt. Door variatie in individueel gedrag of fysiologische conditie van soorten te onderzoeken kunnen effecten van bosversnippering vroeger gesignaleerd worden, en kunnen beheerders vroeger ingrijpen. Insectenetende bosvogels zijn ideale soorten voor dergelijk onderzoek aangezien hun ecologische eisen goed gekend zijn. Deze soorten spelen een belangrijke rol in het voedselweb, en zijn daardoor ook gevoelig aan bosversnippering. In gefragmenteerde landschappen is het aanbod voor insectenetende vogels sterk gewijzigd, maar ook de boomsoortsamenstelling en de bosstructuur hebben een grote invloed op het voedselaanbod in de overblijvende bosfragmenten. Vooral in periodes waar van nature weinig voedsel aanwezig is, zoals tijdens de winter, kan het bijkomende effect van bosversnippering en boomsoortsamenstelling belangrijk zijn.\\

Het doel van deze doctoraatsthesis was om te onderzoeken hoe bosversnippering en de beschikbaarheid aan voedselbronnen, gelinkt aan boomsoortsamenstelling en boomsoortdiversiteit, interageren om de fitness van insectenetende bosvogels te bepalen over hun hele jaarcyclus, en te onderzoeken wat het effect hiervan is op lagere trofische niveaus. Voor dit onderzoek gebruikte ik Koolmezen (\textit{Parus major}) en Pimpelmezen (\textit{Cyanistes caeruleus}). Beide soorten komen voor in diverse bostypes --van grote en diverse bossen tot kleine bosfragmenten en monoculturen-- en hun gedrag kan gedetailleerd opgevolgd worden gedurende het hele jaar doordat ze vaak voedertafels bezoeken en broeden in nestkasten. Verder spelen Kool- en Pimpelmezen een belangrijke rol in het bosvoedselweb, door de grote hoeveelheid insecten die ze eten.\\

Eerst bestudeerde ik het directe effect van bosversnippering en boomsoortsamenstelling op het broedsucces van kool- en pimpelmezen. Daarvoor verzamelde ik data over de hoeveelheid rupsen (die het hoofdvoedsel van beide mezensoorten vormen) in 53 onderzoeksplots langs gradi\"{e}nten van boomsoortsamenstelling en bosversnippering, en volgde ik het broedsucces op in 200 nesten van Koolmezen en 112 nesten van Pimpelmezen in deze plots. Ik toon aan dat zowel de beschikbaarheid aan voedsel, gelinkt met boomsoortsamenstelling, en de versnippering van het bos het broedsucces be\"{i}nvloeden. Vooral welke boomsoorten, en niet zozeer de hoeveelheid boomsoorten, aanwezig waren in de plots had een effect. Het hoogste broedsucces werd voor beide soorten waargenomen in monoculturen van Zomereik (\textit{Quercus robur}). Voor Koolmezen kon dit effect verklaard worden door een hogere overleving van de jongen als er meer rupsen beschikbaar waren, terwijl Pimpelmezen meer eieren legden in monoculturen van Zomereik. Fragmentatie-effecten werden enkel waargenomen in monoculturen van Beuk (\textit{Fagus sylvatica}), waar weinig rupsen te vinden zijn. In monoculturen van Beuk was het broedsucces namelijk lager in kleine bosfragmenten.\\

Ten tweede onderzocht ik hoe bosversnippering en individuele variatie in stress-levels het habitatgebruik van Koolmezen in de winter be\"{i}nvloeden, en welke effecten dit heeft op het broedsucces in de lente. Daarvoor volgde ik 68 Koolmezen in 20 bosfragmenten die varieerden in oppervlakte door middel van radio-telemetrie. Dit onderzoek vond plaats gedurende twee winters, \'{e}\'{e}n waar voedsel talrijk aanwezig was in de bossen (in de vorm van Beukennootjes), en \'{e}\'{e}n waar voedsel schaars was. Daarna voerde ik een experiment uit in \'{e}\'{e}n van deze 20 bosfragmenten, waarbij ik vogels met verschillend habitatgebruik in de winter extra eieren gaf. Op deze manier kon ik testen of habitatgebruik in de winter een effect had op het broedsucces indien de omstandigheden om jongen groot te brengen zwaarder waren. Ik toonde aan dat voedselbeschikbaarheid in bossen en bosversnippering het habitatgebruik van Koolmezen in de winter bepalen. Wanneer er voldoende Beukennootjes in het bos waren, vermeden de gezenderde mezen buiten het bos te foerageren. Koolmezen die in kleinere bossen verbleven legden grotere afstanden af om naar naburige bossen te vliegen en daar te foerageren. Uit de resultaten van het experiment bleek dat wintergedrag een sterke invloed had op het broedsucces in de lente: individuen die meer buiten het bos foerageerden in de winter slaagden er niet goed in om extra jongen groot te brengen, terwijl individuen die de hele winter in het bos bleven dit wel konden. De variatie in wintergedrag bleek echter niet gecorreleerd te zijn met de hoeveelheid stress-hormonen of met de conditie van de individuen.\\

Ten derde, bekeek ik het effect van stress op het sociale gedrag van Koolmezen in de winter. Ik onderzocht het sociale netwerk van Kool- en Pimpelmezen die naar voedertafels kwamen en op een aantal van deze Koolmezen kwantificeerde ik de hoeveelheid corticosterone, het belangrijkste stresshormoon bij vogels, dat opgeslagen zit in staartveren. Hieruit bleek dat individuen met een grotere hoeveelheid stresshormonen in veren gegroeid in de later zomer, meer sociale interacties hadden in de winter, en daardoor een centralere plaats innamen in het sociaal netwerk. Verder vond ik aanwijzingen dat meer sociale interacties mogelijk de stress levels in de winter verlagen, eventueel door een betere toegang tot sociale informatie over waar voedsel te vinden is. Dit kan dan weer effecten hebben op de conditie in de lente en op het broedsucces.\\
 
Ten vierde onderzoek ik hoe bosversnippering en boomsoortsamenstelling samen het functioneren van insectenetende vogels bepalen. Meer specifiek kijk ik naar het effect van vogels op plantenetende insecten, en welke gevolgen dit heeft op de planten zelf. Daarvoor plantte ik drie soorten jonge boompjes (Beuk, Zomereik en Amerikaanse eik (\textit{Q. rubra})) in 53 onderzoeksplots die verschilden in boomsoortsamenstelling en bosversnippering. Een deel van deze boompjes werd afgespannen met netten om te zorgen dat vogels er niet op insecten konden foerageren. Ik bekeek vervolgens hoe predatie door vogels, boomsoortsamenstelling en de afstand tot de bosrand de bladvraat op deze boompjes be\"{i}nvloedde, en hoe dit de groei van de bomen be\"{i}nvloedde. Uit dit veldexperiment bleek dat predatie door vogels een belangrijke rol speelt. Als vogels toegang hebben tot de boom was de insectenvraat lager; waardoor (bij twee van de drie soorten) de jonge boompjes ook beter groeiden. Bij Zomereiken, de soort met de meeste geassocieerde insecten, werd het effect van vogelpredatie sterker dichter bij de bosrand, vermoedelijk omdat daar meer vogels voorkomen. De boomsoortsamenstelling van het bos had enkel een effect bij Zomereik, en enkel op de boompjes waar vogels geen toegang tot hadden, met meer insectenvraat als er meer Zomereiken in de kruinlaag stonden. Deze resultaten tonen het belang van insectenetende vogels aan voor natuurlijke verjonging in bossen.\\

Ik toon aan dat factoren op landschapsschaal --namelijk bosversnippering-- en lokale factoren --voedselbeschikbaarheid, bepaald door de boomsoortsamenstelling-- samen de fitness van twee algemene insectenetende bosvogels bepalen in een sterk versnipperd landschap. Boomsoortsamenstelling en bosversnippering be\"{i}nvloeden het broedsucces zowel door effecten op voedselbeschikbaarheid in de lente, als door onrechtstreekse effecten op het wintergedrag. Bosversnippering heeft ook effect op de controle die vogels uitoefenen op plantenetende insecten, met een hogere predatiedruk op insecten bij Zomereiken dicht bij de bosrand. Verder beperkte een diverse boomsoortsamenstelling bladvraat door insecten als vogels niet aanwezig waren. Mijn onderzoek toont aan dat het streven naar een diverse boomsoortsamenstelling zowel de fitness als het functioneren van insectenetende bosvogels bevordert, in het bijzonder in kleine bosfragmenten.


\cleardoublepage
\hbox{}
\clearpage
%%%%%%%%%%%%%%%%%%%%%%%%%%%%%%%%%%%%% Chapter 1 - General introduction %%%%%%%%%%%%%%%%%%%%%%%%%%%%%%%%%%%%%%%%%

\CenterWallPaper{1}{CH1.jpg}
\newpage{\thispagestyle{empty}\cleardoublepage}
\ClearWallPaper
\pagestyle{mainmatter}\chapter{General introduction} \label{chap:Introduction}
\chaptermark{General introduction}
\lettergroup{\thechapter}	

\begin{flushright} \color{gray}Daan Dekeukeleire\color{black}\end{flushright}

\vspace*{\fill}
\noindent \color{gray} $\lhd$ One of the research plots in a beech forest, photo by Irene Lantman.

\color{black}

\newpage
	
	\section{Biodiversity loss and habitat fragmentation}
	
	
Human actions have a profound impact on biodiversity \citep{Cardinale2012, Pimm2014, Ceballos2015}. Over the last decades and centuries, species and populations are being lost across the globe at a high and ever increasing rate. Such biodiversity loss impairs the functioning\footnote{Ecosystem functions are ecological processes controlling fluxes of energy, nutrients and matter through the environment, such as primary production, decomposition, nutrient cycling and predation rates \citep{Cardinale2012}.}  and stability of ecosystems, both at the local and global level \citep{Hooper2005, Cardinale2012, Lefcheck2015}. Biodiversity loss also imperils the services these ecosystems provide to society (`ecosystem services'; \citealt{Costanza1997}), such as food security \citep{Klein2007, Maine2015}, disease prevention \citep{Gibb2020}, and human wellbeing \citep{Marselle2019}.\\

Biodiversity loss is caused by multiple, often interacting and synergistic, drivers, such as climate change, pollution, invasive species and land use change \citep{Bongaarts2019}. Among these, land use change is widely recognized as a main driver \citep{Pimm2014, Haddad2015}. Land use change results in habitat loss and habitat fragmentation, the process whereby continuous habitat is subdivided into smaller and more isolated fragments separated by a matrix of human-transformed land cover. Although there is still vigorous debate in the scientific literature on the relative importance of the size and the spatial configuration of habitat for biodiversity patterns \citep{Fahrig2013, Hanski2015, Fahrig2017, Haddad2017, Fletcher2018, Fahrig2019, Saura2021}, the consensus is that spatial configuration of habitat (i.e. fragmentation) impacts species occurrences and biodiversity \citep{Fischer2007, Haddad2015, Chase2020}. Moreover, negative impacts of fragmentation are especially severe for habitat-specialist and threatened species \citep{Pfeifer2017}.\\

Habitat fragmentation can impact biodiversity through several, often interacting, mechanisms (also see figure \ref{fig1-1}). First, larger fragments support more species and more individuals than small fragments. This species-area relationship is one of the oldest and most documented patterns in ecology (e.g. \citealt{Arrhenius1921, Preston1962, MacArthur1967}). As population sizes decline with decreasing fragment size, populations experience higher extinction risks due to environmental, demographic and genetic stochasticity \citep{Lande1993, Andren1994}.\\

Second, habitat fragmentation leads to changes in ecological processes and habitat quality in remaining patches (`ecosystem decay'; \citealt{Chase2020}). Habitat fragmentation results in more, and often sharper edges between habitat and matrix, and an increased edge-to-core ratio in the remaining habitat fragments \citep{Fischer2007}. Such edges differ from the core in abiotic conditions (such as microclimate and pollution) and biotic conditions (such as predation pressure or competition). The magnitude of edge effects depends on the type of edge and the surrounding land use \citep{Ries2004}, but such effects can penetrate deep into the remaining habit. For instance, in old-grown forests, edges negatively affect the breeding success of great tits (\textit{Parus major}), a common insectivorous bird, and such effects were detectable up to 500 m from the edge \citep{Wilkin2007}. Nevertheless, the effects of edges on species occurrences are highly variable, and numerous studies even document positive relationships \citep{Fahrig2017}. For instance, bird communities in Western European forest edges are often observed to be more diverse than in the core of the forest \citep{Terraube2016, Melin2018}. However, a recent meta-analysis indicates that negative edge effects mainly occur in specialists or species of conservation concern, while positive responses are more likely to manifest for generalists and invasive species \citep{Pfeifer2017}.\\

Third, decreased connectivity between (sub)populations in fragmented landscapes negatively effects biodiversity. For example, a long-term landscape-scale experiment shows that plant species richness is greater in connected fragments compared to isolated fragments of equivalent size \citep{Haddad2017}. Spatial isolation can lead to reduced dispersal and gene flow between (sub)populations, which can reinforce adverse effects of demographic and genetic stochasticity, and lead to population extinctions. For example, translocation experiments in brown treecreepers (\textit{Climacteris picumnus}), an Australian forest-understorey bird, show that impaired female dispersal in fragmented landscapes is the main cause of lower pairing success in small fragments \citep{Cooper2002}. Moreover, meta-population theory indicates that isolated fragments are less likely to be recolonized when a population in a fragment goes extinct \citep{Hanski1998, Hanski1999}. As a classic example, long-term surveys of Glanville fritillary (\textit{Melitaea cinxia}) occurring in a network of dry grassland patches in Finland show that isolated patches go extinct more often, and are (re)colonized less often compared to larger or well-connected patches (e.g. \citealt{Hanski2017}).\\

Fourth, fragmentation can affect the structure and stability of complex ecological networks that emerge from the interactions between different species, such as plants and their pollinators or seed dispersers, predators and their prey or hosts and their parasites \citep{Hagen2012, Grass2018}. Changes in encounter rates between species in small fragments can therefore increase extinction risks in other species. There is overwhelming evidence that species are not equally important for ecosystem functioning and that some exert disproportionate effects. These include species at high trophic levels (e.g. predators) and abundant or generalist species (reviewed in \citealt{Hagen2012}). For example, hydro-electric works in Venezuela created habitat fragments that were too small for predators to thrive, resulting in exceptionally high levels of herbivores. This in turn led to declining population sizes and local extinctions in plant species \citep{Terborgh2001, Terborgh2006}.\\
	
	
	\begin{figure}[th!]
	\begin{center}
		\includegraphics[width=\textwidth]{fig1-1_v3.png}
	\end{center}
		\caption{Habitat fragmentation can impact species through species-area relationships, edge effects (egdes: light green; core: dark green) can have negative or positive effects), isolation effects or changing biotic interactions in small fragments due to local extinctions (represented by species coloured in grey). Figure after \citep{Boeraeve2019}.  \label{fig1-1}}
	\end{figure}
\clearpage	
	
	\section{Forest fragmentation and management}
	
Effects of historic habitat loss and fragmentation patterns are especially severe in forest ecosystems in Western and Central Europe due to the large-scale deforestation \citep{Bastrup-Birk2016}. Since the introduction of agriculture circa 6000 BP, forest was transformed into agricultural fields and into grazing land for cattle, resulting in large scale forest fragmentation. During this long history of deforestation in Europe, there were periods with renewed cultivation of forests, mainly due to increased demands for firewood and timber \citep{Tack1993}. In many regions in Europe, such recent forests were created far from fragments that were continuously forested. For example in northern Belgium, forest fragments have a median size of only 1.5 hectare, and the number of fragments has strongly increased over the last 250 years. Only 16 \% of the total forest cover remained continually wooded the last 250 years, and many of the small and recent forests were created on former wetlands or heathlands, spatially isolated from other forest fragments \citep{DeKeersmaeker2014}. Many specialized forest species can only colonize such newly created forest patches very slowly, leading to depauperate communities in recent forest fragments (e.g. \citealt{Desender1999, Honnay2002, Boeraeve2018}).\\

In general, European forests are not only highly fragmented, also the tree species composition and structure is highly impacted due to human management. Of the remaining forest cover in Europe, more than 95 \% is currently managed, and only 0.7 \% of forest can be regarded as `undisturbed' or primary forest \citep{Sabatini2018}. Traditional forest management actions such as coppice with standards management or wood pasture management can lead to structurally diverse forests with a high diversity, and can promote rare and specialist forest species \citep{Hedl2010, Mullerova2015, Vandekerkhove2016}, but such management was almost completely abandoned in European forests by the middle of the 20th century. Instead, forests were managed solely to maximize timber and wood production, and therefore largely converted to even-aged stands with relatively short rotation times. The tree species composition in such forests is highly simplified, with more than 29 \% of the forest cover composed of a single tree species \citep{Barsoum2016}. It is thus not surprising that only 14 \% of forests protected under the EU's Habitat Directive --which aims to conserve a wide range of threatened species-- was estimated to have a `favourable' conservation status \citep{EEA2020}.\\

While conservation of the remaining natural habitat should be a priority, reforestation and adequate management of secondary forests will be essential to safeguard biodiversity and the provision of ecosystem services \citep{Perring2018}. Reforestation has the potential to sequester carbon and thus is generally viewed as an important part of strategies to mitigate climate change \citep{Bastin2019}. For example, in northern Belgium, the Flemish administration aims to create 10,000 hectares of new forests in the next decade, of which 4000 hectares in the next four years (\url{www.bosteller.be}). At the same time, management of currently forested areas has shifted from a traditional focus on optimizing economic return in the form of timber production, towards accommodating multiple ecosystem services (e.g. water and soil protection, carbon sequestration and recreational value) and conserving biodiversity \citep{Pawson2013, Puettmann2015, Coll2018}. Recent studies have demonstrated the importance of tree species richness and composition in determining multiple ecosystem functions in forest systems in different environments \citep{Ratcliffe2017, Baeten2019, Hertzog2019}. Yet, selecting the most beneficial species or species mixtures for ecosystem functioning and biodiversity is complicated by the fact that there is no evidence for a super-species or super-species-mixture that provides high levels of functioning and diversity across all contexts \citep{VanderPlas2016}. Thus, there is a need for empirical studies in real-world forest landscapes in order to inform policy makers and practitioners of where reforestation should take place, and of how current and future forest habitats should be managed \citep{Watts2016, Coll2018}.

\clearpage	
	\section{Research gaps in forest fragmentation research}
	
Although forest fragmentation effects on biodiversity receive much scholarly attention, most studies addressing the effects of forest fragmentation use occurrence or count data (reviewed in \citealt{Fardila2017}). Such measures constitute time-lagged responses that may obscure the actual status of populations in fragmented landscapes \citep{Ewers2006}. For instance, several species of ancient forest plants persist in small forests fragments in Belgium more than a century after fragmentation \citep{Vellend2006}, but genetic diversity and recruitment of many of these decline in small fragments \citep{Jacquemyn2002, Jacquemyn2007, Vandepitte2007}. Such extinction debt in small forest fragments can also present opportunities for conservation. As long as a population that is expected to become extinct persists, there is time for conservation measures preventing extinction, such as forest management and landscape restoration \citep{Kuussaari2009}. Yet, identifying these populations, and which management actions should be prioritized, is no trivial matter.\\

Studies focusing on how fragmentation impacts individual behaviour and physiological condition can provide early warnings of population decline, and inform management actions \citep{Cooke2013, Fardila2017}. For example, common toads (\textit{Bufo bufo}) at reproductive sites with small and isolated forests in the surrounding landscape have a lower body condition and higher levels of the stress hormone corticosterone compared to those at sites surrounded by large or many forest fragments, likely impacting reproduction and long-term population persistence \citep{Janin2011}. Yet, such subtler effects of forest fragmentation on individual behaviour and physiology are typically more difficult to measure and can pose methodological challenges, and are therefore rarely assessed \citep{Fardila2017}. 

\clearpage	
	\section{Birds as sentinel species in fragmented forests}
	
Forest-dependent birds are well-suited taxa to serve as sentinel species in real-world landscapes, i.e. species that respond to environmental change in a timely and measurable way, and can indicate an otherwise unobserved change in the ecosystem \citep{Hazen2019}.  Due to their high trophic position in the forest food web, predators such are insectivorous birds are expected to be impacted more by forest fragmentation \citep{VanNouhuys2005}.\\

Forest fragmentation strongly impacts the distribution of critical resources for forest birds. In small fragments, insect food resources can be limited, and this has earlier been shown to limit avian reproductive success \citep{Robinson1998, Burke1998}. For example, in South-Eastern Australia, arthropod numbers on the forest floor were shown to be almost twice as high per unit area in larger forest fragments compared to small fragments. Consequentially, female Eastern yellow robins (\textit{Eopsaltria australis)}, an Australian understorey passerine, can invest less in eggs and can forage less efficiently when breeding in small forest fragments, leading to a lower reproductive success compared to their conspecifics in larger fragments \citep{Zanette2000}. Resource availability does not only depend on the fragmentation context, but also on the local conditions and habitat quality in the remaining forest fragments, such as tree species composition and structural diversity \citep{Mortelliti2010, Humphrey2015}. Both the identity and diversity of trees in the breeding territory can influence the diversity and abundance of arthropods, and thus the food availability for insectivores (e.g. \citealt{Shutt2018, VanSchrojensteinLantman2018}). The number and abundance of arthropods associated with different tree species strongly differs \citep{Kennedy1984, Brandle2001}. For instance, in west-european ecosystems, native deciduous oak species, such as pedunculate oak (\textit{Quercus robur} L.) and sessile oak (\textit{Q. petraea} Liebl.) support several hundred arthropods species, including many lepidoptera, while other species, such as hornbeam (\textit{Caprinus betulus} L) only support circa fifty species. Along these lines, the breeding success of insectivorous birds such as great and blue tits has been shown to be positively related to the abundance of arthropod-rich tree species in the breeding territory \citep{Wilkin2007a, Amininasab2016, Shutt2018}.  Although there are exceptions, e.g. species that are closely related to native species, non-native tree species generally support much lower numbers of arthropod prey \citep{Goßner2004}, and there abundance is negatively related with breeding performance of insectivorous birds (e.g. \citealt{Narango2017} , \citealt{Mackenzie2014}). Next to tree species identity, several studies also show that a greater tree species diversity increases the diversity and abundance of arthropods \citep{Fuentes-Montemayor2012} and the temporal spread of prey availability \citep{Kennedy1984}.\\

Similarly, forest configuration can impact daily movement and social behaviour \citep{Cosgrove2018, He2019}. How fragmentation impacts movement behaviour often depends on the physiological condition of individuals, measured in its most crude form as body condition, and is often mediated by hormonal changes in response to biotic and abiotic stressors \citep{Goossens2020, Creel2013}. Along these lines, compared to individuals living in continuous forests, Carolina chickadees (\textit{Poecile carolinensis}) living in disturbed forests were observed to have a lower body mass and higher faecal corticosterone levels, impacting their foraging behaviour \citep{Lucas2006}. Hormonal changes due to abiotic or biotic stressors also influence social behaviour. Elevated levels of corticosterone decrease neophobia and dominance interactions, and increase the use of social information in songbirds \citep{Spencer2007, Boogert2013}.\\

For resident bird species, forest fragmentation effects can be especially important during winter, when resource availability is low and energy demands are increased (reviewed in \citealt{Williams2015}). Outside of the breeding season, many forest-dependent bird species forage in loose flocks and increase their home range, and, in fragmented landscapes, cross open habitats more often \citep{Kattan1994, Maldonado-Coelho2004}. Moreover, forest fragmentation can increase natural fluctuations in microclimatic conditions, e.g. through colder temperatures at forest edges \citep{Latimer2017}. Such conditions can lead to a lower body condition or increased stress, which can cause carry-over effects on subsequent periods in the annual cycle. Due to methodological and technological advances, such as light-weight radio transmitters or Passive Integrated Transponder-tags, individual birds can be followed meticulously year-round, thus allowing researchers to asses interactions between different periods. Yet, due to the persistent view that considers effects during the breeding season to be paramount in comparison to other periods, such full annual cycle approaches are still rare in fragmentation research \citep{Marra2015}.\\

Several markers can be used to investigate the effects of forest fragmentation and forest composition on individual condition. A first and simple group of marker for stress are body condition indices, i.e. indices that correct mass for size, such as the scaled mass index (SMI; \citealt{Peig2009}). Although still often used in the literature, the use of body condition as a marker for stress has recently been criticized (reviewed in \citealt{Wilder2016}). Body condition values can be difficult to interpret as such values are not always directly positively related to individual fitness and can strongly vary over short time-scales. For example, during winter, fat load of great tits is not maximized, but optimized, at least partly because of a need to trade off starvation risk against predation risk, and can be adjusted over the course of hours of a single day \citep{Macleod2005, Gosler1996}. Recently, physiological markers, such as hormone levels are increasingly used as proxies for environmentally induced stress. Hormone levels are directly related to fitness in animals, and can therefore inform understanding of the physiological mechanisms underlying individual responses in a wide range of behavioural and ecological studies \citep{Wilder2016}. In vertebrates, energetically demanding conditions trigger the release of glucocorticoids mediated by the hypothalamic-pituitary-adrenal (HPA) axis \citep{Wingfield2013}. In birds, the dominant glucocorticoid is corticosterone (CORT), which mediates physiological processes that steer behavioural and metabolic changes and, in turn, enables individuals to better cope with environmental challenges. Yet, long-term chronic elevations of CORT are associated with considerable fitness costs, such as a decreased immunocompetence \citep{Wingfield2013, Creel2013}. Measuring long-term baseline CORT levels is difficult, as levels in faecal samples or blood plasma rapidly increase when an individual is captured or handled, and thus expose acute stress responses rather than the response to long-term chronic stress \citep{Romero2002}. Methodological advances in the last decade now allow the measurement of CORT levels stored in feathers. CORT is passively incorporated in feathers during growth, and, when corrected for growth rate, feather CORT values therefore represent an integrated measure of plasma corticosterone levels during the entire period of feather growth \citep{Bortolotti2008, Romero2016}. Recent experimental studies have confirmed a direct link between daily energetic demands during the period of feather growth and the level of CORT stored in feathers \citep{Johns2018}. Moreover, once the feather is fully grown, CORT levels in feathers are stable over time and can easily sampled many months after the feather has grown, and therefore, be used to investigate links between different parts of an individual's annual cycle \citep{Bortolotti2010}.\\ 

Forest fragmentation and forest composition are also expected to shape the predatory impact of birds on lower trophic levels. Cascading effects of arthropod predation can then benefit plant fitness and growth (reviewed in \citealt{Mantyla2011}). For example, white oak saplings (\textit{Quercus alba} L.) from which birds were excluded, suffered 12 \% more herbivory than control saplings, and produced circa 20 \% less biomass in the subsequent growing season \citep{Marquis1994}. Such trophic cascades therefore have important consequences for ecosystem services, such as tree regeneration, wood production and crop yield, both in forests and in surrounding habitats \citep{Mantyla2011, Maas2016}. Yet, which ecological mechanisms influence trophic cascades is still less well understood, especially in complex real-world food webs \citep{Schmitz2000, Vidal2018}. Forest fragmentation and forest composition can be expected to influence avian predation through several mechanisms. First, a growing number of experimental studies show that the richness of predators enhances prey suppression through niche complementarity, and such effects generally increase with the spatial and temporal scale considered \citep{Duffy2002, Casula2006, Griffin2013}. According to the enemies hypothesis, the number of predators is higher in more diverse plant communities due to a higher diversity of prey species and higher abundance of additional resources such as refuges \citep{Letourneau1987, Root1973, Russell1989}. Indeed, avian diversity is often positively correlated to forest composition and tree species diversity \citep{Giffard2012, Charbonnier2016}, but also to the size of forest fragments (e.g. \citealt{Andren1994}). Along these lines, a recent study showed that a higher abundance and higher functional and phylogenetic diversity of birds in tropical mixed forests resulted in higher attack rates on artificial caterpillars \citep{Nell2018}. Second, forest fragmentation and forest diversity can also influence individual variation within a predator species, and therefore its predatory impact. For instance, forest fragmentation has been correlated to activity levels, space use or food intake in great and blue tits \citep{Naef-Daenzer2000, Bueno-Enciso2016}. Yet, how forest fragmentation and forest composition interact has seldom been considered \citep{Vidal2018}.\\

\clearpage
	\section{Objectives and research questions}
	
The overarching aim of this dissertation is to assess how forest fragmentation and resource availability, shaped by tree species composition and diversity, synergistically affect the fitness of avian insectivores across their full annual cycle, and how this feeds back to lower trophic levels. Note that I investigate the effects of the pattern of forest fragmentation that resulted from the historical process of the subdivision of continuous forest, and therefore, I estimate fragmentation effects on the plot level. I do not investigate habitat fragmentation by comparing different `landscape windows' with different habitat configuration for a given total amount of habitat (fragmentation per se \textit{sensu} \citealt{Fahrig2017, Fahrig2019}). First, I investigate the direct effects of forest fragmentation and food availability during spring on reproductive performance. Second, I investigate how conditions during winter and stress hormone levels affect habitat use and movement behaviour during winter, and how these effects in winter are linked to performance in spring through carry-over effects. Third, I zoom in on stress hormone levels, and how these influence social behaviour during winter. Finally, I investigate the effect of forest fragmentation and resource availability on avian functioning.\\

More specifically, this leads to four main research questions:
\begin{itemize}
	\item Q1: how do forest fragmentation and tree species composition interact to shape resource availability during spring and avian reproductive performance?
	\item Q2: how do forest fragmentation, resource availability during winter and stress hormone levels interact to shape avian movement behaviour and habitat use in winter, and does this lead to carry-over effects on avian reproductive performance in spring?
	\item Q3: how do stress hormone levels influence avian social behaviour during winter, and does this social behaviour influence subsequent stress levels and fitness?
	\item Q4: how do forest fragmentation and tree species composition interact to shape ecosystem functions related to birds, specifically the top down control of arthropod herbivores? 
\end{itemize}

Each of these research questions is addressed in a separate dissertation chapter (see conceptual figure \ref{fig1-2}).
	
\clearpage	
	\section{Dissertation outline}
To address the research questions of this dissertation, I first investigated the communities of vertebrate insectivores (i.e. birds and bats) in our study area (see Box B, page \pageref{boxb}), and selected two species in this community as a study system, namely the great tit (\textit{Parus major}) and blue tit (\textit{Cyanistes caeruleus}). These species are highly suitable to the study of the year-round effects of forest fragmentation and resource availability on avian reproduction and ecosystem functioning, for three main reasons. First, both species prefer forest habitats, especially during the breeding season, although they also utilize other habitats, such as (sub)urban gardens, parks and hedgerows in agricultural areas \citep{Gosler1993, Stenning2018}. Great and blue tits occur in a wide range of forest types, from large old-grown deciduous forests to small fragments and plantations. Second, individuals can be monitored year-round, as both species are resident in Western Europe (although migrants from Eastern and Northern Europe do visit our region in variable numbers; \citealt{Harrap1996}). Yet, habitat use varies over the course of their annual cycle. In autumn and winter, tits form social groups that vary in size and composition and forage mainly on seeds and nuts across larger areas. During this period, they often visit artificial feeders, allowing detailed observations of their winter behaviour. In the spring breeding season, tits become territorial and forage in the immediate surroundings of their nest site \citep{Naef-Daenzer2000}. Both species naturally breed in tree cavities, but readily adopt nest boxes, allowing the monitoring of their reproductive performance in great detail \citep{Lambrechts2010}. Third, during spring and summer, tits prey upon large numbers of caterpillars, and as such, play an important role in the forest food web \citep{Sanz2001, Mols2007}. Using these two study species throughout the dissertation, allows a syntheses across the research chapters and, therefore, a more thorough understanding of the complex relationships between environmental changes, behaviour, and animal condition. Moreover, as great and blue tits have been used as study species in ecological, behavioural and physiological studies for several decades, there is also a large amount of literature available that can be used for comparisons.\\

In \textbf{chapter \ref{chapter2}}, I investigate how forest fragmentation and tree species composition shape resource availability and avian reproductive success (Q1). In order to do so, I collected data on the biomass of caterpillars, the main food source of great and blue tits, in 53 research plots across independent gradients of forest fragmentation and tree species composition, and monitored the reproductive performance in 200 nests of great tits and 112 nests of blue tits in these plots.\\

In \textbf{chapter \ref{chapter3}}, I use a combined observational and experimental design to study how forest fragmentation, resource availability and stress levels affect winter habitat use, and how this impacts subsequent reproductive performance (Q2). First, I investigated winter habitat use by radio tracking great tits in 20 forests fragments varying in size across two winters. One winter, food was abundantly available in the form of beech mast, the other winter, food was scarce. Second, I experimentally increased the brood size of individuals differing in winter habitat use, to investigate if variation in winter movement leads to reproductive costs when conditions to raise young are challenging.\\

In \textbf{chapter \ref{chapter4}}, I investigate how stress hormone levels influence winter social behaviour (Q3). I inferred the winter social network of great and blue tits visiting artificial feeders at one of the studied forest fragments. On a subset of great tits, I quantified feather corticosterone as an archive of stress experienced during feather growth. I relate corticosterone in original feathers (grown in late summer) to social network position in winter, and in turn, relate social network position in winter to corticosterone in induced feathers (grown during winter).\\
 
In \textbf{chapter \ref{chapter5}}, I investigate how forest fragmentation and tree species composition jointly influence the functional role of insectivorous birds (Q4). I focus on top-down control of arthropod herbivores on saplings. In a field experiment, I planted saplings of three tree species (beech, red oak and pedunculate oak) in research plots varying in tree species composition and fragmentation context, and excluded avian predators from a subset of these. I investigate how avian top-down control, overstorey tree species composition and edge effects jointly shape herbivory levels and growth of these saplings.\\

In \textbf{chapter \ref{discussion}}, the general discussion, I synthesize and link the results of the research described in the previous chapters. I also propose guidelines for forest fragment management, and indicate directions for future research.\\

The research in chapters \ref{chapter2} and \ref{chapter5} (as well as the radio tracking in chapter \ref{chapter3}) was performed within the Treeweb platform, a collaborative research platform which is used to investigate the impact of tree species diversity, tree species composition and forest fragmentation upon the functioning of different trophic levels by an interdisciplinary team at Ghent University. This platform is discussed in more detail in \textbf{Box A}, page \pageref{boxa}. Next, in \textbf{Box B}, page \pageref{boxb}, I discuss the communities of vertebrate insectivores, namely birds and bats, in these plots and explore the relationship between insectivore diversity, community and tree species composition and fragmentation.\\

	
	\begin{figure}[bh!]
		\begin{center}
			\includegraphics[width=\textwidth]{fig1-2_v3.png}
		\end{center}
		\caption{Conceptual representation of the four research chapters of the dissertation, with chapter numbers in the figure. Full arrows: investigated relationships; dashed arrows: hypothesized relationships.  \label{fig1-2}}
	\end{figure}
	
	\clearpage
	\newpage
	
%%%%Section BOX A%%%%%%	
	\section*{Box A: The Treeweb platform -- functional \mbox{biodiversity} research across trophic levels in a fragmented landscape}\label{boxa}
	\addcontentsline{toc}{section}{Box A}
	\vspace{1.2cm}
	\textbf{Partially adapted from:}\\
	De Groote S. et al. 2017. \textit{Plant Ecology Evolution} 150, 229-239. \url{https://doi.org/10.5091/plecevo.2017.1331}
	\vspace{1.2cm}
	
Understanding how a higher diversity of species, genes and communities affects ecosystem functioning (Biodiversity Ecosystem Functioning; BEF) is a major question in ecological research. During the past decades, this relationship has mainly been investigated in controlled lab and field experiments. While earlier field experiments mainly focused on grasslands systems (e.g. the Cedar creek experiment; \citealt{Tilman2006}), there has been more attention for complex ecosystems such as forests as of late (e.g. \citealt{Paquette2018}). However, BEF field experiments in forests have only recently been established, and studies in grassland systems show that the importance of biodiversity for ecosystem functioning increases with ecosystem succession \citep{Meyer2016}. This indicates the need for studies in more developed systems. Moreover, current experimental studies often minimize variation in factors other than tree-species diversity, such as landscape and fragmentation context. These factors could, however, be critically important in understanding BEF relationships \citep{DeLaender2016, Liu2018}.\\

The Treeweb platform was established precisely in order to unravel the effects of tree species diversity, tree species composition and forest fragmentation on ecosystem composition and functioning in mature forests. This platform consists of a network of 53 research plots in 19 forest fragments in northern Belgium (figure A.1). The design largely follows the European-scale FunDivEUROPE research platform \citep{Baeten2013}, but in the case of the Treeweb platform, plots were selected to explicitly differ in the level of fragmentation context (table A.1, figure A.2).\\

These research plots were established across a gradient in fragmentation, i.e. some located in the forest interior in large, well-connected forests, and others in small, isolated forests and/or close to the forest edge. We used a high resolution database for habitats and land uses: the Biological Valuation Map, which covers more than 100 habitat and vegetation types \citep{Vriens2011}. To define forest fragments, we selected all forest types including coniferous stands and recent poplar plantations. The minimum distance between two plots within a single fragment was 71.2m. In order to characterize the plot-based level of fragmentation, we calculated twelve commonly used indices that capture different aspects of fragmentation, from local effects of edges, to isolation from neighbouring forest fragments (see table A.2; \citealt{McGarigal2015}). On the basis of this data, a Principle Component Analysis was run, which shows that circa 51 \% of the variation is captured by the first principal axis, and 17 \% in the second principle axis (see figure A.2). The variables with the highest loadings for the first and second axis of the PCA are `forest area in a radius of 500 m of the plot' and `shape index of the forest fragments wherein the plot was situated', respectively. The first PCA axis thus represents a gradient form plots close to the forest edge in small fragments to plots in the forest interior of large fragments. The second axis represents a gradient for plots in compact to elongated forests.\\

Across this gradient of forest fragmentation context, 53 research plots were selected. The Treeweb plots are monocultures or mixtures of three locally important focal tree species: the native pedunculate oak (\textit{Quercus robur} L), beech (\textit{Fagus sylvatica} L) and the invasive red oak (\textit{Q. rubra} L). Each of the seven possible tree species combinations is included in the platform: three monocultures, three mixtures of two species, and one three-species mixture (figure A.1).  Complete dilution is avoided in the design, allowing separation between tree species identity and diversity effects. The plots were established in early 2014. Plots measure 30 $\times$ 30 m and are surrounded by a buffer zone of the same tree species composition of minimum 10 m. As such, plots reach the minimum size for several measurements (e.g. herbivory, litter input) to be ecologically meaningful \citep{Baeten2013}. The location of a plot was chosen with the purpose of minimizing the admixture of non-target tree species ($<$ 5 \% of the basal area) and of maximizing the evenness of the target tree species in mixtures. In order to minimize confounding environmental factors, the 53 selected study plots are all situated on a similar dry sandy-loam soil, have the same land-use history (continuously forested since at least 1850), the same management (high forest, no active management in the last 15 years) and a similar developmental stage (mature stands; $>$ 60 years old).\\

As such, the Treeweb platform makes it possible to study interactions between local variables such as tree species composition and diversity, and landscape context. This dissertation presents the avian research within the Treeweb platform. 

	
	
%% Table A.1
\clearpage
\begin{landscape}
%\begin{sidewaystable}
\begin{table}
	\begin{center}
		\begin{footnotesize}
			\caption*{\textbf{Table A.1}: summary information of the 53 Treeweb plots, including relative basal area of the three focal tree species (Fsyl = European beech (\textit{Fagus sylvatica} L), Qrob = pedunculate oak (\textit{Quercus robur} L), Qrub = red oak (\textit{Q. rubra} L)), fragment area and Euclidian distance to the nearest forest edge. Values given are the mean (minimum--maximum).}  \label{TabA1}
			
	\begingroup
	\setlength{\tabcolsep}{10pt} % Default value: 6pt
	\renewcommand{\arraystretch}{1.5} % Default value: 1
	\begin{tabular}{l c c c c c c c}
				
				\toprule
				& \multicolumn{7}{c}{\textbf{Tree species composition}} \\
				& \textbf{Qrob} & \textbf{Fsyl} & \textbf{Qrub} &  \textbf{Qrob-Fsyl} & \textbf{Qrob-Qrub} & \textbf{Fsyl-Qrub} & \textbf{Qrob-Fsyl-Qrub}\\
				\hline
				Number of plots & 8 & 8 & 8 & 8 & 8 & 6 & 7\\
				\cline{2-8}
				
				\multirow{2}{*}{Basal area Qrob (\%)} & 98.8 & -- & --& 45.3  & 46.5  & -- & 27.5 \\
				& (95--100) &&&(33.8--57.4) & (27.6--59.9) &&(17.9--38.7)\\
				\cline{2-8}
				
				\multirow{2}{*}{Basal area Fsyl (\%)} &  \multirow{2}{*}{--} & 99.6 & \multirow{2}{*}{--} & 53.8  & \multirow{2}{*}{--} & 47  & 33.7 \\
				&& (96.9--100) &&(42.6--65.2)&&(35.8--59.3)&(24.1--44.8)\\
				\cline{2-8}
				\multirow{2}{*}{Basal area Qrub (\%)} &  \multirow{2}{*}{--} & \multirow{2}{*}{--} & 98.1 & \multirow{2}{*}{--}  & 49.7 & 55.3  & 35.99 \\
				&&& (92.1--100) &&(36.7--67.5)&(40.7--64.2)&(25.3--49.4)\\
				\cline{2-8}
				\multirow{2}{*}{Fragment area (ha)} & 39.1 & 75.7 & 55.9 &48.0&30.7&73.0&47.3\\
				&(3--90.4)&	(3.5--90.4)&(35.5--90.4)&(1.3--90.4)&(19.3--53.4)&(29.8--90.4)&(10.7--90.4)\\
				\cline{2-8}
				\multirow{2}{*}{Closest edge (m)}& 52.1 & 75.7 & 92.2 & 87 & 75 & 104.4 & 70.5\\
				&(21.6--101)	& (43--215.5) & (7--200)	&	(33.6--197) & (12.9--199) & (53.1--171) & (25.9--135.7)\\
				\bottomrule
				
				

				
			\end{tabular}\endgroup
		\end{footnotesize}
	\end{center}
%\end{sidewaystable}
\end{table}
\end{landscape}
\clearpage

%Figure A1
\begin{figure}[th]
	\begin{center}
		\includegraphics[width=\textwidth]{figA1.png}
	\end{center}
	\caption*{\textbf{Figure A.1:} Map of the study area in northern Belgium. Green patches represent forest fragments, and the 53 Treeweb plots are indicated by coloured symbols, showing plot-level tree species composition. Fsyl $=$ beech (\textit{Fagus sylvatica}), Qrob $=$ pedunculate oak (\textit{Quercus robur}), Qrub $=$ red oak (\textit{Q. rubra}). C: Detailed map of a three species-mixture plot (30 $\times$ 30 m). Polygons represent crown projection and are coloured according to tree species.  \label{FigA1}}
\end{figure}


%Figure A2
\begin{figure}[th]
	\begin{center}
		\includegraphics[width=\textwidth]{figA2.png}
	\end{center}
	\caption*{\textbf{Figure A.2:} biplot the first two PCA axes of 12 fragmentation metrices (see definition in table A.2) encapsulating different aspects of fragmentation on the fragment and plot level. Colour and shape of the 53 Treeweb plots indicates plot-level tree species composition (see legend in figure A.1). \label{FigA1}}
\end{figure}


%%Table A2%%%
\clearpage
%\begin{landscape}
	%\begin{sidewaystable}
	\begin{table}
		\begin{center}
			\begin{footnotesize}
				\caption*{\textbf{Table A.2}: definition, mean value and range of twelve metrices for fragmentation context of the different Treeweb plots.}  \label{TabA2}
				
				\begingroup
				\setlength{\tabcolsep}{10pt} % Default value: 6pt
				\renewcommand{\arraystretch}{1.5} % Default value: 1
				\begin{tabular}{p{2cm} p{6cm} c l}
					
					\toprule
					\textbf{Fragmentation variable} & \textbf{Definition} & \textbf{Mean} & \textbf{Range} \\
					\hline
					Frag\_area & Area of the forest fragment in which the plot is situated (ha) & 47.68 & 1.31 -- 90.36\\
					Frag\_shape & Shape index calculated as the ratio of fragment perimeter divided by the minimum perimeter possible for a maximally compact fragment of the same size (unit less) & 2.13 & 1.03 -- 3.58\\
					Frag\_prox & Fragment proximity index calculated as the sum of fragment area divided by the nearest edge to edge distance squared between a fragment and the focal fragment for all fragments whose edges are within 1000 m of the focal fragment (unit less) & 48.21 & 1.04 -- 165.50\\
					Frag\_nearestneighb & Euclidian distance to nearest forest fragment, from border to border (m) & 84.54 & 15.81 -- 301.87\\
					Frag\_edgelenght & Length of the total edge of the forest fragment in which the plot is located (km) & 5.88 & 0.51 -- 10.93\\
					Frag\_edgedens & Edge density calculated as the edge length divided by the fragment area (m/ha) & 147.68 & 76.22 -- 389.93\\
					Plot\_forest100m & Forest cover in a 100 m radius of the centre of a plot (ha) & 2.61 & 1.31 -- 27.03\\
					Plot\_forest300m & Forest cover in a 300 m radius of the centre of a plot (ha) & 16.35 & 3.16 -- 27.03\\
					Plot\_forest500m & Forest cover in a 500 m radius of the centre of a plot (ha) & 30.92 & 3.23 -- 51.67\\
					Plot\_edge100m & Total length of all forest edges within a 100 m radius of the centre of a plot (m) & 176.74 & 0.00 -- 490.54\\
					Plot\_edge300m & Total length of all forest edges within a 300 m radius of the centre of a plot (km) & 1.63 & 0.65 -- 2.93\\
					Plot\_closestedge & Shortest Euclidian distance from the centre of the plot to the closest forest edge (m) & 78.79 & 6.97 -- 215.49\\
					
					
					
					
					\bottomrule
					
					
					
					
				\end{tabular}\endgroup
			\end{footnotesize}
		\end{center}
		%\end{sidewaystable}
	\end{table}
%\end{landscape}
\clearpage


	
	\newpage
	\section*{Box B: Bird and bat communities in the Treeweb project}\label{boxb}
	\addcontentsline{toc}{section}{Box B}
	\vspace{2cm}
	
In the Treeweb project (see Box A), both bird and bat communities were investigated in the context of syntheses analyses relating tree species composition and fragmentation to multiple ecosystem functions and the diversity of multiple communities across different trophic levels \citep{Hertzog2020, Perring2021}. In this box, I give a brief overview and exploration of these data sets.\\

Birds and bats play an important role as mobile vertebrate insectivores in forest food webs \citep{Sekercioglu2012, Russo2016}. Both landscape level factors, such as fragmentation context, and local factors, such as forest structure and composition, can influence the diversity of birds and bats (e.g. \citealt{Charbonnier2016, Fuentes-Montemayor2013, Fuentes-Montemayor2017, Whytock2018, Penone2019, Barbaro2019}). In this exploratory analysis I relate bird and bat species richness to the tree species composition and to the forest fragment area. I expect the highest diversity in tree species compositions with the highest abundance of arthropod prey, i.e. pedunculate oak monocultues or mixtures containing this species, and in larger forest fragments. Species-specific responses depending on the foraging ecology and mobility likely shape this pattern \citep{Fuentes-Montemayor2013, Barbaro2019}.\\

\subsection*{Material and methods}

Bird and bat communities were investigated in all 53 Treeweb plots in 2018. The bird community was sampled using point-counts during the morning chorus in spring. All plots were visited three times: in the first half of March, the first half of April and the first half of May. During the first 4 h after sunrise, on days without adverse weather conditions, each plot was visited for 10 min (which is a sufficient amount of time to allow plot-level bird diversity assessments in temperate forests; \citealt{Bonthoux2012}). During this time period, all species and the number of individuals that could be seen or heard were noted. Each plot was visited once in the first hour after sunrise, once between the second and third hour after sunrise, and once during the fourth hour after sunrise. Due to logistical constraints, plots within the same forest fragment were visited on the same morning.\\

Bat activity was sampled using nine calibrated automatic bat recorders (D500x detector, Pettersson Electronics, Sweden; settings following \citealt{Spoelstra2017}). Detectors were placed at ground-level in the centre of the plot, with the microphone pointed upwards. While most observed species are only detectable when flying in the plot or the buffer zone of 10 m with the same tree species composition, note that three species, namely noctule bat (\textit{Nyctalus noctula}), Leisler's bat (\textit{N. leisleri}) and serotine (\textit{Eptesicus serotinus}), can also be recorded when flying outside the plot and the buffer zone \citep{Barataud2015}, but when these species were recorded, calls were very loud, indicating the bat flew close to the detector. All plots were sampled one full night (from sunset-sunrise) during the first week of July 2018, on nights without rain and with minimum temperature above 15$^{\circ}$C. Although a single night of sampling per plot provides only a coarse description of local bat assemblages, previous work has shown that such sampling can be successfully used to identify differences in bat communities in forests (e.g. \citealt{Fuentes-Montemayor2013, Charbonnier2016}). Recordings were all manually validated in Batexplorer 2.1 (Elekon AG, Switzerland). Bat species were identified to the finest possible taxonomic level (species, or species-pairs in the case of whiskered/Brandt's bat (\textit{Myotis mystacinus/brandtii}) and brown/grey long-eared bat (\textit{Plecotus auritus/austriacus})). As proxy for bat activity, we counted the number of 5s recordings with two or more calls per species.\\

To explore which variables drive the bird and bat community composition, a redundancy-analysis (RDA) was performed using the function rda in the R-package `vegan' \citep{Oksanen2018}. As response variable, we used a matrix containing the bird and bat species, which were Hellinger-transformed to prevent overabundant species influencing results \citep{Legendre2001}. The explanatory variables were the number of tree species, the relative basal area of the three focal tree species of the Treeweb platform (beech (\textit{F. sylvatica}), pedunculate oak (\textit{Q. robur}) and red oak (\textit{Q. rubra})), the total basal area, the number of tree cavities, the cover of the shrub layer, the cover of the herb layer, the size of the forest fragment, the distance to the closest edge and the edge density (see \citealt{DeGroote2017}; and Box A, page \pageref{boxa}) for information on how these environmental variables were collected). The explained variance of the bird and bat RDA (adjusted R$^2$) was 0.06 and 0.11, respectively.\\

To test for effects of tree species composition (as a factor with seven levels) and forest fragment area, we fitted models for the number of species for both birds and bats. The number of species was investigated with a glmer with Poisson distribution. Models were run using the `lme4' package \citep{Bates2015}, with p-values obtained with `lmerTest' \citep{Kuznetsova2017}. In all models, fragment ID was included as a random factor to account for the non-independence of plots in the same forest fragment.\\

\subsection*{Results and discussion}
In total, we recorded 27 bird species and 9 bat species, with great tit (\textit{Parus major}), closely followed by blue tit (\textit{Cyanistes caeruleus}), and common pipistrelle (\textit{Pipistrellus pipistrellus}) as the most abundant bird and bat species, respectively. On average, we recorded 7.3 bird species and 5.6 bat species in the plots. Several species showed relationships with specific overstorey species or forest attributes (figure B.1A and B.1B). For example, the number of singing Eurasian wrens (\textit{Troglodytes troglodytes}) and activity of Natterer's bat (\textit{M. nattereri}) increased with the basal area of pedunculate oak, while serotine (\textit{E. serotinus}) activity increased with beech basal area. Eurasian wren numbers were positively correlated to understorey vegetation cover, while whiskered bat (\textit{M. mystacinus/brandtii}) activity was positively correlated to shrub cover.\\

Contrary to the initial expectation, forest fragment area did not significantly influence the number of species in both the bird and the bat communities. However, I did observe tree species composition effects. The estimated bird species richness was highest in pedunculate oak monocultures and in mixtures of beech and pedunculate oak. Bird species richness was lowest in red oak and beech monocultures. Bird species richness in tree species mixtures did not significantly differ with the richness observed in the constituent monocultures, with the exception of the mixture beech -- pedunculate oak, which was higher than beech monocultures (figure B.2). Bat species richness was highest in beech and pedunculate oak monocultures, and was significantly lower in red oak stands (figure B.2). As was the case for birds, bat species richness in all tree species mixtures did not significantly differ from other tree compositions (figure B.2).\\

The responses of birds and bats to forest composition has mostly been as explained by composition-dependent variation in the abundance of arthropod prey (e.g. \citealt{Charbonnier2016, Penone2019}). Along these lines, \citet{Barbaro2019} show that models including both arthropod diversity and forest composition explain more variation in bird and bat diversity than models only including forest composition. Thus, the high species richness of birds and bats in pedunculate oak monocultures is likely due to the high number of arthropod species associated with this tree species \citep{Kennedy1984, Brandle2001}. Similarly, the low species richness of both taxonomical groups in monocultures of red oak is expected, given the low abundance and diversity of arthropods associated with this tree species in its invasive range (e.g. compare with \citealt{VanSchrojensteinLantman2020}, in the same study area). In beech monocultures, both taxonomic groups show divergent responses, with a low species richness for birds and a high species richness of bats (figure B.2). This difference can likely be explained by different responses to the low shrub cover and open structure in beech monocultures. A diverse understorey and shrub layer has earlier been shown to increase bird diversity in European forests \citep{Renner2018, Penone2019}. On the other hand, the open structure and low shrub cover of beech stands likely makes such forest stands also ideal foraging sites for bat species adapted to foraging in semi-open habitats (e.g. serotine (\textit{E. serotinus}), noctule (\textit{N. noctula}); see also \citealt{Fuentes-Montemayor2013}), while species adapted to cluttered environments were also still present.\\

Contrary to the initial expectations and to previous studies (e.g. \citealt{Fuentes-Montemayor2017, Whytock2018}), the size of the forest fragment in which the plot were situated was not related to the number of species observed in the plot for both taxonomic groups. This could be because the gradient of forest fragment sizes in our study area was relatively small (1.3 -- 90.4 ha), or because more area-sensitive or forest specialist species (e.g. wood warbler (\textit{Phylloscopus sibilatrix}), Bechstein's bat (\textit{M. bechsteinii})) are no longer present in the region.\\

\begin{figure}[h!]
	\begin{center}
		\includegraphics[width=\textwidth]{figB1A.png}\hfill
		\includegraphics[width=\textwidth]{figB1B.png}
	\end{center}
	\caption*{\textbf{Figure B.1}: Redundancy analysis of (A) the breeding bird and (B) foraging bats in the 53 Treeweb plots. The RDA for bird  Birds (cyan) and bat species (blue) are visualized, with outliers captioned.}
\end{figure}

\begin{figure}[h!]
	\begin{center}
		\includegraphics[width=0.9\textwidth]{figB2.png}
	\end{center}
	\caption*{\textbf{Figure B.2}: estimated species richness for (A) birds and (B) bats in the different tree species compositions. Model estimate for an average forest fragment size (full circles), 95\%--confidence intervals (lines) and raw data (open circles). Fsyl= beech (\textit{Fagus sylvatica}), Qrob: pedunculate oak (\textit{Quercus robur}); Qrub: red oak (\textit{Q. rubra}).}
\end{figure}
\clearpage

%\cleardoublepage
%\cleardoublepage
%\hbox{}
%\thispagestyle{plain}
%\clearpage
%%%%%%%%%%%%%%%%%%%%%%%%%%%%%%%% CHAPTER TWO  %%%%%%%%%%%%%%%%%%%%%%%%%%%%%%%%%%%
\thispagestyle{plain} % empty 
\CenterWallPaper{1}{CH2.jpg}
\newpage{\thispagestyle{empty}\cleardoublepage}
\ClearWallPaper
\pagestyle{mainmatter}
\chapter{Forest fragmentation and tree species composition jointly shape breeding performance of two avian insectivores} \label{chapter2}
\chaptermark{Forest fragmentation and tree species composition shape breeding performance}
\lettergroup{\thechapter}

\begin{flushright} \color{gray}Daan Dekeukeleire\\
	Lionel Hertzog\\ Pieter Vantieghem\\ Irene van Schrojenstein Lantman\\ Bram K. Sercu\\
 Roschong Boonyarittichaikij\\ An Martel\\ Kris Verheyen\\ Dries Bonte\\ Diederik Strubbe \\ Luc Lens

\vspace*{1cm}
Adapted from: Dekeukeleire et al. (2019) \textit{Forest Ecology and Management}, \textbf{443}:95-105. DOI: 10.1016/j.foreco.2019.04.023
\end{flushright}

\vspace*{\fill}
\noindent \color{gray} $\lhd$ Fledglings, such as this blue tit, were ringed and weighted. Photo by Stephanie Schelfhout.
	
\color{black}


\newpage

	\section{Abstract}
	
Habitat fragmentation and forestry practices affect forest structure and composition, and hence, their intrinsic value for biodiversity conservation. While higher tree species diversity is commonly proposed to result in habitat of higher quality for forest species, how these tree diversity and tree composition effects interact with forest fragmentation in terms of critical resources and demographic effects on forest birds remains poorly understood. We investigated here possible synergistic effects of forest fragmentation and tree species composition on the breeding performance of two common, insectivorous forest birds in a human-dominated landscape in northern Belgium. We monitored the breeding performance of great tits and blue tits in 53 plots across independent gradients of tree species composition and forest fragmentation. In addition, data on the biomass of the main food source of these two species (i.e. caterpillars) was collected during the breeding season. Both tree composition and habitat fragmentation impacted the breeding performance of great and blue tits. Effects of tree species composition were mainly driven by tree species identity, rather than by tree species diversity, and the highest breeding performance was obtained in monocultures of pedunculate oaks. Fragmentation effects were only observed in resource-poor beech monocultures with breeding performance declining with reduction in forest area. Structural Equation Modelling revealed diverse and species-specific pathways: for great tits tree composition effects on breeding performance were driven by resource availability while for blue tits these effects were driven by variation in maternal condition, as measured by clutch size. Thus, forestry practices aiming at promoting forest-dependent birds could benefit from including tree species that support high arthropod numbers and by maintaining forest patches of larger sizes.\\
\clearpage
	
	\section{Introduction}

Forest management has long focused on optimizing productivity and wood quality, but managers are increasingly challenged to maintain forests that provide multiple ecosystem services, such as protection of soil and water, carbon stock optimization, and biodiversity conservation \citep{Pawson2013, Puettmann2015}. However, tree species composition and structure of managed forests is often highly simplified. For instance, about 29 \% of Europe's forests are composed of a single tree species only \citep{Barsoum2016}, raising concerns for biodiversity conservation and forest resilience in the face of global changes \citep{Thompson2009}. Likewise, only about 14 \% of the forest habitats covered by the EU's Habitat Directive --which aims to ensure the conservation of a wide range of rare, threatened or endemic animal and plant species-- were estimated to have a `favourable' conservation status \citep{EEA2020}. Hence, next to reconnecting isolated forest patches through reforestation or via establishing ecological corridors \citep{Humphrey2015, Newmark2017}, recent studies stress the importance of tree species mixtures for the deliverance of multiple ecosystem services and the conservation of forest biodiversity (e.g. \citealt{Brockerhoff2017, VanderPlas2016, Hertzog2019}). A higher number of tree species increases structural diversity in managed forest stands, and hence, the availability of microhabitats for forest fauna \citep{Brockerhoff2017, Penone2019}. Along these lines, bird diversity and functional richness have been shown to increase with the number of deciduous tree species in mixed forest stands in the temperate and boreal regions across Europe \citep{Charbonnier2016}.\\

As in many temperate lowlands worldwide, European forests have a long history of fragmentation and land use change. In northern Belgium for example, forest fragments have a median size of 1.5 ha only, and the number of fragments has strongly increased over the last 250 years \citep{DeKeersmaeker2014}. Avian population sizes in small habitat patches are generally lower compared to those in larger patches, especially in landscapes with little habitat left \citep{Andren1994}. These population are therefore subject to higher extinction risks due to demographic, environmental and/or genetic stochasticity. Moreover, spatial isolation may lead to reduced dispersal and gene flow between subpopulations which, in turn, may reinforce adverse effects of demographic and genetic stochasticity \citep{Cooper2002}. Forest fragmentation can also directly impair avian reproductive success through smaller clutch sizes (e.g. \citealt{Wilkin2007a}), increased nest predation (e.g. \citealt{Deng2005, Huhta2004}), decreased nestling growth and body condition (e.g. \citealt{Bueno-Enciso2016}) and decreased fledging success (e.g. \citealt{Hinsley2009}) in fragmented forests. For example, in North-American fragmented forests, the level of brood parasitism by the brown-headed cowbird (\textit{Molothrus ater}) varies with both local edge effects and the landscape context, leading to sink populations in small forest fragments in landscapes with low forest cover \citep{Lloyd2005}. Furthermore, fragmentation can also influence breeding performance through impacts on parental behaviour. In highly fragmented landscapes, forest specialist species may fail to maintain the size of their foraging home ranges due to unsuitable conditions in the surrounding landscape matrix. For instance, Northern saw-whet owls (\textit{Aegolius acadicus}) breeding in small forest patches with large inter-patch distances occupy smaller home ranges and show lower food provisioning rates than individuals breeding in larger forests, resulting in chronically stressed nestlings and lower nestling survival \citep{Hinam2008}.\\

Aside from these well-described effects of habitat fragmentation, resource availability in remaining habitat patches is a key driver of species' population dynamics and distributions \citep{Mortelliti2010}. For instance, in a study on middle spotted woodpeckers (\textit{Dendrocopos medius}), the density of old-grown native oaks explained more variation in patch occupancy dynamics than did patch size \citep{Robles2012}. Likewise, a study on Hazel Dormice (\textit{Muscardinus avellanarius}) found that the amount of available food resources, measured as the cover of fruiting understory shrubs, shaped patch colonization dynamics, while extinction was mainly affected by patch size \citep{Mortelliti2014}. Thus, both landscape context and forest composition can codetermine the persistence and dynamics of animal populations in human-dominated landscapes. So far, however, most studies investigating the role of patch-level resource availability in fragmented landscapes only measured effects on patch occupancy (e.g. see \citealt{Franken2004, Vogeli2010, Robles2012, Cunningham2016}; but see \citealt{Mortelliti2014}). Yet, understanding how fragmentation and tree species composition effects jointly shape demographic processes underlying patch occupancy, such as through differences in reproductive success, is crucial for formulating effective species conservation strategies in fragmented landscapes.\\

Secondary cavity nesting birds such as Paridae are particularly well suited to study synergistic effects of tree species composition and forest fragmentation on demographic processes. Common species such as great tit (\textit{Parus major}) and blue tit (\textit{Cyanistes caeruleus}) rely on large quantities of arthropods --mainly during the ephemeral peak of caterpillars-- to rear their young \citep{Ceia2016}. Because of their high trophic position, these species can play an important role in the forest food web \citep{Sanz2001, Dekeukeleire2019}. Although both species can occur in multiple habitats, such as (sub)urban gardens, parks or hedgerows in agricultural areas, forest is the preferred habitat, especially during the breeding season \citep{Gosler1993, Stenning2018}. Great and blue tits occur in a wide range of forest types and readily adopt nest boxes, allowing to monitor reproductive performance in great detail \citep{Lambrechts2010}. Tits were earlier shown to rear lower numbers of fledglings, or fledglings with lower body condition, in small fragments and at forest edges (\citealt{Hinsley2009, Wilkin2009, Bueno-Enciso2016}; but see \citealt{Nour1998}). Native pedunculate oaks (\textit{Quercus robur} L) are generally assumed to be the optimal breeding habitat for both species, with some studies defining territory quality on the basis of the number of oak trees they contain \citep{Wilkin2007, Bell2014}. However, tits can achieve a high breeding performance in other forests too \citep{Shutt2018}, as long as an adequate supply of suitable arthropod prey are available. In general, in deciduous stands, clutch sizes and fledgling numbers are approximately 30 \% larger compared to coniferous or sclerophyllous forest stands (e.g. \citealt{VanBalen1973, Lambrechts2004, Tremblay2005}).\\

Here, we use great and blue tits as sentinel species for assessing interacting effects of tree species composition and forest fragmentation on reproductive performance within the context of recent management strategies to promote mixed forest stands. To assess how food resource availability shapes breeding performance across these gradients, we collected data on the abundance of caterpillars, the most important food resource for breeding tits. We expect lower breeding performance in small and resource-poor forest patches, driven by food availability during nesting. Specifically, we predict (i) the highest fledging success and fledgling condition in monocultures of pedunculate oaks or admixtures with this species, independent of forest fragment area, (ii) intermediate fledging success and fledgling condition in admixtures without pedunculate oak, independent of forest fragment area; and (iii) lowest fledging success and fledgling condition in small, fragmented monocultures of other tree species.\\

\clearpage
	\section{Material \& Methods}
		\subsection*{Study sites}
		
		This study was conducted in the Treeweb platform, which consists of 53 research plots (30 $\times$ 30m) across 19 forest fragments in northern Belgium (see Box A, page \pageref{boxa}). The landscape in this region is primarily agrarian, interspersed with residential areas and highly fragmented deciduous forests. The tree layer in the plots of the Treeweb platform consists of three regionally common species; pedunculate oak (\textit{Quercus robur} L), beech (\textit{Fagus sylvatica} L), red oak (\textit{Q. rubra} L). In contrast to the native pedunculate oak and beech, the American red oak, is an invasive species in Belgian forests \citep{Branquart2007}. To establish the plots, six to eight replicates were selected for each of the seven possible tree species combinations; i.e. 3 monocultures, 3 two-species mixtures and 1 three-species mixture. These replicates were spread along a fragmentation gradient; some plots being situated close to the forest edge in small and isolated forest fragments while others being situated in the forest interior in large and well-connected forest fragments. To minimize confounding effects from the presence of adjacent, different tree stands, each selected plot was surrounded by a buffer zone of minimum 10 m consisting of the same tree species composition. To minimize other potentially-confounding environmental effects, all plots were established in forest stands with similar land-use history (continually wooded since at least 1850), management history (mature stands where no forestry management took place in the last decade) and soil (dry sandy loam) (see \citealt{DeGroote2017} for full details on plot selection procedure). Beech plots in this study area have a poorly developed shrub layer, while red oak plots have a shrub layer consisting of red oak saplings. Pedunculate oak plots have a rich shrub layer with hazel (\textit{Corylus avellana} L) and rowan (\textit{Sorbus aucuparia} L). Tree species mixtures show intermediate shrub cover with the respective monocultures \citep{DeGroote2017}.\\
		
		We derived 12 fragmentation measures for every study plot (see Box A, page \pageref{boxa}). Because of the high correlations between these measures, we selected two measures for further analyses: (i) fragment-level forest area in which each study plot was embedded, and (ii) plot-level edge density. Fragment area was calculated as the total surface area of the forest fragment in which a study plot is situated (range: 1.3 to 90.4 ha). This variable was selected as it is a key component of habitat fragmentation \citep{Ewers2007, Fischer2007}. Forest fragments were defined as physically continuous patches covered by forest (all tree species, including coniferous (\textit{Pinus}, \textit{Larix} and \textit{Picea} species) or poplar (\textit{Populus} species) plantations, and all stand ages) based on detailed land use GIS layers \citep{Vriens2011}. Edge density quantifies the intensity of edge effects on individual plots, with higher values being characteristic for more fragmented forests. Edge density was defined as the total length of all edges of forest with other land use classes (such as agricultural land or residential areas) within a radius of 300 m of the plot (range: 655 m -- 2932.4 m).\\
		
		
		\subsection*{Breeding data collection}
		
		At the four corners of each study plot, standard wooden nest boxes (height 1.5 m, dimensions 23 $\times$ 9 $\times$ 12 cm, entrance diameter 32 mm) were placed in autumn 2014 on the tree closest to the corner. During the breeding seasons (April--June) of 2015 and 2016, all nest boxes were checked at least twice a week to determine occupancy by blue or great tits, and to record date of the first laid egg (the laying date was back-calculated assuming one egg was laid per day; \citealt{Perrins1965}), the total number of eggs produced (clutch size), the number of hatchlings and the number of fledglings. When nestlings were 7 to 10 days old, parents were captured when feeding their young in order to gather data on parental condition (as measured by tarsus length and body mass). Captured birds were fitted with numbered metal rings of the Belgian Ringing scheme. Nestlings were measured (tarsus and body mass) and ringed at an age of 15 days \citep{Matthysen2011}. Body condition of adults and fledglings was calculated using the scaled-mass index (SMI), which adjusts the mass of all individuals to that which they would have had if all had the same body size, using the equation of the linear regression of ln-body mass on ln-tarsus length estimated by type 2 (standardized major axis; SMA) regression \citep{Peig2009}. In line with the literature (e.g. \citealt{Matthysen2011, Wilkin2006}), second clutches were rare, and only first clutches were considered.
		
		\subsection*{Frass collection}
		
		As a proxy for caterpillar biomass, frass (i.e. caterpillar droppings) was collected in 52 plots during 2016 (in one plot, permission was not granted by the owners). In each plot, 3 frass traps, consisting of a 50 $\times$ 50 cm wooden frame with a white cotton cloth weighted down in the centre \citep{Zandt1994}, were placed 50 cm above the ground. One frass trap was placed in the centre of the plot, and two others halfway on a line connecting the centre with the northern corners. In none of the plots, these locations were situated under clearings or canopy gaps. Frass traps were put in the field on April 12th, and emptied once every week until June 7th. To separate frass from larger plant debris, collected material was first dried (at 60$^{\circ}$C for 24 h) and then sieved (mesh size 1 mm$^2$). Next, every sample was weighted (to the nearest 0.001 g) and the relative content of frass vs. small plant material (e.g. pollen and stamens) was estimated under a stereo microscope (magnification 10$\times$). The mass of the frass was then calculated by multiplying the estimated percentage of frass with the total mass of the sample. To test the validity of this method, a subset of 15 samples (5 random samples per tree species monoculture) were sorted by hand under a stereo microscope (magnification 10$\times$), and then weighed again. This subsample confirmed that the aforementioned method produces a reliable estimate of the frass mass of a sample (R$^2$ adj.$=$ 0.86, p $<$ 0.001, see also supplementary material S2.1). To obtain an estimate of the amount of caterpillar biomass available to breeding tits, we calculated `frass mass per nest' as the total mass of frass in all three traps combined, summed over the 15 day period between hatching and ringing of the nestlings (assuming equal frass fall in all days between sampling dates). This way, we quantified the food availability during the period nestlings were provisioned by their parents for each occupied nest box.\\
		
		\subsection*{Data analyses}
		
		All statistical analyses were performed in the R statistical environment (\citealt{RCoreTeam2018}, version 3.5.1). First, to test for the effects of tree species composition and fragmentation on breeding performance during 2015 and 2016, we applied a set of (Generalized) Linear Mixed Models ((G)LMM) with fledging success (proportion per nest of the hatchlings that survived until fledgling state) and average fledgling body condition (SMI) as response variables. Models were run for both species separately. Proportional data (i.e. fledging success) were analysed using a Generalized Linear Mixed Model (GLMM) with a beta-binomial distribution to account for the under-dispersed nature of our data (Estimated dispersion parameter great tit: 0.629; blue tit: 0.754). GLMM's were fitted using maximum likelihood estimation via Template Model Builder in the package glmmTMB \citep{Brooks2017}. Fledgling body condition was analysed using a Linear Mixed-Effect Model (LMM) using the, lme4 package \citep{Bates2015}, with p-values obtained through the lmerTest package \citep{Kuznetsova2017}. In each of these models, we included tree species composition (as a factor with seven levels), a measure of forest fragmentation (log fragment area or log edge density), the interaction between tree species composition and fragmentation, laying date (Julian day) and year as explanatory variables. These models were run with one fragmentation measure at a time, as models with both metrics did not converge despite applying standard practices to ensure convergence (see \url{https://rstudio-pubs-static.s3.amazonaws.com/33653_57fc7b8e5d484c909b615d8633c01d51.html}). All models included forest fragment ID as a random term to account for the possible non-independence of nests present within the same forest fragment. In models with body condition of fledglings as a response, the number of nestlings was included as an additional fixed covariate. Standard errors and 95\%-confidence intervals for group-level effects were obtained with the package addSE (\url{https://github.com/lionel68/addSE}). All statistical inferences reported are those for the full models. Nine great tit and four blue tit males, and nine great tit and two blue tit females were encountered in both years. We performed all analyses with and without these nests and found no qualitative differences in the results (data not shown). Thus, we decided to treat each adult as independent in the analyses (following \citealt{Bueno-Enciso2016}). In addition, we also verified whether tree species composition and fragmentation affected other breeding parameters, namely nest box occupation, parental body condition (SMI), laying date, clutch size and hatching success (see supplementary material, table S2.2).
		
		\subsection*{Ethics statement}
		
		Capture and ringing permits were granted by the Belgian Ringing Scheme and the Flemish authorities (Agentschap voor Natuur en Bos; ANB/BL-FF/V15-00034 and ANB/BL-FF/V16-00003). All protocols were approved by the Ethical Committee VIB Ghent (EC2015-023).
		
%Figure 2.1 Conceptual diagram chapter 2	
\begin{figure}[h!]
	\begin{center}
		\includegraphics[width=\textwidth]{fig2-1.png}
	\end{center}
	\caption{Hypothesised structural equation model linking tree composition and fragment area to fledging success and fledgling body condition through changes in maternal condition (clutch size) and resource availability (frass mass). Note that for clarity the interaction between tree species composition and fragment area is indicated by an $\ast$ on the figure.  \label{fig2-1}}
\end{figure}
\clearpage
		
\clearpage		
	\section{Results}
		\subsection*{Fragmentation and tree species composition effects on breeding performance}
		
		A total of 200 nests of great tits and 112 nests of blue tits were monitored during both breeding seasons (table \ref{tab2-1}). There were no plots where none of the nest boxes were occupied. On average, two out of four nest boxes in a plot were occupied by great tit, and one out of four by blue tit. In great tit nests, fledging success was significantly lower in beech monocultures (estimate in a fragment with average area: 0.57; CI: 0.39--0.74) and red oak monocultures (estimate in a fragment with average area: 0.38; CI: 0.14--0.70) compared to pedunculate oak monocultures (estimate in a fragment with average area: 0.91; CI: 0.79--0.96; figure \ref{fig2-2}a). Fledging success in all tree species mixtures did not significantly differ from other tree species compositions (figure \ref{fig2-2}b). In addition, the interaction between fragmentation (log fragment area) and tree species composition was significant, but there was only an effect for beech monocultures. In beech monocultures, estimated fledging success increased from 0.12 in 3.5 ha forest fragments area to 0.79 in 90.3 ha forest fragments. Models including fragment area explained more variation in fledging success than models including edge density (see supplementary material, table S2.3). The average body condition (SMI) of fledglings per nest decreased with an increasing number of nestlings (LMM estimate -0.0998, Z $=$ -2.50, p $=$ 0.014), but did not significantly vary with tree species composition.\\
		
		Blue tits showed no significant differences in fledging success or fledgling body condition between any of the tree species compositions, nor were fledging success or fledgling body condition correlated to fragment area or edge density (figure \ref{fig2-3}). Fledging success was also not correlated to the number of nestlings in blue tits. For both species, no significant differences between tree species compositions or relationships with fragment area and edge density were found for other breeding parameters, such as nest box occupancy, parental body condition (SMI), laying date or hatching success. Clutch sizes of blue tits were significantly larger in pedunculate oak monocultures compared to other tree species compositions, but did not vary with forest area (see supplementary material, table S2.2). 
		
		
		
%%%%%Table 2.1%%%%%
\clearpage
\thispagestyle{empty}
\begin{landscape}
	%\begin{sidewaystable}
	\begin{table}
		\begin{center}
			\begin{footnotesize}
				\caption{Summary of the reproductive parameters observed of great tits (\textit{Parus major}) and blue tits (\textit{Cyanistes caeruleus}) in different tree species combinations in forest fragments south of Ghent (Belgium). Fsyl = beech (\textit{Fagus sylvatica}), Qrob = pedunculate oak (\textit{Quercus robur}), Qrub = red oak (\textit{Q. rubra}). Values given are mean $\pm$ SE.}  \label{tab2-1}
				
				\begingroup
				\setlength{\tabcolsep}{6pt} % Default value: 6pt
				\renewcommand{\arraystretch}{1.5} % Default value: 1
				\hspace*{-1.8cm}
				\begin{tabular}{m{1.8cm} >{\centering}m{1cm} >{\centering}m{1cm} >{\centering}m{1cm} >{\centering}m{1cm} >{\centering}m{1cm} >{\centering}m{1cm} >{\centering}m{1cm} >{\centering}m{1cm} >{\centering}m{1cm} >{\centering}m{1cm} >{\centering}m{1cm} >{\centering}m{1cm} >{\centering}m{1cm} >{\centering}m{1cm} p{0.001cm}}
					
				\cline{0-15}
				& \multicolumn{2}{c}{\textbf{Qrob}} & \multicolumn{2}{c}{\textbf{Fsyl}} & \multicolumn{2}{c}{\textbf{Qrub}} & \multicolumn{2}{c}{\textbf{Fsyl-Qrob}}& \multicolumn{2}{c}{\textbf{Fsyl-Qrub}}& \multicolumn{2}{c}{\textbf{Qrob-Qrub}}& \multicolumn{2}{c}{\textbf{Fsyl-Qrob-Qrub}}&\\
					
				\cmidrule(lr){2-3}\cmidrule(lr){4-5}\cmidrule(lr){6-7}\cmidrule(lr){8-9}\cmidrule(lr){10-11}\cmidrule(lr){12-13}\cmidrule(lr){14-15}
				
				& Great tit & Blue tit &  Great tit & Blue tit & Great tit & Blue tit & Great tit & Blue tit & Great tit & Blue tit & Great tit & Blue tit & Great tit & Blue tit&\\
				\cline{0-15}
				\textbf{Number of breeding pairs} & 
				36 & 16 & 31 & 15 & 31 & 14 & 27 & 23 & 13 & 8 & 34 & 19 & 28 & 17&\\
				\textbf{Fledging \hyphenation{succes}} & 0.94 $\pm$ 0.19 & 0.87 $\pm$ 0.26 & 0.52 $\pm$ 0.46 & 0.82 $\pm$ 0.32 & 0.57 $\pm$ 0.38 & 0.83 $\pm$ 0.27 & 0.82 $\pm$ 0.34 & 0.82 $\pm$ 0.32 &  0.59 $\pm$ 0.49 & 0.82 $\pm$ 0.37 & 0.80 $\pm$ 0.31 & 0.72 $\pm$ 0.44 & 0.79 $\pm$ 0.35 & 0.85 $\pm$ 0.28&\\
				\textbf{Mean Fledgling body condition (SMI)} & 17.4 $\pm$ 1.46 & 11.2 $\pm$ 0.983 & 17.6 $\pm$ 4.09 & 10.6 $\pm$ 1.04 & 17 $\pm$ 1.92 & 11.1 $\pm$ 0.88 & 17.8 $\pm$ 2.01 & 10.9 $\pm$ 0.94 & 18 $\pm$ 2.13 & 11.5 $\pm$ 1.16 & 16.8 $\pm$ 1.5 & 11.1 $\pm$ 0.83 & 16.9 $\pm$ 1.95 & 11.7 $\pm$ 0.70&\\
				\textbf{Nest box occupation} & 0.50 $\pm$ 0 & 0.47 $\pm$ 0.09&0.47 $\pm$ 0.09 & 0.50 $\pm$ 0 & 0.50 $\pm$ 0 & 0.47 $\pm$ 0.09 & 0.50 $\pm$ 0 & 0.50 $\pm$ 0 & 0.4 $\pm$ 0.14 & 0.44 $\pm$ 0.13 & 0.50 $\pm$ 0 & 0.47 $\pm$ 0.09 & 0.50 $\pm$ 0 & 0.50 $\pm$ 0&\\
				\textbf{Laying date (1$^{st}$ April $=$ 1)} & 14.8 $\pm$ 7.35 & 9.44 $\pm$ 4.26 & 18.6 $\pm$ 7.23 & 15.1 $\pm$ 4.82 & 16.3 $\pm$ 6.4 & 14.3 $\pm$ 4.7 & 16.1 $\pm$ 6.55 & 14.9 $\pm$ 7.86 & 18.8 $\pm$ 7.46 & 18.0 $\pm$ 7.89 & 14.8 $\pm$ 6.43 & 12.2 $\pm$ 6.64 & 15.8 $\pm$ 5.24 & 16.1 $\pm$ 9.01&\\
				\textbf{Parental condition} & 17.3 $\pm$ 0.95 & 11 $\pm$ 0.339 & 17.1 $\pm$ 0.92 & 10.6 $\pm$ 0.55 & 17.7 $\pm$ 1.27 & 10.8 $\pm$ 0.64 & 17.5 $\pm$ 1.28 & 10.8 $\pm$ 0.87 & 17.3 $\pm$ 1.09 & 10.8 $\pm$ 0.57 & 17.4 $\pm$ 1.00 & 11 $\pm$ 0.893 & 17.4 $\pm$ 1.05 & 11.1 $\pm$ 0.72&\\
				\textbf{Clutch size} & 7.83 $\pm$ 7.35 & 11.5 $\pm$ 4.26 & 7.16 $\pm$ 7.23 & 8.87 $\pm$ 4.82 & 7.68 $\pm$ 6.4 & 9.21 $\pm$ 4.7 & 7.74 $\pm$ 6.55	 &  10.3 $\pm$ 7.86 & 7.77 $\pm$ 7.46 & 8.38 $\pm$ 7.89 & 8.32 $\pm$ 6.43 & 10.5 $\pm$ 6.64 & 7 $\pm$ 5.24 & 9.24 $\pm$ 9.01&\\
				\textbf{Hatching success} & 0.84 $\pm$ 0.27 & 0.84 $\pm$ 0.24 & 0.72 $\pm$ 0.27 & 0.65 $\pm$ 0.41 & 0.86 $\pm$ 0.27 & 0.96 $\pm$ 0.07 & 0.79 $\pm$ 0.26 & 0.79 $\pm$ 0.28 & 0.74 $\pm$ 0.35 & 0.87 $\pm$ 0.072 & 0.85 $\pm$ 0.25 & 0.72 $\pm$ 0.33 & 0.77 $\pm$ 0.37 & 0.83 $\pm$ 0.33&\\	
				\cline{0-15}	
				\end{tabular}\endgroup
			\end{footnotesize}
		\end{center}
		%\end{sidewaystable}
	\end{table}
\end{landscape}
\clearpage

%Figure 2.2
\begin{figure}[h!]
	\begin{center}
		\includegraphics[width=\textwidth]{fig2-2.png}
	\end{center}
	\caption{Estimated effects for tree composition and fragment area on (a) fledging success and (b) mean fledgling body condition (SMI) of great tits (\textit{Parus major}). Results from GLMM and LMM show the estimate for an average forest fragment area and slope with log forest fragment area with 95\% confidence intervals for the different tree species compositions. Fsyl $=$ beech (\textit{Fagus sylvatica}), Qrob $=$ pedunculate oak (\textit{Quercus robur}), Qrub $=$ red oak (\textit{Q. rubra}). \label{fig2-2}}
\end{figure}

%Figure 2.3
\begin{figure}[h!]
	\begin{center}
		\includegraphics[width=\textwidth]{fig2-3.png}
	\end{center}
	\caption{Estimated effects for tree composition and fragment area on (a) fledging success and (b) mean fledgling body condition (SMI) of blue tits (Cyanistes caeruleus). Results from GLMM and LMM show the estimate for an average forest fragment area and slope with log forest fragment area with 95\% confidence intervals for the different tree species compositions. Fsyl $=$ beech (\textit{Fagus sylvatica}), Qrob $=$ pedunculate oak (\textit{Quercus robur}), Qrub $=$ red oak (\textit{Q. rubra}). \label{fig2-3}}
\end{figure}
		
\clearpage		
		\subsection*{Indirect composition effects via resource availability and parental quality on breeding success}
		
		In great tits, both structural equation models showed a good fit, indicating no singificant paths were missing (SEM fledging success: Fisher's C $=$ 0.37, df $=$ 2, p $=$ 0.82; SEM fledgling mass: Fisher's C $=$ 5.79, df $=$ 8, p $=$ 0.67). In both SEMs (figure \ref{fig2-4}), tree species composition, but not fragment area, affected frass mass, and hence, food availability. Frass mass was highest in monocultures of pedunculate oak, and significantly lower in both beech and red oak monocultures, while mixtures showed intermediate values (figure \ref{fig2-5}). Clutch size was not affected by composition or fragment area in great tits, whereas frass mass had strong positive effects on fledging success (figure \ref{fig2-4}a). Nestling body condition was negatively correlated to clutch size, but was not related to frass mass (see supplementary material S2.4).\\
		
		In blue tits, both structural equation models showed a good fit, indicating no singificant paths were missing (SEM fledging success: Fisher's C $=$ 0.6, df $=$ 2, p $=$ 0.74; SEM fledgling mass: Fisher's C $=$ 3.95, df $=$ 8, p $=$ 0.86). As in the great tit SEMs, tree species composition affected frass mass, and hence, food availability. In blue tits, however, tree species composition also affected clutch size, which was significantly higher in pedunculate oak monocultures compared to beech monocultures and, marginally, to three-species mixtures (figure \ref{fig2-6}). Contrary to great tits, fledging success in blue tits was not correlated to frass mass, but was correlated to clutch size (figure \ref{fig2-4}b). Nestling body condition was not correlated to either clutch size nor frass mass (see supplementary material S2.4).\\

		
%Figure 2.4
\begin{figure}[h!]
	\begin{center}
		\includegraphics[width=\textwidth]{fig2-4.png}
	\end{center}
	\caption{Structural equation model (SEM) path diagrams for (a) great tit fledging success, (b) blue tit fledging success. Arrows and number near arrows (standardized effect sizes), direction and relative magnitude of the relationship for significant variables, respectively; dashed arrows, non-significant relationships specified in the a priori model. Coefficients of determination (R$^2$) are shown for all response variables. Note that for clarity the (non-significant) interaction between tree species composition and fragment area is indicated by a $\ast$ on the figure. \label{fig2-4}}
\end{figure}

%Figure 2.5
\begin{figure}[h!]
	\begin{center}
		\includegraphics[width=0.6\textwidth]{fig2-5.png}
	\end{center}
	\caption{Differences in frass mass for different tree species compositions, calculated with the partial linear mixed-effect model included in the piecewise SEM analysis. Model estimate (full circles), 95\%-confidence intervals (line) and data (open circles) are shown. Fsyl $=$ beech (\textit{Fagus sylvatica}), Qrob $=$ pedunculate oak (\textit{Quercus robur}), Qrub $=$ red oak (\textit{Q. rubra}). Results shown are those for the partial model of the great tit SEMs, but do not differ qualitatively for the blue tit results. \label{fig2-5}}
\end{figure}

%Figure 2.6
\begin{figure}[h!]
	\begin{center}
		\includegraphics[width=0.6\textwidth]{fig2-6.png}
	\end{center}
	\caption{Differences in clutch sizes of blue tits for different tree species compositions, calculated with the partial generalized linear mixed-effect model included in the piecewise SEM analysis. Model estimate (full circles), 95\%-confidence intervals (line) and data (open circles) are shown. Fsyl $=$ beech (\textit{Fagus sylvatica}), Qrob $=$ pedunculate oak (\textit{Quercus robur}), Qrub $=$ red oak (\textit{Q. rubra}).  The species combination Fsyl-Qrub was dropped in this analysis because of sample size issues. \label{fig2-6}}
\end{figure}		

\clearpage
	\section{Discussion}
	
	In this study, we found that both tree species composition and fragmentation impact the breeding performance of two insectivorous bird species in managed forests. Breeding performance was more strongly related to tree species identity than to diversity, as for both great and blue tits, fledging success was highest in pedunculate oak monocultures, and lowest in beech and --in the case of great tit-- red oak monocultures. Breeding performance in tree species mixtures was intermediate between the level of the constituent monocultures. Fragmentation effects were detected in beech monocultures only, where great tit breeding success increased with fragment area. In accordance with these results, caterpillar frass mass varied with tree species composition but not with fragment area. SEMs further revealed that tit reproductive success was related to tree species composition and fragmentation via multiple and species-specific pathways. In great tits, tree species composition shaped food availability during the nestling period, which in turn shaped fledging success. Edge effects and the inability to expand foraging home-ranges likely underlied poor breeding performance in small fragments consisting of food-deprived beech monocultures. For blue tits, however, the effect of tree species composition on fledging success was mediated through clutch size, rather than food availability during nesting, suggesting the presence of seasonal carry-over effects.\\
	
	The stronger effect of tree species identity compared to tree diversity on breeding performance was in line with expectations (e.g. \citealt{Naef-Daenzer2000, Shutt2018}), as pedunculate oak has a very high number of associated arthropod species, including many Lepidoptera \citep{Kennedy1984, Brandle2001}. Moreover, pedunculate oak stands are often characterized by a diverse shrub layer with species such as hazel and rowan (e.g. \citealt{DeGroote2018}), which can also harbor many prey species for tits and can provide additional cover against cold temperatures \citep{Latimer2017}. We did not detect clear effects of tree diversity on prey availability or breeding performance. The caterpillar abundance in tree mixtures was intermediate between the abundances in the respective monocultures. Likewise, other studies also found additive effects but no diversity effects of tree mixtures on soil properties (e.g. \citealt{DeGroote2018}), vegetation (e.g. \citealt{Barsoum2016}) or arthropod communities (e.g. \citealt{Koricheva2006}). This suggest that these properties might not be mechanistically linked to tree diversity per se, but rather relate to the presence of particular tree species or combinations of species with certain traits \citep{Forrester2015}.\\ 
	
	In great tits, the effect of tree species composition on fledging success was shaped by food availability during the nestling stage. Clutch sizes were similar across tree compositions, but nestling survival increased with the ephemeral peak in caterpillar availability. Previously, great tits breeding closer to pedunculate oak trees were shown to feed their young more high-quality prey \citep{Wilkin2007a} and to have a lower daily energy expenditure, likely due to shorter foraging flight distances \citep{Hinsley2008}. In blue tits, however, the effect of tree species composition was shaped indirectly through higher clutch sizes, rather than through the ephemeral peak in food availability. Clutches were significantly larger in monocultures of pedunculate oak compared to beech, which was in accordance with results from another study \citep{Amininasab2016}. Experimental studies earlier showed that female blue tits in a better condition can invest in larger clutches \citep{Slagsvold1990}. This could indicate that the higher breeding performance of blue tits in pedunculate oak monocultures is shaped by maternal condition, which could be affected by carry-over effects from outside the breeding season or by selective territory settlement of individuals in better condition. Our results hence support the hypothesis that increased food availability in high quality habitat, such as pedunculate oak stands, results in a higher reproductive success, either through improved fledging success (great tits) or increased clutch sizes (blue tits). Other factors besides food availability are less likely be important in our study. Nest predation was rare (three out of 312 nests were predated by great spotted woodpecker (\textit{Dendrocopos major})), while necropsy of a subset of deceased pulli did not indicate the presence of pathogens.\\
	
	Results from this study hence plead for the inclusion of pedunculate oak, or similar arthropod-rich trees such as sycamore (\textit{Acer pseudoplatanus} L) and birch (\textit{Betula} spec.) \citep{Shutt2018}, to ensure the provisioning of multiple forest ecosystem services (compare \citealt{VanderPlas2016}). However, tree species can also vary in their provision of suitable cavities for breeding \citep{Remm2011} and arthropod-rich trees do not necessarily provide most suitable nest sites. The latter can be the limiting factor for populations of secondary cavity breeding birds, especially so in heavily-managed forest stands \citep{Newton1994, Robles2012}. In forests without nest boxes, mixing arthropod-rich with cavity-rich tree species likely creates the most optimal habitat for forest insectivores.\\
	
	Fragmentation effects were only observed in beech monocultures, as in larger beech-dominated fragments, a higher proportion of great tit nestlings successfully fledged. A lower food availability in small fragments has been shown to limit breeding performance of insectivorous birds in several temperate forests (\citealt{Burke1998, Zanette2000}; but see \citealt{Lampila2005}), while in small Mediterranean forest fragments, great tits nestlings were shown to receive fewer caterpillars and be smaller at fledging, compared to larger fragments (\citealt{ Bueno-Enciso2016}; but see \citealt{Nour1998}). However, as frass mass did not increase with fragment area in our study, other mutually non-exclusive mechanisms may underlie the observed fragmentation effect.\\
	
	Firstly, individuals breeding in large fragments may compensate for low food availability by expanding the size of their foraging home ranges, while this may not be possible in small fragments. Home ranges of great and blue tits are on average ca 0.3 ha \citep{Naef-Daenzer1994} but can vary considerably. For instance, in sclerophyllous forest, blue tits have been shown to double their mean foraging travel distance, and even regularly forage up to 500 m from their nest, compared to deciduous forest \citep{Tremblay2005}. This, however, resulted in a similar biomass provisioned to nestlings in lower compared to higher quality habitats. Conceivably, such a behavioural flexibility could increase nestling survival in larger forest fragments with low arthropod densities, but is not always possible in the smallest fragments of our study ($<$ 1.5 ha), which were isolated within an open agricultural landscape with low prey abundance. Studies on breeding great tits earlier showed that crossing such open landscape gaps to forage in other fragments strongly increases energetic costs \citep{Hinsley2008}.\\
	
	Secondly, due to the typical absence of a buffering shrub layer in beech monocultures \citep{Dolle2017, Sercu2017}, edge effects could be more pronounced and penetrate deeper than in other tree species compositions \citep{Latimer2017}. Great tits generally produce fewer fledglings and/or fledglings with a lower body condition closer to forest edges \citep{Deng2005, Wilkin2007}. Edge effects such as an increased cooling of breeding cavities at night \citep{Wiebe2001} can directly impact nestling survival and condition \citep{Bleu2017}, and could be most prominent in small fragments of beech monocultures.\\
	
	Studies investigating signatures of forest fragmentation on insectivorous birds in Europe often fail to find strong effects (reviewed in \citealt{Lampila2005}), and this was commonly explained by the long history of anthropogenic fragmentation in European forests. However, most of these studies did not take local resource availability --shaped by tree species composition and forest structure-- into account. For instance, in a highly-modified landscape in northern Belgium, breeding performance of great tits, blue tits and Eurasian nuthatches (\textit{Sitta europaea}) did not vary with forest fragment area \citep{Matthysen1998, Nour1998}, but all forest fragments investigated in these studies were arthropod-rich pedunculate oak stands. Instead, our study shows that even common forest-dependent birds may suffer from forest fragmentation in human-dominated landscapes when patch-level resource availability is low, and that such effects may be underestimated when not taking the latter into account.\\
	
	Our results further suggest that effects of habitat fragmentation and tree species composition may interact in shaping reproductive success via resource availability, and potentially also via seasonal carry-over effects. In forest stands with low resource availability, landscape fragmentation negatively affected fledging success, probably by restricting the amount of habitat that can effectively be searched for food. Our results hence show that effects of landscape context should be considered when managing forests for biodiversity and ecosystem functioning. In addition, careful selection of tree species with complementary life-history characters, rather than increasing tree diversity per se, may provide the most direct benefits for biodiversity conservation. On a landscape scale, our results stress the need to also conserve small forest fragments with high resource availability, and to improve the habitat quality of such fragments by promoting tree species that provide arthropod food during the critical breeding period. \\
	
	\clearpage
	\subsection*{Funding}
	
	Financial support for this research was provided via the UGent GOA project ``Scaling up Functional Biodiversity Research: from Individuals to Landscapes and Back (TREEWEB)''.
	
	\subsection*{Acknowledgements}
	
	We would like to thank the private forest owners and the Flemish Forest and Nature Agency (ANB) for granting access to their property. We would like to thank Robbe De Beelde and Hans Matheve for help with the fieldwork, Eline Lorer, Rhea Maesele and Laurian Van Maldeghem for help with weighing the frass samples and Liesbeth De Neve for advise on the study design. Furthermore, we would like to thank Femke Batsleer for useful comments on the first draft of this manuscript.
	
	\subsection*{Data Accessibility}
	
	Breeding data, frass mass data and fragmentation data are available on GitHub: \url{https://doi.org/10.5281/zenodo.2641799}
	
	\subsection*{Author contributions}
	Luc Lens, Diederik Strubbe and Daan Dekeukeleire conceived the study; Luc Lens, Dries Bonte, Kris Verheyen and An Martel acquired funding; Daan Dekeukeleire, Pieter Vantieghem, Irene van Schrojenstein Lantman, Bram K. Sercu and Roschong Boonyarittichaikij collected the data, Daan Dekeukeleire and Lionel Hertzog performed the analyses; Daan Dekeukeleire led the writing of the manuscript and all authors contributed significantly to the drafts.\\
	
	\clearpage
	\section{Supplementary material}
	
%Figure S2.1
\begin{figure}[th]
	\begin{center}
		\includegraphics[width=0.8\textwidth]{figS2-1.png}
	\end{center}
	\caption*{\textbf{Figure S2.1}: Results of a linear model of the frass mass calculated by multiplying the estimated percentage of frass with total mass of the sample obtained after sorting by hand of 15 samples. R$^2$ adj.$=$ 0.8632, p $<$ 0.001. Frass was estimated under a stereo-microscope at 10x magnification by Daan Dekeukeleire and Irene van Schrojenstein Lantman.}
\end{figure}	
	
	\clearpage
	\thispagestyle{plain}
	\begin{landscape}
	%\begin{sidewaystable}
	\begin{table}
		\begin{center}
			\begin{footnotesize}
				\caption*{\textbf{Table S2.2}: Results of the GLMMs and LMMs testing the effects of Tree species composition and fragmentation on nest box occupation (binomial distribution), parental body condition (SMI) (normal distribution), laying date (normal distribution), clutch size (Conway-Maxwell Poisson distribution) and hatching success (betabinomial distribution) for great tit (\textit{Parus major}) and blue tit (\textit{Cyanistes caeruleus}). Here we present the estimated effects of tree species composition in an average fragment (intercept) and slopes with fragmentation metrices, including lower and upper limits of the 95\% confidence interval. Furthermore, we give the standard deviation of the random effect (Forest fragment id). Fsyl $=$ beech (\textit{Fagus sylvatica}), Qrob $=$ pedunculate oak (\textit{Quercus robur}), Qrub $=$ red oak (\textit{Q. rubra}). (Continued on next pages; 1/5)}
				
				\begingroup
				\setlength{\tabcolsep}{10pt} % Default value: 6pt
				\renewcommand{\arraystretch}{1.5} % Default value: 1
				\begin{tabular}{c l c c c c c c}
					\toprule
					& \textbf{1) Nest box occupation} & \multicolumn{3}{c}{\textbf{Great tit}} & \multicolumn{3}{c}{\textbf{Blue tit}}\\
					
					& \textbf{Fixed effect}s & \textbf{Estimate} & \multicolumn{2}{c}{\textbf{95\%--CI}} & \textbf{Estimate} & \multicolumn{2}{c}{\textbf{95\%--CI}}\\
					\cline{2-8}
					
					\multirow{7}{*}{\rotatebox[origin=c]{90}{\parbox[c]{3cm}{\centering Intercept (value for  average fragment area)}}}
					
					 & 3 species mixture & 0.603 & 0.445 & 0.742 & 0.348 & 0.216 & 0.509\\
					& Fsyl & 0.595 & 0.434 & 0.738 & 0.307 & 0.184 & 0.465\\
					& Fsyl \& Qrob & 0.546 & 0.385 & 0.697 & 0.381 & 0.254 & 0.527\\
					& Fsyl \& Qrub & 0.375 & 0.152 & 0.667 & 0.329 & 0.091 & 0.706\\
					& Qrob & 0.745 & 0.593 & 0.854 & 0.402 & 0.236 & 0.594\\
					& Qrob \& Qrub & 0.538 & 0.366 & 0.702 & 0.368 & 0.229 & 0.533\\
					& Qrub & 0.405 & 0.174 & 0.688 & 0.353 & 0.097 & 0.734\\
					\cline{2-8}
					\multirow{7}{*}{\rotatebox[origin=c]{90}{\parbox[c]{3cm}{\centering Slopes)}}} & slope 3 species mixtures:log fragment area & 0.439 & -0.384 & 1.261 & -0.401 & -1.134 & 0.332\\
					& slope Fsyl:log fragment area & -0.17 & -0.665 & 0.325 & 0.088 & -0.463 & 0.64\\
					& slope Fsyl \& Qrob:log fragment area & 0.104 & -0.292 & 0.501 & 0.075 & -0.275 & 0.425\\
					& slope Fsyl \& Qrub:log fragment area & 0.817 & -0.913 & 2.548 & -0.035 & -1.906 & 1.835\\
					& slope Qrob:log fragment area & 0.214 & -0.383 & 0.811 & -0.031 & -0.519 & 0.458\\
					& slope Qrob \& Qrub:log fragment area & -1.494 & -3.377 & 0.388 & 0.362 & -1.612 & 2.337\\
					& slope Qrub:log fragment area & 1.006 & -1.111 & 3.123 & -0.096 & -2.791 & 2.6\\
					\cline{2-8}
					
					& standard deviation of random effect& 	0.305&&&$<$ 0.001&&\\	
					\bottomrule
				\end{tabular}\endgroup
			\end{footnotesize}
		\end{center}
		%\end{sidewaystable}
	\end{table}
	\end{landscape}
	\clearpage
	
\clearpage
\thispagestyle{plain}
\begin{landscape}
	%\begin{sidewaystable}
	\begin{table}
		\begin{center}
			\begin{footnotesize}
				\caption*{\textbf{Table S2.2}: continued (2/5)}
				
				\begingroup
				\setlength{\tabcolsep}{10pt} % Default value: 6pt
				\renewcommand{\arraystretch}{1.5} % Default value: 1
				\begin{tabular}{c l c c c c c c}
					\toprule
					& \textbf{2) Parental body condition (standardized)} & \multicolumn{3}{c}{\textbf{Great tit}} & \multicolumn{3}{c}{\textbf{Blue tit}}\\
					
					& \textbf{Fixed effect}s & \textbf{Estimate} & \multicolumn{2}{c}{\textbf{95\%--CI}} & \textbf{Estimate} & \multicolumn{2}{c}{\textbf{95\%--CI}}\\
					\cline{2-8}
					
					\multirow{7}{*}{\rotatebox[origin=c]{90}{\parbox[c]{3cm}{\centering Intercept (value for  average fragment area)}}}
					
					& 3 species mixture & -0.046 & -0.497 & 0.404 & 0.258 & -0.424 & 0.94\\
					& Fsyl & -0.226 & -0.701 & 0.25 & 0.05 & -0.812 & 0.912\\
					& Fsyl \& Qrob & 0.052 & -0.447 & 0.551 & -0.203 & -0.851 & 0.445\\
					& Fsyl \& Qrub & -0.331 & -1.189 & 0.527 & 1.184 & -0.845 & 3.212\\
					& Qrob & -0.075 & -0.528 & 0.379 & 0.16 & -0.604 & 0.924\\
					& Qrob \& Qrub & 0.12 & -0.423 & 0.663 & 0.466 & -0.459 & 1.391\\
					& Qrub & 0.898 & -0.052 & 1.848 & -0.84 & -2.228 & 0.548\\
					\cline{2-8}
					\multirow{7}{*}{\rotatebox[origin=c]{90}{\parbox[c]{3cm}{\centering Slopes)}}} & slope 3 species mixtures:log fragment area & -0.002 & -0.659 & 0.655 & -0.192 & -0.917 & 0.532\\
					& slope Fsyl:log fragment area & -0.102 & -0.262 & 0.465 & 0.11 & -0.817 & 1.038\\
					& slope Fsyl \& Qrob:log fragment area &0.143 & -0.203 & 0.489 & 0.214 & -0.198 & 0.626\\
					& slope Fsyl \& Qrub:log fragment area & 0.731 & -0.411 & 1.874 & -1.392 & -3.669 & 0.886\\
					& slope Qrob:log fragment area & 0.343 & -0.056 & 0.743 & 0.029 & -0.541 & 0.599\\
					& slope Qrob \& Qrub:log fragment area & 0.877 & -0.505 & 2.258 & 2.318 & -0.499 & 5.135\\
					& slope Qrub:log fragment area & -0.875 & -2.461 & 0.71 & 1.421 & -0.608 & 3.45\\
					\cline{2-8}
					
					& standard deviation of random effect& 	0.379&&& 0.659&&\\	
					\bottomrule
				\end{tabular}\endgroup
			\end{footnotesize}
		\end{center}
		%\end{sidewaystable}
	\end{table}
\end{landscape}
\clearpage

\clearpage
\thispagestyle{plain}
\begin{landscape}
	%\begin{sidewaystable}
	\begin{table}
		\begin{center}
			\begin{footnotesize}
				\caption*{\textbf{Table S2.2}: continued (3/5)}
				
				\begingroup
				\setlength{\tabcolsep}{10pt} % Default value: 6pt
				\renewcommand{\arraystretch}{1.5} % Default value: 1
				\begin{tabular}{c l c c c c c c}
					\toprule
					& \textbf{3) Laying date (standardized)} & \multicolumn{3}{c}{\textbf{Great tit}} & \multicolumn{3}{c}{\textbf{Blue tit}}\\
					
					& \textbf{Fixed effect}s & \textbf{Estimate} & \multicolumn{2}{c}{\textbf{95\%--CI}} & \textbf{Estimate} & \multicolumn{2}{c}{\textbf{95\%--CI}}\\
					\cline{2-8}
					
					\multirow{7}{*}{\rotatebox[origin=c]{90}{\parbox[c]{3cm}{\centering Intercept (value for  average fragment area)}}}
					
					& 3 species mixture & 0.236 & -0.151 & 0.623 & 	0.117 & -0.443 & 0.677\\
					& Fsyl & 0.591 & 0.199 & 0.983 & 	-0.062 & -0.633 & 0.51\\
					& Fsyl \& Qrob & 0.312 & -0.093 & 0.718 & 	-0.126 & -0.624 & 0.372\\
					& Fsyl \& Qrub & 0.423 & -0.387 & 1.232 & 	0.069 & -1.3 & 1.438\\
					& Qrob & 0.107 & -0.253 & 0.467 & 	-0.962 & -1.584 & -0.341\\
					& Qrob \& Qrub & 0.085 & -0.367 & 0.536 & 	-0.477 & -1.057 & 0.103\\
					& Qrub & 0.288 & -0.53 & 1.106 &  	-0.304 & -1.844 & 1.236\\
					\cline{2-8}
					\multirow{7}{*}{\rotatebox[origin=c]{90}{\parbox[c]{3cm}{\centering Slopes)}}} & slope 3 species mixtures:log fragment area & -0.394 & -0.941 & 0.152 & 	0.188 & -0.416 & 0.793\\
					& slope Fsyl:log fragment area & -0.067 & -0.363 & 0.23 & 	-0.14 & -0.649 & 0.369\\
					& slope Fsyl \& Qrob:log fragment area & 0.152 & -0.118 & 0.422 & 	-0.143 & -0.463 & 0.176\\
					& slope Fsyl \& Qrub:log fragment area & 0.348 & -0.747 & 1.443 & 	0.391 & -1.28 & 2.062\\
					& slope Qrob:log fragment area & 0.355 & 0.005 & 0.705 & 	-0.177 & -0.54 & 0.186\\
					& slope Qrob \& Qrub:log fragment area & 0.111 & -1.035 & 1.256 & 	0.098 & -1.441 & 1.637\\
					& slope Qrub:log fragment area & -0.041 & -1.587 & 1.506 &  	0.235 & -2.387 & 2.857\\
					\cline{2-8}
					
					& standard deviation of random effect& 	0.139&&& 0.151&&\\	
					\bottomrule
				\end{tabular}\endgroup
			\end{footnotesize}
		\end{center}
		%\end{sidewaystable}
	\end{table}
\end{landscape}
\clearpage


\clearpage
\thispagestyle{plain}
\begin{landscape}
	%\begin{sidewaystable}
	\begin{table}
		\begin{center}
			\begin{footnotesize}
				\caption*{\textbf{Table S2.2}: continued (4/5)}
				
				\begingroup
				\setlength{\tabcolsep}{10pt} % Default value: 6pt
				\renewcommand{\arraystretch}{1.5} % Default value: 1
				\begin{tabular}{c l c c c c c c}
					\toprule
					& \textbf{4) Clutch size} & \multicolumn{3}{c}{\textbf{Great tit}} & \multicolumn{3}{c}{\textbf{Blue tit}}\\
					
					& \textbf{Fixed effect}s & \textbf{Estimate} & \multicolumn{2}{c}{\textbf{95\%--CI}} & \textbf{Estimate} & \multicolumn{2}{c}{\textbf{95\%--CI}}\\
					\cline{2-8}
					
					\multirow{7}{*}{\rotatebox[origin=c]{90}{\parbox[c]{3cm}{\centering Intercept (value for  average fragment area)}}}
					
					& 3 species mixture & 7.368 & 6.667 & 8.144 & 	10.582 & 9.549 & 11.727\\
					& Fsyl & 7.497 & 6.782 & 8.287 & 	9.712 & 8.776 & 10.748\\
					& Fsyl \& Qrob & 8.050 & 7.281 & 8.900 & 	11.221 & 10.354 & 12.161\\
					& Fsyl \& Qrub &8.012 & 6.527 & 9.836 & 	10.971 & 8.611 & 13.977\\
					& Qrob & 8.145 & 7.473 & 8.877 & 	13.004 & 11.806 & 14.324\\
					& Qrob \& Qrub & 8.155 & 7.288 & 9.127 & 	11.235 & 10.212 & 12.360\\
					& Qrub & 7.551 & 6.144 & 9.280 & 	10.018 & 7.619 & 13.174\\
					\cline{2-8}
					\multirow{7}{*}{\rotatebox[origin=c]{90}{\parbox[c]{3cm}{\centering Slopes)}}} & slope 3 species mixtures:log fragment area & -0.048 & -2.395 & 2.298 & 	-0.043 & -0.154 & 0.067\\
					& slope Fsyl:log fragment area & 0.006 & -0.070 & 0.082 & 	0.057 & -0.040 & 0.155\\
					& slope Fsyl \& Qrob:log fragment area & 0.003 & -0.064 & 0.070 & 	-0.016 & -0.068 & 0.035\\
					& slope Fsyl \& Qrub:log fragment area & 0.029 & -0.244 & 0.302 & 	-0.073 & -0.379 & 0.232\\
					& slope Qrob:log fragment area & -0.027 & -0.111 & 0.058 & 	0.036 & -0.021 & 0.092\\
					& slope Qrob \& Qrub:log fragment area & -0.260 & -0.542 & 0.021 & 	-0.108 & -0.365 & 0.148\\
					& slope Qrub:log fragment area & 0.123 & -0.263 & 0.510 & 	0.017 & -0.452 & 0.485\\
					\cline{2-8}
					
					& standard deviation of random effect& $<$ 0.001 &&& $<$ 0.001 &&\\	
					\bottomrule
				\end{tabular}\endgroup
			\end{footnotesize}
		\end{center}
		%\end{sidewaystable}
	\end{table}
\end{landscape}
\clearpage

\clearpage
\thispagestyle{plain}
\begin{landscape}
	%\begin{sidewaystable}
	\begin{table}
		\begin{center}
			\begin{footnotesize}
				\caption*{\textbf{Table S2.2}: continued (5/5)}
				
				\begingroup
				\setlength{\tabcolsep}{10pt} % Default value: 6pt
				\renewcommand{\arraystretch}{1.5} % Default value: 1
				\begin{tabular}{c l c c c c c c}
					\toprule
					& \textbf{5) Hatching success} & \multicolumn{3}{c}{\textbf{Great tit}} & \multicolumn{3}{c}{\textbf{Blue tit}}\\
					
					& \textbf{Fixed effect}s & \textbf{Estimate} & \multicolumn{2}{c}{\textbf{95\%--CI}} & \textbf{Estimate} & \multicolumn{2}{c}{\textbf{95\%--CI}}\\
					\cline{2-8}
					
					\multirow{7}{*}{\rotatebox[origin=c]{90}{\parbox[c]{3cm}{\centering Intercept (value for  average fragment area)}}}
					
					& 3 species mixture & 0.846 & 0.737 & 0.915 & 	0.921 & 0.856 & 0.958\\
					& Fsyl & 0.701 & 0.581 & 0.799 & 	0.675 & 0.415 & 0.859\\
					& Fsyl \& Qrob & 0.769 & 0.651 & 0.855 & 	0.888 & 0.742 & 0.956\\
					& Fsyl \& Qrub &0.714 & 0.462 & 0.879 & 	0.876 & 0.51 & 0.979\\
					& Qrob &0.83 & 0.741 & 0.893 & 	0.933 & 0.802 & 0.98\\
					& Qrob \& Qrub & 0.849 & 0.733 & 0.921 & 	0.795 & 0.568 & 0.92\\
					& Qrub & 0.739 & 0.437 & 0.912 &  	0.977 & 0.64 & 0.999\\
					\cline{2-8}
					\multirow{7}{*}{\rotatebox[origin=c]{90}{\parbox[c]{3cm}{\centering Slopes)}}} & slope 3 species mixtures:log fragment area & -0.225 & -1.213 & 0.763 & 	0.72 & -0.383 & 1.822\\
					& slope Fsyl:log fragment area & 0.11 & -0.265 & 0.484 & 	-0.596 & -1.58 & 0.388\\
					& slope Fsyl \& Qrob:log fragment area &-0.017 & -0.392 & 0.358 & 	0.21 & -0.358 & 0.778\\
					& slope Fsyl \& Qrub:log fragment area & 0.158 & -1.268 & 1.584 & 	0.51 & -1.998 & 3.017\\
					& slope Qrob:log fragment area & 0.004 & -0.512 & 0.52 & 	0.665 & -0.053 & 1.383\\
					& slope Qrob \& Qrub:log fragment area & -0.051 & -1.912 & 1.809 & 	2.494 & -1 & 5.987\\
					& slope Qrub:log fragment area & 1.129 & -1.443 & 3.702 &  	-0.811 & -5.519 & 3.898\\
					\cline{2-8}
					
					& standard deviation of random effect& $< $0.001&&& 0.751&&\\	
					\bottomrule
				\end{tabular}\endgroup
			\end{footnotesize}
		\end{center}
		%\end{sidewaystable}
	\end{table}
\end{landscape}
\clearpage

\clearpage
\thispagestyle{plain}
\begin{landscape}
	%\begin{sidewaystable}
	\begin{table}
		\begin{center}
			\begin{footnotesize}
				\caption*{\textbf{Table S2.3}: Akaike information criterion (AIC) and difference with the lowest AIC for the models ($\bigtriangleup$AIC) for models using Fragment area and models using Edge density. Models explaining most variation are indicated in bold.}
				
				\begingroup
				\setlength{\tabcolsep}{10pt} % Default value: 6pt
				\renewcommand{\arraystretch}{1.5} % Default value: 1
				\begin{tabular}{l l c c c c}
					\toprule
					\textbf{Response variable} & &  \multicolumn{2}{c}{\textbf{Great tit}} & \multicolumn{2}{c}{\textbf{Blue tit}}\\
					\hline
					
					& & \textbf{AIC} & \textbf{$\bigtriangleup$AIC}& \textbf{AIC} & \textbf{$\bigtriangleup$AIC}\\
					
					\cmidrule(lr){3-4}\cmidrule(lr){5-6}
					
					Fledging success & \textbf{Fragment area} & \textbf{505.13} & \textbf{0} & 	304.99 & 0\\
					& Edge Density & 507.36 & 2.23 & 	311.99 & 7\\
					
					Average fledgling body condition & \textbf{Fragment area} & \textbf{300.53} & \textbf{0} & 211.93 & 0\\
					& Edge Density & 304.98 & 4.45 & 213.53 & 1.6\\
					
					\cmidrule(lr){1-1}
					\cmidrule(lr){2-2}
					\cmidrule(lr){3-4}
					\cmidrule(lr){5-6}
					
					Nest box occupation & Fragment area & 272.3 & 0 & 	198.5 & 0\\
					&Edge Density & 275.28 & 2.98 &199.7 & 1.2\\
					
					Laying date & Fragment area & 560.47 & 3.49 & 	331.21 & 10\\
					& Edge Density & 556.98 & 0 & 	321.21 & 0\\
					
					Parental body condition & Fragment area & 592.19 & 8.69 & 354.4 & 0\\
					&Edge Density & 583.5 & 0 & 	354.87 & 0.47\\
					
					Clutch size & Fragment area & 873.52 & 0.06 & 	484.56 & 7.07\\
					&Edge Density & 873.46 & 0 & 	477.49 & 0\\
					
					Hatching success & Fragment area & 669.71 & 0.44 & 	429.3 & 0\\
					&Edge Density & 669.27 & 0 & 	-- & --\\
					
					
					
					\bottomrule
				\end{tabular}\endgroup
			\end{footnotesize}
		\end{center}
		%\end{sidewaystable}
	\end{table}
\end{landscape}
\clearpage


\clearpage
\thispagestyle{plain}
\begin{landscape}
%\begin{sidewaystable}
\begin{table}
	\begin{center}
		\begin{footnotesize}
			\caption*{\textbf{Table S2.4}: Results for the SEMs linking tree composition and fragment area to fledgling body condition through changes in maternal condition (clutch size) and resource availability (frass mass) for (a) Great Tit (\textit{Parus major}) and (b) Blue Tit (\textit{Cyanistes caeruleus}). Here we present the estimated intercept and slopes, including lower and upper limits of the 95\% confidence interval for all composite models of the SEM. Fsyl = beech (\textit{Fagus sylvatica}), Qrob $=$ pedunculate oak (\textit{Quercus robur}), Qrub = red oak (\textit{Q. rubra}). (Continued on next pages; 1/3)}
			
			\begingroup
			\setlength{\tabcolsep}{10pt} % Default value: 6pt
			\renewcommand{\arraystretch}{1.5} % Default value: 1
			\begin{tabular}{l c c c c c c}
				\toprule
				\multicolumn{7}{l}{\textbf{1) frass $\sim$ tree species combination $$ log fragment area}}\\
				& \multicolumn{3}{c}{\textbf{Great tit}} & \multicolumn{3}{c}{\textbf{Blue tit}}\\
				\cmidrule(lr){2-4}\cmidrule(lr){5-7}
				
				& \textbf{Estimate} & \multicolumn{2}{c}{95\%--CI} & \textbf{Estimate} & \multicolumn{2}{c}{95\%--CI} \\
				\cmidrule(lr){2-7}
				
				3 species mixture & 0.170 & -0.449 & 0.790 & 	-0.310 & -0.870 & 0.251\\
				Fsyl & -0.755 & -1.431 & -0.080 & 	-0.981 & -1.662 & -0.299\\
				Fsyl \& Qrob & -0.338 & -0.999 & 0.323 & 	-0.597 & -1.180 & -0.014\\
				Qrob &1.252 & 0.676 & 1.828 & 	0.868 & 0.263 & 1.474\\
				Qrob \& Qrub & 0.034 & -0.612 & 0.681 & 	0.708 & 0.080 & 1.336\\
				Qrub & -0.800 & -1.786 & 0.186 & 	-0.597 & -2.455 & 1.261\\
				\cline{1-7}
				3 species mixtures:log fragment area & -0.349 & -1.141 & 0.442 & 	-0.136 & -0.930 & 0.658\\
				Fsyl:log fragment area & 0.133 & -0.565 & 0.832 & 	0.030 & -0.674 & 0.733\\
				Fsyl \& Qrob:log fragment area &0.037 & -0.387 & 0.461 & 	-0.072 & -0.450 & 0.306\\
				Qrob:log fragment area & 0.277 & -0.274 & 0.828 & 	0.020 & -0.424 & 0.463\\
				Qrob \& Qrub:log fragment area & -0.474 & -2.328 & 1.379 & 	1.464 & -0.301 & 3.229\\
				Qrub b:log fragment area & 0.488 & -1.118 & 2.095 & 	0.220 & -2.469 & 2.910\\
				
				\bottomrule
			\end{tabular}\endgroup
		\end{footnotesize}
	\end{center}
	%\end{sidewaystable}
\end{table}
\end{landscape}
\clearpage



\clearpage
\thispagestyle{plain}
\begin{landscape}
	%\begin{sidewaystable}
	\begin{table}
		\begin{center}
			\begin{footnotesize}
				\caption*{\textbf{Table S2.4}: continued (2/3)}
				
				\begingroup
				\setlength{\tabcolsep}{10pt} % Default value: 6pt
				\renewcommand{\arraystretch}{1.5} % Default value: 1
				\begin{tabular}{l c c c c cc}
					\toprule
					\multicolumn{7}{l}{\textbf{2) clutch size $\sim$ tree species combination $\ast$ log fragment area}}\\
					& \multicolumn{3}{c}{\textbf{Great tit}} & \multicolumn{3}{c}{\textbf{Blue tit}}\\
					\cmidrule(lr){2-4}\cmidrule(lr){5-7}
					
					& \textbf{Estimate} & \multicolumn{2}{c}{95\%--CI} & \textbf{Estimate} & \multicolumn{2}{c}{95\%--CI} \\
					\cmidrule(lr){2-7}
					
				3 species mixture &7.759 & 6.786 & 8.872 & 	9.453 & 8.442 & 10.585\\
				Fsyl & 7.442 & 6.584 & 8.412 & 	8.607 & 7.483 & 9.899\\
				Fsyl \& Qrob &8.382 & 7.373 & 9.531 & 	9.820 & 8.866 & 10.876\\
				Qrob &7.954 & 7.058 & 8.965 & 	11.778 & 10.520 & 13.186\\
				Qrob \& Qrub & 8.562 & 7.423 & 9.875 & 	10.670 & 9.675 & 11.767\\
				Qrub & 7.350 & 5.304 & 10.184 & 	9.505 & 6.068 & 14.888\\
				\cline{1-7}
				3 species mixtures:log fragment area & 0.150 & -0.039 & 0.338 & 	-0.015 & -0.163 & 0.132\\
				Fsyl:log fragment area & -0.025 & -0.130 & 0.079 & 	0.155 & 0.028 & 0.283\\
				Fsyl \& Qrob:log fragment area &0.040 & -0.041 & 0.120 & 	-0.043 & -0.117 & 0.030\\
				Qrob:log fragment area & -0.082 & -0.184 & 0.020 & 	0.050 & -0.038 & 0.137\\
				Qrob \& Qrub:log fragment area & -0.299 & -0.699 & 0.101 & 	-0.180 & -0.489 & 0.129\\
				Qrub b:log fragment area & 0.192 & -0.359 & 0.743 & 	0.079 & -0.595 & 0.754\\
					
					\bottomrule
				\end{tabular}\endgroup
			\end{footnotesize}
		\end{center}
		%\end{sidewaystable}
	\end{table}
\end{landscape}
\clearpage

\clearpage
\thispagestyle{plain}
\begin{landscape}
	%\begin{sidewaystable}
	\begin{table}
		\begin{center}
			\begin{footnotesize}
				\caption*{\textbf{Table S2.4}: continued (3/3)}
				
				\begingroup
				\setlength{\tabcolsep}{10pt} % Default value: 6pt
				\renewcommand{\arraystretch}{1.5} % Default value: 1
				\begin{tabular}{l c c c c cc}
					\toprule
					\multicolumn{7}{l}{\textbf{3)average fledgling body condition  $\sim$ clutch size $+$ tree species combination $\ast$ log fragment area}}\\
					& \multicolumn{3}{c}{\textbf{Great tit}} & \multicolumn{3}{c}{\textbf{Blue tit}}\\
					\cmidrule(lr){2-4}\cmidrule(lr){5-7}
					
					& \textbf{Estimate} & \multicolumn{2}{c}{95\%--CI} & \textbf{Estimate} & \multicolumn{2}{c}{95\%--CI} \\
					\cmidrule(lr){2-7}
					
					3 species mixture &19.041 & 17.191 & 20.892 & 	11.629 & 10.050 & 13.208\\
					Fsyl & 17.527 & 15.280 & 19.775 & 	11.165 & 9.285 & 13.045\\
					Fsyl \& Qrob &20.788 & 18.677 & 22.898 & 	11.284 & 9.461 & 13.107\\
					Qrob &20.244 & 18.213 & 22.275 & 	11.482 & 9.558 & 13.405\\
					Qrob \& Qrub &18.762 & 16.340 & 21.185 & 	11.137 & 9.430 & 12.845\\
					Qrub & 20.377 & 17.000 & 23.754 & 	11.947 & 9.100 & 14.794\\
					\cline{1-7}
					3 species mixtures:log fragment area &-0.224 & -1.637 & 1.189 & 	-0.198 & -0.970 & 0.574\\
					Fsyl:log fragment area & -1.612 & -2.706 & -0.519 & 	-0.198 & -0.970 & 0.574\\
					Fsyl \& Qrob:log fragment area &0.240 & -0.404 & 0.885 & 	-0.519 & -0.519 & -0.519\\
					Qrob:log fragment area & -0.334 & -1.240 & 0.571 & 	0.885 & 0.885 & 0.885\\
					Qrob \& Qrub:log fragment area & -2.703 & -7.155 & 1.749 & 	0.571 & 0.571 & 0.571\\
					Qrub:log fragment area & -0.069 & -4.698 & 4.561 & 	1.749 & 1.749 & 1.749\\
					Clutch size & -0.355 & -0.570 & -0.139 & 	-0.015 & -0.169 & 0.139\\
					
					\bottomrule
				\end{tabular}\endgroup
			\end{footnotesize}
		\end{center}
		%\end{sidewaystable}
	\end{table}
\end{landscape}


\cleardoublepage
\thispagestyle{plain}
\hbox{}
\clearpage

	%%%%%%%%%%%%%%%%%%%%%%%%%%%%%%%%%%%%% CHAPTER THREE  %%%%%%%%%%%%%%%%%%%%%%%%%%%%%%%%%%%%%%%%%%
	\CenterWallPaper{1}{CH3.jpg}
	\newpage{\thispagestyle{empty}\cleardoublepage}
	\ClearWallPaper
	\pagestyle{mainmatter}
	\chapter{Winter foraging outside forests carries a reproductive cost in a resident forest bird} \label{chapter3}
	\chaptermark{Winter foraging outside forests carries a reproductive cost}
	\lettergroup{\thechapter}

	\begin{flushright} \color{gray} Daan Dekeukeleire\\ Diederik Strubbe\\ Luc Lens\\ 
	
\end{flushright}

\vspace*{\fill}
\noindent \color{gray} $\lhd$ To increase brood size in a challenge experiment, females were induced to lay extra eggs by collecting the first two eggs. These eggs were stored in nests of moss before returning them to the nest box. Photo: Daan Dekeukeleire.

\color{black}
\newpage
	
	
		\section{Abstract}
		
Non-fatal effects during one part of an individual's annual cycle can lead to carry over effects during the next season. In resident species, variation in movement behaviour and habitat use are expected to be especially notable when resource availability is low, such as during winters and in in fragmented landscapes. Yet, how such variation in winter behaviour affects reproductive success remains poorly understood. In this study, we use a combined observational and experimental approach to investigate how winter movement behaviour and habitat use in a spatially-structured population of great tits (\textit{Parus major}) varies with food availability, forest fragment size and phenotypic stress markers, and to what extent this affects their subsequent reproductive success and physiological condition during spring. Therefore, we radio-tracked 67 individuals in 20 forest fragments varying in size from 1.3 to 225 hectares in a highly human-modified landscape in northern Belgium. In one medium sized forest fragment, we experimentally increased brood size to investigate how variation in winter forest use impacts reproductive success. We show that individuals exposed to low food availability in forests during winter spend more time foraging in the landscape matrix. Foraging outside the forest in winter was associated with a considerable reproductive cost, as females foraging outside the forest in winter had fewer young surviving until fledging. However, foraging outside the forest was unrelated to phenotypic stress markers. Our study stresses the importance of individual-level and full annual data for an understanding of population dynamics, information vital for management and conservation actions. 		
		
		
\clearpage
	
	\section{Introduction}
	
	
Animals are predicted to maximize their fitness by adjusting their movement behaviour in response to the spatial and temporal heterogeneity of the environment \citep{Nathan2008}. Such optimization may occur at different spatial and temporal scales, ranging from daily foraging within territory boundaries to landscape-level dispersal at intergenerational scales \citep{Jeltsch2013}. The ability of animals to move across landscapes can be impacted by the process of anthropogenic habitat fragmentation, the subdivision of continuous habitat in small, isolated fragments separated by a matrix of human-transformed land cover (reviewed in \citealt{Cosgrove2018, Fischer2007}). At a landscape level, impaired movement among isolated habitat fragments can lead to a reduction in gene flow and demographic exchange between subpopulations, and ultimately, compromise local survival in small habitat fragments (e.g. \citealt{Cooper2002, Delaney2010}). However, habitat fragmentation can also affect movement patterns at the scale of individual home ranges, with direct and/or indirect fitness effects \citep{Cosgrove2018}. Depending on a species' ability to monopolize critical resources and cross the landscape matrix, home range sizes can change in different directions under increasing habitat fragmentation (reviewed in \citealt{Harris2002}). If individuals regularly venture into the matrix at no apparent cost, home range sizes can be expected to increase in fragmented landscapes \citep{Hansbauer2008, Beasley2010, Snell2020}. As an example, the Eastern bettong (\textit{Bettongia gaimardi}), a forest-specialized marsupial, occupied larger home ranges in landscapes comprising multiple forest fragments \citep{Gardiner2019}. Yet, in many species, movements across the matrix do bear direct or indirect costs, such as an increased predation risk, an elevated energy expenditure or a decreased nutritional condition, which can be expected to result in smaller home ranges and lower levels of resource availability in highly fragmented landscapes \citep{Redpath1995, Doherty2003, Verbeylen2009}. As an example, Northern saw-whet owls (\textit{Aegolius acadicus}) breeding in small forest patches with large inter-patch distances occupied smaller home ranges and provided fewer food to their offspring than individuals breeding in larger forests, resulting in chronically stressed nestlings and a lower fledging success in the former \citep{Hinam2008}. Likewise, great tits (\textit{Parus major}) that regularly crossed forest gaps of 50--100 m during the breeding season had a higher daily energy expenditure, and reared fewer and lighter fledglings, than individuals that always foraged inside the forest interior \citep{Hinsley2000, Hinsley2008}.\\

In most species, breeding and non-breeding periods follow each other in annual cycles, whereby non-fatal effects, such as poor physical condition, during one period may carry over and affect vital processes, such as reproductive success or survival, during the next one. This makes subsequent periods inextricably linked \citep{Harrison2011, Marra2015}. In support of this, American Redstarts (\textit{Setophaga ruticilla}) that held winter territories in habitat with higher insect abundance, were in better condition in spring, left their wintering grounds earlier, and had a higher reproductive success compared to individuals wintering in low-quality habitat \citep{Norris2004, Studds2005, Reudink2009}. However, due to methodological challenges to study individuals year-round and a persistent belief that events occurring during the breeding season have the strongest effects on individual fitness, studies investigating seasonal carry-over effects are still comparatively rare and largely observational \citep{OConnor2014, Marra2015}. Moreover, carry-over effects are often only manifested when individuals face environmental challenges \citep{OConnor2015}. For instance, experimental elevation of the stress hormone corticosterol in a wild fish population during autumn did not lead to observable carry-over effects under normal weather conditions but caused earlier mortality during an extreme winter \citep{OConnor2010}. Likewise, habitat fragmentation signatures on movement behaviour can be expected to be strongest when conditions are most challenging, such as during winter in temperate regions of the northern hemisphere. During this period, high energetic costs of thermoregulation, in combination with low resource availability, may force individuals of mobile species, such as birds, to expand their home ranges, resulting in increased foraging movement across the landscape matrix when their natural habitat is fragmented (e.g. \citealt{Wiktander2001}). In addition, habitat fragmentation may intensify natural seasonal fluctuations in resource availability or environmental conditions \citep{Ewers2013}. Along these lines, bird populations residing in small and isolated forest remnants with a large amount of forest edge have been shown to experience colder temperatures and higher levels of energy expenditure during winter compared to those from large, continuous forest fragments \citep{Latimer2017}. Yet, also in the field of movement ecology, potential carry-over effects between different stages of the annual cycle remain poorly documented.\\

In vertebrates, social or environmental stressors, such as food shortages, trigger the release of glucocorticoid hormones, mediated by the hypothalamic-pituitary-adrenal (HPA) axis \citep{Wingfield2013, Creel2013}. In birds, the main glucocorticoid is corticosterone (CORT), which mediates physiological processes that steer behavioural and physiological changes and, in turn, enables individuals to better cope with stressors. Baseline CORT levels therefore mainly vary between individuals depending on their nutritional condition and energy expenditure. Although acute CORT responses have been linked to dominance \citep{Pravosudov2003}, dominants and subordinates often show similar baseline CORT levels in birds \citep{Creel2013}. As the main cues of CORT production are known to trigger movement, it is not surprising that CORT levels are also strongly associated with movement decisions (reviewed in \citealt{Goossens2020}). CORT is passively incorporated in feathers during growth, making feather CORT values an integrated measure of plasma CORT levels, during the entire period of feather growth \citep{Bortolotti2008, Romero2016}. Moreover, once the feather is fully grown, CORT levels in feathers are stable over time and can easily be sampled many months after the feather has grown, and therefore, be used to investigate links between different parts of an individual's annual cycle \citep{Bortolotti2010}.\\

Forest-dependent tits and chickadees (Paridae) residing in fragmented landscapes are very well suited to study carry-over effects of winter movement behaviour on subsequent breeding performance. During autumn and winter, these species form loose mixed-species flocks that roam forests to forage on ephemeral and patchy food sources such as seeds and nuts \citep{Ekman1989, Farine2012}. Such flocks generally avoid crossing open landscapes, due to real or perceived predation or energetic costs \citep{Hinsley2000, Rodriguez2001}, but do forage outside forests when food resources are scarce. For example, North American black-capped chickadees (\textit{Poecile atricapillus}) that inhabit small forest fragments ventured farther into the open matrix to forage at experimental feeders, compared to individuals inhabiting large fragments of the same tree species composition, or small fragments where food was experimentally supplemented \citep{Turcotte2003}. Variation in winter movement has also been linked to dominance hierarchies, with dominant male great tits staying more in forest and subordinate females and first-year individuals venturing more into the matrix and residential gardens, during winters with low food availability \citep{Gosler1987}. It has been predicted that such individual variation in movement behaviour during winter can play an important role in shaping variation in breeding performance \citep{Montreuil-Spencer2019}.\\

We here study how winter movements and habitat use in a spatially-structured population of great tits vary with extrinsic factors, namely food availability and forest fragment size, and intrinsic factors, namely sex, age, body condition and CORT levels, and to what extent this affects their subsequent reproductive success and physiological condition during spring. We hypothesize that (i) individuals in small forests with low food availability during winter spend more time foraging in the landscape matrix, and (ii) the latter negatively affects their subsequent breeding success and/or physiological condition (see conceptual figure \ref{fig3-1}). To test the first hypothesis, we selected 20 deciduous forest fragments in a highly human-modified landscape in northern Belgium, ranging in size between 1.3 and 225.6 ha. In this set of fragments, we radio-tracked 67 individuals, each for 12 h, over the course of two winters: one during a beech mast year when food was abundant, and one during a non-mast year when food was much scarcer. We predict more foraging outside forests by individuals from small fragments and during the non-mast year (extrinsic factors), and by subordinate individuals such as females, first-year birds, and individuals with high CORT levels (intrinsic factors). We further predict larger distances travelled when foraging outside compared to inside forests. To test the second hypothesis, we selected one medium-sized forest fragment out of the 20 investigated where we combined observational with experimental data. To increase the number of tracked individuals and the length of the tracking period, we inferred winter movement behaviour of great tits residing in this forest from the visits of 163 PIT-tagged individuals at feeding stations placed both inside and outside the forest during 83 days. During the following spring, we experimentally altered brood sizes of a number of PIT-tagged females to assess potential reproductive carry-over costs from previous winter when conditions to raise offspring are challenging. We predict lower reproductive success in females that foraged more outside forests during winter. In addition to assessing reproductive costs of winter foraging in the form of low nestlings survival until fledging (fledging success) or poor fledgling condition (which in tits predicts post-fledging survival; \citealt{Perrins1965, Monros2002}), we also tested carry-over effects on maternal blood parasite load. Tits most often harbor chronic infections of haemosporidian parasites, which are transferred by blood-sucking diptera. Yet blood parasite load increases when immune functioning is compromised due to increased parental effort and is a good predictor of post-breeding survival \citep{Christe2012, Puente2010}.\\

	
%Figure 3.1
\begin{figure}[h]
	\begin{center}
		\includegraphics[width=\textwidth]{fig3-1.png}
	\end{center}
	\caption{Conceptual figure showing hypothesized links between extrinsic factors (beech mast and fragment size) and intrinsic factors (body condition, age, sex and feather CORT), winter movement and habitat use, and the effects on reproductive success and physiological condition. The part of these links investigated through the observational radio tracking study is indicated in blue, while the part investigated in the challenge experiment is indicated in red.}\label{fig3-1}
\end{figure}	
\clearpage	
	
	
	\section{Material \& Methods}
	\subsection*{Study area}
	
Our study was carried out in a landscape in the north of Belgium that is highly modified by human activities. Land cover data for the study area were obtained from a high resolution land use database, the Biological Valuation Map (BVM), covering more than 100 habitat and vegetation types \citep{Vriens2011}. Shapefiles were separated into four habitat types, i.e. `forest', `agricultural land', `residential areas' and `other' (comprising orchards, wetlands, moorlands, etc.), for which raster layers were created (resolution of 2 m). Within this human-dominated landscape matrix, we selected 20 mixed deciduous forest fragments, ranging from 1.31 to 225.61 ha (mean 37.47 ha) and dominated by three regionally common species: the native beech and pedunculate oak, and the non-native red oak (figure \ref{fig3-2}). All selected forests consisted of mature forest stands ($>$ 60 years), with similar land-use history (continually forested since at least 1850), management regime (high forest) and soil conditions (sandy loam). Based on these similarities, we considered fragment size as a justified proxy for resource availability. Each forest was equipped with standard nest boxes for great tits (height 1.5 m, dimensions 23 $\times$ 9 $\times$ 12 cm, entrance 32 mm) which were frequently used as night roosts during winter. One of the forests, a 39.51 ha fragment in Gontrode, Melle (50.975$^{\circ}$N, 3.799$^{\circ}$E) (henceforth: experimental site) was selected for a detailed within-population study and challenge experiment. At this site, a total of 85 standard nest boxes were put up during autumn 2015.\\

Beech mast was used as a proxy for the availability of natural food in winter \citep{Perdeck2000}. Beech mast is itself an important winter food source for great tits, but is also positively correlated with the seed set of several other tree species, such as alder (\textit{Alnus}), birch (\textit{Betula}), hazel (\textit{Corylus}), hawtorn (\textit{Crataegus}) and spruce (\textit{Picea}) \citep{Perdeck2000, Tinbergen1985}. Data on beech mast availability were obtained from the Flemish `Forest condition monitoring' program ran by the Institute for Nature and Forest (INBO). This long-term dataset is collected at 11 forest sites in northern Belgium, among which our experimental site \citep{Sioen2020}. As beech mast production is synchronized over large regions \citep{Perdeck2000, Chamberlain2007}, this dataset can be considered representative for all 20 selected forests.

	
%Figure 3.2
\begin{figure}[h!]
	\begin{center}
		\includegraphics[width=\textwidth]{fig3-2.png}
	\end{center}
	\caption{Map of the study area in northern Belgium. Green patches represent forest fragments. The 20 fragments selected for radio tracking are coloured in dark green. The inset depicts the experimental site, were the location of the nest boxes (open circles) and the RFID-feeding stations (red triangles) are indicated.}\label{fig3-2}
\end{figure}



	\subsection*{Winter habitat use and movement behaviour}
	
In each of the 20 selected forests, we captured 3--4 individuals during nightly nest box controls during December till February of 2015--16 and 2016--17. All individuals were measured (tarsus length, to the nearest 0.01 mm; wing length, to the nearest 0.5 mm; and body mass, to the nearest 0.01 g), sexed and aged (first year or adult) based on plumage characteristics \citep{Svensson1992}, and fitted with a unique metal rings of the Belgian Ringing scheme. Prior to release, a radio tag (0.60 g, 150-151 MHz, 60 bpm, 10 cm antenna, Telemetrie-Service Dessau, Germany) was glued to the central tail feathers (`tail-mount', \citealt{Kenward2000}). After release, each of the 67 tagged individual was allowed 24 hours of habituation before radio tracking started. All individuals were resident birds, and spent each night in the forest, albeit not always in the same nest box. Each individual was tracked for a total of 12 hours during the next two to four days, divided over six two-hour tracking sessions that started when the focal individual was encountered first during 30 min of random searching effort. We choose a tracking duration of 12 hours to maximize the amount of individuals we could track. Two sessions took place between 8--10 am, two between 10:30--12:30 am, and two between 2--4 pm. Radio tracking was carried out by one person using a three-element yagi-antenna and Sika receiver (Biotrack Ltd, UK). Locations were fixed every 10 min with `homing in' to attain the closest proximity and most accurate location \citep{Kenward2000}, whenever possible with visual confirmation (without disturbance). Based on these locations, we used integrated Step-Selection Functions (iSSFs) in r package `amt' \citep{Signer2019} as a metric for habitat use. We therefore generated 20 random steps (with an observed location fix as the starting point) for each observed step (i.e. two consecutive location fixes) by fitting a gamma distribution to the observed step lengths and a von Mises distribution to the turning angles (following \citealt{Signer2019}). We only included data for which the time between consecutive location fixes was shorter than 30 min (to account for the different tracking sessions).\\

In the experimental forest, we derived winter habitat use from visiting rates of PIT-tagged individuals at feeding stations placed inside and outside of the forest. Here, great tits were captured both with mist nets and nest box controls (night) during November 2016. All individuals were measured, sexed and aged (see previous page), and the left and right second-to-outer tail feathers were collected for feather corticosterone analysis (see next page). In addition to a numbered metal ring, each trapped individual was fitted with a plastic ring containing a unique PIT-tag (IB Technology, Aylesburyon, UK). During winter, we ran six sunflower feeding stations equipped with a RFID-antenna (ANTC100, 10 cm diameter) connected to a stationary decoder (LID650, Dorset ID, The Netherlands). All feeding stations were established within the mean foraging distance of the radio tracked great tits (762.2 m $\pm$ 76.6 m), with two stations located in the centre of the forest and four others located in residential gardens near the forest where birds had already been fed in previous winters. To maximize detection of foraging activity outside the forest, feeding stations were positioned at different sides of the forest; i.e. one north of the forest, two west and one south (east of the forest there were only open fields; see figure \ref{fig3-2})). As tagged individuals were observed on a maximum of two feeding stations per day only, we consider detection probability inside and outside of the forest to be equal. All feeding stations were set up during the first week of December 2016, and after a 15-day habituation period, the time and location of each visit of 163 tagged individuals were logged from dawn until dusk during 83 consecutive days. A total of eight PIT-tagged individuals were additionally equipped with radio tags (tracking method as before). This confirms that individuals that were only recorded on forest feeders used forest habitat during the 12 hours of radio-tracking, while individuals that were recorded on feeding stations outside the forest also visited feeders in nearby gardens, hence supporting the use of PIT-tag data for studying winter movement behaviour (see supplementary material S3.1). Winter forest use was calculated as the proportion of total number of visits that took place at feeding stations located inside the forest.

	
	
	
	\subsection*{Challenge experiment}
	
From March 1st 2017 onwards, all nest boxes at the experimental forest were visited at least three times per week to monitor nest building, and were inspected daily from March 25th until June 15th. The presence and identity of PIT-tagged females was inferred from a distance with a hand held PIT-tag reader (LID-575ISO, Trovan, UK), i.e. without disturbing the individual. A total of 32 PIT-tagged females, for which winter movement data were available and that occupied a breeding nest box in the forest, were selected for the brood enlargement treatment. We therefore collected the first two eggs on the morning they were laid as this was earlier shown to trigger a female to lay one extra egg to compensate the induced loss \citep{Oppliger1996, Visser2001}. Following the procedure outlined in \citet{Visser2001}, the removed eggs were numbered with a non-soluble marker pen and kept in a bed of moss at ambient temperature in darkness, and turned twice a day. Both eggs were returned to the original nest when a further five eggs had been laid and additionally supplemented with two additional eggs (also marked with non-soluble pen) collected from a synchronous nest in a nearby forest fragment. As a result of this treatment, broods in all experimental nests were increased by three eggs, of which one was produced by the female herself. A total of 10 females that also bred in nest boxes in the forest but had not been equipped with a PIT-tag during autumn, were selected for a control treatment. Their nests were also visited daily, and the first two eggs were also marked with non-soluble pen. On the fifth day, both eggs were swapped with eggs from another synchronous nest, but the brood size was unaltered. Compared to the control group, experimental nests had on average three more eggs, indicating that the treatment had been successful. Laying date, parental age, parental wing length (which in great tits predicts dominance; \citealt{Sandell1991}) and hatching success (proportion of the eggs that hatched) did not differ between control and experimental nests (supplementary material S3.2).\\

When nestlings were 7 to 10 days old, both parents were captured while feeding their young and were measured (tarsus length, wing length and body mass). Furthermore, a blood sample (75 $\mu$l) was collected in a heparinized capillary tube via brachial vein puncture and stored at -20$^{\circ}$C for analysis of blood parasites (see next page). To account for potential daily variation in parasite prevalence (e.g. \citealt{VanHemert2019}), adults were always captured in the morning. As females bear the largest reproductive costs (e.g. \citealt{Visser2001}), we only included females in further analyses. Nestlings were measured (tarsus and body mass) and ringed at an age of 15 days \citep{Matthysen2011}. 

	
	
	\subsection*{Metrics of individual condition and stress}
	
Body condition of individuals was quantified using the scaled-mass index (SMI; \citealt{Peig2009}). This index adjusts the mass of all individuals for a similar body size, using the equation of the linear regression of log-body mass on ln-tarsus length estimated by type 2 (standardized major axis; SMA) regression. SMA regression was calculated using the smatr package \citep{Warton2012}. We calculated the SMI for (i) all radio tracked individuals, measured when they were tagged during winter, (ii) all PIT-tagged individuals, measured when they were captured during autumn prior to tracking, and (iii) all nestlings, when they were measured at age 15.\\

Feather corticosterone (CORTf) provides a retrospective view on stress experienced during feather growth \citep{Bortolotti2008, Johns2018}. CORTf was quantified in 20 females at the experimental site. These individuals were captured during November 2017, and their two second-to-outer tail feathers were then collected. In our population, these original feathers are grown during post-breeding and post-juvenile moult. Note that in our population all first-year birds moult all their tail feathers, and moult typically finishes between the end of August and the first week of October \citep{Dhondt1973}. Average growth bar widths of each feather were measured to obtain a metric for feather growth rate \citep{Brodin1993, Grubb2006}. Finally, CORTf was quantified using a validated ultra-performance liquid chromatography coupled to tandem mass spectrophotometry (UPLC-MS/MS) method \citep{SallehHudin2018}. Full details on the procedure can be found in sectie \ref{mm4}. To account for the time-dependent deposition of CORT in feathers during growth \citep{Romero2016}, we divided CORTf values by the average feather growth bar width. In our population, the standard approach, i.e. dividing CORTf values by feather length, is not appropriate as feather length was not related to the feather growth rate (data not shown).
	

	\subsection*{Blood parasite screening}
	
Blood samples were screened for parasite infections using a nested PCR protocol (\citealt{Hellgren2004} modified as in \citealt{Jenkins2015}). Briefly, we performed a first PCR using HaemNF1 and HaemNR3 primer pair targeting a conserved region of the haemosporidian (\textit{Plasmodium}, \textit{Haemoproteus} and \textit{Leucocytozoon}) cytochrome b gene. Using the product of the first PCR as template, we then amplified a 479 bp fragment with the \textit{Plasmodium}- and \textit{Haemoproteus}-specific HaemF and HaemR2 primers pair and the \textit{Leucocytozoon}-specific HaemFL and HaemR2L primers pair. We visualized PCR products with agarose gel electrophoresis (2 \%). Nested PCR were performed in duplicates and a sample was considered positive if we observed amplification in at least one run. PCR products that were amplified by HaemF and HaemR2 primers were sequenced in both directions by Sanger sequencing (Microsynth AG, Balgach, Switzerland) in order to determine whether the infection was caused by a \textit{Plasmodium} sp. or a \textit{Haemoproteus} sp. \textit{Haemoproteus} was only encountered in a single sample, while \textit{Plasmodium} was encountered in 60 \% of the samples. Next, we performed quantitative PCR on samples infected by \textit{Plasmodium} sp. to measure relative parasitaemia (\citealt{Christe2012} modified as in \citealt{Jenkins2015}). Host and parasite genes were amplified in different wells but in the same plate, in triplicates. A series of twofold dilutions of two pooled infected samples (starting from 20ng/$\mu$L) was used to establish a standard curve, computed as the mean $Ct$ (cycle threshold value) as a function of the common logarithm of the dilution factor. The standard curve was also amplified on the three plates. Reactions were run in a final volume of 40$\mu$L, including 20$\mu$l of Takyon Low ROX Probe 2X MasterMix (Eurogentec, Seraing, Belgium), 4 $\mu$l of genomic DNA (5ng/$\mu$L), 0.9$\mu$M of each primer, 0.2$\mu$M of each probe, and 8$\mu$L of ultrapure water. qPCR was performed in a 7500 real-time PCR System (Applied Biosystem, Foster City, CA, United States) with the following thermal profile: 2 min at 50$^{\circ}$C, 10 min at 95$^{\circ}$C, followed by 48 cycles of 15 sec at 95$^{\circ}$C and 1 min at 54$^{\circ}$C. $Ct$ value was estimated as the mean of the three replicates. \\

\noindent Host and parasite DNA concentrations ($\alpha$) were calculated as:
\begin{equation*}
	\alpha = 10^{\frac{Ct-I}{m}}
\end{equation*}
$Ct$ being the mean of the measured $Ct$, $I$ the intercept of the standard curve and $m$ the slope of the standard curve. Relative parasitaemia (R) was calculated as the ratio between parasite and host DNA:
\begin{equation*}
	R = \frac{\alpha_{parasite}}{\alpha_{host}}
\end{equation*}
$R$ was log-transformed in order to normalise the distribution.


	\subsection*{Statistical analyses}
	
To test our first hypothesis, we modelled the habitat use by radio-tagged individuals in relation to extrinsic and intrinsic factors. Therefore, we constructed a generalized linear mixed models (GLMM) with a Poisson distribution using the glmmTMB package \citep{Brooks2017}. As the interaction between year and habitat type was significant (p $<$ 0.001), models were run for each year separately. In each model, we compared characteristics of observed steps with characteristics of 20 alternative steps sharing the same starting point. Explanatory variables were the habitat type (factor with four levels) of the endpoint of the observed or alternative step (forest, agricultural, residential, other), forest fragment size, and the age, sex and body condition (SMI) of the tagged individual. To investigate differences in habitat use among individuals residing in forest fragments of different size, a two-way interaction between fragment size and habitat type was included in the model. As the distance covered and the turning angle inherently influence the habitat type of the endpoint of a step, we included step length and turning angle in the model to account for this, as advised by \citet{Forester2009}. Continuous variables (fragment size and SMI) were mean-centered and scaled prior to running the model, and individual ID was included as a random effect to account for individual variation. Following \citet{Muff2020}, we fixed the random intercept variance to $10^6$ to avoid shrinkage and subsequent bias. To model distance covered (step length, i.e. distance between two consecutive location fixes in 10 min time) in relation to extrinsic and intrinsic factors, we constructed a linear mixed effect (LMM) in the package lme4 \citep{Bates2015} with p-values obtained using the package lmerTest \citep{Kuznetsova2017}. Habitat type (forest vs non-forest), age, sex, body condition and fragment size were included as explanatory variables. To investigate differences in step length between habitats in both winters (and, hence, mast availability), a two-way interaction between habitat type and year was included in the model. To account for individual variation, individual ID was included as a random effect.\\

To model the propensity of PIT-tagged individuals to forage outside the experimental forest in relation to intrinsic factors (Prediction ii), we constructed a GLMM with a binomial distribution with age, sex, CORTf and SMI as explanatory variables. To test carry-over effects of foraging behaviour outside the forest during winter on reproductive success and physiological condition during the subsequent spring (prediction iii), we used (generalized) linear models ((G)LMMs) to model two reproductive parameters (i.e. fledgling success and the body condition of fledglings (SMI)) and one physiological parameter (i.e. maternal blood parasite load) as a function of winter forest use. In all models, Julian date was included as a fixed co-variate to account for potential differences due to the timing of breeding, and the forest type (Pruno-Fraxinetum, Fago-Quercetum and Stellario-Carpinetum; categories based on the Biological Valuation Map, \citealt{Vriens2011}), were included to account for potential differences in habitat quality. The original clutch size was included as a fixed co-variate to account for variation in maternal investment. Fledging success (the proportion of hatched young that survived until fledging) was modelled using a Generalized Linear Model with a betabinomial distribution in glmmTMB package v. 0.2.3 \citep{Brooks2017}. Fledgling body condition was modelled using a Linear Mixed Effect Model (LMM). This model also included the number of fledglings as a fixed co-variate, and nest-ID as a random effect to account for non-independence of individuals in the same nest. Maternal blood parasite load (log of the relative parasitaemia) was modeled using a linear model. All analyses were performed in R v. 3.5.1 \citep{RCoreTeam2018}.\\

For each model, post-fitting checks were performed to ensure homogeneity of residual variance, adequate residual distribution (Gaussian models) and absence of outlying observations using the DHARMa package \citep{Hartig2019}. 
	

	\subsection*{Ethics statement}

All protocols were approved by the Ethical Committee VIB Ghent (EC2015-023). Capture and ringing permits were granted by the Belgian Ringing Scheme and the Flemish authorities (Agentschap voor Natuur en Bos; ANB/BL-FF/V15-00034 and ANB/BL-FF/V16-00003).\\

Radio tags weighted 2.86 \% to 3.63 \% (average 3.29 \%) of the individuals' body mass. Such an additional load has earlier been shown not to affect great tit survival, range use or manoeuvrability \citep{Naef-Daenzer2001}. After radio tracking, we could recapture 36 out of the 67 tagged birds during nest box controls, and then removed their tag. As tags were glued to the tail feather, the tags of all other individuals fell off after about two weeks.

\clearpage	
	\section{Results}
	\subsection*{Habitat use and movement behaviour in winter}
	
A total of 4945 location fixes (on average 74 $\pm$ 9.3 (sd) per individual) were obtained during both winters (32 birds were radio-tracked during 2015--16 and 35 birds during 2016--17). Individuals moved on average 787.85 m during each two-hour tracking session. While residential area and forest habitat were equally used during the first year, individuals avoided residential gardens during the second winter with high mast availability (table \ref{tab3-1}, figure \ref{fig3-3}). The magnitude of avoidance of agricultural areas and other land use was also higher during the second year. Intrinsic factors (sex, age or body condition (SMI)) did not significantly explain variation in habitat use, nor did the size of the forest fragment in which each individual resided (table \ref{tab3-1}). At the experimental forest, winter forest use did not significantly vary with age, sex, SMI or feather CORT either (table 3.2a).\\

Distance covered (step length) during foraging was significantly larger outside compared to inside forests (outside: est. 78.92 [95\%--CI: 63.40--94.45]; inside: est. 56.93 [95\%--CI: 41.42--72.44]) and tended to decrease with increasing forest size, albeit marginally non-significant (p $=$ 0.053; table \ref{tab3-2}). Intrinsic factors (age, sex and body condition (SMI)) did not significantly explain variation in distances covered by the tracked individuals. 


	\subsection*{Challenge experiment: foraging outside the forest leads to reproductive costs}
	
Nestling mortality in the experimental forest was very high, likely due to an unusual cold spell in early May 2017, with an average fledging success of 48.23 \% in the experimental group and 43.93 \% in the control group (not significantly different p $=$ 0.183, see supplementary material S3.2). Fledging success significantly decreased with time spent outside the forest during winter (table \ref{tab3-3}b; figure \ref{fig3-4}). Estimated fledging success decreased from 98.7 \%, if a female was consistently recorded feeding inside the forest during winter, to 42.7 \%, if a female was consistently recorded feeding outside (figure \ref{fig3-4}). Nestling mass (measured at an age of 15 days) did not significantly vary with female forest use, nor did the maternal blood parasite load (table \ref{tab3-3}b).

%%Table 3.1	
	\clearpage
	\begin{landscape}
		%\thispagestyle{empty}
		%\begin{sidewaystable}
		\begin{table}
			\begin{center}
				\begin{footnotesize}
					\caption{summary of the test statistics for the GLMM investigating winter habitat use of 67 radio tracked great tits. Intercept: land use forest. All estimates (est.), standard errors (SE) and p-values refer to the fixed effects in the model.}  \label{tab3-1}
					
					\begingroup
					\setlength{\tabcolsep}{8pt} % Default value: 6pt
					\renewcommand{\arraystretch}{1.5} % Default value: 1
					%\hspace*{-3em}
					\begin{tabular}{l r r r r r r}
						
						\toprule
						& \multicolumn{3}{c}{\textbf{Winter 2015--2016}} & \multicolumn{3}{c}{\textbf{Wint 2016--2017}} \\
						& \multicolumn{3}{c}{\textbf{(non-mast year)}} & \multicolumn{3}{c}{\textbf{(mast year)}}\\
						\cmidrule(r){2-4}
						\cmidrule(r){5-7}
						& \textbf{est.} & \textbf{SE} & \textbf{p-value} & \textbf{est.} & \textbf{SE} & \textbf{p-value}\\
						\cmidrule(r){2-4}
						\cmidrule(r){5-7}
						
						 Intercept & 0.000 & 5.122 & 0.999 \color{white}*\color{black} & 0.000	& 3.477 & 0.999 \color{white}*\color{black}\\
						 Land use agricultural & -0.921 & 0.078 & $<$ 0.001 $\ast$ & -1.372 & 0.092 & $<$ 0.001 $\ast$\\
						 Land use residential & 0.125 & 0.069 & 0.071 \color{white}*\color{black} & -0.263 & 0.095 & 0.006 $\ast$\\
						 Land use other & -0.233 & 0.107 & 0.029 $\ast$ & -0.678 & 0.153 & $<$ 0.001 $\ast$\\
						 Forest fragment size & -0.001 & 182.800 & 0.894 \color{white}*\color{black} & 0.000 & 192.500 & 0.981 \color{white}*\color{black}\\
						 Age & 0.000 & 498.500 & 1.000 \color{white}*\color{black} & 0.000 & 444.200 & 1.000 \color{white}*\color{black}\\
						 Sex & 0.000 & 357.700 & 1.000 \color{white}*\color{black} & 0.000 & 401.200 & 1.000 \color{white}*\color{black}\\
						 SMI & 0.000 & 147.100 & 1.000 \color{white}*\color{black} & 0.000 & 258.100 & 1.000 \color{white}*\color{black}\\
						 Log(step length) & 0.038 & 0.015 & 0.015 $\ast$ & 0.071 & 0.020 & $<$ 0.001 $\ast$\\
						 cos(turning angle) & -0.730 & 0.033 & $<$ 0.001 $\ast$ & -0.682 & 0.032 & $<$ 0.001 $\ast$\\
						 agricultural:fragment size& 0.053 & 0.077 & 0.492 \color{white}*\color{black}& -0.246 & 0.138 & 0.075 \color{white}*\color{black}\\
						 residential:fragment size & -0.013 & 0.062 & 0.836 \color{white}*\color{black}& -0.003 & 0.073 & 0.962 \color{white}*\color{black}\\
						 other:fragment size & -0.432 & 0.176 & 0.014 $\ast$ & 0.094 & 0.097 & 0.332 \color{white}*\color{black}\\
						
						\bottomrule
					\end{tabular}\endgroup
				\end{footnotesize}
			\end{center}
			%\end{sidewaystable}
		\end{table}
	\end{landscape}	

\begin{figure}[t]
	\begin{center}
		\includegraphics[width=0.8\textwidth]{fig3-3.png}
	\end{center}
	\caption{Population level selection coefficients (with 95\% confidence interval) for different habitat types compared to forest, during 2015--16 (winter with low mast availability) and 2016--17 (winter with high mast availability).}\label{fig3-3}
\end{figure}


%%Table 3.2
%\begin{landscape}
	%\thispagestyle{empty}
	%\begin{sidewaystable}
	\begin{table}[b]
		\begin{center}
			\begin{footnotesize}
				\caption{summary of the test statistics for the LMM investigating distances covered by 67 radio tracked great tits. Intercept: use of forest habitat. All estimates (Est.), standard errors (SE) and p-values refer to the fixed effects in the model.}  \label{tab3-2}
				
				\begingroup
				\setlength{\tabcolsep}{8pt} % Default value: 6pt
				\renewcommand{\arraystretch}{1.5} % Default value: 1
				%\hspace*{-3em}
				\begin{tabular}{l r r r}
					
					\toprule
					& \textbf{Est.} & \textbf{SE} & \textbf{p-value} \\ 
					\cmidrule(r){2-4}
					\textbf{Intercept} & 56.930 & 7.913 & $<$ 0.001 $\ast$\\
					\textbf{Non-forest habitat} & 21.990 & 0.716 & $<$ 0.001 $\ast$\\
					\textbf{Year} & -3.982 & 5.920 & 0.504 \color{white}*\color{black}\\
					\textbf{Non-forest: Winter} & 10.790 & 1.011 & $<$ 0.001 $\ast$\\
					\textbf{Fragment size} & -5.922 & 3.012 & 0.054 \color{white}*\color{black}\\
					\textbf{Age} & 0.557 & 7.507 & 0.941 \color{white}*\color{black}\\
					\textbf{Sex} & -1.223 & 5.929 & 0.837 \color{white}*\color{black}\\	
					\textbf{SMI} & -0.002 & 2.967 & 0.999 \color{white}*\color{black}\\	
					
					
					\bottomrule
				\end{tabular}\endgroup
			\end{footnotesize}
		\end{center}
		%\end{sidewaystable}
	\end{table}
%\end{landscape}
	

%%Table 3.3
\clearpage
\begin{landscape}
\thispagestyle{empty}
%\begin{sidewaystable}
\begin{table}
	\begin{center}
		\begin{footnotesize}
			\caption{summary of the test statistics for the GLMM investigating a) winter forest use of 163 PIT-tagged great tits at the experimental site, and b) reproductive and physiological parameters of 32 nests included in the challenge experiment. All estimates (Est.), standard errors (SE) and p-values refer to the fixed effects in the model.}  \label{tab3-3}
			
			\begingroup
			\setlength{\tabcolsep}{8pt} % Default value: 6pt
			\renewcommand{\arraystretch}{1.5} % Default value: 1
			%\hspace*{-3em}
			\begin{tabular}{l r r r r r r r r r}
				
				\toprule
				\textbf{a)} & \multicolumn{3}{c}{\textbf{Wint forest use}} &&&&&&\\
				\cmidrule{2-4}
				& \textbf{Est.} & \textbf{SE} & \textbf{p-value} &&&&&&\\ 
				\cmidrule(r){1-4}
				Intercept & 12.472 & 10.300 & 0.226 &&&&&&\\
				SMI & -0.678 & 0.589 & 0.250 &&&&&&\\
				Sex & 0.138 & 0.970 & 0.887&&&&&&\\
				Age & 1.206 & 0.996 & 0.226 &&&&&&\\
				CORTf & 137.299 & 248.504 & 0.581 &&&&&&\\
				\cmidrule{1-4}
				\textbf{b)} & \multicolumn{3}{c}{\textbf{Fledging success}} & \multicolumn{3}{c}{\textbf{Fledging mass}} & \multicolumn{3}{c}{\textbf{Blood parasite load}}\\
				\cmidrule(r){2-4} \cmidrule(r){5-7} \cmidrule(r){8-10}
				& \textbf{Est.} & \textbf{SE} & \textbf{p-value} & \textbf{Est.} & \textbf{SE} & \textbf{p-value} & \textbf{Est.} & \textbf{SE} & \textbf{p-value}\\
				\hline
				Intercept & 14.159 & 9.790 & 0.148 \color{white}*\color{black}& 14.309 & 7.823 & 0.084 \color{white}*\color{black} & 1.876 & 12.442 & 0.887 \color{white}*\color{black}\\
				Winter Forest use & -6.106 & 1.830 & $<$ 0.001 $\ast$ & -0.344 & 0.905 & 0.708 \color{white}*\color{black} & -5.094 & 3.881 & 0.260 \color{white}*\color{black}\\
				Original clutch size & -0.133 & 0.099 & 0.182 \color{white}*\color{black} & -0.169 & 0.230 & 0.472 \color{white}*\color{black} & -0.021 & 0.168 & 0.906 \color{white}*\color{black}\\
				Laying date &  -0.042 & 0.124 & 0.732 \color{white}*\color{black} & -0.002 & 0.079 & 0.984 \color{white}*\color{black} & -0.006 & 0.123 & 0.961 \color{white}*\color{black}\\
				Forest type (Fago-Quercetum) & 4.334 & 1.363 & 0.001 $\ast$ & 1.479 & 0.969 & 0.146 \color{white}*\color{black} & 3.881 & 3.122 & 0.282 \color{white}*\color{black}\\
				Forest type (Pruno-Fraxinetum) & 2.534 & 0.932 & 0.007 $\ast$ & 0.406 & 0.727 & 0.585 \color{white}*\color{black} & 0.900 & 1.054 & 0.441 \color{white}*\color{black}\\
				Pulli number &&&& 0.266 & 0.228 & 0.259 \color{white}*\color{black}
				 &&&\\
				\bottomrule
				
			\end{tabular}\endgroup
		\end{footnotesize}
	\end{center}
	%\end{sidewaystable}
\end{table}
\end{landscape}

\begin{figure}[h]
	\begin{center}
		\includegraphics[width=\textwidth]{fig3-4_v2.png}
	\end{center}
	\caption{Relationship between winter forest use (proportion of the feeder records at feeding stations outside the forest) and fledging success (proportion of hatched pulli that successfully fledged) of female great tits. Points show raw data per female and the lines show model prediction with 95\%-confidence intervals (grey) assuming the average values of other explanatory variables.}\label{fig3-4}
\end{figure}

\clearpage	
	
	\section{Discussion}
	
When food availability was high during winter, individuals spent less time foraging in the landscape matrix, which is in line with our first hypothesis. Against our expectations, the proportion of foraging time in the landscape matrix was independent of the size of the forest in which they resided, nor did it vary with its age, sex, body condition or feather CORT. However, as predicted, distances travelled while foraging in the landscape matrix exceeded those in forests. In line with the second hypothesis, females that spent more time in the landscape matrix during winter were less efficient in coping with increased offspring demand during spring, as reflected in an inverse relationship with fledging success (i.e. the number of pulli that survived until fledging). In contrast, both fledgling condition (a proxy for post-fledging survival; \citealt{Perrins1965, Monros2002}) and maternal blood parasite load (a proxy for post-breeding survival; \citealt{Puente2010}) were unrelated to winter movement behaviour.\\

Winter movement and habitat use were hence mainly explained by food availability, providing support for the widely-accepted hypothesis that temporal and spatial heterogeneity in resource distribution shapes mobility and space use in animals (reviewed in \citealt{Boutin1990}). The propensity to forage outside forest boundaries during winter did not vary with forest fragment size in our radio tracking study, but we did observe that individuals in smaller forests tended to cover larger foraging distances, due to the fact that such birds regularly crossed large forest gaps when moving between neighboring forest fragments. At the experimental site, we observed that individual variation in forest use was spatially structured. Individuals that used nest boxes in the north-western part of the forest (and likely also occupied this territory during winter), spend more time at the feeders situated in the forest in winter, compared to individuals nesting at sites closer to residential areas (see supplementary material S3.4). The fact that individuals ventured less into the landscape matrix when food conditions in the forest were favorable, further provides indirect evidence that foraging outside forests is costly. In support of this, we showed that that distances travelled were significantly larger when foraging outside forest boundaries, which has earlier been shown to correlate with energy expenditure in birds \citep{Hinsley2000, Hinsley2008}. In addition to increased energy expenditure, small songbirds are highly vulnerable to predation when flying in the open \citep{Lima1990, Desrochers1997}. Although difficult to quantify, common avian predators in our study area such as the Eurasian sparrowhawk (\textit{Accipiter nisus}), may pose a significant threat to small passerines crossing open landscapes. Yet, as levels of stress hormones (measured in feathers grown during winter) and body condition (measured during spring) did not significantly vary with winter habitat use (supplementary material S3.3), it is unlikely that increased energy expenditure or predation-related physiological stress explained the observed reproductive cost in our challenge experiment.\\

Alternatively, movement in the landscape matrix may have indirect effect through induced nutritional costs. While food supplemented at feeding stations inside and outside forests did not differ in nutritional value and the number of feeder visits was not correlated to winter habitat use (supplementary material S3.3), it is likely that individuals that foraged more in the landscape matrix fed more on supplemented food. In support of this assumption, tracking data of eight PIT-tagged birds showed that individuals that were mainly recorded on forest feeders spent much time foraging on the forest floor or in dense understory, while those that were mainly recorded on garden feeders often visited feeders in neighboring residential areas as well (supplementary material S3.3). Note that, apart from our feeders, no supplementary food was available in the forest. While fatty foods provide ample energy during winter, they generally lack vitamins and carotenoids that accumulate in subcutaneous fat and liver tissues and can be mobilised when demand is high during breeding \citep{Plummer2013}. In support of this, blue tits (\textit{Cyanistes caeruleus}) that had better access to supplemented fatty food during winter produced fewer fledglings compared to not-supplemented populations (\citealt{Plummer2013a}; but see \citealt{Robb2008a, Crates2016}). Seeds and nuts are the main food source during winter, but arthropods, which hold many of these micronutrients, still form a substantial part of the winter diet of great tits under natural conditions \citep{Velky2011}. Such arthropod food sources are likely more available to individuals foraging in forests compared to those foraging in residential gardens.\\

As opposed to extrinsic factors, such as food quantity and quality, winter habitat use and movement behaviour did not vary with any of the intrinsic traits measured in this study (body condition, sex, age and feather CORT). Yet, this does not exclude the possibility that both winter mobility and breeding performance could be influenced by consistent between-individual variation in exploratory behaviour (`personality' \textit{sensu} \citealt{Dingemanse2004, Spiegel2017, Both2005}). In support of this, `slower exploring' individuals have been found to benefit in unpredictable environments where behavioural flexibility is favored. Such slow explorers typically respond more quickly to changes in food distribution, preferring to search novel food sources within their home-range \citep{Verbeek1994, VanOverveld2010, Arvidsson2016}, possibly providing a fitness advantage over `fast exploring' individuals that rely on exploring new areas in search of known food sources. While we did not quantify personalities in our study, these findings suggest the hypothesis that slow exploring individuals may also cope better with raising offspring under challenging conditions \citep{Both2005}, hence providing an alternative mechanism underlying the observed reproductive cost of foraging in the landscape matrix during winter.\\

Apart from providing evidence for cross-seasonal carry-over effects of movement behaviour in heterogeneous landscapes, our results also contribute to the long-lasting debate on the extent to which wild animals may (fail to) adapt to large-scale anthropogenic changes to their environment, such as urban sprawl. Great tits are one of the most common insectivorous birds in European modified landscapes, and have often been regarded as generalists that cope well with landscape transformation and urbanisation (e.g. \citealt{Kark2007}). Yet, recent studies have shown that even low levels of landscape transformation may already result in reduced avian breeding performance (e.g. \citealt{Hinsley2009, Seress2018, DeSatge2019}). Our study reveals that the latter may not only result from inadequate food supplies during breeding \citep{Derryberry2020, Seress2020} but also from winter habitat use. We therefore plead for incorporating all stages of the annual cycle when informing management and conservation actions, also for non-migratory birds. 

	
	\subsection*{Funding}

Financial support for this research was provided via the UGent GOA project ``Scaling up Functional Biodiversity Research: from Individuals to Landscapes and Back (TREEWEB)''.
	
	\subsection*{Acknowledgements}
	
We thank the private forest owners, the Province of Oost-Vlaanderen and the Flemish Forest and Nature Agency (ANB) for granting access on to their property. Furthermore, we thank Robbe de Beelde, Kathryn Godfrey, Bruno Trappeniers, Pieter Vantieghem and Siebe Verholle for invaluable help with the fieldwork, Liesbeth de Neve for advise on the study design and Ren\'{e} Janssen for technical advice and support. Finally, we thank the family Haesaert, family Verheyen and the family Van Hoeylandt for allowing us to install feeding stations in their gardens.
	
	\subsection*{Data Accessibility}
	
All movement data is available on Movebank (id: 731068668).
	
	\subsection*{Author contributions}
	 Luc Lens, Diederik Strubbe and Daan Dekeukeleire conceived the study; Luc Lens, Dries Bonte, Kris Verheyen and An Martel acquired funding; Daan Dekeukeleire collected and analysed the data; Johan Aerts performed the feather corticosterone analyses; Camille Sophie Cozzarolo and Philippe Christe performed the blood parasite analyses; Daan Dekeukeleire led the writing of the manuscript with significant contributions of Diederik Strubbe, Femke Batsleer and Luc Lens.\\
	 \clearpage
	
	\section{Supplementary material}
	
	\subsection*{Supplementary material S3.1: radio tracking of PIT-tagged individuals at the experimental site}
	
\begin{figure}[h!]
	\begin{center}
		\includegraphics[width=0.75\textwidth]{FigS3-1.png}
	\end{center}
\caption*{\textbf{Figure S3.1}: Radio tracking data for eight PIT-tagged individuals that were followed for 12h. Dots represent location fixes, lines connect consecutive location fixes. Triangles represent feeding stations with RFID-antenna, red triangles represent feeding stations were the PIT tag was recorded.}
\end{figure}
\clearpage

	\subsection*{Supplementary material S3.2: more information on the challenge experiment}

\begin{figure}[h]
	\begin{center}
		\includegraphics[width=0.85\textwidth]{FigS3-2-1}
	\end{center}
	\caption*{\textbf{Figure S3.2.1}: Map of nest boxes with in the experimental forest size. Nest boxes are coloured according to the treatment: control treatment (n: 10) and brood enlargement treatment (n: 32).}
\end{figure}

\begin{figure}[b]
	\begin{center}
		\includegraphics[width=0.7\textwidth]{figS3-2-2.png}
	\end{center}
\caption*{\textbf{Figure S3.2.2}: Laying date of the nests in the control treatment and brood enlargement treatment. The laying date did not significantly differ (linear model; p $=$ 0.885).}
\end{figure}


%%%Table S3.2.1	
\begin{table}[h]
	\begin{center}
		\begin{footnotesize}
			\caption*{\textbf{Table S3.2.1}: Paternal age of the males and the females in the nests control treatment (n: 10) and brood enlargement treatment (n: 32). Age did not significantly differ between treatments ($\chi^2$ test; males: $\chi^2$ $=$ 0.002, p $=$ 0.999; females: $\chi^2$ $=$ 1.756, p $=$ 0.189)} 
			
			\begingroup
			\setlength{\tabcolsep}{8pt} % Default value: 6pt
			\renewcommand{\arraystretch}{1.5} % Default value: 1
			%\hspace*{-3em}
			\begin{tabular}{l r r r r}
				
				\toprule
				& \multicolumn{2}{c}{\textbf{Control}} & \multicolumn{2}{c}{\textbf{Brood size}}\\
				&&& \multicolumn{2}{c}{\textbf{enlargement}}\\
				\cmidrule(r){2-3} \cmidrule(r){4-5}
				& Female & Male & Female & Male\\
				\hline
				\textbf{Adult} & 1 & 5 & 13 &18\\
				\textbf{First years} & 8 & 2 & 18 &6 \\
				\bottomrule
			\end{tabular}\endgroup
		\end{footnotesize}
	\end{center}
	%\end{sidewaystable}
\end{table}


\begin{figure}[h]
	\begin{center}
		\includegraphics[width=0.7\textwidth]{figS3-2-3.png}
	\end{center}
	\caption*{\textbf{Figure S3.2.3}: Wing length of the parents in the control treatment and brood enlargement treatment. The wing length did not significantly differ (linear model; p $=$ 0.615).}
\end{figure}

\begin{figure}[h]
	\begin{center}
		\includegraphics[width=0.7\textwidth]{figS3-2-4.png}
	\end{center}
	\caption*{\textbf{Figure S3.2.4}: Body condition (SMI) of the parents in the control treatment and brood enlargement treatment. The SMI did not significantly differ (linear model; p $=$ 0.615).}
\end{figure}

\begin{figure}[h]
	\begin{center}
		\includegraphics[width=0.7\textwidth]{figS3-2-5.png}
	\end{center}
	\caption*{\textbf{Figure S3.2.5}: Clutch size for the control treatment and brood enlargement treatment in the challenge experiment. On average, nests in the control treatments had eight eggs, while the nests in the enlarged broods had eleven eggs.}
\end{figure}

\begin{figure}[h]
	\begin{center}
		\includegraphics[width=0.7\textwidth]{figS3-2-6.png}
	\end{center}
	\caption*{\textbf{Figure S3.2.6}: Hatching success (proportion of the eggs that hatched) for the control treatment and brood enlargement treatment in the challenge experiment. Hatching success did not significantly vary between the treatments (generalized linear model with binomial distribution; p $=$ 0.113).}
\end{figure}

\begin{figure}[h]
	\begin{center}
		\includegraphics[width=0.7\textwidth]{figS3-2-7.png}
	\end{center}
	\caption*{\textbf{Figure S3.2.7}: Fledging success (proportion of the hatched eggs that fledged) for the control treatment and brood enlargement treatment in the challenge experiment. Fledging success did not significantly vary between the treatments (generalized linear model with binomial distribution; p $=$ 0.183).}
\end{figure}


\clearpage
\subsection*{Supplementary material S3.3: winter forest use, feeder visits and feather CORT in the induced feather}

\begin{figure}[h]
	\begin{center}
		\includegraphics[width=\textwidth]{figS3-3-1.png}
	\end{center}
	\caption*{\textbf{Figure S3.3.1}: The number of records at feeding stations was not significantly related to winter forest use.}
\end{figure}


%%%Table S3.3.1	
\begin{table}[h!]
	\begin{center}
		\begin{footnotesize}
			\caption*{\textbf{Table S3.3.1}: results from a linear model with the average number of daily records at feeding stations as response variable and winter forest use and age as explanatory variables.}
			
			\begingroup
			\setlength{\tabcolsep}{8pt} % Default value: 6pt
			\renewcommand{\arraystretch}{1.5} % Default 
			\begin{tabular}{p{4cm} r r}
				
				\toprule
				& \textbf{Estimate} & \textbf{p-value}\\
				\cmidrule{2-3}
				\textbf{Intercept} & 24.143 & 0.006\\
				\textbf{Winter forest use} & 3.921 & 0.506\\
				\textbf{Age} & -2.696 & 0.503\\
				\bottomrule
			\end{tabular}\endgroup
		\end{footnotesize}
	\end{center}
\end{table}


%Figure S3.3.2
\begin{figure}[t]
	\begin{center}
		\includegraphics[width=\textwidth]{FigS3-3-2.png}
	\end{center}
	\caption*{\textbf{Figure S3.3.2}: The body condition in spring (SMI) was not significantly related to winter forest use.}
\end{figure}


%%%Table S3.3.2
\begin{table}[h!]
	\begin{center}
		\begin{footnotesize}
			\caption*{\textbf{Table S3.3.2}: results from a linear model with SMI in spring as response variable and winter forest use, age and number of pulli as explanatory variables.}
			
			\begingroup
			\setlength{\tabcolsep}{8pt} % Default value: 6pt
			\renewcommand{\arraystretch}{1.5} % Default 
			\begin{tabular}{p{4cm} r r}
				
				\toprule
				& \textbf{Estimate} & \textbf{p-value}\\
				\cmidrule{2-3}
				\textbf{Intercept} & 18.022 & $<$ 0.001\\
				\textbf{Winter forest use} & 30.144	&0.774\\
				\textbf{Age} & 0.731 & 0.052\\
				\textbf{Number of pulli} & -0.067 & 0.506\\
				\bottomrule
			\end{tabular}\endgroup
		\end{footnotesize}
	\end{center}
\end{table}


\begin{figure}[t]
	\begin{center}
		\includegraphics[width=\textwidth]{FigS3-3-3.png}
	\end{center}
	\caption*{\textbf{Figure S3.3.3}: The CORT values in the feathers grown during winter (induced feather) were not significantly related to winter forest use.}
\end{figure}

%%%Table S3.3.3
\begin{table}[h!]
	\begin{center}
		\begin{footnotesize}
			\caption*{\textbf{Table S3.3.3}: results from a linear model with CORTf in the induced feather as response variable and winter forest use, age and feather growth bar length as explanatory variables.}
			
			\begingroup
			\setlength{\tabcolsep}{8pt} % Default value: 6pt
			\renewcommand{\arraystretch}{1.5} % Default 
			\begin{tabular}{p{4cm} r r}
				
				\toprule
				& \textbf{Estimate} & \textbf{p-value}\\
				\cmidrule{2-3}
				\textbf{Intercept} & -17.617 & 0.144\\
				\textbf{Winter forest use} & 0.686 & 0.740\\
				\textbf{Feather growth bar length} & 1.399 & 0.207\\
				\textbf{Age} & -0.870 & 0.580\\
				\bottomrule
			\end{tabular}\endgroup
		\end{footnotesize}
	\end{center}
\end{table}

\begin{figure}[th!]
	\begin{center}
		\includegraphics[width=\textwidth]{FigS3-3-4.png}
	\end{center}
	\caption*{\textbf{Figure S3.3.4}: Map of the fledging success of nests in the control and brood enlargement treatment at the experimental site. Note that fledging success was not spatially auto correlated (Moran's I: 0.051, expected: -0.042; p-value $=$ 0.073).}
\end{figure}

%
\clearpage
\subsection*{Supplementary material S3.4: Spatial autocorrelation in the models investigating fledging succes}

\begin{figure}[h!]
	\begin{center}
		\includegraphics[width=\textwidth]{figS3-4-1.png}
	\end{center}
	\caption*{\textbf{Figure S3.4.1}: spatial plot of the residuals of (A) a model investigating fledging success without including forest type, and (B) a model investigating fledging success with forest type (which is itself spatially structured) as covariate. Dots represent nest box locations, size represents the residual size and colour represents positive (grey) or negative (white). While the residuals of the first model are clearly spatially auto correlated (all residuals in the north-western part are negative), including forest type partially accounts for this.}
\end{figure}

\begin{figure}[h!]
	\begin{center}
		\includegraphics[width=\textwidth]{figS3-4-2.png}
	\end{center}
	\caption*{\textbf{Figure S3.4.2}: relationship between time spend outside the forest in winter and distance between the nest box used during the breeding season and the closest residential area.}
\end{figure}



\cleardoublepage
\thispagestyle{plain}
\hbox{}
\clearpage
	%%%%%%%%%%%%%%%%%%%%%%%%%%%%%%%%%%%%% Chapter 4 - Top-down control %%%%%%%%%%%%%%%%%%%%%%%%%%%%%%%%%%%%%%%%%	
	\CenterWallPaper{1}{CH4.jpg}
	\newpage{\thispagestyle{empty}\cleardoublepage}
	\ClearWallPaper
	\pagestyle{mainmatter}
	\chapter{Cross-seasonal relationships between social network position and stress in a free-ranging songbird} \label{chapter4}
	\chaptermark{Cross-seasonal relationships between social network position and stress}
	\lettergroup{\thechapter}

		\begin{flushright} \color{gray}Daan Dekeukeleire*\\ Lionel Hertzog*\\ Johan Aerts\\ Martijn L. Vandegehuchte\\ Pieter Vantieghem\\ An Martel\\
			 Kris Verheyen\\ Dries Bonte\\ Diederik Strubbe\\ Luc Lens\\
			 \vspace{1cm} * equal contribution
		
	\end{flushright}
		
		\vspace*{\fill}
\noindent \color{gray} $\lhd$ Great tit with a PIT-tag on its leg visiting one of the experimental feeders. Photo by Gert Arijs.
		

	\color{black}
	\newpage
	
	
	\section{Abstract}
	
Social interactions have important fitness consequences for individuals, for example through transfer of pathogens or information about food resources. Social network position may therefore affect the level of stress experienced by an individual, which in turn can trigger behavioural changes in order to decrease perceived levels of stress, leading to new network positions. How important such feedback loops are under natural conditions, however, remains unclear. In this study, we investigated reciprocal relationships between social network position and stress hormone levels across seasons in a great tit (\textit{Parus major}) population. Using Passive-Integrated-Transponder tags, we inferred the winter social network structure of great and blue tits in a forest fragment in Belgium, based on visits of tagged birds to artificial feeders. Of a subset of these individuals, we quantified feather corticosterone as an archive of stress experienced during feather growth. We sampled original feathers grown before flocking during post-breeding moult in late summer and induced feathers regrown from the same follicle during the subsequent winter. Individuals with higher corticosterone levels in original feathers had more central positions in the winter social network. In turn, birds that associated with more individuals in the winter network tended to have lower corticosterone levels in induced feathers, possibly reflecting better access to social information on winter food resources. Our study hence suggests a feedback process, in which birds characterized by high corticosterone post-breeding feathers adjust their social behaviour, which influences their network position over the next winter. This could in turn potentially lead to reduced stress levels in winter, with potential carry-over effects to the subsequent breeding season.
\clearpage
	
	\section{Introduction}
	
The social environment can fundamentally shape the life history of individuals \citep{Krause2002}. Within a social group, each individual can be seen as part of a network of social interactions varying in frequency, duration and type, from dominance to neutral associations. Social interactions can determine the access an individual has to information on food resources or roost sites \citep{Fortuna2009, Aplin2012, Webster2013}, as well as how well it can exploit such resources \citep{Firth2016}, and can impacts their ability to successfully reproduce \citep{Oh2010}. Such fitness consequences not only depend on pairwise interactions between individuals, but also on the wider network in which each individual is embedded \citep{Brent2015}. The social structure that arises from these interactions also shapes population processes, such as spread of information or the spread of diseases and parasites \citep{Bull2012, MacIntosh2012}. It is well-known that individuals can show a large degree of spatial and temporal variation in their social interactions, but the ecological and evolutionary processes that underlie this variation are less well understood \citep{Sih2009, Farine2015, Croft2016}. Therefore, unravelling mechanisms and identifying ecological factors underlying the social structure of populations are of paramount importance \citep{Wey2008, Farine2015}.\\

Stress-related hormones, in particular glucocorticoids, have been proposed as underlying mediators of associations within social networks \citep{Boogert2014, Solomon-Lane2015, Brandl2019}. Environmental stressors, such as food shortage or adverse weather conditions, often result in energetically demanding conditions, which, in vertebrates, can trigger the release of glucocorticoids mediated by the hypothalamic-pituitary-adrenal (HPA) axis \citep{Wingfield2013}. In birds, the dominant glucocorticoid is corticosterone (CORT), which mediates physiological processes that steer behavioural and metabolic changes and, in turn, enables individuals to better cope with environmental challenges. For example, unpredictable food supplies during winter were earlier shown to trigger long-term, moderate elevations of plasma CORT levels in mountain chickadees (\textit{Poecile gambeli}), which in turn increased food-caching behaviour and improved spatial memory \citep{Pravosudov2001, Pravosudov2003a}. Furthermore, experimental exposure of birds to elevated CORT levels was shown to affect behavioural traits that shape social interactions, such as dominance, neophobia \citep{Spencer2007} and social information use \citep{Boogert2013}. Stress-induced glucocorticoids might hence influence an individual's attraction to conspecifics or an individual's maintenance of social relationships, and this may ultimately lead to changes in network position. Along these lines, captive zebra finches (\textit{Taeniopygia guttata}) exposed to experimentally increased CORT levels as nestling were less `choosy' with whom to forage after fledging, resulting in a more central network position compared to siblings in a control group \citep{Boogert2014}. Causal relationships between glucocorticoid-mediated actions and network position may also exist in the opposite direction. The social buffering hypothesis states that an individual's position within a social network may determine the degree of stress it experiences, e.g. through differential access to food resources \citep{Webster2013}, pathogen or parasite load \citep{Bull2012, MacIntosh2012}, or intensity of competition \citep{Oh2010, Fisher2016}. In support of this hypothesis, wild Barbary macaques (\textit{Macaca sylvanus}) with stronger social associations were shown to exhibit lower glucocorticoid increases in response to social aggression and cold stress during winter \citep{Young2014}. While not all factors that influence network position are yet fully understood, recent studies suggest that individuals can adjust their social behaviour, influencing their network position, to acquire social information \citep{Kulahci2018, Kulahci2019}. It is hence possible that individuals respond to environmentally-induced stress by adjusting their social behaviour --which influences their network position-- in order to be better equipped to deal with biotic or abiotic stressors \citep{Croft2016}. However, to our knowledge, such pathway has never been empirically tested.\\

Low availability of food is an important environmental stressor for wild animals (e.g. \citealt{Kitaysky1999}), and in this respect, resident tits and chickadees (Paridae) are highly suitable taxa to study relationships between stress and social interactions. In temperate and boreal forests, food availability during winter is generally the main factor affecting their survival \citep{Jansson1981, Brittingham1988, Perdeck2000}. While tits are territorial during the breeding season, they form loose mixed-species flocks in autumn and winter and roam forests to forage on ephemeral and patchy food sources such as seeds and nuts \citep{Ekman1989, Farine2012}. Such flocking behaviour has been linked to a higher foraging efficiency and reduced risk of predation (reviewed in \citealt{Sridhar2009}). For example, great tits foraging in flocks located hidden food faster by copying the foraging location of flock members \citep{Krebs1972}, while blue tits (\textit{Cyanistes caeruleus}) decreased their predator vigilance in flocks with great tits \citep{Telleria2001}. Mixed Paridae flocks typically display high fission-fusion dynamics, recurrently splitting into subgroups that regularly re-merge \citep{Farine2015b}. As a consequence, individuals may show strong temporal variation in the number of other individuals and subgroups they associate with. Social information use is crucial to this decision-making and may have other important fitness consequences such as in terms of foraging efficiency \citep{Sueur2011}. Individuals occupying a more central position in the network may have better access to social information, and hence, may be better informed and able to access novel food resources faster \citep{Aplin2012, Farine2015a, Firth2016}.\\

Based on the growing correlative and experimental evidence for the central role of social information in optimizing feeding efficiency, we hypothesize a feedback mechanisms, in which the corticosterone levels individuals experience would affect their social behaviour, which in turn would influence their access to food source through social information, and thus their subsequent stress levels, with possible carry-over effects to the breeding season. To investigate his, we inferred a social network of wild great and blue tits from visits of birds tagged with Passive Integrated Transponders (PIT) to artificial feeders during winter using Gaussian Mixture Models \citep{Psorakis2012}, and quantified feather corticosterone (CORTf) in feathers grown post breeding (hereafter original feathers) as well as in their homologue feathers grown during winter (hereafter induced feathers), on a subset of these individuals. CORTf levels provide a retrospective view on HPA-axis (re)activity, and as such, on stress experienced during feather growth \citep{Romero2016}. We predict that (i) individuals characterized by high levels of CORTf in their original feathers grown during the post-breeding period occupy more central social network positions in winter; (ii) individuals that occupy more central network positions in winter have better access to food, as inferred from a higher feeding frequency at the feeders; and (iii) individuals that have better access to food will exhibit lower CORTf in their induced feathers regrown during winter. A graphical visualization of our hypothetical framework is shown in figure \ref{fig4-1}.\\

\begin{figure}[h!]
	\begin{center}
		\includegraphics[width=\textwidth]{fig4-1.png}
	\end{center}
	\caption{Conceptual figure showing hypothesized links between stress level during post-breeding moult, winter social network position, and stress level in winter.}\label{fig4-1}
\end{figure}

\clearpage
	\section{Material \& Methods}\label{mm4}
	
		\subsection*{Study system and fieldwork}
	
Our study was carried out in the Aelmoeseneie forest, a 39.5 ha mixed deciduous forest surrounded by residential areas and agricultural fields in Gontrode (Melle), Belgium (50.975$^{\circ}$N, 3.799$^{\circ}$E) during the winter of 2016 and 2017. Since 2015, a total of 466 great tits and 410 blue tits were captured, measured (tarsus length, wing length, and body mass), aged (first year or adult) and sexed (based on plumage characteristics; \citealt{Svensson1992}) at this location. Upon capture, each individual was fitted with a metal ring of the Belgian ringing scheme with a unique code on one leg and a plastic ring containing a unique PIT-tag (IB Technology, Aylesburyon, UK) on the other leg. Capture-mark-recapture analysis indicated that circa 75 \% of the resident winter population of great tits and circa 61 \% of the resident winter population of blue tits were tagged at the time of our study (Supplementary material S4.1). \citet{Silk2015} indicate that marking circa 30 \% of a population of this size already yields accurate measures of an individual's relative connectedness within a social network, suggesting that our capture effort was more than sufficient to derive robust individual connectivity measures.\\

During the first week of December 2016, six sunflower seed feeding stations were installed and equipped with a RFID-antenna (ANTC100, 10 cm diameter) connected to a stationary decoder (LID650, Dorset ID, The Netherlands). Four of these feeding stations were installed in private gardens near the forest where bird feeding takes places every winter, while two were located in the forest. All stations were located within close proximity of each other (range 239 to 975 m; see figure \ref{fig3-2}), well within the average winter foraging range of tits (\citealt{Aplin2013}; chapter \ref{chapter3}). After a 15-day habituation period, the time and location of each visit of all tagged individuals of both species were logged from dawn until dusk during 83 consecutive days.\\

171 great tits were captured through extensive mist netting and nightly controls of 85 nest boxes on 20 days during November-December 2016 (before the feeders were installed). Of each individual, the left and right second-to-outer tail feathers were collected upon first capture. In our population, these original feathers are grown during post-breeding and post-juvenile moult \citep{Dhondt1973}. Note that in our population all first-year birds moult all their tail feathers, and moult typically finishes between the end of August and the first week of October \citep{Dhondt1973}. By plucking these original feathers, such homologue feather growth was induced from the same follicle during winter in approximately six weeks \citep{Talloen2008}. Note that CORTf levels in these induced feathers were not related to the time the original feather was collected (p-value $=$ 0.55, see online code). Upon recapture through mist netting or nightly nest box checks (March 2017) or when parents were provisioning their young (April-May 2017), left and right induced tail feathers from 28 of the original 171 individuals (7 adult males, 8 adult females, 6 first-winter males, 7 first-winter females) were collected.\\



	\subsection*{Social Network}
	
	As visits of flocks to a feeder can vary in duration depending on the size of the flock and the time of the day \citep{Farine2015}, we used a Gaussian Mixture Model developed by \citet{Psorakis2012} to detect temporal clusters of bird visits (`gathering events'). By applying a method that avoids selecting groups based on arbitrary time windows, recordings of tagged great and blue tits were assigned to a group according to the temporal cluster to which they belonged. Subsequently, we constructed the social network for the entire winter using the R package asnipe v. 1.1.11 \citep{Farine2013}. As social information on food resources in mixed species flocks of tits can be transferred between both conspecifics and heterospecifics \citep{Farine2015, Firth2016}, both great and blue tits were included in the network. Days on which one or more feeders did not log visits due to technical failure (e.g. low battery) were excluded, resulting in 59 days of recording. Associations between individuals, or `edges', were defined using the simple ratio index (i.e. the probability of observing both individuals together given that one has been observed, \citealt{Whitehead1995}). From the estimated social network, we calculated the following four network measures that were previously shown to predict important ecological properties such as disease and parasite transmission, transfer of information about food resources, and stress physiology \citep{Godfrey2009, Aplin2012, Bull2012, MacIntosh2012, Boogert2014}: (i) degree: the number of unique individuals an individual associates with; (ii) eigenvector centrality; a measure of an individual's connectedness to other well-connected individuals; (iii) betweenness centrality; a measure for an individual's role in connecting otherwise distinct groups of individuals, calculated as the number of shortest paths that go through a focal individual; and (iv) strength: the weighted degree, i.e. the overall sum of all the associations of an individual in a network. All these measures were derived using igraph v1.1.11 \citep{Csardi2006}. We quantified to what extent individuals were consistent in their network positions during winter by making networks for every day and deriving all four network centrality measures for each individual on a daily basis and assessing the level of consistency using the package rptR \citep{Stoffel2017}. 
	
	\subsection*{Feather measurements and CORTf quantification}
	
CORTf was quantified for the two original and two induced feathers of the 28 individuals that were captured both in autumn and spring. Prior to feather measurements, all dirt was meticulously removed from the collected feathers using tweezers. Subsequently, every feather was measured to the nearest 0.01 mm and weighted to the nearest 0.001 g. Growth bar width was measured to obtain a measure of feather growth rate. Growth bars are alternating dark and light bands on feathers, which denote a 24 h period of growth \citep{Brodin1993, Grubb2006}. We pinned each feather on a white card and marked a point at a distance of 7/10 from its proximal end. Subsequently, the proximate and distal ends of five consecutive growth bars between this point and the distal end of the feather was marked by the same observer with an ultrafine mounting pin. Each marked card was then scanned (Oc\'{e} OP1130, the Netherlands) and growth-bar widths were automatically measured in ImageJ \citep{Schneider2012}. Growth bar widths could be measured in all investigated feathers. Fault bars, i.e. narrow translucent bands in feathers perpendicular to the rachis due to malformations, which are very common in rectrices grown during the nesting stage in great tit, but are very rare in rectrices grown during the post-juvenile moult \citep{Pap2007}, were never present.\\

After performing these measurements, the calamus of each feather was removed and the feather was homogenized for CORTf quantification using a validated ultra-performance liquid chromatography coupled to tandem mass spectrophotometry (UPLC-MS/MS) method \citep{SallehHudin2018}. Briefly, after feather homogenization, the targeted glucocorticoids, corticosterone and 11-deoxycorticosterone, were extracted with methanol and were ultra-purified using solid phase extraction. Chromatographic analysis was performed on an Acquity Xevo-TQS mass spectrometer (Waters, Belgium) used in the multiple reaction monitoring mode to achieve optimal sensitivity and selectivity. For both targeted compounds as well as the internal standard corticosterone-d8, two precursor fragment ion transitions were determined, enabling the determination of the ratio between both transitions, which was used together with the relative retention time for the accurate identification and quantification of each compound according to the requirements of the Commission Decision No. 2002/657/EC. Analysis results were reported as the value ($\mu$g kg-1). As CORTf levels in left and right feathers were highly correlated (see supplementary material S4.2), we used the average value of the two feathers in all analyses. In three cases where one of the feathers produced an outlying high value, only the value of the other feather was retained (see supplementary material S4.3). In addition, in one original and three induced feather pairs, both feathers showed CORTf concentrations below the detection capability according to Commission Decision No. 2002/657/EC and were therefore set to zero. Since gamma distributions (see statistical section below) do not accommodate 0 values, a small constant value of 0.0001 was added to all measured feathers in the subsequent analyses.\\
	
	
	\subsection*{Statistical Analyses}
	
All analyses were performed in R v 3.5.1 \citep{RCoreTeam2018}. First, to test whether CORTf levels in the original feather (grown during post-breeding moult) correlated with the social network position in the subsequent winter, we used a set of Linear Models (LMs). Each model included one of the four centrality measures as response variable and CORTf in the original feather (log-transformed), sex, age, feeding frequency, and growth bar width as explanatory variables. Growth bar width was included as a fixed effect to account for the time-dependent deposition of CORT in feathers during growth \citep{Bortolotti2010, Jenni-Eiermann2015, Romero2016}. Due to the non-independence of network measures \citep{Farine2015, Farine2017}, we used permutation tests to assess the significance of these models' parameters. We conducted 10,000 permutations of our network. To keep potentially confounding spatial patterns intact, such as preferences from certain individuals for certain feeders, we swapped observations between individuals visiting the same feeder on the same day (following \citealt{Moyers2018}). Subsequently, the same LMs as before were fitted for all these permutated networks. If a given variable's parameter estimate from our observed network fell outside the 95\% range of the estimates from the permuted networks, the effect of that variable was considered significantly different from random expectations.\\

Second, to test whether the network position correlated with the feeding frequency at the feeders, we calculated daily average frequencies of feeder visits for each individual in the network. Two sets of GLMs were then fitted. In a first set, each linear model (Gaussian distribution) included the feeding frequency as a response variable and one of the four centrality measures, sex and age as explanatory variables.\\

Finally, to test whether the winter network position was associated with the stress levels experienced during winter and whether higher feeding frequencies were related to lower stress levels during winter, we used two sets of generalized linear models (GLM) with gamma distribution (log-link). In a first model, CORTf values of induced feathers were regressed against feeding frequencies and growth bar widths in a GLM (gamma distribution, log-link). Sex and age were also included as explanatory variables in these models. The second set of models included CORTf in the induced feather as response variable and one of the four centrality measures together with sex, age and feather growth bar width as explanatory variables. As CORTf levels in the original feather were not correlated to CORTf levels in the induced feather (see supplementary material S4.4) we did not include the CORTf levels in the original feather as co-variate in these models. To ensure model convergence an observation with a CORTf value in the induced feather of 12 $\mu$g (median 0.05) had to be dropped.
For each model, post-fitting checks were performed to ensure homogeneity of residual variance, adequate residual distribution (Gaussian models) and absence of outlying observations using the DHARMa package \citep{Hartig2019}. 
	

	\subsection*{Ethics statement}
	
Field sampling protocols were approved by the Ethical Committee VIB Ghent (EC2015-023). Birds were captured and ringed under permits granted by the Belgian Ringing Scheme and the Flemish Authorities (Agentschap voor Natuur en Bos; ANB/BL-FF/V16-00003 and ANB/BL-FF/V16-00189).

\clearpage
	\section{Results}
	
A total of 138,743 unique feeder visits by 163 great tits and 115 blue tits were logged between 15 December 2016 and 28 February 2017 (83 days). The network algorithm identified 2256 gathering events at the feeders, these had a median duration of 48 min and 75 \% of the events lasted between 20 and 90 min. The emerging network from these gathering events is shown in figure \ref{fig4-2}. The average number of visits per day to one of the six feeding stations ranged from 1.00 to 39.08, and 43 \% of all individuals were logged only on a single feeder. Repeatability scores corrected for daily variation in network size were consistent to the ones previously reported for great tit (\citealt{Aplin2015a}; see online code), implying consistent network positions during winter for all tracked individuals.\\

\textit{Relationship between CORTf in the original feather and network position}: of the four centrality measures tested, betweenness, eigenvector and degree centrality were all significantly positively related to CORTf in original feathers (figure \ref{fig4-3}, table \ref{tab4-1}). For example, when controlling for the effects of age, sex, and feather growth, a doubling in CORTf in the original feather corresponded to an increase of 3 interaction partners in the winter network (figure \ref{fig4-3}C). When performing 10,000 permutations to control for the non-independence of network data, all relationships were significantly different from random expectation (see online code). The fourth centrality measure, strength, was not significantly correlated to CORTf in original feathers (figure \ref{fig4-3}, table \ref{tab4-1}).\\

\textit{Relationship between winter network position and CORTf in induced feathers}: CORTf in induced feathers was not significantly related to any of the four network metrices (figure \ref{fig4-4}, table \ref{tab4-2}),   although there was a trend for degree to increase with CORTf in the induced feahter (p-value $=$ 0.072). When controlling for the effect of age, sex, and feather growth, respectively, an increase of 10 interactions corresponded with a reduction of 15 \% in CORTf.\\

\textit{Relationship between winter network position and feeding frequency}: the four network centrality measures were all significantly and positively related to feeding frequency during winter (figure \ref{fig4-5}, table \ref{tab4-4}).\\

\textit{Relationship between feeding frequency and CORTf in induced feathers}: CORTf in induced feathers was not significantly related to feeding frequency (table \ref{tab4-4}, online code).\\


\begin{figure}[h!]
	\begin{center}
		\includegraphics[width=\textwidth]{fig4-2a.png}
	\end{center}
	\caption{Relationship between social network position in 278 free-ranging great and blue tits and feather corticosterone (CORTf) in a subset of 28 great tits. (A) winter social network where colour represents CORTf values of the original feathers, grown post-breeding. Grey nodes are individuals of which the CORTf values are unknown and white nodes are outliers that were not retained for further analysis. Node sizes of individuals with CORTf data are proportional to their eigenvector centrality.}\label{fig4-2}
\end{figure}

\begin{figure}[h!]
	\begin{center}
		\includegraphics[width=\textwidth]{fig4-2b.png}
	\end{center}
	\caption*{\textbf{Figure 4.2 (continued)}: Relationship between social network position in 278 free-ranging great and blue tits and feather corticosterone (CORTf) in a subset of 28 great tits. (B) winter social network where colour represents CORTf values of the induced feathers, grown during winter. Grey nodes are individuals of which the CORTf values are unknown and white nodes are outliers that were not retained for further analysis. Node sizes of individuals with CORTf data are proportional to their eigenvector centrality.}
\end{figure}

\begin{figure}[h!]
	\begin{center}
		\includegraphics[width=\textwidth]{fig4-3.png}
	\end{center}
	\caption{Relationship between CORTf in original tail feathers and (a) eigenvector centrality, (b) betweenness centrality, (c) degree, and (d) strength (summed edge weights) in the winter network. Points represent raw data. Lines show model predictions with a 95\% confidence interval. Great tits with higher CORTf in their original tail feathers, grown post breeding, are more central in the winter social network. P$_{rand}$-values denote the proportion of the 10,000 permutated networks with more extreme values than the observed value.}\label{fig4-3}
\end{figure}

\clearpage	
%%%Table 4.1	
\begin{table}[h!]
	\begin{center}
		\begin{footnotesize}
			\caption{Summary statistics of four models with CORTf in the original feather as explanatory variable (log-transformed) and one centrality measure in the winter network as response variable. p$_{rand}$-values denote the proportion of the 10,000 permutated networks with more extreme values than the observed value.} \label{tab4-1} 
			
			\begingroup
			\setlength{\tabcolsep}{8pt} % Default value: 6pt
			\renewcommand{\arraystretch}{1.5} % Default value: 1
			%\hspace*{-3em}
			\begin{tabular}{p{2cm} l r r r }
				
				\toprule
				\textbf{Centrality measure} & \textbf{Parameter} & \textbf{Estimate} & \textbf{Std. Error} &\textbf{ p$_{rand}$-value}\\
				\hline
				\multirow{5}{2cm}{Eigenvector centrality \mbox{R$^2$: 0.04}} & Intercept & 2.00 & 1.02 & $<$ 10$^{-4}$\\
				& CORTf original & 0.03 & 0.03 & $<$ 10$^{-4}$\\
				& sex (male) & 0.08 & 0.12 & $<$ 10$^{-4}$\\
				& age (adult) & 0.09 & 0.13 &  $<$ 10$^{-4}$\\
				& growth bar length & -0.11 & 0.06 &  $<$ 10$^{-4}$\\
				\hline
				\multirow{5}{2cm}{Betweenness centrality \mbox{R$^2$: 0.25}} & Intercept & 13.19 & 6.78 & $<$ 10$^{-4}$\\
				& CORTf original & 0.57 & 0.22 & $<$ 10$^{-4}$\\
				& sex (male) & 2.26 & 0.85 & $<$ 10$^{-4}$\\
				& age (adult) & -0.33 & 0.89 & $<$ 10$^{-4}$\\
				& growth bar length & -0.58 & 0.44 & $<$ 10$^{-4}$\\
				\hline
				\multirow{5}{2cm}{Degree \mbox{R$^2$: 0.13}} & Intercept & 251.20 & 77.48 & 0.117\\
				& CORTf original & 5.35 & 2.64 & $<$ 10$^{-4}$\\
				& sex (male) & 15.88 & 9.74 & $<$ 10$^{-4}$\\
				& age (adult) & -3.88 & 10.18 & $<$ 10$^{-4}$\\
				& growth bar length & -9.37 & 5.06 & 0.010\\
				\hline
				\multirow{5}{2cm}{Strength \mbox{R$^2$: 0.11}} & Intercept & 47.33 & 16.35 & $<$ 10$^{-4}$\\
				& CORTf original & 0.73 & 0.53 & 0.390\\
				& sex (male) & 1.99 & 2.05 & $<$ 10$^{-4}$\\
				& age (adult) & 0.51 & 2.15 & $<$ 10$^{-4}$\\
				& growth bar length & -2.40 & 1.07 & $<$ 10$^{-4}$\\
				\bottomrule
			\end{tabular}\endgroup
		\end{footnotesize}
	\end{center}
	%\end{sidewaystable}
\end{table}

\clearpage
\begin{figure}[h!]
	\begin{center}
		\includegraphics[width=0.8\textwidth]{fig4-4.png}
	\end{center}
	\caption{Relationship between CORTf in tail feathers induced to grow during winter and (a) eigenvector centrality, (b) betweenness centrality, (c) degree and (d) strength in the winter social network. Points represent raw data. Lines show model prediction with a 95\% confidence interval.}\label{fig4-4}
\end{figure}


\clearpage	
%%%Table 4.2	
\begin{table}[h!]
	\begin{center}
		\begin{footnotesize}
			\caption{Summary statistics of four models with one centrality measure in the winter network as explanatory variable and CORTf in the induced feather as response variable.} \label{tab4-2} 
			
			\begingroup
			\setlength{\tabcolsep}{8pt} % Default value: 6pt
			\renewcommand{\arraystretch}{1.5} % Default value: 1
			%\hspace*{-3em}
			\begin{tabular}{p{2cm} l r r r }
				
				\toprule
				\textbf{Centrality measure} & \textbf{Parameter} & \textbf{Estimate} & \textbf{Std. Error} &\textbf{p-value}\\
				\hline
				\multirow{5}{2cm}{Eigenvector centrality \mbox{pseudo--R$^2$: 0.27}} & Intercept & -12.72 & 3.23 & $<$ 0.001\\
				& eigenvector & -0.31 & 0.74 & 0.670\\
				& sex (male) & 0.23 & 0.54 & 0.676\\
				& age (adult) & -0.46 & 0.50 & 0.370\\
				& growth bar length & 0.93 & 0.29 & 0.003\\
				\hline
				\multirow{5}{2cm}{Betweenness centrality \mbox{pseudo--R$^2$: 0.30}} & Intercept & -13.70 & 3.17 & $<$ 0.001\\
				& betweenness & -0.10 & 0.10 & 0.317\\
				& sex (male) & 0.26 & 0.52 & 0.619\\
				& age (adult) & -0.47 & 0.45 & 0.311\\
				& growth bar length & 1.02 & 0.29 & 0.002\\
				\hline
				\multirow{5}{2cm}{Degree \mbox{pseudo--R$^2$: 0.32}} & Intercept & -10.77 & 3.07 & 0.002\\
				& degree & -0.01 & 0.01 & 0.077\\
				& sex (male) & 0.23 & 0.51 & 0.648\\
				& age (adult) & -0.52 & 0.45 & 0.256\\
				& growth bar length & 0.88 & 0.26 & 0.003\\
				\hline
				\multirow{5}{2cm}{Strength \mbox{pseudo--R$^2$: 0.29}} & Intercept & -12.53 & 3.19 & $<$ 0.001\\
				& strength & -0.02 & 0.04 & 0.545\\
				& sex (male) & 0.24 & 0.53 & 0.653\\
				& age (adult) & -0.43 & 0.48 & 0.379\\
				& growth bar length & 0.92 & 0.28 & 0.003\\
				\bottomrule
			\end{tabular}\endgroup
		\end{footnotesize}
	\end{center}
	%\end{sidewaystable}
\end{table}	

\clearpage
%%%Table 4.3	
\begin{table}[h!]
	\begin{center}
		\begin{footnotesize}
			\caption{Summary statistics of the models with one centrality measure in the winter network as explanatory variable and feeder visits as response variable} \label{tab4-3} 
			
			\begingroup
			\setlength{\tabcolsep}{8pt} % Default value: 6pt
			\renewcommand{\arraystretch}{1.5} % Default value: 1
			%\hspace*{-3em}
			\begin{tabular}{p{2cm} l r r r }
				
				\toprule
				\textbf{Centrality measure} & \textbf{Parameter} & \textbf{Estimate} & \textbf{Std. Error} &\textbf{p-value}\\
				\hline
				\multirow{4}{2cm}{Eigenvector centrality \mbox{R$^2$: 0.20}} & Intercept & 11.945 & 1.006 & $<$ 0.001\\
				& eigenvector & 12.377 & 1.672 & $<$ 0.001\\
				& sex (male) & -0.624 & 1.057 & 0.55\\
				& age (adult) & -1.646 & 1.067 & 0.124\\
				\hline
				\multirow{4}{2cm}{Betweenness centrality \mbox{R$^2$: 0.17}} & Intercept & 11.660 & 1.064 & $<$ 0.001\\
				& betweenness & 1.384 & 0.202 & $<$ 0.001\\
				& sex (male) & -1.052 & 1.074 & 0.328\\
				& age (adult) & -0.594 & 1.083 & 0.584\\
				\hline
				\multirow{4}{2cm}{Degree \mbox{R$^2$: 0.09}} & Intercept & 8.526 & 1.786 & $<$ 0.001\\
				& degree & 0.077 & 0.016 & $<$ 0.001\\
				& sex (male) & -0.512 & 1.126 & 0.650\\
				& age (adult) & -0.481 & 1.137 & 0.194\\
				\hline
				\multirow{4}{2cm}{Strength \mbox{R$^2$: 0.44}} & Intercept & 6.054 & 1.037 & $<$ 0.001\\
				& strength & 1.062 & 0.081 & $<$ 0.001\\
				& sex (male) & -0.895 & 0.884 & 0.313\\
				& age (adult) & -1.766 & 0.892 & 0.049\\
				\bottomrule
			\end{tabular}\endgroup
		\end{footnotesize}
	\end{center}
	%\end{sidewaystable}
\end{table}

%%%Table 4.4	
\begin{table}[h!]
	\begin{center}
		\begin{footnotesize}
			\caption{Summary statistics of the model with feeder visits as explanatory variable and CORTf in the induced feather as response variable.} \label{tab4-4} 
			
			\begingroup
			\setlength{\tabcolsep}{8pt} % Default value: 6pt
			\renewcommand{\arraystretch}{1.5} % Default value: 1
			%\hspace*{-3em}
			\begin{tabular}{p{2cm} l r r r}
				
				\toprule
				& \textbf{Parameter} & \textbf{Estimate} & \textbf{Std. Error} &\textbf{p-value}\\
				\cmidrule{2-5}
				\multirow{3}{2cm}{CORTf in the induced feather \mbox{pseudo-R$^2$: 0.35}} & Intercept & -13.137 & 2.272 & $<$ 0.001\\
				& Feeding frequency & -0.047 & 0.028 & 0.100\\
				& Growth bar length &1.011 & 0.224 & $<$ 0.001\\
				\bottomrule
			\end{tabular}\endgroup
		\end{footnotesize}
	\end{center}
	%\end{sidewaystable}
\end{table}

\begin{figure}[h!]
	\begin{center}
		\includegraphics[width=\textwidth]{fig4-5.png}
	\end{center}
	\caption{Relationship between feeder visits (i.e. average daily number of visits to a feeder) and (a) degree, (b) eigenvector centrality, (c) betweenness centrality, and (d) strength in the winter social network. Points represent raw data. Lines show model predictions with a 95\% confidence interval (dark grey) and prediction interval (i.e., taking into account residual standard deviation in light grey).}\label{fig4-5}
\end{figure}


\clearpage	
	\section{Discussion}
	
Great tits characterized by higher CORTf values in original feathers grown during their natural post-breeding moult occupied more central network positions in winter flocks, in accordance with our first prediction. More central individuals were more frequently recorded at feeders during winter, in support of our second prediction, but CORTf values in induced feathers did not significantly correlate with the recorded feeding frequencies. Individuals associating with more individuals, as indicated by a higher degree, showed somewhat lower CORTf values in their induced feathers grown during winter, although this relationship was not significant.\\

The fact that great tits with higher CORTf levels in original feathers grown after breeding occupied more central social positions in winter flocks supports the general notion that HPA-axis (re)activation, in particular stress-induced glucocorticoid levels, can shape complex social behaviour across seasons in birds. Similarly, in wild populations of barn swallows (\textit{Hirundo rustica}) \citep{Levin2016} and house finches (\textit{Haemorhous mexicanus}) \citep{Moyers2018}, individuals with higher plasma CORT levels were shown to occupy more central positions in the social network in the weeks after the CORT levels were measured. Moreover, both in captive and wild populations of zebra finches, higher CORT levels during the nestling stage led to less social differentiation after reaching nutritional independence during foraging \citep{Boogert2014, Brandl2019}. Results from our study further suggest that relationships between glucocorticoid hormones and social network position do not only operate between different life stages, but also between different seasons. Great tits interacting with many other individuals showed a tendency of lower CORTf in their induced feathers grown during winter. In winter flocks of tits, central individuals may have more access to social information about food resources and can therefore discover novel food patches faster \citep{Aplin2012, Farine2015b, Firth2016}. CORT values in late summer could influence winter behavior through several, non-mutually exclusive, mechanisms. First, behavioural adjustments mediated by increased CORT levels can remain for several months, as has been shown in aviary experiments \citep{Spencer2007, Boogert2013}. Second, CORT levels in late summer could directly influence social organization in late summer, which in turn could shape social organization in winter. Although less extensive, tits do form flocks in late summer, which constitute a social network \citep{Ekman1989}. In great tits, social network position has been observed to be stable within the winter season (this study; \citealt{Aplin2015a}), suggesting it could be possible that social network position is also stable between seasons.\\

Possibly, stress levels measured in the original feathers in adults reflect differences in reproductive investment, as moulting starts directly after breeding \citep{Dhondt1973}. These feathers thus grow in a period when insect food resources are declining but still relatively high \citep{Naef-daenzer2008}. Differences in CORTf levels in the induced feathers, however, could reflect nutritional condition, as this is the main stressor during the period over which these feathers grow \citep{Jansson1981, Brittingham1988, Perdeck2000}.\\

While access to food is the main determinant of winter survival for resident birds in temperate zones, pathogens such as \textit{Mycoplasma gallisepticum} or \textit{Salmonella} have been shown to infect passerines and can also cause high winter mortality \citep{Robb2008}. More central individuals in social networks are, in general, also more likely to acquire pathogens and parasites \citep{Godfrey2009, Bull2012}. Individuals may hence need to trade off access to food resources for pathogen exposure, with stressed individuals likely prioritizing direct access to food and those in better condition likely prioritizing pathogen avoidance. In our study, we were only able to quantify CORTf in individuals that survived the winter. Hence, we lack data on CORTf levels during the post-breeding period, as well as CORTf levels and network centrality during winter, of non-survivors. An earlier study on house sparrows (\textit{Passer domesticus}) showed that individuals with higher CORTf in feathers grown post breeding were less likely to survive the winter \citep{Koren2012}. Further research on the interaction between stress hormones and social networks should therefore test whether great tits experiencing high stress levels in the post-breeding period also occupy more central network positions, and if so, whether this central network position affects their pre-mortality pathogen load.\\

While network centrality was positively correlated with the number of feeder visits across individuals, large inter-individual variation for a given level of network centrality occurred. For instance, some individuals always visited a feeder alone (high feeding frequency but low centrality) while others only made a few visits per day yet always accompanied by many other birds (low feeding frequency but high centrality). The fact that some frequent visitors showed low levels of network centrality corroborates that a central position could be caused by an adjustment in social behaviour -- rather than constituting a simple by-product of frequently visiting feeders. Nevertheless, and contrary to our expectations, we did not observe a direct correlation between the number of feeder visits and CORTf in induced feathers, nor was any of the centrality metrices investigated significantly related with CORTf in induced feathers, although we did observe a trend for degree, i.e. the number of unique individuals an individual associates with. In our study system, food was available at feeders from the second half of December, and was thus abundant and not variable in time. Moreover, our study took place in a beech mast year, and thus also natural food was abundantly available. Thus the overabundance of food during winter may have limited the applicability of feeder visits as a proxy for access to food.\\

Alternatively, or complementarily, the observed relationship between CORTf in original feathers and social network position could be caused by other, non-mutually-exclusive factors. First, in several bird species, dominant individuals tend to occupy more central positions in social networks \citep{Krause2002} and can have higher CORT responses to stressors \citep{Pravosudov2003}. However, previous studies on great tits did not show a correlation between dominance and centrality in mixed species flocks \citep{Farine2012}. Furthermore, in our study, CORTf of both the original and induced feathers were not correlated to wing length (supplementary material S4.4), which is a predictor of dominance in great tits \citep{Sandell1991, Gosler1996}. Thus, differences in dominance are less likely to explain relationships between glucocorticoids and position in the social network of great tits studied here. Second, both network position and stress physiology have previously been linked to animal personality, i.e. consistent between-individual variation in suites of behaviour. Earlier studies showed that great tit individuals that are more `exploratory' in standardized novel-environment tests have more social associations and switch more between flocks, resulting in a more central position in the social network \citep{Aplin2013, Snijders2014}. However, more `exploratory' individuals were also shown to have a lower increase in CORT in response to acute stressors in great tits \citep{Baugh2013} as well as other passerines \citep{Lendvai2011, Moyers2018}. Taken together, more central individuals would then be expected to have lower CORTf, while we observed an opposite correlation between CORTf values in the original feathers and network centrality. It would therefore be highly interesting to combine personality tests, stress physiology, and social network analyses in further studies.\\

In conclusion, we here provide evidence for relationships between stress, quantified through feather glucocorticoid levels as a retrospective read-out of HPA-axis (re)activation, and social interactions in a natural bird population. Great tits with higher stress hormone levels after the breeding season occupied more central social network positions in the subsequent winter. Correlations between centrality and winter stress levels were not significant, but our data do suggest that birds interacting with many flock members could have lower levels of stress hormone in winter. This could indicate a temporal feedback mechanism whereby individuals experiencing elevated stress levels adjust their social interactions over winter, hence reducing subsequent stress levels. While these results support the notion of changes in social network positioning as a consequence of behavioural strategies in response to stress, earlier studies on great tits revealed consistent network positions across different winters \citep{Aplin2015a}. Similar levels of consistency were also reported in other species, even in the case of profound changes in the environmental conditions or in group composition \citep{Baigger2013, Godfrey2013, Krause2017, Blaszczyk2018}, while other studies found evidence for temporal variation in networks in response to experimental manipulation of prey abundance \citep{StClair2015}, and experimental pathogen introduction \citep{Stroeymeyt2018}. Hence, to unravel the extent to which network positions represent static or plastic behavioural strategies, there is a need for innovative field experiments on natural populations in which stress hormone levels, social network positions, and available food resources can be manipulated independently.\\
\clearpage
	
	\subsection*{Funding}
	
This work was supported by the Ghent University GOA project ``Scaling up Functional Biodiversity Research: from Individuals to Landscapes and Back (TREEWEB''.
	
	\subsection*{Acknowledgements}
	
We thank Robbe De Beelde for invaluable help with the field work, Kathryn Godfrey for processing feather samples, Luc Willems for technical support, and Liesbeth De Neve for contributing to the overall study design. We thank Jolien Scheerlinck of the Stress Physiology Research Group for helping with sample preparation and glucocorticoid analyses. We thank the family Haesaert, family Verheyen, and the family Van Hoeylandt for allowing us to install feeders in their gardens.
	
	\subsection*{Data Accessibility}
	
All data and code to reproduce the analyses is available on GitHub: \url{https://doi.org/10.5281/zenodo.3582958}


	
	\subsection*{Author contributions}
Luc Lens, Diederik Strubbe, Lionel Hertzog and Daan Dekeukeleire conceived the study; Luc Lens, Dries Bonte, Kris Verheyen and An Martel acquired funding; Daan Dekeukeleire and Pieter Vantieghem collected the data; Johan Aerts performed the CORTf analyses; Lionel Hertzog performed the network analyses; Daan Dekeukeleire led the writing of the manuscript and all authors contributed significantly to the drafts. \\

\clearpage	
	\section{Supplementary material}
	
	\subsection*{Supplementary material S4.1: population size estimates}
	
We used the Jolly-Seber method for open populations \citep{Jolly1965, Seber1965} to estimate the population size during winter. Briefly, we used capture data from March 2015 to November 2017; with for each year three sampling events: mist netting in November, mist netting in March, and captures at nest boxes during the breeding season. We estimated the population sizes of both great and blue tits at two sampling events: November 2016 and March 2017 (table S4.1). We first estimated the number of tagged individuals in our population ($M_i$), using the formula:\\
\begin{equation*}
	M_i = \frac{n_i \cdot z_i}{r_i} + m_i
\end{equation*}	
with $n_i$ the total number of individuals captured during the sampling event; $z_i$ the number of individuals tagged before the sampling event, not captured during the sampling event, but captured in a later sampling event; $r_i$ the number of individuals captured during the sampling event that were captured in a later sampling event; and $m_i$ the number of tagged individuals captured during the sampling event.\\

Then, we estimated the population size during the sampling event ($N_i$) using the formula:
\begin{equation*}
	N_i = \frac{n_i \cdot M_i}{m_i}
\end{equation*}
We then calculated the ratio of tagged individuals in the population ($M_i$ $+$ newly tagged individuals) to the total population and took the mean of November and March to estimate the proportion of the winter population size that was tagged.

\begin{table}[h!]
	\begin{center}
		\begin{footnotesize}
			\caption*{\textbf{Table S4.1}: Population size estimates} 
			
			\begingroup
			\setlength{\tabcolsep}{8pt} % Default value: 6pt
			\renewcommand{\arraystretch}{1.5} % Default value: 1
			%\hspace*{-3em}
			\begin{tabular}{p{3cm} c c c c}
			\toprule
			& \multicolumn{2}{c}{Great tit} & \multicolumn{2}{c}{Blue tit}\\
			\cmidrule(r){2-3} \cmidrule(r){4-5}
			& November 2016 & March 2017 & November 2016 & March 2017\\
			\cmidrule{1-5}
			n & 142 & 49 & 74 & 20\\
			m & 56 & 34 & 23 & 15\\
			r & 62 & 29 & 14 & 10\\
			z & 20 & 48 & 23 & 22\\
			M & 101.8 & 115.1 & 144.6 & 59\\
			N & 258.2 & 165.9 & 456.1 & 78.7\\
			Estimated proportion of tagged individuals in the population & 0.72 & 0.78 & 0.42 & 0.81\\
			\bottomrule
			\end{tabular}
		\end{footnotesize}\endgroup
	\end{center}
\end{table}
\clearpage

	\subsection*{Supplementary material S4.2: CORTf levels}
	
\begin{figure}[h!]
	\begin{center}
		\includegraphics[width=\textwidth]{figS4-2.png}
	\end{center}
	\caption*{Figure showing the CORTf values in the left and the right second to outer tail feather. CORTf values in both feather were highly correlated (including outliers: Pearson's correlation, R $=$ 0.757, p $<$ 0.001), and thus the mean was used in further analyses. In three out of the 72 feather pairs, marked as triangles, the difference between the two feathers was unlikely high (\citealt{Aharon-Rotman2017, Lattin2011}; but see \citealt{Harris2016}). Despite the fact that the respective values were analytically according the set criteria (see material and methods), these were omitted from the statistical analysis only the lowest value was kept.}
\end{figure}

\clearpage
	\subsection*{Supplementary material S4.3: CORTf levels in original and induced feathers}
	
	\begin{figure}[h!]
		\begin{center}
			\includegraphics[width=\textwidth]{figS4-3.png}
		\end{center}
		\caption*{Figure showing the CORTf values in the original and the induced second to outer tail feather. We investigated if CORTf in the induced was significantly correlated to CORTf in the original feathers in a linear models, were growth bar width was included as a fixed effect to account for the time-dependent deposition of CORT in feathers during growth. There is no significant effect of CORTf in the original feather on the CORTf in the induced feather (n $=$ 37, df $=$ 26, t-value $=$ -0.255, p $=$ 0.801). Individuals are coloured according to their age (red $=$ first year; blue $=$ adult).}
	\end{figure}

\clearpage
	\subsection*{Supplementary material S4.4: CORTf and wing length}
	
	We investigated if CORTf was significantly correlated to wing length in the induced and original feathers in two separate linear models. In both models, growth bar width was included as a fixed effect to account for the time-dependent deposition of CORT in feathers during growth. There is no significant effect of wing length on CORTf in the original feather (n $=$ 37, df $=$ 34, t-value $=$ -0.358 , p $=$ 0.722), nor in the induced feather (n $=$ 37, df $=$ 34, t-value $=$ -0.183, p $=$ 0.856).\\
	
\begin{figure}[h!]
	\begin{center}
		\includegraphics[width=0.6\textwidth]{figS4-4.png}
	\end{center}
	\caption*{
		Figure showing the raw data for (a) the original feather CORTf and (b) the induced feather CORTf.}
\end{figure}	
	
\thispagestyle{plain}
\hbox{}
\clearpage
	%%%%%%%%%%%%%%%%%%%%%%%%%%%%%%%%%%%%% Chapter 5 %%%%%%%%%%%%%%%%%%%%%%%%%%%%%%%%%%%%%%%%%	
\thispagestyle{plain} % empty
\CenterWallPaper{1}{CH5.jpg}
\newpage{\thispagestyle{empty}\cleardoublepage}
\ClearWallPaper
\pagestyle{mainmatter}
\chapter{Avian top-down control affects invertebrate herbivory and sapling growth more strongly than overstorey species composition in temperate forest fragments} \label{chapter5}
\chaptermark{Avian top-down control affects invertebrate herbivory and sapling growth}
\lettergroup{\thechapter}

\begin{flushright} \color{gray}Daan Dekeukeleire*\\ Irene M. van Schrojenstein Lantman*\\ Lionel Hertzog\\ Martijn L. Vandegehuchte\\ Diederik Strubbe\\
	 Pieter Vantieghem\\ An Martel\\ Kris Verheyen\\ Dries Bonte \\ Luc Lens\\
	\vspace{0.5cm} * equal contribution 
	\vspace{0.5cm}
	
	Adapted from: Dekeukeleire et al. (2019) \textit{Forest Ecology and Management}, \textbf{442}:1-9. DOI: 10.1016/j.foreco.2019.03.055
	
	\vspace*{\fill}		\end{flushright}
\noindent \color{gray} $\lhd$ Caterpillar of the butterfly \textit{Favonius quercus}, feeding on pedunculate oak. Photo by Pieter Vantieghem.
	

\color{black}
\newpage


	\section{Abstract}
	To better understand natural regeneration of trees and forest dynamics it is important to gain insight into the drivers of invertebrate herbivory. In mature forests, associational resistance of trees resulting from a high diversity of neighbouring trees is common, and can have cascading effects on tree growth through resource concentration effects or through changes in top-down control. While the underlying biological processes are known to be influenced by the forest's spatial properties, we lack insights on how resource concentration, top-down control and fragmentation jointly affect sapling performance in fragmented landscapes. We therefore experimentally quantified effects of the proportion of conspecific trees in the overstorey (resource concentration), avian top-down control (natural enemies) and distance to the forest edge on invertebrate herbivory levels and sapling growth. The assessments were made on planted saplings of beech (\textit{Fagus sylvatica}), pedunculate oak (\textit{Quercus robur}) and red oak (\textit{Quercus rubra}) in 53 experimental plots and birds were excluded by means of exclosures from a subset of these saplings. Excluding avian top-down control increased herbivory on each tree species. Increased herbivory led to decreased sapling growth in beech and red oak. On pedunculate oak saplings, top-down control was stronger closer to the forest edge. Furthermore, in this species, herbivory inside the exclosures increased with an increasing proportion of conspecific trees in the overstorey, while such a resource concentration effect was not observed outside the exclosures. Our results show the importance for forest management of conserving insectivorous birds and promoting a mixed overstorey, which can decrease sapling herbivory when bird abundance is low. More generally, our study provides insight into the complex, multitrophic interactions that drive sapling growth in forest stands located within fragmented landscapes.\\
	\clearpage
	
	\section{Introduction}
	Invertebrate herbivores play a key role in forest food webs and the resulting nutrient cycling \citep{Duffy2002} and herbivory affects the survival and growth of trees, in particular of saplings. As herbivorous invertebrates affect forest succession and dynamics \citep{Bagchi2014}, understanding the drivers of herbivory is important to better understand natural regeneration patterns and predict forest dynamics. Herbivory levels on a focal tree can be strongly influenced by the composition and diversity of the neighbouring tree community \citep{Barbosa2009}. While a higher diversity of neighbours may both increase herbivory levels through `associational susceptibility' or decrease herbivory levels through `associational resistance', meta-analyses and several reviews suggest that associational resistance is most common \citep{Andow1991, Balvanera2006}, especially in mature forests \citep{Castagneyrol2014, Jactel2007}.\\
	
	Associational resistance is most often explained by two --mutually non-exclusive-- hypotheses: the Resource Concentration Hypothesis and the Enemies Hypothesis \citep{Jactel2007}. According to the Resource Concentration Hypothesis \citep{Root1973}, specialist herbivores have a lower probability of finding their matching host species in more diverse plant communities, resulting in lower herbivore population densities and less herbivory damage (reviewed for temperate forests by \citealt{Castagneyrol2014, Jactel2005}). In support of this hypothesis, young Norway spruce (\textit{Picea abies} L) had a lower probability of having galls under higher canopy cover of other tree species in mixed stands \citep{Muiruri2017}. Similarly, the pine processionary moth (\textit{Thaumetopoea pityocampa}) was less efficient in locating pine trees due to volatiles of neighbouring birch trees, which decreased the abundance of caterpillars and thus defoliation of pines growing in mixed forest stands \citep{Jactel2011}. According to the Enemies Hypothesis \citep{Letourneau1987, Root1973, Russell1989}, the number of predators and parasites is higher in more diverse plant communities due to a higher diversity of prey species and higher abundance of additional resources such as refuges or food. Stronger top-down control by predators and parasites then increases plant fitness by reducing herbivore abundance and herbivory. Although empirical support for the Enemies Hypothesis in forests is mixed \citep{Grossman2018, Zhang2011}, predation pressure on herbivores has been widely associated with tree species diversity and forest composition \citep{Giffard2012, Muiruri2016, Yang2018}. A recent study in tropical forests recorded higher abundance, and higher functional and phylogenetic diversity, of birds in tree-species mixtures than predicted based on the corresponding single-species monocultures, which resulted in higher attack rates on artificial caterpillars in these mixtures \citep{Nell2018}. Due to intensive foraging, vertebrate predators, such as insectivorous birds, can reduce the abundance of herbivorous invertebrates and thus increase plant fitness or growth \citep{Mantyla2011, Mooney2010, Schmitz2000}. For example, white oak saplings (\textit{Quercus alba} L) from which birds were excluded, suffered 12 \% more herbivory than control saplings, and produced circa 20 \% less biomass in the subsequent growing season \citep{Marquis1994}.\\
	 
	Despite the solid theoretical basis for associational resistance, effects thereof remain notoriously difficult to quantify in the field, and the relative importance of the different underlying mechanisms remains unclear \citep{Barbosa2009, Grossman2018, Muiruri2016}. While associational resistance is essentially a local process, it nevertheless depends on the spatial context, which is surprisingly seldom considered as an important driver \citep{Nadrowski2010}. Globally, more than 20 \% of forest area is currently located within 100 m of a forest edge, which has far-reaching effects on species distributions and ecosystem functioning \citep{Haddad2015, Pfeifer2017}. Impacts of forest edges on herbivory are highly variable, and still not well understood. Both higher (e.g. \citealt{Castagneyrol2019, Terborgh2006}) and lower (e.g. \citealt{Ruiz-Guerra2010, Simonetti2007}) levels of herbivory have been reported closer to the forest edge and in smaller fragments. Lower rates of herbivory near forest edges may result from higher top-down control by insectivorous bird species, the diversity and activity of which has been shown to be higher near temperate forest edges \citep{Hofmeister2017, Melin2018, Terraube2016}. In support of this, experimental studies with artificial caterpillars revealed higher predation levels at edges than in the forest interior (\citealt{Barbaro2014, Gonzalez-Gomez2006}; but see \citealt{Peter2015}). In synergy with patterns of overstorey species composition, shifts in abundance of natural enemies in relation to the distance from forest edges may hence shape associational effects on herbivory in fragmented landscapes.\\
	
	In this study, we experimentally assessed the relative effects of overstorey species composition and avian top-down control on invertebrate herbivory of tree saplings in deciduous forest fragments, and how these changes in herbivory levels in turn affect sapling growth. We therefore planted saplings of three focal tree species (beech (\textit{Fagus sylvatica} L), pedunculate oak (\textit{Quercus robur} L) and red oak (\textit{Q. rubra} L)) in plots along independent gradients of forest fragmentation and tree species composition. We experimentally excluded insectivorous birds from a subset of these saplings, and monitored herbivory levels in both the presence and absence of birds. We predicted (i) higher herbivory levels on saplings inside exclosures than on those outside exclosures; (ii) stronger effects of the exclusion of avian predators closer to forest edges; (iii) higher herbivory levels on saplings under an overstorey with a larger proportion of conspecific trees and (iv) reduced sapling growth under higher herbivory levels.
	
\clearpage
	\section{Material \& Methods}
	
		\subsection*{Study area}
		
		This study was conducted within the Treeweb platform, which consists of 53 study plots (30 m $\times$ 30 m) spread across 19 forest fragments in northern Belgium (50.899$^{\circ}$ N, 3.946$^{\circ}$ E -- 50.998$^{\circ}$ N, 3.584$^{\circ}$ E; see also Box A, page \pageref{boxa}). The tree layer in the plots consists of three regionally common and economically relevant species: the native beech (\textit{F. sylvatica}) and pedunculate oak (\textit{Q. robur}), and the non-native red oak (\textit{Q. rubra}). Spread along a forest fragmentation gradient, six to eight replicate plots were selected for each of the possible tree species combinations; i.e. three monocultures, three two-species mixtures and one three-species mixture. To minimize the effects of potentially confounding environmental factors, all plots were established in mature stands ($>$ 60 years old; \citealt{DeGroote2018}), with a similar land-use history (continually forested since at least 1850), management (no forestry management in the last decade) and soil conditions (dry sandy loam; see \citealt{DeGroote2017} for more details). The herb and shrub layer in these plots is relatively species-poor, with \textit{Rubus} sp., \textit{Pteridium aquilinum} and saplings of red oak as the most common understorey species \citep{DeGroote2017}. Large mammalian herbivores, such as roe deer, are rare and only present in low numbers ($<$ 10) in the largest forest fragments in our study area. The most common insectivorous birds in the study plots are great tit (\textit{Parus major}), blue tit (\textit{Cyanistes caeruleus}), wren (\textit{Troglodytes troglodytes}), robin (\textit{Erithacus rubecula}), and nuthatch (\textit{Sitta europaea}).\\
		
		In every plot, the diameter at breast height of all trees with a diameter larger than 15 cm was measured to calculate the relative basal area of each tree species per plot (range: 0 -- 1). Furthermore, we calculated the Euclidean distance from the centre of each plot to the nearest forest edge (range: 7.0 m -- 215.5 m) as a proxy for edge effects.\\
		
		\subsection*{Field experiment}
		In autumn 2014, we planted 18 saplings (3-year-old, 40 -- 120 cm tall) in each plot (n $=$ 53) in three clusters (figure \ref{fig5-1}A). Each cluster consisted of six saplings planted in pairs: two beech saplings, two red oak saplings and two pedunculate oak saplings. Before bud burst in early April 2016 we covered one cluster of saplings per plot with a commercial bird exclusion net (dark green polypropylene nets, mesh size 10 $\times$ 10 mm) to exclude avian predators but allow invertebrates to enter. We attached the nets to a 2.5 m long pole in the centre of the sapling cluster and secured the edge of the net to the ground to prevent birds from entering from below (figure \ref{fig5-1}B). Exclosures were installed in 52 of the 53 plots (no permission granted in one plot). Note that our exclosures only excluded vertebrates; invertebrate predators were not excluded and may even have benefited from the exclusion of vertebrate insectivores (e.g. \citealt{Bosc2018, Grass2017}).\\
				
%Figure 5.1
\begin{figure}[h!]
	\begin{center}
		\includegraphics[width=\textwidth]{fig5-1_cut.png}
	\end{center}
	\caption{A) Example of a detailed map of a three-species plot (30 m $\times$ 30 m), with the crown projection area for the overstorey trees (coloured according to the species), location of the sapling clusters (black triangles) and exclosure (grey rectangle). B) Exclosure with a cluster of six saplings. \label{fig5-1}}
\end{figure}
		
		We scored herbivory levels and measured the growth of 466 saplings (see supplementary material S5.1 for sapling survival). After the main herbivory peak, between June 28th and June 30th, 2016, we randomly sampled ten leaves per sapling to estimate herbivory levels. A sample of ten leaves corresponds to circa 8 \% of the leaves of a beech sapling, circa 24 \% for red oak and circa 30 \% for pedunculate oak. Two authors (DD, IvSL) independently estimated the leaf area loss (percentage of leaf consumed by chewers and leaf miners) per leaf in the field visually in categories nearest to 0 \%, 1 \%, 2 \%, 5 \% and every consecutive 5 \%. In cases the two estimates differed, the average value was used. At the end of the growing season (September 2015 and 2016), we measured the height of all saplings to the nearest cm.\\
		
		\subsection*{Statistical Analyses}
		
		To investigate (in)direct relationships between the overstorey tree species composition, distance to the forest edge, top-down control by insectivorous birds, and sapling growth, we used structural equation models (SEM). For each sapling species, we built a separate SEM (figure \ref{fig5-2}) as a piecewise combination of (generalized) linear mixed effect models using piecewiseSEM \citep{Lefcheck2016}. This approach allows for the incorporation of random effects and flexible distributions of the response variable. As independent (exogenous) variables we included in each model: (i) the relative basal area (proportion) of the conspecific tree species in the overstorey of a plot (range 0 -- 1), (ii) the Euclidian distance of the plot centre to the forest edge (range 6.97 m -- 215.49 m) and (iii) the absence or presence of an exclosure (0 or 1). As dependent (endogenous) variables we considered `herbivory level', calculated as the mean leaf area loss of the ten sampled leaves per sapling and `sapling growth', calculated as the difference in height between September 2016 and September 2015 divided by the initial sapling height in September 2015. The first model in each SEM modelled herbivory level (response variable) as a function of the relative basal area of the conspecific tree species, the edge distance (scaled around their mean; \citealt{Schielzeth2010}), the exclosure treatment and the interaction between exclosure treatment and edge distance and between exclosure treatment and conspecific basal area as explanatory variables. These models had a negative binomial response variable distribution (log-link) with the variance increasing quadratically with the mean, as this distribution gave a better fit and convergence than a Poisson distribution. The second model in each SEM modelled sapling growth (response variable) as a function of the scaled edge distance, conspecific basal area (relative to total basal area) and herbivory level as explanatory variables and these models had a normal response variable distribution. All statistical inferences reported are those for the full models.\\
		
		All models were run using the glmmTMB package v. 0.2.3 \citep{Brooks2017}. In each model, sapling cluster ID, nested within plot ID, nested within forest fragment ID was included as a random term to account for the possible dependence of the herbivory levels on saplings planted within the same cluster located within the same plot located within the same forest fragment. Marginal R$^2$ (i.e. proportion of the variation explained by the fixed effects) were derived using the function r2 from the package sjPlot v. 2.6.1 \citep{Ludecke2018}. In some models the random effect variance was estimated at 0 and to derive the R$^2$ values we had to drop these random terms. SEM fits were evaluated using Shipley's test of directed separation (Fisher's C statistic) calculated through the significance of all missing paths. Values of p $>$ 0.05 indicated that the model included all important relations between the observed variables \citep{Shipley2009}. SEMs were fitted using the package piecewiseSEM v. 2.0.2 \citep{Lefcheck2016}. Effect sizes were standardized by multiplying the raw coefficients with the standard deviation of the independent variable divided by the standard deviation of the dependent variable. All analyses were performed in R v. 3.5.1 \citep{RCoreTeam2018}.\\
\clearpage	
	\section{Results}
	
	Herbivory levels were, on average, higher for pedunculate oak (est. 15.8 \%; 95\%--CI: 12.5 -- 19.0) and red oak (est. 18.6 \%; 95\%--CI: 20.3 -- 16.9) than for beech (est. 10.4 \%; 95\%--CI: 9.1 -- 11.7; supplementary material S5.2). The three structural equation models (figure \ref{fig5-2}) showed a good fit, indicating that no significant paths were missing (SEM beech: Fishers's C $=$ 10.31, df $=$ 6, p $=$ 0.11; SEM pedunculate oak: Fishers's C $=$ 9.93, df $=$ 6, p $=$ 0.13; SEM red oak: Fisher's C = 7.79, df $=$ 6, p $=$ 0.24).\\
	
	For beech saplings, excluding birds led to 4.8 \% more leaf area loss (figure \ref{fig5-3}, table \ref{tab5-1}, est. herbivory level exclosure: 17.5 \%, 95\%--CI: 6.1 -- 28.9 vs. control: 12.7 \%, 95\%--CI: 4.4 -- 21.0), which in turn led to a decrease of 0.005 in growth (height increase relative to height the year before) (figure \ref{fig5-4}A, table \ref{tab5-1}, est. growth exclosure: -0.002, 95\%---CI: -0.030--0.025 vs. control: 0.003, 95\%--CI: -0.025--0.031). Edge distance and the proportion of beech in the overstorey was not associated with herbivory level or growth for beech (figure \ref{fig5-2}, figure \ref{fig5-4}B, figure \ref{fig5-5}, table \ref{tab5-1}).\\
	
%Figure 5.2
\begin{figure}[h!]
	\begin{center}
			\includegraphics[width=0.8\textwidth]{fig5-2.png}
	\end{center}
	\caption{Structural equation model (SEM) path diagrams for A) beech (\textit{Fagus sylvatica}), B) pedunculate oak (\textit{Quercus robur}) and C) red oak (\textit{Quercus rubra}). Arrows show direction, dashed arrows show interactions, and numbers near arrows (standardized effect sizes) show relative magnitude of the relationship for significant variables. Grey arrows show non-significant relationships specified in the a priori model. Coefficients of determination (marginal R$^2$) are shown for all response variables. \label{fig5-2}}
\end{figure}

%Figure 5.3
\begin{figure}[h!]
	\begin{center}
		\includegraphics[width=0.8\textwidth]{fig5-3.png}
	\end{center}
	\caption{Raw data for the herbivory level (average \% leaf area loss for ten leaves per sapling) for beech (\textit{Fagus sylvatica}), pedunculate oak (\textit{Quercus robur}) and red oak (\textit{Quercus rubra}) from which birds were excluded (exclosure treatment) and for control saplings. Herbivory levels significantly increased inside the exclosures for all saplings species.\label{fig5-3}}
\end{figure}

%Figure 5.4
\begin{figure}[h!]
	\begin{center}
		\includegraphics[width=\textwidth]{fig5-4.png}
	\end{center}
	\caption{Relationship between growth (increase in height relative to initial height) and A) herbivory levels (\% leaf area loss) and B) relative concentration of conspecific trees in the overstorey for saplings of beech (\textit{Fagus sylvatica}), pedunculate oak (\textit{Quercus robur}) and red oak (\textit{Quercus rubra}). Points show raw data per sapling and the lines show model prediction with 95\%-confidence intervals (grey) assuming the average values of other explanatory variables. Note that some saplings decreased in height due to damage. A) Sapling growth significantly decreased with increasing herbivory levels for beech and red oak (full lines), while this relationship was not significant for pedunculate oak (dashed line). B) Sapling growth significantly increased with increasing concentration of red oak in the overstorey (full line), while this relationship was not significant for the other species (dashed lines).\label{fig5-4}}
\end{figure}


%%Table 5.1
\clearpage
\begin{landscape}
	\thispagestyle{empty}
%\begin{sidewaystable}
\begin{table}
	\begin{center}
		\begin{footnotesize}
			\caption{Summary of test statistics for models included in the SEM, ordered per sapling species. All estimates (est.), standard errors (SE) and p-values refer to the fixed effects in models with either herbivory level or sapling growth as response variable.}  \label{tab5-1}
			
			\begingroup
			\setlength{\tabcolsep}{8pt} % Default value: 6pt
			\renewcommand{\arraystretch}{1.5} % Default value: 1
			%\hspace*{-3em}
			\begin{tabular}{p{0.2cm} p{2cm} r r r r r r r r r p{0.001cm}}
				
				\cmidrule{1-11}
				& & \multicolumn{3}{c}{\textbf{\textit{F. sylvatica}}} & \multicolumn{3}{c}{\textbf{\textit{Q. robur}}} & \multicolumn{3}{c}{\textbf{\textit{Q. rubra}}} &\\
				\cmidrule(r){3-5}
				\cmidrule(r){6-8}
				\cmidrule(r){9-11}
				& & \textbf{est.} & \textbf{SE} & \textbf{p-value} & \textbf{est.} & \textbf{SE} & \textbf{p-value} & \textbf{est} & \textbf{SE} & \textbf{p-value} &\\
				\multirow{6}{*}{\rotatebox[origin=c]{90}{\parbox[c]{3cm}{\centering \textbf{Herbivory level}}}} & Intercept & 2.001 &0.141&$<$ 0.001 $\ast$& 2.294& 0.169 & $<$ 0.001 $\ast$ & 2.829 & 0.126 & $<$ 0.001 $\ast$& \\
				& Exclosure &0.465 & 0.143 & 0.001 $\ast$ & 0.399 &0.167 & 0.016 $\ast$ & 0.108 & 2.263 & 0.023 $\ast$&\\
				& Edge distance &0.085 & 0.105 & 0.418 \color{white}*\color{black} & 0.332 & 0.118 & 0.005 $\ast$ & 0.027 & 0.076 & 0.726 \color{white}*\color{black} &\\
				& Relative basal area &0.418 & 0.297 & 0.160 \color{white}*\color{black} & -0.346 & 0.361 & 0.338 \color{white}*\color{black} & -0.081 & 0.213 & 0.702 \color{white}*\color{black} &\\
				& Edge dist : Exclosure &0.025 & 0.123 & 0.838 \color{white}*\color{black} & -0.371 & 0.132 & 0.005 $\ast$ & -0.015 & 0.078 & 0.853 \color{white}*\color{black} &\\
				& Rel. basal area : Exclosure &-0.267 & 0.312 & 0.392 \color{white}*\color{black} & 1.497 & 0.132 & $<$ 0.001 $\ast$ & -0.01 & 0.221 & 0.966 \color{white}*\color{black} &\\
				\cmidrule{3-11}
				\multirow{4}{*}{\rotatebox[origin=c]{90}{\parbox[c]{2.2cm}{\centering \textbf{Sapling growth}}}} & Intercept &0.018 & 0.007 & $<$ 0.001 $\ast$ & 0.013 & 0.014 & $<$ 0.001 $\ast$ & 0.058 & 0.013 & $<$ 0.001 $\ast$ &\\
				& Herbivory level &-0.001 & $<$ 0.001 & 0.031 $\ast$ & $<$ 0.001 & $<$ 0.001 & 0.360 \color{white}*\color{black} & -0.002 & $<$ 0.001 & 0.001 $\ast$&\\
				& Edge distance &-0.003 & 0.005 & 0.545 \color{white}*\color{black} & 0.013 & 0.009 & 0.122 \color{white}*\color{black} & -0.007 & 0.007 & 0.329 \color{white}*\color{black} &\\
				& Relative basal area &-0.004 & 0.016 & 0.805 \color{white}*\color{black} &  -0.007 & 0.027 & 0.787 \color{white}*\color{black} &  0.041 & 0.019 & 0.038 $\ast$&\\
				\cmidrule{1-11}
			\end{tabular}\endgroup
		\end{footnotesize}
	\end{center}
	%\end{sidewaystable}
\end{table}
\end{landscape}
\clearpage	


Herbivory levels on pedunculate oak increased by 18.4 \% when birds were excluded (figure \ref{fig5-3}, table \ref{tab5-1}, est. herbivory level exclosure: 26.9 \%, 95\%--CI: 12.2 -- 41.6 vs. control: 8.5 \% 95\%--CI: 3.5 -- 13.6) and were higher further from the forest edge (figure \ref{fig5-2}, table \ref{tab5-1}). Moreover, the exclosure treatment significantly interacted with both the distance from the forest edge and the proportion of pedunculate oak in the overstorey (table \ref{tab5-1}). Excluding birds largely cancelled out the increase in herbivory levels with distance from the forest edge (figure \ref{fig5-5}A, table \ref{tab5-1}). Excluding birds also led to higher herbivory levels when the proportion of pedunculate oak in the overstorey increased (figure \ref{fig5-5}B, table \ref{tab5-1}). The level of herbivory did not significantly affect the growth of pedunculate oak saplings (figure \ref{fig5-4}A, table \ref{tab5-1}, est. growth exclosure: -0.004, 95\%--CI: -0.051 -- 0.042 vs. control: 0.005, 95\%--CI: -0.043 -- 0.053).\\

Excluding birds from red oak saplings led to 5.7 \% more leaf area loss (figure \ref{fig5-3}, table \ref{tab5-1}, est. herbivory level exclosure: 26.5 \%, 95\%--CI: 14.0 -- 39.0 vs. control: 20.8 \%, 95\%--CI: 11.1 -- 30.4), which decreased relative sapling growth by 0.009 (figure \ref{fig5-4}A, table \ref{tab5-1}). The level of herbivory was not associated with edge distance or proportion of red oak in the overstorey (figure \ref{fig5-2}, table \ref{tab5-1}). However, sapling growth was positively and directly associated with the proportion of red oak in the overstorey (figure \ref{fig5-4}B), with an increase in monocultures of red oak (est. 0.068, 95\%--CI 0.016 -- 0.120) compared to stands where red oak was absent in the overstorey (est. 0.028 95\%--CI -0.023 -- 0.079).\\

%Figure 5.5
\begin{figure}[h!]
	\begin{center}
		\begin{subfigure}[c]{\textwidth}
			\includegraphics[width=\textwidth]{fig5-5A.png}
		\end{subfigure}
		\\
		\begin{subfigure}[c]{\textwidth}
			\includegraphics[width=\textwidth]{fig5-5B.png}
		\end{subfigure}
	\end{center}
	\caption{Relationship between herbivory levels (\% leaf area loss) and A) distance to the forest edge and B) relative basal area of conspecific trees for saplings of beech (\textit{Fagus sylvatica}), pedunculate oak (\textit{Quercus robur}) and red oak (\textit{Q. rubra}). Data points show raw data per sapling and the lines show model prediction with 95\%-confidence intervals (grey), assuming the average value for other explanatory variables. For pedunculate oak saplings herbivory levels significantly increased with A) edge distance outside the exclosure (full line) and B) with relative concentration of pedunculate oak in the overstorey inside the exclosure (full line). Other relationships were not significant (dashed line). \label{fig5-5}}
\end{figure}
\clearpage
	
	
	

	\section{Discussion}
	
	Saplings of all three study species showed higher herbivory levels within exclosures in accordance with prediction i, but only pedunculate oak, showed a larger effect of excluding avian predators closer to the forest edge (prediction ii) and if birds were excluded, higher herbivory levels under an overstorey with more conspecific trees (prediction iii). Higher herbivory levels only led to reduced sapling growth for red oak and beech (prediction iv), and red oak unexpectedly showed increased growth under an overstorey with a larger proportion of conspecific trees.\\
	
	Based on our experiments, resource concentration effects were only apparent in pedunculate oak saplings when birds were excluded. This result implies that the presence of natural enemies may hinder the detection of resource concentration effects in observational studies. Pedunculate oak hosts a high density and diversity of herbivores, among which many caterpillars that form a crucial prey source for insectivorous birds \citep{Naef-Daenzer2000}. Insectivorous birds have been shown to use leaf damage on oaks \citep{Gunnarsson2018, Heinrich1983} or herbivore-induced oak volatiles \citep{Amo2013} as cues to find such prey. Although insectivorous birds such as tits (Paridae) sample all trees in their territory, they preferably forage on the trees with the highest prey density \citep{Naef-Daenzer2000}. This could counteract the increased herbivory levels found on pedunculate oak saplings underneath pedunculate oak overstories, and explain why resource concentration effects on this species were only present in the absence of birds. The bird community could potentially also explain the increased herbivory on pedunculate oak saplings planted in monocultures compared to mixtures, which has been observed in young plantations (e.g. \citealt{Alalouni2014, Setiawan2014}), given that young forests do not have a similar bird community as mature forests (e.g. \citealt{Whytock2018}). Contrary to pedunculate oak, beech and red oak support low abundances of herbivores, which are most often generalists \citep{Branco2015, Brandle2001, Goßner2004}. As the latter do not show strong responses to resource concentration \citep{Castagneyrol2014, Jactel2005}, differences in diet niche breadth of associated herbivores may hence explain variation in effect sizes of overstorey composition on the regeneration potential of tree species.\\
	
	Herbivory levels on pedunculate oak saplings outside the exclosures increased further from the forest edge, but this effect was much weaker for saplings growing inside the exclosures. Together, these results indicate that top-down control by avian predators is stronger near forest edges, which conforms with results from earlier studies based on experiments with artificial caterpillars \citep{Barbaro2014, Gonzalez-Gomez2006}. In our study, this pattern was only apparent in pedunculate oak saplings, possibly driven by the preference for this tree species by insectivorous birds discussed earlier. Yet, similar edge effects were not found in an earlier study on mature pedunculate oak trees in the same study plots \citep{VanSchrojensteinLantman2018}. As resistance against herbivory starts to build up in saplings and reaches its highest level in mature trees \citep{Boege2005}, the ontogeny of resistance may explain why findings differ between both studies. In support of this, herbivory levels on mature trees were lower than on saplings for red oak (9.25 \% vs. 18.6 \%) and beech (2.67 \% vs. 10.4 \%), but not on pedunculate oak (12.9 \% vs. 11.3 \%; \citealt{VanSchrojensteinLantman2018}).\\
	
	Higher herbivory levels led to a reduced growth of beech and red oak saplings, but not of pedunculate oak. This could be due to the limited time span of our study, as negative effects of herbivory on tree growth are known to increase in subsequent growing seasons \citep{Marquis1994, Mooney2007}. Hence, it is possible that the effect could emerge for pedunculate oak in the subsequent growing seasons as well. Additional to the negative effects of herbivory on sapling growth, a higher proportion of red oak in the overstorey showed a direct positive relationship with the growth of red oak saplings, while such a link was not apparent in both other tree species. Our results indicate that this relationship is not mediated by differences in herbivory levels, and can thus not be explained by a reduced herbivore community on this tree species in its introduced range \citep{Branco2015, Goßner2004}. Alternatively, increased sapling growth under higher red oak cover could result from higher light transmission, which has earlier been shown to be a limiting factor for red oak growth \citep{Dey1996}. Another recent study in the same plots showed higher light transmission during leaf expansion in red oak overstories than in both other tree species, although this effect disappeared after leaves had fully flushed \citep{Sercu2017}. Leaf burst of tree saplings is often earlier than that in the overstorey, and light availability during this period can be critically important for sapling fitness \citep{Augspurger2005}. Other potential factors such as competitive vegetation, soil or litter conditions seem less likely to explain the observed sapling growth pattern in red oak, as plots with red oak overstories were characterized by a dense shrub layer of red oak saplings and a low litter quality \citep{DeGroote2017, DeGroote2018}.\\
	
	An often-cited mechanism facilitating non-native plant invasions is proposed by the enemy-release hypothesis (reviewed by \citealt{Liu2006}), which states that once introduced to a non-native region, plants experience a decrease in regulation by herbivores and other natural enemies. While it has been found that compared to native \textit{Quercus} species, red oak is avoided by generalist and specialist seed predators in Europe (e.g. \citealt{Myczko2014, Bogdziewicz2018}), we show that saplings of red oak suffer from herbivory at similarly high levels as native pedunculate oak (see also \citealt{Wein2016, VanSchrojensteinLantman2018}), and that this herbivory decreases sapling growth. Comparable information on herbivory pressure on red oak in the native range would be necessary to formally test for enemy release as an underlying driver of the spread of this species in European forests, yet our results suggest that native herbivores may be a factor slowing down the early stages of red oak invasions. However, we also observed that red oak sapling growth increased with an increasing abundance of red oak in the overstorey. Along the same lines, a recent study in Poland observed that the recruitment of red oak is higher in red oak monocultures than in mixtures with native sessile oak (Q\textit{. petraea}) \citep{Bogdziewicz2019}. These findings suggest a positive density dependence once red oak is established, which can accelerate invasion speeds once high densities have been reached \citep{Brooker2007}. 
	Guidelines for sustainable forest management emphasize the importance of natural regeneration \citep{DenOuden2010}. Compared to sapling plantations, natural regeneration has considerable economic and ecological advantages \citep{Burgi2003, Vranckx2014}, stressing the importance of optimizing sapling survival and growth under the canopy of managed forests. Our study indicates that insectivorous birds can control herbivory on saplings and we found direct cascading effects on sapling growth. These combined results hence provide direct empirical support for the important ecosystem services provided by birds \citep{Sekercioglu2012}, and plead for forest management strategies that promote a high functional diversity of insectivorous birds. This could be achieved by creating, or maintaining, gradual forest edges \citep{Melin2018, Terraube2016} and low canopy cover and by conserving large trees and deadwood in managed stands \citep{Bereczki2014, Penone2019}. Note, however, that insectivorous birds may also benefit red oak saplings and hence bolster their invasion potential. When insectivorous birds are absent or occur at low densities, a diverse overstorey may compensate for the lower top-down control by decreasing herbivory levels on saplings of tree species with a high number of associated specialist herbivores, such as pedunculate oak in our study.\\
	
	In conclusion, our study shows that herbivory on saplings in temperate forests can be influenced by top-down control --especially at forest edges-- and overstorey species composition with consequences for sapling growth and natural forest regeneration. More generally, our study provides insight into the complex, multitrophic interactions that drive sapling growth in forest stands located within fragmented landscapes.\\
	\clearpage

	\subsection*{Funding}
	
	Financial support for this research was provided via the UGent GOA (Concerted Research Actions) project  ``Scaling up Functional Biodiversity Research: from Individuals to Landscapes and Back (TREEWEB)''.

	\subsection*{Acknowledgements}
	
	We thank the private forest owners and the Flemish Forest and Nature Agency (ANB) for granting access to their property. We also thank Liesbeth De Neve for advice on the study design and Bram Sercu and Robbe De Beelde for help with the field work. Furthermore, we thank Margot Vanhellemont and two anonymous reviewers for useful comments on the first draft of this manuscript.

	\subsection*{Data Accessibility}
	Herbivory level and growth per sapling are available on GitHub: \url{https://doi.org/10.5281/zenodo.2616014}
	
	\subsection*{Author contributions}
	Luc Lens, Dries Bonte, Irene van Schrojenstein Lantman and Daan Dekeukeleire conceived the study; Luc Lens, Dries Bonte, Kris Verheyen and An Martel acquired funding; Daan Dekeukeleire, Irene van Schrojenstein Lantman and Pieter Vantieghem collected the data, Daan Dekeukeleire and Irene van Schrojenstein Lantman performed the analyses; Daan Dekeukeleire and Irene van Schrojenstein Lantman led the writing of the manuscript and all authors contributed significantly to the drafts.
	\clearpage
	
	\section{Supplementary material}
%table S5.1
%\begin{landscape}
	%\begin{sidewaystable}
	\begin{table}[h!]
		\begin{center}
			\begin{footnotesize}
				\caption*{\textbf{Table S5.1}: Summary of the survival of the different saplings species in the different overstorey species compositions. Note that we present the data here per tree species composition, while we use relative conspecific basal area in the analyses. Survival per sapling species is the proportion of saplings that survived since planting and before placing exclosures. During the course of the experiment 58 sapling died, which were excluded from the analyses. N per species is the number of saplings initially planted (6 in each plot). Abbreviations for the overstorey tree species composition: Qrob: \textit{Quercus robur}; Qrub: \textit{Q. rubra}, Fsyl: \textit{Fagus sylvatica}.}  \label{TabA1}
				
				\begingroup
				\setlength{\tabcolsep}{10pt} % Default value: 6pt
				\renewcommand{\arraystretch}{1.5} % Default value: 1
				\hspace*{-0.8cm}
				\begin{tabular}{p{0.2cm} p{1.8cm} >{\centering}m{0.8cm} >{\centering}m{0.8cm} >{\centering}m{0.8cm} >{\centering}m{0.8cm} >{\centering}m{0.8cm} >{\centering}m{0.8cm} >{\centering}m{0.8cm} p{0.001cm}}
					
					\cmidrule{1-9}
					& & \multicolumn{7}{c}{\textbf{Overstorey Tree species composition}} &\\
					& & \textbf{Qrob} & \textbf{Fsyl} & \textbf{Qrub} &  \textbf{Qrob-Fsyl} & \textbf{Qrob-Qrub} & \textbf{Fsyl-Qrub} & \textbf{Qrob-Fsyl-Qrub}&\\
					\cmidrule{1-9}
					& \textbf{N per species} & 48 & 48 & 48 & 48 & 36 & 48 & 42&\\
					\cmidrule{2-9}
					
					\multirow{3}{*}{\rotatebox[origin=c]{90}{\parbox[c]{1.5cm}{\centering \textbf{Sapling species}}}}
					& \textbf{\textit{Q. robur}} & 0.15 & 0.46& 0.33 & 0.02  & 0.11 & 0.15 & 0.12 &\\
					& \textbf{\textit{Q. rubra}} & 0.77 & 0.85 & 0.96 & 0.81 & 0.89 & 0.92 & 0.76&\\
					& \textit{\textbf{F. sylvatica}} & 0.71 & 0.52 & 0.81 & 0.40 & 0.64 & 0.77 & 0.33&\\
					\cmidrule{1-9}
					
				\end{tabular}\endgroup
			\end{footnotesize}
		\end{center}
		%\end{sidewaystable}
	\end{table}
%\end{landscape}
\clearpage	


%table S5.2
\clearpage
\begin{landscape}
\thispagestyle{empty}
%\begin{sidewaystable}
\begin{table}
	\begin{center}
		\begin{footnotesize}
			\caption*{\textbf{Table S5.2}: Summary of (a) the herbivory levels (mean leaf area loss based on ten leaves) and (b) height growth observed in the different overstorey species compositions of the different saplings species within the exclosures and of control saplings. Abbreviations for the overstorey tree species composition: Qrob: \textit{Quercus robur}; Qrub: \textit{Q. rubra}, Fsyl: \textit{Fagus sylvatica}. Note that we present the data here per tree species composition, while we use relative conspecific basal area in the analyses. Values given are mean ($\pm$ standard deviation). For sapling growth on \textit{Q. robur} some standard deviations could not be calculated.}
			
			\begingroup
			\setlength{\tabcolsep}{10pt} % Default value: 6pt
			\renewcommand{\arraystretch}{1.6} % Default value: 1
			\hspace*{-1.8cm}
			\begin{tabular}{p{0.2cm} p {0.2cm}p{1.9cm} >{\centering}m{1.9cm} >{\centering}m{1.9cm} >{\centering}m{1.9cm} >{\centering}m{1.9cm} >{\centering}m{1.9cm} >{\centering}m{1.9cm} >{\centering}m{1.9cm} p{0.001cm}}
				
				\cmidrule{1-10}
				& & & \multicolumn{7}{c}{\textbf{Overstorey Tree species composition}} &\\
				& & & \textbf{Qrob} & \textbf{Fsyl} & \textbf{Qrub} &  \textbf{Qrob-Fsyl} & \textbf{Qrob-Qrub} & \textbf{Fsyl-Qrub} & \textbf{Qrob-Fsyl-Qrub}&\\
				\cmidrule{1-10}
				\multicolumn{10}{l}{\textbf{a) Sapling herbivory levels}}& \\
				\multirow{6}{*}{\rotatebox[origin=c]{90}{\parbox[c]{3cm}{\centering \textbf{Sapling species}}}} & \multirow{2}{*}{\rotatebox[origin=c]{90}{\parbox[c]{1cm}{\centering \textbf{Qrob}}}} & \textbf{exclosure} &48.33 $\pm$ 20.21 & 13.06 $\pm$ 10.71 & 22.65 $\pm$ 13.60 & - & 15.60 $\pm$ 9.33 & 25.43 $\pm$ 10.46 & 9.4 $\pm$ 0.42 &\\
				& & \textbf{control} &6.95 $\pm$ 2.05 & 6.25 $\pm$ 3.12 & 15.05 $\pm$ 8.81 & - & 18.20 $\pm$ 4.24 & - & 10.53 $\pm$ 0.85&\\
				& \multirow{2}{*}{\rotatebox[origin=c]{90}{\parbox[c]{1cm}{\centering \textbf{Qrub}}}} & \textbf{exclosure} & 19.62 $\pm$ 10.08 & 15.45 $\pm$ 13.69 & 16.05 $\pm$ 12.48 & 28.55 $\pm$ 19.92 & 25.15 $\pm$ 10.92 & 31.24 $\pm$ 19.91 & 11.94 $\pm$ 7.76 &\\
				& & \textbf{control} &18.39 $\pm$ 14.53 & 15.27 $\pm$ 15.56 & 15.84 $\pm$ 12.85 & 16.66 $\pm$ 12.88 & 13.01 $\pm$ 9.03 & 24.79 $\pm$ 15.70 & 18.95 $\pm$ 10.23 &\\
				& \multirow{2}{*}{\rotatebox[origin=c]{90}{\parbox[c]{1cm}{\centering \textbf{Fsyl}}}} & \textbf{exclosure} & 8.41
				$\pm$ 6.67 & 12.02 $\pm$ 11.66 & 11.22 $\pm$ 7.38 & 15.07 $\pm$ 10.63 & 16.66 $\pm$ 9.68 & 14.96 $\pm$ 7.86 & 11.92 $\pm$ 6.98 &\\
				& & \textbf{control} &7.55 $\pm$ 7.96 & 12.05 $\pm$ 12.29 & 7.05 $\pm$ 6.61 & 7.74 $\pm$ 6.59 & 10.95 $\pm$ 7.91 & 10.91 $\pm$ 13.48 & 6.34 $\pm$ 5.94 &\\
				\cmidrule{1-10}
				\multicolumn{10}{l}{\textbf{b) Sapling growth}}& \\
				\multirow{6}{*}{\rotatebox[origin=c]{90}{\parbox[c]{3cm}{\centering \textbf{Sapling species}}}} & \multirow{2}{*}{\rotatebox[origin=c]{90}{\parbox[c]{1cm}{\centering \textbf{Qrob}}}} & \textbf{exclosure} & 0.004 $\pm$ 0.015 & -0.021 $\pm$ 0.055 & 0.025 $\pm$ 0.048 & - & -0.044 $\pm$ 0.113 & 0.001 $\pm$ 0.044 & -0.023 &\\
				& & \textbf{control} &-0.018 $\pm$ 0.038 & 0.019 $\pm$ 0.023 & 0.044 $\pm$ 0.043 & - & -0.087 $\pm$ 0.123 & 0.009 & 0.019 &\\
				& \multirow{2}{*}{\rotatebox[origin=c]{90}{\parbox[c]{1cm}{\centering \textbf{Qrub}}}} & \textbf{exclosure} & 0.040
				$\pm$ 0.079 & 0.041 $\pm$ 0.070 & 0.089 $\pm$ 0.058 & 0.008	$\pm$ 0.129 & 0.050 $\pm$ 0.064 & 0.059 $\pm$ 0.046 & 0.089 $\pm$ 0.058 &\\
				& & \textbf{control} &0.032 $\pm$ 0.107 & 0.053 $\pm$ 0.121 & 0.068	$\pm$ 0.075 & 0.036	$\pm$ 0.080 & 0.030	$\pm$ 0.146 & 0.015	$\pm$ 0.153 & 0.068	$\pm$ 0.075	&\\
				& \multirow{2}{*}{\rotatebox[origin=c]{90}{\parbox[c]{1cm}{\centering \textbf{Fsyl}}}} & \textbf{exclosure} & 0.068 $\pm$ 0.087 & 0.020	$\pm$ 0.065 & 0.031	$\pm$ 0.109 & 0.004	$\pm$ 0.018 & -0.013 $\pm$ 0.036 & 0.004 $\pm$ 0.077 & 0.008 $\pm$ 0.031 &\\
				& & \textbf{control} &0.008	$\pm$ 0.080 & 0.007 $\pm$ 0.084 & -0.024 $\pm$ 0.163 & 0.001 $\pm$ 0.139 & 0.062 $\pm$ 0.143 & -0.008 $\pm$ 0.070 & 0.041 $\pm$ 0.024	&\\
				\cmidrule{1-10}
				
			\end{tabular}\endgroup
		\end{footnotesize}
	\end{center}
	%\end{sidewaystable}
\end{table}
\end{landscape}


\clearpage
\thispagestyle{plain}
\hbox{}
\clearpage

	%%%%%%%%%%%%%%%%%%%%%%%%%%%%%%%%%%%%% Chapter 6 %%%%%%%%%%%%%%%%%%%%%%%%%%%%%%%%%%%%%%%%%	

 % empty 
\CenterWallPaper{1}{CH6.jpg}
\newpage{\thispagestyle{empty}\cleardoublepage}
\ClearWallPaper
\pagestyle{mainmatter}
\chapter{General discussion and conclusion}
\label{discussion}
\chaptermark{General discussion and conclusion}
\lettergroup{\thechapter}

\begin{flushright} \color{gray}Daan Dekeukeleire\color{black}\end{flushright}

	\vspace*{\fill}
\noindent \color{gray} $\lhd$ Breeding great tit in one of the nest boxes. Photo by Pieter Dierckx.

\color{black}
\newpage

Land use change increasingly leads to the loss of forest habitats globally \citep{Hansen2013}. The remaining forest cover is highly fragmented and the tree species compositions are often highly simplified, threatening biodiversity and the deliverance of ecosystem services \citep{Cardinale2012, Haddad2015}. Due to their high trophic level, insectivorous birds are ideal sentinel species in forest fragmentation research. In this dissertation I aimed to assess how forest fragmentation and local resource availability, shaped by tree species composition and diversity, affect the fitness and functioning of great tits and blue tits across their full annual cycle in a highly human-modified landscape in Northern Belgium.\\

First, I focused on the effects of forest fragmentation and resource availability during spring. In chapter \ref{chapter2}, I show that caterpillar abundance, the main food source for tits, is shaped by local tree species composition, and influences breeding success. Structural Equation Modelling revealed diverse and species-specific pathways. For great tits, tree composition effects on breeding performance were driven by caterpillar frass, and thus by food resource availability. For blue tits, these effects were driven by variation in clutch size, which suggests the presence of seasonal carry-over effects. Fragmentation effects were only observed in resource-poor beech monocultures, with breeding performance declining in accordance with reduction in forest fragment size.\\

Second, I focused on how environmental conditions during winter affect winter behaviour, and how behavioural effects during winter are linked to breeding performance during spring. In chapter \ref{chapter3}, I show that resource availability in winter, and to a lesser extent forest fragment size, influence winter habitat use and movement behaviour of great tits. When food resources were abundantly present in forests in a mast year of beech, individuals preferred to forage in forests, and avoided other habitat types. Individuals residing in smaller forests flew larger distances to forage in neighbouring forest fragments compared to those residing in large fragments. Foraging outside the forest in winter had a clear reproductive cost in spring, i.e. a lower fledging success, but was not related to phenotypic stress markers, i.e. stress hormone levels or body condition.\\

Third, I focused on the effect which stress --specifically stress hormone levels-- has on social behaviour and social interactions. In chapter \ref{chapter4}, I show that levels of corticosterone, the main avian stress-hormone, correlate with social behaviour in winter. Individuals with higher corticosterone levels in feathers grown during late summer associated with more individuals during winter, and thus occupied more central positions in the winter social network. This could in turn lead to reduced stress levels, through increased access to social information on food resources, with possible carry-over effects to the subsequent breeding season.\\

Finally, I investigated how forest fragmentation and tree species composition affect avian functioning. In chapter \ref{chapter5}, I show that avian top-down control decreases herbivory on young trees and --in the case of two of the three investigated tree species-- increased growth. On pedunculate oak saplings, top-down control was stronger closer to the forest edge. Effects of tree species composition on herbivory levels were observed in pedunculate oak, with herbivory increasing with the amount of conspecifics in the overstorey, but only in the absence of birds.\\

Below, I discuss and link the results of the research described in the these chapters, and indicate directions for future research. Specifically, I discuss carry-over effects, social interactions and movement behaviour, and multi-trophic interactions in fragmented forests. Finally, I discuss implications of my research for forest management and biodiversity conservation.\\

	\section{Stress levels, carry-over effects and the importance of full annual cycle research}
	
The results presented in this dissertation stress the importance of full annual data for a comprehensive understanding of the dynamics of populations: information that can be vital for conservation actions and for predicting how populations might respond to global change \citep{Marra2015}. In birds, seasonal carry-over effects have mostly been investigated in migratory bird species \citep{Harrison2011, OConnor2014, Marra2015}, but my data demonstrate that effects during winter are important for resident species as well (see also \citealt{Williams2015}). In tits, carry-over effects on breeding performance have previously been investigated mainly on population level (e.g. \citealt{Robb2008a, Plummer2013a, Plummer2013}; but see \citealt{Crates2016}). My data demonstrates the importance of individual-level data in investigating carry-over effects. However, although I observed a clear carry-over effect of winter habitat use on the spring breeding performance, it remains unclear which mechanisms cause such carry-over effects.\\

Carry-over effects could be caused by increased stress hormone levels (mainly corticosterone; CORT) in response to abiotic or biotic stressors in different seasons. In winter, numerous studies indicate that the main factor impacting stress in small passerines is food availability \citep{Jansson1981, Brittingham1988, Perdeck2000}. However, elevated CORT levels have also been linked to other factors in passerines which could additionally be important in winter, such as cold temperatures, which increases energetic needs \citep{Cirule2017}, or increased predation risk \citep{Scheuerlein2001, Clinchy2011}. The main factor that could cause variation in stress hormone levels during spring and summer is likely reproductive investment. In birds, CORT levels are commonly upregulated during breeding, and correlational evidence suggests this may prepare both sexes for increased energetic needs during provisioning \citep{Love2014, Ouyang2013a}. For example, great tits with large clutch sizes and individuals that feed their nestlings at high rates have higher baseline CORT concentrations than those with small broods and low feeding rates \citep{Ouyang2013}. Such elevations in CORT can be higher when conditions for rearing young are more difficult. For instance, blue tits have higher CORT levels during adverse weather conditions in the breeding season and when breeding in low quality habitat \citep{Henderson2017}. In the context of global change, it would be very interesting to investigate how increasing temperatures in both winter and spring influence stress levels in birds. Global warming decreased the year-to-year variability of seed production and the reproductive synchrony among beech trees (and likely other tree species as well), which benefits seeds predators \citep{Bogdziewicz2020}. Together with milder winter temperatures, this increased stability in food availability could potentially lead to lower stress levels for resident birds in winter. During the spring breeding season, insectivorous birds such as great and blue tit time their breeding so that the peak of nestling food demand corresponds to the ephemeral peak in caterpillar abundance. Due to rising temperatures, trees leaf-out earlier and caterpillars develop faster, leading to an earlier peak abundance of caterpillars. Birds advance their breeding as well, but at a slower rate than caterpillars, leading to a phenological mismatch \citep{Visser1998, Sanz2002}, and therefore to higher energy demands during nestling provisioning \citep{TeMarvelde2011}, which could increase chronic stress levels during breeding and post-breeding \citep{Henderson2017}. \\

Chronically high levels of such hormones negatively impact body condition, immunity and physiology (e.g. \citealt{Gao2017}). Although I did not find a relationship between feather CORT levels in feathers grown during winter and subsequent breeding success or body condition (chapter \ref{chapter3}), feather CORT levels have previously been shown to predict subsequent body condition, breeding success and survival in small insectivorous birds (e.g. \citealt{Koren2012, Harms2015, Boves2016, Monclus2020}). Note that feather CORT levels can only be interpreted in relative terms, i.e. within a population. The feathers investigated in chapters \ref{chapter3} and \ref{chapter4} reflect the conditions in early winter. Possibly, the conditions during late winter are more important than those during early winter (e.g. as shown for farmland granivorous birds; \citealt{Siriwardena2008}). Alternatively, or additionally, in my study design, food was abundantly available at feeders from the second half of December on, which could potentially allow chronically stressed individuals to survive as well. The role of long-term alleviation of stress levels, as measured by CORT stored in feathers, in causing carry-over effect thus deserve further research attention. Stress hormones have previously also been linked to habitat characteristics in forest-dependent passerines, such as forest structure \citep{Suorsa2003a} and tree species composition \citep{Cirule2017, Henderson2017}, but due to logistical and financial constraints, I only measured feather CORT levels in a single forest fragment. In this context, it would be highly interesting to further investigate variation in year-round feather CORT levels, or other phenotypic stress makers, across a gradient of tree species composition and forest quality, and relate this to variation in subsequent life-history.\\

A second possible mechanism underlying carry-over effects between winter and spring could be variation in nutritional condition. Small passerines such as tits cannot store fat to any great extent \citep{Drent1980}, but micronutrients such as vitamins and carotenoids can accumulate in subcutaneous fat and liver tissues, and can be mobilised when demand is high during breeding \citep{Plummer2013a}. Previous studies indicate that an increased carotenoid and vitamin intake resulting from supplemented food during egg production in spring, increases investment in eggs, hatching success and nestling immunocompetence \citep{Saino2003, Møller2008}. Note that in my study area, blue tits in resource rich plots were able to invest in a bigger clutch size and were therefore likely in better condition (see chapter \ref{chapter2}). In order to ascertain the potential role of nutritional conditions in winter on subsequent fitness, controlled experiments investigate effects of different diets on condition and phenotypic stress markers in aviaries, for instance as conducted by \citet{SallehHudin2016}. This could prove to be very interesting.\\


	\section{Social interactions and movement behaviour in fragmented landscapes}
	
Determining the ecological drivers which underlie social structure and social interactions in animal populations is of key importance to understanding processes such as social information transfer and pathogen spread \citep{Farine2015, Croft2016}. In chapter \ref{chapter4}, I show that stress hormones such as corticosterone (CORT) likely play an important role in mediating social interactions. Several approaches could further help understand social interactions and their fitness consequences for birds living in fragmented landscapes. First, in order to confirm the causal role of stress hormones, it would be very interesting to experimentally increase energetic costs or stress levels (e.g. through handicap experiments; \citealt{Johns2018}), or to experimentally directly increase hormone levels (e.g. through CORT implants; \citealt{Lattin2011}), and to subsequently follow social interactions, as well as social information transfer and pathogen transfer. Such an approach would also allow to determine the extent to which network positions represent static or plastic behavioural strategies in response to environmental stressors \citep{StClair2015, Aplin2015a}.\\
 
Second, it would be highly interesting to investigate the role which forest configuration and fragmentation context play in shaping social networks. In chapter \ref{chapter3}, I show that forest configuration impacts the decisions individuals make about when and where to move. Along these lines, previous studies have demonstrated that vegetation structures such as treelines, shrub cover and hedgerows influence the movement of tits, both inside \citep{Farine2016} and outside forests \citep{Cox2016}. Consequently, physical features of the environment can impact which other individuals an individual encounters more or less frequently. Currently, the role of habitat configuration in shaping the social system of group-living animals remains largely overlooked \citep{He2019}. Observations during radio tracking (chapter \ref{chapter3}) indicate that within a flock, certain members did not cross forest gaps, while others did. It could thus be expected that forest fragmentation leads to increased fission in flocks, which in turn would impact social information use and epidemiology. Such fission-fusion dynamics in bird flocks could result from active decision-making processes, but are more generally a result of individuals following social interaction rules, such as social attraction, alignment and repulsion \citep{Couzin2003, Silk2014}. Decision-making processes and social interaction rules vary depending on individual personality \citep{Couzin2003} and on the social and ecological context which individuals experience \citep{Hoare2004}. More individual level data across multiple years and across different environmental contexts would be needed to test the hypothesis that forest fragmentation increases fission in winter tit flocks, and that such fission-fusion dynamics can allow individuals to avoid majority decisions that are not in their favour, as shown in mammal species living in tightly connected social groups \citep{Kerth2006, Stueckle2008}.\\

In this context, the use of proximity tags would be highly interesting. In chapter \ref{chapter4}, the social network was constructed using visits from PIT-tagged individuals at feeding stations equipped with RFID-antennas, which is the most often used method in song birds (e.g. \citealt{Boogert2014, Aplin2015, Farine2015b, Moyers2018}). Yet, this method does not provide any information on social associations when birds are not foraging at artificial feeders with RFID-antennas. Such information could be obtained using recently developed light-weight proximity loggers ($<$ 2 g), which are able to monitor individual associations and flight trajectories at an unpreceded spatial and temporal resolution, even in closed forests \citep{Ripperger2016, Ripperger2020}. 


	\section{Multi-trophic interactions in fragmented forests}
	
The question of how herbivores are controlled is one of most debated in community ecology \citep{Pace1999, Schmitz2000, Borer2005}. Many studies support the hypothesis that predators control herbivore numbers, and that this has cascading effects on plant growth (the green world hypothesis; \citealt{Hairston1960}), most notably for terrestrial herbivorous arthropods and their predators (see \citealt{Vidal2018} for a recent meta-analysis). Yet, which ecological mechanisms influence such trophic cascades remainsl less understood, especially in complex real-world food webs \citep{Schmitz2000, Mantyla2011, Vidal2018}. My study (chapter \ref{chapter5}) is among the first to simultaneously consider the effects on trophic cascades of both local tree species composition and fragmentation context. My results demonstrate the importance of predators, as avian top-down control decreased herbivory in all investigated tree species, and increased growth in two of these (see also \citealt{Mantyla2011}). Effects of conspecific trees in the overstorey on sapling herbivory (and thus resource concentration, \textit{sensu} \citealt{Root1973}) were only observed in the absence of predators on one of the three tree species, i.e. pedunculate oak. This pattern is likely caused by the higher diversity and abundance of host-specific herbivores associated with pedunculate oak \citep{Kennedy1984}. On pedunculate oak, top down control was higher near forest edges, possibly due to an increased avian (functional) diversity or abundance at forest edges (\citealt{Barbaro2014, Terraube2016}; but see \citealt{Chan2020}).\\

In order to better understand the mechanisms shaping top-down control and to predict how this changes across different landscapes and environmental contexts, it would be very interesting to investigate the underlying individual level effects. In addition to changes in predator community, edge effects and forest fragmentation can also shape predatory impact through individual variation in a predator's activity level, space use or foraging efficiency \citep{Naef-Daenzer2000, Bueno-Enciso2016, Jarrett2020}. High resolution DNA meta-barcoding now makes it possible to monitor the prey choice of insectivorous birds such as tits in great depth \citep{Jedlicka2017, Rytkonen2019, Jarrett2020, Shutt2020}. Moreover, by combining markers for both arthropods and plants, one can detect both the arthropod prey species and the plant species on which these prey have fed \citep{Silva2019}. Such dietary variation could then be linked to variation in individual condition and environmental factors.\\

Future studies should also broaden the scope of predators investigated \citep{Vidal2018}. In my study (chapter \ref{chapter5}), I did not collect data on the identity of the insectivorous predators or of the herbivores, apart from an anecdotal observation of a great tit foraging on caterpillars on experimental saplings. Although my study took place during the peak breeding season of forest-dependent birds, and thus during their peak activity period, other insectivores could potentially also play a role in top-down control in temperate forests, especially later in the season. First, invertebrate predators, such as spiders, carabid beetles and hymenopterans, deserve more attention. Earlier exclosure experiments have demonstrated that birds can predate on such invertebrate predators, and such intraguild predation can have indirect consequences for herbivore populations \citep{Polis1992}. For instance, recent studies in grasslands and agricultural systems show that bird predation on spiders can decrease total herbivore control even if birds predate on herbivores too \citep{Bosc2018, Grass2017}. Studying the extent to which such processes also occur in temperate zone forests, and what effects --if any-- landscape context has on this, would be necessary to better understand trophic cascades and the relative importance of vertebrate and invertebrate insectivores.\\

Second, other vertebrate insectivores, specifically bats, could also play an important role in top-down control of arthropods in temperate forests (reviewed in \citealt{Maas2016} for tropical systems). To avoid day-time predators (such as birds or hymenopterans), many temperate zone herbivorous species of caterpillar and other arthropods restrict foraging to night-time, and spend the daytime hidden (reviewed in \citealt{Heinrich1979}). Dietary analyses indicate that gleaning bat species, such as brown long-eared bat, Natterer's bat and Bechstein's bat in Europe, feed upon such prey \citep{Wolz2013, Hope2014}, but due to a lack of experimental work, the functional role of bats in temperate zone forests remains unclear \citep{Russo2016}. Contrastingly, in tropical rainforests and agroforestry systems, multiple field experiments using night-time and daytime exclosure treatments clearly show that bats control herbivorous arthropods and cascading effects benefit plants \citep{Kalka2008, Williams-Guillen2008, Maas2013}. In temperate zone forests, birds, bats and invertebrate predators are differently affected by forest composition and landscape-level factors (e.g. \citealt{Penone2019}), thus the role of different insectivorous groups can depend on both local and landscape-level factors \citep{Maas2016}.\\

Hence, in order to better understand mechanisms that influence trophic cascades, field experiments are needed that compare the effects of excluding day-time predators, night-time predators or both, if possible replicated across a gradient of forest fragmentation. It would be highly interesting to monitor how such treatments affect herbivory, and to simultaneously  also investigate changes in invertebrate predator numbers and the cascading effects on plant fitness. Such studies would also provide critical insight in management actions needed to promote biological suppression of pest arthropods in forest ecosystems as well as in surrounding agricultural systems.\\


	\section{Management implications}
	
	Based on the results presented in this dissertation, several management actions can be proposed:\\
	\subsection*{(a) Promoting tree species mixtures}
	
	My work is part of a growing number of studies that demonstrate the importance of tree species diversity and composition in determining multiple ecosystem functions in forest systems (e.g. \citealt{Gamfeldt2013, VanderPlas2016, Ratcliffe2017, Baeten2019, Hertzog2019}). Research shows that there is no evidence for a super-species providing high levels of functioning and diversity across all contexts \citep{VanderPlas2016}. Although individual functions can reach the highest level in certain tree species monocultures, due to trade-offs, these will have low values for certain other functions \citep{Gamfeldt2013, VanderPlas2016, Hertzog2020}. Therefore, tree species mixtures lead to intermediate levels across these different functions, and therefore outperform monocultures when multiple functions are considered (`jack of all trades'-effect; \citealt{VanderPlas2016}).\\
	
	In this work, I studied the winter habitat use (chapter \ref{chapter3}) and reproductive success (chapter \ref{chapter2}) of tits, and the avian top down control on herbivorous arthropods (chapter \ref{chapter5}). Reproductive performance was highest in pedunculate oak monocultures, either through improved fledging success (great tit) or due to increased clutch sizes (blue tit), and was lowest in beech monocultures (and red oak in the case of great tits). Frass data demonstrates that these effects are likely caused by the higher amount of food resources linked to pedunculate oak in the form of caterpillars. At the same time, beech mast is an important winter food source for resident forest-dependent birds, including great and blue tits \citep{Perdeck2000, Chamberlain2007}. The availability of beech mast has long been linked to winter survival in resident forest-dependent bird species \citep{Kallander1981, Perdeck2000, Bouwhuis2015}, but as I outline in chapter \ref{chapter3}, high beech mast availability can also be advantageous for great tits --and likely for other forest-dependent resident birds-- by allowing them to avoid potential risks and costs associated with foraging outside of the forest in winter. This results in a better reproductive performance in the following spring.\\
	
	These individual functions, i.e. spring resource availability and winter resource availability, thus reach the highest levels in monocultures, but these monocultures have low values for other functions. Together, the results from chapters 2 and 3 indicate that mixtures of tree species that deliver high levels of different functions (instead of a higher tree species diversity per se) create the most optimal habitat for great and blue tits, and likely for other bird species that similarly depend on arthropods during spring and summer and on tree seeds during autumn and winter. More specifically, my data indicates that mixtures of both arthropod-rich tree species --e.g. pedunculate oak, sycamore or birch \citep{Shutt2018}-- and tree species that provide winter food --e.g. beech or hazel-- would be optimal. Nevertheless,  currently data is lacking on the relative contribution of spring food and winter food to recruitment. I could be possible that the benefits in term of food abundance of a pedunculate oak monoculture during spring outweigh the benefits of winter food availability in mast years. Note that also the bird species richness did not significantly differ between these two tree species compositions (see Box B, page \pageref{boxb}). Previous studies have linked fledgling body condition and fledging success to the amount of oaks around a nest box \citep{Wilkin2007a, Shutt2018}. Yet, in my study system, fledgling  body condition and fledging success did not statistically differ between pedunculate oak monocultures and mixtures of pedunculate oak and beech. This indicates that the occurrence of beech in the territory of tits does not negatively influence the breeding success, as long as oaks are present. Results from chapter \ref{chapter5} show that promoting such tree species mixtures could also decrease herbivory and therefore increase natural tree regeneration (as also shown by \citealt{Alalouni2014, Setiawan2014} in young plantations). \\
	
	Recently, within the field of biodiversity-ecosystem functioning research, it has been argued that in order to be practically relevant, ecosystem functions should be linked to specific value-based perspectives that can depend on different stakeholder perspectives \citep{Slade2017, Manning2018, Hertzog2019}. For instance, a higher abundance of herbivorous insects can be regarded as positive from a conservationist perspective, as in chapter \ref{chapter2}, or as a negative component when forest regeneration is pursued, as in chapter \ref{chapter5}. Interestingly, a recent study within the Treeweb framework indicates that promoting mixed forest stands promotes both ecosystems functioning and biodiversity for both conservationist and economical perspectives \citep{Hertzog2020}.\\
	
	These guidelines to establish mixed tree species stands are most relevant and applicable for forests that are managed in order to deliver multiple ecosystem services, which form the largest part of the total forest cover in Europe \citep{Brockerhoff2017, Sabatini2018}. The promotion of tree species mixtures in managed forests does not negate the need to increasingly protect forest areas where natural processes can take place (e.g. development of veteran trees, coarse woody debris, canopy gaps, and natural tree regeneration) so as to conserve forest-dependent biodiversity.\\
	
	
	\subsection*{(b) Management of red oaks}
	The results of my research are also relevant to the debate on the management of invasive tree species. One of the tree species investigated in this dissertation was the red oak, a species native to Eastern North America. Red oak was introduced in Europe for timber production and as an ornamental tree in parks, and currently covers over 350 000 ha in the Europe \citep{Nicolescu2020}. However, red oaks can be invasive, especially on sandy soil, and can negatively affect forest biodiversity \citep{Goßner2004, Campagnaro2018, Chmura2013, VanSchrojensteinLantman2020}. Along these lines, it was not unexpected that I observed a lower reproductive success of great tits breeding in red oak monocultures (see chapter \ref{chapter2}) and a lower bat and bird diversity (see Box B, page \pageref{boxb}). However, I did not observe such significant negative effects in the two or three-species mixtures containing red oak: this suggests that negative effects could be limited as long as red oaks do not reach dominance. Red oak sapling growth increased in accordance with an increasing abundance of red oak in the overstorey (see chapter \ref{chapter5}), which demonstrates that invasion speeds can accelerate with increasing densities. My results thus support management actions that eradicate red oaks or at the very least prevent red oak from reaching dominance. As red oak saplings in the understorey can quickly grow and come to dominate when gaps in the canopy are formed \citep{Dey1996}, management actions should aim for a gradual eradication in the case of mixed stands, e.g. through selective thinning.\\ 
	
	\subsection*{(c) Promoting landscapes with both large and mixed small forest fragments}
	Results from my study reveal both positive and negative effects of forest fragmentation. Top-down herbivore control, which leads to increased sapling growth, was stronger closer to forest edges (chapter \ref{chapter5}), which confirms results from earlier studies \citep{Gonzalez-Gomez2006, Barbaro2014}. Similarly, multiple other ecosystem functions and services measured in the Treeweb plots, such as soil carbon stocks and tree biomass, were higher near forest edges \citep{Hertzog2019}. Such results suggest a higher level of forest ecosystem functions and services in small fragments, compared to a similar area in large fragments. At the same time, I detected negative fragmentation effects on breeding performance of great tits, at least in beech monocultures (chapter \ref{chapter2}, also compare \citealt{Wilkin2007, Bueno-Enciso2016}), and greater distances covered outside forests in winter (and thus likely greater predation risk and energetic costs; \citealt{Lima1990, Hinsley2000}) for great tits living in small fragments (chapter \ref{chapter3}). Similarly, in the Treeweb project, the diversity of several arthropod groups decreased in plots with higher edge effects \citep{Hertzog2020}. Thus, while the level of multiple functions and the ecosystem service deliverance can be higher in more fragmented landscapes compared to landscapes with an equal but more continuous forest cover, this can be at the expense of forest biodiversity \citep{Hertzog2020, Valdes2020}.\\
	
	Together, these results suggest that a landscape with large forest fragments is optimal, but small fragments can also still be valuable, especially for provision ecosystem services (see also \citealt{ArroyoRodriguez2020, Valdes2020}). For biodiversity conservation, the buffering and connection of existing fragments should be prioritized. This would limit negative edge effects and would allow specialized species to (re)colonize forest fragments (e.g. ancient forest plants: \citealt{Honnay2002}; short-winged carabid beetles: \citealt{Desender1999}; mycorrhizal fungi: \citealt{Boeraeve2018}). Gradual edges and edges with low-intensity land use can also help to decrease negative effects on biodiversity \citep{Ries2004}. Managers should therefore maintain or create gradual forest edges and open patches of semi-natural grassland or heathland imbedded in forests (and thus promote `internal' forest edges). Such open spaces in forest landscapes also increase the recreational and cultural value of forest landscapes \citep{Tew2019}. At the same time, newly created small or isolated forest fragments (e.g. as proposed by the `one hectare forest initiative' in the framework of the EU's common agricultural policy) could be still be useful for the provision of certain ecosystem services, even if their value for biodiversity conservation is limited. As shown in chapter \ref{chapter2}, negative effects of forest fragmentation depended on the local tree species composition, and were only present in monocultures of resource-poor tree species. This demonstrates that diversifying stand is especially important in small forest fragments \citep{Hertzog2019}.\\
	
	\subsection*{(d) Winter bird feeding in human-dominated landscapes}
	The research presented in this dissertation demonstrates that even common forest-dependent birds, such as great and blue tit, suffer from landscape transformation, especially when resource availability in the remaining forests is low (chapters \ref{chapter2} and \ref{chapter3}). In chapter \ref{chapter3}, I great tits foraging outside the forest in winter and feeding on supplementary food in residential gardens, had a lower breeding success than the individuals that stayed in the forest to a greater extent. Although feeding birds in winter is increasingly popular and often is promoted by nature conservation NGOs (e.g. \url{www.natuurpunt.be/het-grote-vogelweekend}), the ecological consequences are not well understood \citep{Robb2008, Jones2011, Cox2018}. Yet, my results do not necessarily indicate that feeding forest-dependent birds in (sub)-urban environments in winter is harmful. It remains unclear whether visiting feeders outside the forest actually leads to lower success, or whether it revealed variation among individuals that is reflected in both feeder use and breeding success. Moreover, feeding birds is an important tool to increase the engagement of people, especially children, with wildlife and biodiversity \citep{Robb2008a}. Surveys indicate that feeding wild birds leads to an increased connection with biodiversity, which in turn increases people's willingness to conserve biodiversity at large \citep{Cox2018, White2018}. Dissuading winter bird feeding therefore does not seem appropriate, but previous studies indicate that campaigns which advocate winter bird feeding should stimulate the use of high quality food and simultaneously promote the planting of native trees and shrubs in (sub)urban environments. First, such actions could improve connectivity and limit predation risk, as earlier work indicates that tits prefer to use vegetation corridors (e.g. tree lanes, hedgerows) to move in (sub)urban environments \citep{Cox2016}. Second, native trees and shrubs would help enhance arthropod populations and therefore increase the supply of high-quality food, both during the breeding season \citep{Seress2020} as well as during winter.
	
\clearpage	
	\section{Conclusion}

In this dissertation, I demonstrate that landscape level factors --forest-fragmentation-- and local factors --resource availability, shaped by tree species composition-- jointly affect the fitness of two common forest-dependent insectivorous birds in a highly human-modified landscape. My data shows that tree species-linked food resource availability can affect breeding performance of tits, both directly through arthropod availability during spring, and indirectly through carry-over effects from winter habitat use, linked to food availability in winter. Fragmentation effects on breeding success were only observed in resource poor forests fragments. Moreover, both forest fragmentation and tree species composition affected avian functioning, as top down control of herbivorous arthropods and its cascading effects on plant growth were influenced by forest edges and by tree species composition. My study indicates that diversifying forest stands, especially in small forest fragments, represents a management strategy that promotes both avian fitness and avian ecosystem functioning in fragmented landscapes.\\

                        
\clearpage
%%%%% DANKWOORD %%%%
	
	\chapter*{Acknowledgments -- Dankwoord}
	\pagestyle{mainmatter}
	\chaptermark{Acknowledgments -- Dankwoord}
	\addcontentsline{toc}{chapter}{Acknowledgments -- Dankwoord}
	\label{Acknowledgments}
	
Hoewel mijn naam alleen staat op deze doctoraatsthesis zou die er niet liggen zonder de hulp en steun van v\'{e}\'{e}l mensen.\\

Eerst en vooral wil ik mijn promotor en co-promotor bedanken om me te kans te geven dit onderzoek uit te voeren. Luc, ongelooflijk bedankt voor je enthousiaste en persoonlijke begeleiding, voor al je feedback, inspiratie en motivatie. Je optimisme gaf me altijd energie om ervoor te gaan. Diederik, merci voor alle hulp, expertise en tips, maar ook voor al je humor en je energie. Ik ben ongelooflijk blij dat je bij dit project kwam, en me zo hard hielp. \textit{I also want to thank the member of my PhD jury, professor Kirsty Park, professor Erik Matthysen, professor Lander Baeten and Dr. Lieze Rouffaer, for critically and comprehensively assessing my dissertation and providing very helpful comments, that improved the manuscript. Thank you all!} Dries, ik vond het een eer jou als voorzitter van jury te hebben, en super hard bedankt voor alle raad bij mijn onderzoek, en om, samen met Luc, voor zo'n goede sfeer te zorgen op de Terec.\\

Verder ook een grote merci ook aan de vele collega's, (ex-)collega's en vrienden van op de Terec, Eon en Limno. De koffiepauzes op het 10de of in de vos, de wekelijkse pizza lunch in de prima donna, de middagpauzes in de plantentuin of de after-work drinks waren niet alleen ongelooflijk plezant, maar het was ook inspirerend en motiverend om jullie als collega's te hebben. Bedankt Ruben, Katrien, Frederik, Thomas, Bram D, Jasmijn, Johan, Lynda, Pei, Lore, Thijs, Jorunn, Viki, Stefano, Liliana, Matthew, Maurice en iedereen anders! \textit{Svana, thank you for your friendship, for your hospitality in Spain and for our weekly coffee skype-chats these last months. I'm looking forward to your defense (it'll be great), and I hope we can soon visit you again and go bird watching together in Urdaibai!} \textit{Laurence and Alex, thanks for making our trip to EOU congres in Cluj so nice, for the fun coffee breaks (in real life or online), and in the case of Alex, the many debates, and in case of Laurence for being such a cool office mate!} Hans, merci voor vogel- en natuurgesprekken, voor alle flat whites, espresso's en pintjes samen, en de hulp in t veld. Steven, bedankt voor de fijne gesprekken over hommels, vleermuizen, radio tracking en fysiologie, en de gezelschapspelletjes met pintjes. Karen, merci om zo'n fijn team te vormen om samen de groepjes studenten te begeleiden op de stages. \textit{Jonathan, thanks for the discussions on science and Italian cuisine. Jiao, it was really nice to be in the same boat and talk to you during the weeks before our submission deadline. You'll do great on your defense!} Martijn, om ook altijd te helpen met tips rond statistiek, schrijven en wetenschap, en de motiverende babbels over je onderzoek, je reizen en natuur. En merci ook om de excursies zo goed in elkaar te steken (met voorsprong het leukste lesgeven). Merci ook aan Angelica om het labo recht te houden, en altijd met allles te helpen (en voor zo vaak koffie te zetten!). Ten slotte wil ik ook de dames van de broodjesbar in de ledeganck bedanken, voor me altijd een extra stukje mozerella of een extra dikke laag veggie-spread op mijn broodjes te smeren, en voor altijd een bionade voor me apart te houden.\\

Ik wil ook heel hard Mike, Iris, Sanne Vdb, Eva, Sanne G, Stephanie S, Margot, Pieter VGB, Pieter S, Haben, Pallieter en de andere mensen van het Fornalab bedanken! Altijd ongelooflijk fijn om bij jullie in Gontrode te gast te zijn tussen mijn veldwerk door, en er een koffie of drie, een zakje oxfam-chips en een babbel mee te pikken. En ook een speciale merci aan Luc W voor het timmerwerk aan feeders, en de babbels over vogels.\\

Het was ook in een ongelooflijk fijn en veelzijdig team als het Treeweb team te kunnen werken. Kris, bedankt me gastvrij te ontvangen in Gontrode, voor al je inzichten, hulp en de constructieve comments om mijn drafts. Lies, je was niet zo lang bij het treeweb project, maar je het me ongelooflijk goed op weg gezet in het begin, merci daarvoor! \textit{Pudsa, thank you for the nice thai gifts, the cooperation and the fun times during field work (and I'm sorry for all the times you had to run after me through the brambles in the dark forest)}. Ik wil ook de andere collega's van de dierengeneeskunde bedanken, in het bijzonder Elin voor de vlotte samenwerking. Robbe, een ongelooflijk dikke merci voor al je enthousiaste hulp bij het veldwerk, en voor alle fijne momenten in het bos! Ik kijk ernaar uit om eens samen in de Kalkense meersen te gaan wandelen. Irene, ook jou wil ik extra hard bedanken. Het was super fijn om met jou samen onderzoek uit te denken, en uit te voeren. Het deed deugd om er samen voor te gaan, en elkaar bij te springen. Bedankt ook om er te zijn voor mij, en zo'n goede bureaugenoot te zijn. En ook super bedankt voor je hulp bij dit boekje en je super mooie foto's! (en sorry voor de vele keren dat --onbewust--  \'{e}\'{e}n van jouw balpennen op mijn bureau terecht kwam)! Ik kijk er naar uit om jou gauw eens in Noord Nederland te komen bezoeken!\\

\textit{And Lionel, a special and big Merci for you. As postdoc, you came at just the right moment into the project, and elevated it to a higher level. And more importantly, you helped and inspired so much! And chats, walks, and playing board games or cards with you was also just very fun! Thank you for everything!} Pieter, als veldwerker in het project dacht je altijd mee, en dacht je altijd verder. Hoe vaak je niet met nuttige papers of inzichten kwam... Je kennis over natural history en je inzichten hielpen me steeds verder: merci! En t was ook gewoon enorm fijn om samen vogels te vangen, of te praten over vlinders en andere insecten. Ik wil ook alle studenten bedanken die me hielpen met onderzoek tijdens hun master- of bachelorthesis: Pieter D, Pieter S, Kathy, Brenn, Eline, Laurian, Rhea, Ward en Siebe. Ik vond t echt fijn om jullie te begeleiden en leerde ook zelf van jullie veel bij!\\

Ik wil ook nog een aantal collega's van de laatste maanden bij Natuurpunt bedanken. Kris, een bijzondere merci voor jouw hulp en steun, ook (en vooral) van in de vleermuizenwerkgroep. Pieter BQW, Kevin, Carlos, Laurie, Matti, Stefan en anderen: merci voor t bij Natuurpunt fijn te maken en voor de steun.\\

Maar de PhD zou ik ook nooit hebben kunnen schrijven zonder alle mensen van buiten (of deels buiten) het werk. Ik wil Marc en Nadine, Jaap, Ayla (het was zalig met jou naar klimaatbetogingen te gaan), Sarah en Leander, en iedereen anders bedanken voor de steun de afgelopen jaren. Pieter, jou wil ik bedanken om de interesse in natural history aan te wakkeren bij mij in de JNM, en voor me als 12-jarige mee te nemen op vleermuistelling (en alle vleermuisacties die erop volgden). Ook alle wandelingen in bos t' Ename, en alle discussies en gesprekken over biodiversiteit, historische ecologie en bosbeheer hebben me ongelooflijk ge\"{i}nspireerd, en mijn interesse in (onderzoek naar) bosecosystemen opgewekt. Ren\'{e}, een bijzonder dikke merci voor de hele tijd mee te denken en met oplossingen, hulp en tips te komen voor antennes, receivers, statieven, cameravallen, zenders, loggers, etc. `t voelde vaak aan alsof er een extra veldwerker voor mijn PhD in Stein was op wie ik altijd kon rekenen voor hulp en raad (en die indien nodig ook gewoon afkwam). En vooral ook merci om een keigoede maat te zijn met wie ik samen allerlei leuke vleermuis-projecten kon uitdenken en uitvoeren!\\

Ik wil ook alle vrienden bedanken voor alle ontspanning en steun, voor de vele excursies en weekendjes in de Gaume, de Viroin of de Hoge Venen, voor alle avondjes met choufkes en kazekes in de spinnenkop of voor alle babbels aan het kampvuur. Jonas, voor `t zweefvliegen zoeken of samen aan zee te zijn, al van in de bachelor! Ik hoop dat we vlug m\'{e}\'{e}r op excursie kunnen, ook met Mauro erbij. Didi, merci voor de excursies samen, en voor de babbels over bosbeheer in Steenbergen, dat maakte een deel van mijn onderzoek concreet. Ook een bijzondere dikke merci aan de oude zakken van de NWG: Ward, Hans (merci voor de periodieke mees-meme ter motivatie!), Sander, Proesmans, Pepijn, Dora, Hanne, Michiel, Neeltje, Jefke... Ook een extra dankwoord aan Margaux, voor naast de fijne excursies en ook voor de gesprekken over (bos)ecologie en jouw onderzoek aan de KuLeuven. De KuLeuven (of meer specifiek de stage in de Brenne) was ook de plek waar ik Matthias en Aurora leerde kennen, en Sofie beter leerde kennen. Merci voor alle leuke momenten samen de afgelopen tijd, en voor t wiezen in de Walrus of online vanuit Zuid-Afrika en Australi\"{e}. Sofie, ik duim voor jou, en dat de afwerking van je PhD vlot loopt! Matthias, ik kijk uit om je eindelijk eens in Afrika te komen bezoeken. En Aurora, een speciale merci voor me altijd extra hard te steunen, voor altijd een luisterend oor te zijn, voor je humor en je enthousiasme! Sybryn, ik wil jou ook heel hard bedanken voor je steun en vriendschap! Dan wil ik nog Bram S en Dries VDL bedanken. Ik vond het fantastisch om een bureau te delen met twee goede maten die ik al zo lang kende. Bedankt ook om er altijd te zijn voor mij, en om de sfeer zo goed te maken. Ik kijk er naar uit om opnieuw plannen te smeden om samen voor weekendjes weg, samen vogels kijken of planten zoeken, of caf\'{e}-avondjes, zoals vroeger op de bureau!\\

Dan wil ik ook nog mijn familie bedanken. Thijs, my big bro, super bedankt voor alles. Het was fijn om ter zelfder tijd aan een doctoraat te werken, en het over jouw ervaringen en inzichten vanuit de humanities te hebben. Ik wil je extra hard bedanken voor de hulp, zeker ook het nalezen mijn Engelse teksten, en je vele tips! En ook bedankt om, samen met Adel, mij zo goed te steunen! Ik wil ook heel hard mijn ouders bedanken. Moeke, vake, jullie hebben me zoveel kansen gegeven, om te groeien tot wie ik nu ben! Ik ben ongelooflijk dankbaar voor al jullie steun en hulp, voor al jullie inspiratie. Jullie wakkerden mijn interesse in de natuur zo sterk aan. Ik herinner me nog goed hoe jullie me in het eerste leerjaar me extra motiveerden om met vogels bezig te zijn, en me mijn eerste verrekijker en vogelboek cadeau gaven, waarna de juf me verplaatste in de klas omdat ik meer naar de staartmezen buiten keek dan naar het bord. Jullie namen mij en Thijs mee op al die fantastische reizen naar prachtige en inspirerende landen vol natuur en geschiedenis, jullie zorgden voor zo'n warme en veilige thuis, waar ik me altijd goed kon voelen en kon groeien. Bedankt voor alles!\\

Lieve Femke, ik weet niet goed hoe ik moet beginnen om jou te bedanken. Je bent mijn allergrootste liefde, maar ook mijn collega-onderzoekster, mijn bondgenote, mijn vriendin, mijn grootste inspiratie- en motivatiebron. Je steunt me zo hard en je bent er steeds voor mij, persoonlijk, maar ook met hulp bij dit doctoraat. Ongelooflijk bedankt voor al je inzichten over ecologie, voor het nalezen, voor de motiverende babbels laat `s avonds, voor de prachtige lay-out en voor al je doctoraats-steun. En ook bedankt voor alle momenten samen, voor samen de natuur in te trekken en vleermuizen, vlinders, vogels en amfibie\"{e}n te zoeken. Ik zie je zo graag!\\

\begin{flushright}
Daan Dekeukeleire\\
februari 2021
\end{flushright}


\clearpage
\thispagestyle{plain}
\hbox{}
\clearpage
%%%%% List of publications %%%%

\thispagestyle{plain} % empty 
\CenterWallPaper{1}{PublicationList.jpg}
\newpage{\thispagestyle{empty}\cleardoublepage}
\ClearWallPaper
\pagestyle{mainmatter}
\chapter*{List of publications}
\addcontentsline{toc}{chapter}{List of publications}
\chaptermark{List of publications}

\begin{large}\textbf{Articles in international peer-reviewed journals}\end{large}\\
\begin{footnotesize}* Shared first authorship \end{footnotesize}

	\begin{enumerate}
		\item Goossens E, Boonyarittichaikij R, \textbf{Dekeukeleire D}, Van Praet S, Bonte D, Verheyen K, Lens L, Martel A, Verbrugghe E. Exploring the faecal microbiome of the Eurasian nuthatch (\textit{Sitta europaea}). \textit{Archives of Microbiology} (in press)
		
		\item \textbf{Dekeukeleire D}, Janssen R, Delbroek R, Raymaekers S, Batsleer F, Belien T, Vesterinen EJ. First molecular evidence of an invasive agricultural pest, Drosophila suzukii, in the diet of a common bat, Pipistrellus pipstrellus, in Belgian orchards. \textit{Barbastella, the Journal of Bat Research and Conservation} (in press)
		
		\item Perring M et al. (16 authors, including \textbf{Dekeukeleire D}) (2021) Overstorey composition shapes across-trophic level community relationships in deciduous forests regardless of fragmentation context. \textit{Journal of Ecology} \url{https://doi.org/10.1111/1365-2745.13580}
		
		\item Batsleer F, Bonte D, \textbf{Dekeukeleire D}, Goossens S, Poelmans W, Van der Cruyssen E, Maes D, Vandegehuchte ML (2020) The neglected impact of tracking devices on terrestrial arthropods. \textit{Methods in Ecology and Evolution} \url{https://doi.org/10.1111/2041-210X.13356}
		
		\item Boonyarittichaikij R, Pomian B, \textbf{Dekeukeleire D}, Lens L, Bonte D, Verheyen K, Pasmans F, Martel A, Verbrugghe E (2020) Season as a discriminating factor for faecal metabolomics composition of great tits (\textit{Parus major}). \textit{Belgian Journal of Zoology} \url{https://doi.org/10.26496/bjz.2020.79}
		
		\item Batsleer F, Portelli E, Borg JJ, Kiefer A, Veith M, \textbf{Dekeukeleire D} (2019) Maltese bat fauna shows phylogeographic affiliation with North-Africa: implications for conservation. \textit{Hystrix, the Italian Journal of Mammalogy} \url{https://doi.org/10.4404/hystrix-00237-2019}
		
		\item \textbf{Dekeukeleire D} et al. (11 authors) (2019) Forest Fragmentation and tree species composition jointly shape breeding performance of two avian insectivores. \textit{Forest Ecology and Management} \url{https://doi.org/10.1016/j.foreco.2019.04.023}
		
		\item \textbf{Dekeukeleire D}*, van Schrojenstein Lantman I*, Hertzog L, Vandegehuchte ML, Strubbe D, Vantieghem P, Martel A, Verheyen K, Bonte D, Lens L (2019) Avian top-down control affects invertebrate herbivory and sapling growth more strongly than overstorey species composition in temperate forest fragments. \textit{Forest Ecology and Management} \url{https://doi.org/10.1016/j.foreco.2019.03.055}
		
		\item Hertzog L et al. (14 authors, including \textbf{Dekeukeleire D}) (2019) Forest fragmentation modulates effects of tree species richness and composition on ecosystem multifunctionality. \textit{Ecology} \url{https://doi.org/10.1002/ecy.2653}
		
		\item Mckee C et al. (11 authors, including \textbf{Dekeukeleire D}) (2019) Host phylogeny, geographic overlap, and roost sharing shape parasite communities in European bats. \textit{Frontiers in Ecology and Evolution} \url{https://doi.org/10.3389/fevo.2019.00069}
		
		\item Thomaes A, Dhont PJ,\textbf{ Dekeukeleire D}, Vandekerkhove K (2018) Dispersal behaviour of female stag beetles (\textit{Lucanus cervus}) in a mosaic landscape: when should I stay and where should I go? \textit{Insect Conservation and Diversity} \url{https://doi.org/10.1111/icad.12325}
		
		\item Boonyarittichaikij R et al. (12 authors, including \textbf{Dekeukeleire D}). (2018) Mitigating the impact of microbial pressure on great (\textit{Parus major}) and blue (\textit{Cyanistes caeruleus}) tit hatching success through maternal immune investment. \textit{PLoS One} \url{https://doi.org/10.1371/journal.pone.0204022}
		
		\item Swinnen K et al. (18 authors, including \textbf{Dekeukeleire D}) (2018) Non-native plant and animal occurrences in Flanders and the Brussels capital region, Belgium. \textit{Bio-Invasions Records} \url{https://doi.org/10.3391/bir.2018.7.3.17}
		
		 \item van Schaik J*, \textbf{Dekeukeleire D}*, Gazaryan S, Natradze I, Kerth G. (2018) Comparative phylogeography of a vulnerable bat and its ectoparasite reveals dispersal of a non-mobile parasite among distinct evolutionarily significant units of the host. \textit{Conservation Genetics} \url{https://doi.org/10.1007/s10592-017-1024-9}
		 
		\item Boonyarittichaikij R et al. (14 authors, including \textbf{Dekeukeleire D}) (2017) Salmonella Typhimurium DT193 and DT99 are present in great and blue tits in Flanders, Belgium. \textit{PLoS One} \url{https://doi.org/10.1371/journal.pone.0187640}
		
		\item De Groote S et al. (18 authors including \textbf{Dekeukeleire D}) (2017) Tree species identity outweighs the effects of tree species diversity and forest fragmentation on understorey diversity and composition. \textit{Plant Ecology and Evolution} \url{https://doi.org/10.5091/plecevo.2017.1331}
		
		\item \textbf{Dekeukeleire D}, Janssen R, Haarsma AJ, Bosch T, van Schaik J (2016) Swarming behaviour, catchment area and seasonal movement patterns of Bechstein's bats: implications for conservation. \textit{Acta Chiropterologica}\\ \url{https://doi.org/10.3161/15081109ACC2016.18.2.004}
		
		\item Nyssen P, Smits Q, Van de Sijpe M, Vandendriessche B, Halfmaerten D, \textbf{Dekeukeleire D} (2015) First records of \textit{Myotis alcathoe} von Helversen \& Heller, 2001 in Belgium. \textit{Belgian Journal of Zoology} 145, 131-137.
		
		\item van Schaik J, \textbf{Dekeukeleire D}, Kerth G (2015) Host and parasite life history interplay to yield divergent population genetic structures in two ectoparasites living on the same bat species. \textit{Molecular Ecology} \url{https://doi.org/10.1111/mec.13171}
		
		\item \textbf{Dekeukeleire D}, Janssen R, van Schaik, J (2013) Frequent melanism in Geoffroy's bat (\textit{Myotis emarginatus}, Geoffroy 1806). \textit{Hystrix, the Italian Journal of Mammalogy} \url{https://doi.org/10.4404/hystrix-24.2-8770}
		
	\end{enumerate}

\vspace*{1cm}
\begin{large}\textbf{Articles in other peer-reviewed journals}\end{large}\\

\begin{enumerate}
	\item Boeraeve M, Batsleer F, Vermeire, H, Thomaes A, Opstaele B, \textbf{Dekeukeleire D} (2019) Winterverblijfplaatsen voor vleermuizen. Het belang van bunkergordels, ijskelders en forten in Oost-Vlaanderen. \textit{Natuur.focus} 18(4):136-144.
	
	\item Batsleer F, \textbf{Dekeukeleire D}, Batsleer M, Verbelen D (2019) Kleurafwijkende vuursalamander in Belgi\"{e}. \textit{Ravon} 21(2): 20-22.
	
	\item Hertzog L, \textbf{Dekeukeleire D}, Bonte D, Martel A, Verheyen K, Lens L, Baeten L (2019) Kleine bossen functioneren beter wanneer ze gemengd zijn. \textit{Bosrevue} 82(a): 1-7.
	
	\item Hertzog L et al. (14 authors including \textbf{Dekeukeleire D}) (2019) Forest Fragmentation Modulates Effects of Tree Species Richness and Composition on Ecosystem Multifunctionality. \textit{Bulletin of the Ecological Society of America} 100(3).
	
	\item \textbf{Dekeukeleire D}, Janssen R. (2014) A large maternity colony of 85 Bechstein's bats (\textit{Myotis bechsteinii}) in an invasive tree, the red oak (\textit{Quercus rubra}). \textit{Lutra} 57: 40-55.
	
	\item \textbf{Dekeukeleire D}, De Knijf G, Boers K, Gyselings R, Paelinckx D (2014) Vleermuizen gaan achteruit in Vlaanderen: resultaten van de rapportering 2013 van de Europese beschermde soorten en habitats. \textit{Natuur.focus} 13(2): 59-65.
	
	\item \textbf{Dekeukeleire D}, Mortelmans J, Van de Loock D, Sercu BK (2014) First record for Belgium of \textit{Eurytoma longipennis} Walker, 1832 (Hymenoptera: Chalcidoidea), a gall-forming wasp on \textit{Ammophila arenaria} (L). \textit{Bulletin SRBE/KBVE} 150: 40-42.
	
	\item Mortelmans J, Boeraeve M, Tamsyn W, Proesmans W, \textbf{Dekeukeleire D} (2014) Thirteen new Agromyzidae for Belgium (Diptera: Agromyzidae). \textit{Bulletin SRBE/KBVE} 150: 141-148.
	
	\item Mortelmans J, \textbf{Dekeukeleire D}, Baugn\'{e}e J-Y (2013) Four leaf mining flies on \textit{Ranunculus }sp. new for Belgium (Diptera: Agromyzidae). \textit{Bulletin SRBE/KBVE} 149: 29-33.
	
	\item \textbf{Dekeukeleire D}, Janssen R, Boers K, Willems W (2011) Zwermende vleermuizen bij Antwerpse forten. \textit{Natuur.focus} 10(3):104-109.
	
	\item \textbf{Dekeukeleire D} (2010) First record of soprano pipistrelle (\textit{Pipistrellus pygmaeus} Leach, 1825; Chiroptera: Vespertilionidae) in Wallonia (Belgium). \textit{Lutra} 53: 105-107.
	
	\item Blond\'{e} P, Peccue B, \textbf{Dekeukeleire D} (2009) Vleermuizen in en rond Bos t' Ename. \textit{Natuur.focus} 8(2): 56-61.

\vspace*{2cm}	
\end{enumerate}

\begin{large}\textbf{Peer-reviewed book chapters}\end{large}\\

\begin{enumerate}
	\item Janssen R, \textbf{Dekeukeleire D} (2016) Bechsteins vleermuis \textit{Myotis bechsteinii} -- in: Broekhuizen S, Spoelstra K, Thissen JBM, Canters KJ, Buys JC (editors) De Nederlandse Zoogdieren. Natuur van Nederland 12, Naturalis Biodiversity Center, Leiden: 201-202.
\end{enumerate}



	%%%%%%%%%%%%%%%%%%%%%%%%%%%%%%%%%%%%%  Bibliography  %%%%%%%%%%%%%%%%%%%%%%%%%%%%%%%%%%%%%%%%%%%%%%%%
%	\thispagestyle{empty}
\setlength{\thumbwidth}{0cm}
\setlength{\thumbheight}{0cm}
	%\chapter*{Bibliography}
	\bibliographystyle{customPhDEcology}%elsarticle-harv_M2
	\begin{footnotesize}
		\bibliography{bibliography0602v2}
	\end{footnotesize}

\includepdf{backcover.pdf}	
\end{document}