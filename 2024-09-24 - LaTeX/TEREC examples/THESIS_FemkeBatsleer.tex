%%%%%%%%%%%%%%%% PRESETTINGS %%%%%%%%%%%%%%%%%%%%%%%


\documentclass[10pt, twoside]{book} %

\usepackage[paperheight=24cm,paperwidth=17cm,margin=2.3cm]{geometry}
%\usepackage[margin=2.3cm,b5paper]{geometry} %lmargin=2.5cm,rmargin=4cm,tmargin=3cm,bmargin=3cm
%\pdfpagewidth 17cm
%\pdfpageheight 24cm




\usepackage[table]{xcolor}

\usepackage{amsmath,amssymb,mathtools,graphicx,textcomp,booktabs,url,setspace,xcolor,soul,eurosym}
\usepackage{multirow,listings,setspace,gnuplottex,latexsym,keyval,ifthen,moreverb,lscape,forest}
\usepackage[pagewise]{lineno}
\usepackage{wallpaper}
\graphicspath{{fig/}}

\usepackage[figuresright]{rotating}

\usepackage[]{microtype} %activate={true,nocompatibility},final,kerning=true,spacing=true,factor=1100,stretch=10,shrink=10
\usepackage{booktabs}
\usepackage{romannum}
\usepackage{textgreek}

\usepackage{lmodern}
\usepackage{tabularx}
\usepackage[most]{tcolorbox}

\usepackage{tikz,pgfplots}

\usetikzlibrary{calc,shapes,arrows,intersections,shadows}
\usepackage{textcomp}

\usepackage[most]{tcolorbox}
\usepackage{longtable}

\usepackage[]{threeparttable}
\usepackage{array}
%\usepackage[all]{nowidow} %avoid hanging lines


\usepackage[indention=0.5cm,labelsep=colon,font={sf,small},labelfont={bf,sf}]{caption}
\usepackage[indention=0.5cm,font={sf,small},labelfont={bf,sf}]{subcaption}

\definecolor{lightgray}{gray}{0.90} %not always visible on a dell screen!!!
\definecolor{darkgray}{gray}{0.3}







\usepackage{fancyhdr}  
\pagestyle{fancy} 
\renewcommand{\chaptermark}[1]{\markboth{#1}{}}
\fancypagestyle{frontmatter}{%    
	\fancyhf{} 	
	\fancyfoot[RO,LE]{\textsf{\thepage}} %Page 
	\fancypagestyle{plain}{
		\renewcommand{\headrulewidth}{0pt}
		\fancyhead{}
		\fancyfoot[RO,LE]{\textsf{\thepage}} %Page
	} 
}
\fancypagestyle{mainmatter}{%    
	\fancyhf{} 	
	\fancyhead[RO,LE]{\nouppercase{\small \textsf{\leftmark}}}
	\fancyfoot[RO,LE]{\textsf{\thepage}} %Page 
}
\setlength{\headheight}{10pt}

\usepackage{pdfpages}

\newcounter{letternum}
\newcounter{lettersum}
\setcounter{lettersum}{13}
\newlength{\thumbtopmargin}
\setlength{\thumbtopmargin}{1cm}
\newlength{\thumbbottommargin}
\setlength{\thumbbottommargin}{3cm}
\newlength{\thumbheight}
\pgfmathsetlength{\thumbheight}{%
	(\paperheight-\thumbtopmargin-\thumbbottommargin)/\value{lettersum}}

\newlength{\thumbwidth}
\setlength{\thumbwidth}{1.2cm}
\setlength{\thumbheight}{1cm}


\tikzset{
	thumb/.style={
		%   draw=black,
		fill=light-gray,
		text=black,
		minimum height=\thumbheight, %\thumbheight,
		text width=\thumbwidth,
		outer sep=0pt,%   outer sep=10pt,
		font=\sffamily\Large,
	}
}
\newcommand{\oddthumb}[1]{%
	\begin{tikzpicture}[remember picture, overlay]
		\node [thumb,text centered,anchor=north east,] at ($%
		(current page.north east)-%
		(0,\thumbtopmargin+\value{letternum}*\thumbheight)%
		$) {#1};
	\end{tikzpicture}
}
\newcommand{\eventhumb}[1]{%
	\begin{tikzpicture}[remember picture, overlay]
		\node [thumb,text centered,anchor=north west,] at ($%
		(current page.north west)-%
		(0,\thumbtopmargin+\value{letternum}*\thumbheight)%
		$) {#1};
	\end{tikzpicture}
}
% create a new command to set a new lettergroup
\newcommand{\lettergroup}[1]{%
	\fancyhead[LO]{\oddthumb{#1}}%
	\fancyhead[RE]{\eventhumb{#1}}%
	\fancypagestyle{chapterstart}{%
		\renewcommand{\headrulewidth}{0pt}
		\renewcommand{\footrulewidth}{0pt}
		\fancyhf{}
		\chead{\oddthumb{#1}}% chapters start only on odd pages
		\fancyfoot[RO,LE]{\textsf{\thepage}} %\fancyfoot[RO,LE]{\textsf{Page \thepage}}
	}
	\thispagestyle{chapterstart}
	\stepcounter{letternum}%
}


%%%%%create own labels for intro and discussion
\newcommand{\oddthumbID}[1]{%
	\begin{tikzpicture}[remember picture, overlay]
		\node [thumb,text centered,anchor=north east,] at ($%
		(current page.north east)-%
		(0,\thumbtopmargin+\value{letternum}*\thumbheight)%
		$) {#1};
	\end{tikzpicture}
}
\newcommand{\eventhumbID}[1]{%
	\begin{tikzpicture}[remember picture, overlay]
		\node [thumb,text centered,anchor=north west,] at ($%
		(current page.north west)-%
		(0,\thumbtopmargin+\value{letternum}*\thumbheight)%
		$) {#1};
	\end{tikzpicture}
}
\newcommand{\lettergroupID}[1]{%
	\fancyhead[LO]{\oddthumbID{#1}}%
	\fancyhead[RE]{\eventhumbID{#1}}%
	\fancypagestyle{chapterstart}{%
		\renewcommand{\headrulewidth}{0pt}
		\renewcommand{\footrulewidth}{0pt}
		\fancyhf{}
		\chead{\oddthumbID{#1}}% chapters start only on odd pages
		\fancyfoot[RO,LE]{\textsf{\thepage}} %\fancyfoot[RO,LE]{\textsf{Page \thepage}}
	}
	\thispagestyle{chapterstart}
	\stepcounter{letternum}%
}


%%%% CREATE BOX CHAPTER

\usepackage[english]{babel}
\usepackage[newparttoc,explicit, clearempty]{titlesec}%
\usepackage[titles]{tocloft}
\renewcommand{\cftpartpresnum}{BOOK\enspace}
\renewcommand{\cftchapaftersnum}{.}
\renewcommand\cftchapdotsep{\cftdotsep}

\titleformat{\part}[display]{\bfseries\filcenter \def\partname{Book}}{\Huge\MakeUppercase{\partname}\enspace\thepart}{10pt}{\Huge #1}[\thispagestyle{empty}]%

\titleformat{\chapter}[display]{\filcenter\bfseries}{\LARGE\MakeUppercase{\chaptername}~\thechapter}%
{1\baselineskip}
{\huge#1}%
\titleformat{name=\chapter, numberless}[block]{\filcenter\bfseries}{}%
{0pt}{\huge#1\ifstrequal{#1}{\contentsname}{}{\addcontentsline{toc}{chapter}{#1}}}%


\usepackage[titletoc]{appendix} %
\AtBeginEnvironment{appendices}{\def\chaptername\appendixname}
\AtEndEnvironment{appendices}{\def\chaptername\oldchaptername}
\newenvironment{newchapterbox}{%
	\def\chaptername{BOX}\def\appendixname{BOX}\appendices}%
{\endappendices}



%\makeatletter\renewcommand\tableofcontents{%
	%\chapter*{\contentsname}%
	%\@starttoc{toc}%
	%}
%\makeatother
%%% END


\usepackage[nottoc]{tocbibind}	%numbib


%\usepackage[final]{pdfpages}

\usepackage[]{natbib} %numbers,sort&compress
\setlength{\bibsep}{0.2pt plus 0.3ex}

\usepackage[Sonny]{fncychap} %Sonny





\setlength{\parindent}{2em} 
\renewcommand{\contentsname}{Table of Contents}
\renewcommand{\listfigurename}{List of Figures}
\renewcommand{\listtablename}{List of Tables}
\renewcommand{\appendixname}{}

%numbering of sections
%\numberwithin{section}{chapter}
%\numberwithin{subsection}{section}

\makeatletter
\newenvironment{chapquote}[2][2em]
{\setlength{\@tempdima}{#1}%
	\def\chapquote@author{#2}%
	\parshape 1 \@tempdima \dimexpr\textwidth-2\@tempdima\relax%
	\itshape}
{\par\normalfont\hfill--\ \chapquote@author\hspace*{\@tempdima}\par\bigskip}
\makeatother

\makeatletter
\def\mainmatter{%
	\cleardoublepage
	\@mainmattertrue
	\pagenumbering{arabic}
	\def\mainmatter{\cleardoublepage\@mainmattertrue}
}
\makeatother



\definecolor{light-gray}{gray}{0.70} %not always visible on a dell screen!!!
\definecolor{mygreen}{HTML}{23A48B}
\definecolor{myyellow}{HTML}{F49F1F}
\definecolor{myred}{HTML}{C24133}
\definecolor{mygray}{gray}{0.90}
\setlength{\parindent}{2em}

\usepackage{pdflscape}
\usepackage{afterpage}

\usepackage[colorlinks,linkcolor=black,urlcolor=black,citecolor=black,hypertexnames=false]{hyperref} %load hyperref after fncychap
\hypersetup{%
	pdftitle = {},
	pdfsubject = {PhD thesis},
	pdfkeywords = {},
	pdfauthor = {Femke Batsleer},
	pdfcreator = {\LaTeX\ with package \flqq hyperref\frqq},
}
%\usepackage[]{cleveref} % load cleveref after hyperref
\usepackage{bookmark}

%\renewcommand{\arraystretch}{1.7}




%%%%%%%%%%%%%%%% BEGIN DOCUMENT %%%%%%%%%%%%%%%%%%%%%%%
\begin{document}
%\includepdf[pages=-]{../cover/Cover.pdf}
%\cleardoublepage
\thispagestyle{mainmatter} % empty
	\frontmatter
	\pagestyle{frontmatter}
	\lstset{language=Perl}
	%%%%%%%%%%%%%%%%  BEGIN TITLEPAGE  %%%%%%%%%%%%%%%%%%
	\begin{titlepage}
		
		\begin{center}	
			
			\thispagestyle{empty}
			
			\vspace*{3.00cm}
			
			{\huge \textbf{Spatial ecology, gene flow and conservation of the digger wasp \textit{Bembix rostrata}}}\\
			
			\vspace{7.0 cm}
			
		\end{center}
		

		
	\end{titlepage}

\newpage
		
	\color{black}
	\newpage 
	\thispagestyle{empty}

	\vspace*{\fill}

	\begin{small}

	\noindent \textcopyright 2023 Femke Batsleer

	\vspace{0.5cm}	

	\noindent Batsleer F. (2023). \textit{Spatial ecology, gene flow and conservation of the digger wasp \textit{Bembix rostrata}}. Ph.D. thesis, Ghent University, Ghent, Belgium.

	\vspace{0.5cm}	

\noindent Printed by: University Press, Wachtebeke, Belgium\\
Lay-out, photos and cover: Femke Batsleer \\


	\vspace*{0.5cm}
	
	\noindent The research presented in this study was financially supported by Research Foundation-Flanders (FWO) --- PhD Fellowship Fundamental Research.
	
	\vspace{1cm}
\end{small}	

	
	\newpage{\thispagestyle{empty}\cleardoublepage}
	\color{black}
	\newpage 
	\thispagestyle{empty}
\begin{center}
			\thispagestyle{empty}
			
			\begin{figure}[h!]
				\centering
				\includegraphics[width=0.18\textwidth]{figures/UGent_logo.png}\hfill
				\includegraphics[width=0.25\textwidth]{figures/FacultySciences_logo.png}\hfill
				\includegraphics[width=0.25\textwidth]{figures/FWO_logo.jpg}
				\end{figure}
			
			\vspace*{3.50cm}
			
			{\Large Spatial ecology, gene flow and conservation of the digger wasp \textit{Bembix rostrata}}

			
			\vspace{5.5 cm}
			
			{\normalsize Femke Batsleer} 
			
			\vspace{1.0 cm}
			
			{\normalsize 2023}	
			
			\vspace{2.0 cm}
			
			{\footnotesize Ghent University, Faculty of Sciences, Department of Biology, Terrestrial Ecology Unit}
			
			\vspace{0.5cm}
			
			{\footnotesize Thesis submitted in fulfillment of the requirements for the degree of\\
 			Doctor (Ph.D.) in Science: Biology}

\end{center}
\newpage
		
	\color{black}
	\newpage 
	\thispagestyle{empty}

		
	{\small \noindent \textbf{Supervisors:} \\
			\hspace{10mm}Prof. Dr. Dries Bonte\\
			\hspace{10mm}Prof. Dr. Dirk Maes}\\

	\vspace*{1.0cm}
	
	{\small \noindent \textbf{Examination committee:}\\
		\hspace{10mm}Prof. Dr. Annemieke Verbeken (chair) \\
		\hspace{10mm}Prof. Dr. Jan Van Uytvanck\\
		\hspace{10mm}Dr. Laurence Cousseau\\
		\hspace{10mm}Prof. Dr. Josep D. Asís Pardo\\
		\hspace{10mm}Dr. Viktoriia Radchuk} \\
	
\newpage
\begin{center}	
	\thispagestyle{empty}		
	\vspace*{5.00cm}
	{\footnotesize \textit{Voor mama en papa,}\\
	\textit{jullie kleintje (of die kleine)}\\}
	\vspace*{2cm}
	{\footnotesize \textit{Voor meme,}\\
	\textit{voor altijd uwen `hoaze'}}
	
\end{center}
		
	
	%%%%%%%%%%%%%%%%  BEGIN LISTS   %%%%%%%%%%%%%%%%%%%%%%%
	\newpage{\thispagestyle{empty}\cleardoublepage}
	{\setstretch{0.98}\tableofcontents}
	%\tableofcontents
	



%\cleardoublepage
%\thispagestyle{empty} % empty 
%\hbox{}
\clearpage

\CenterWallPaper{1}{pictures/INTRO-1.png}
\newpage{\thispagestyle{empty}\clearpage}
\cleardoublepage

\ClearWallPaper

\CenterWallPaper{1}{pictures/INTRO-2.png}
\newpage{\thispagestyle{empty}\clearpage}
\hbox{}
\clearpage
\ClearWallPaper

\CenterWallPaper{1}{pictures/INTRO-3.jpg}
\newpage{\thispagestyle{empty}\cleardoublepage}
\ClearWallPaper
% TEMPORARY
%\linenumbers
%%%%%%%%%%%%%%%%%%%%%%%%%%%%%%%%%%%%%General introduction %%%%%%%%%%%%%%%%%%%%%%%%%%%%%%%%%%%%%%%%%
\mainmatter
\pagestyle{mainmatter}

%%%%other type of boxes for intro
\setlength{\thumbwidth}{0.8cm}
\setlength{\thumbheight}{1cm}
\tikzset{
	thumb/.style={
		%draw=black,
		fill=gray,%white
		text=gray,
		minimum height=\thumbheight, %0cm, %\thumbheight,
		text width=\thumbwidth, %0cm,
		outer sep=0pt,%   outer sep=10pt,
		font=\sffamily\Large,
	}
}
\chapter*{General introduction} \label{Introduction}
\chaptermark{General introduction}
\addcontentsline{toc}{chapter}{General introduction}
\lettergroupID{\thechapter}

\begin{flushright} \color{gray}Femke Batsleer\color{black}\end{flushright}

\vspace*{\fill}
\noindent \color{gray} $\lhd$ A male \textit{B. rostrata} on guard.\\
\noindent$\lhd\lhd$ A female at her nest in a grey dune.


\color{black}

\newpage
\renewcommand\thesection{I.\arabic{section}}
\noindent In this thesis I research ecological processes involved in nest spatial patterns and functional connectivity in the digger wasp \textit{Bembix rostrata}. The research of this thesis is situated in the (sub)fields and intersection of spatial ecology, behavioural ecology, conservation research, population genetics, and landscape genetics. First, I start this introduction with a few paragraphs on general ecological concepts: movement and its relation to dispersal and gene flow, fragmentation, spatial patterns of nest clustering, and methods to measure movement. Second, I discuss coastal and inland dune ecosystems in Belgium, including their biodiversity value and fragmentation history. Third, I introduce the study species the digger wasp \textit{Bembix rostrata} and describe its recognition, ecology and conservation status. Lastly, in the final section, I explain the objectives and research questions of this thesis in relation to the general ecological framework and conservation of the species.

	\section{Movement, dispersal and gene flow}
	\textbf{Movement} is a fundamental component of various ecological and evolutionary processes across spatial and temporal scales \citep{nathan2008}. Movement determines---among many other things---spatial patterns of individuals within populations \citep{lopez2012, liu2013}, patterns of species distributions \citep{bonte2004, kokko2006}, metapopulation dynamics \citep{hanski1994, hanski1998} and outcomes of range shifts or expansions \citep{neubert2000, higgins2003, pagel2012, hodgson2022}. In essence, movement shapes the distribution of species, individuals and genes across space and time. Many types of movement exist, such as movements to forage, search for a mate or nest location, defend a territory, migrate (move from and to locations at regular intervals), or disperse (move to another location or environment for reproduction) \citep{jeltsch2013}. When studying the causes and consequences of movement types, it is crucial to define the relevant spatial and temporal scales, the drivers considered and the connections to other types of movement. This promotes the integration within the conceptual framework of movement ecology, defining external and internal factors---including navigation and motion capacities---resulting in movement patterns \citep{nathan2008a, jeltsch2013, goossens2020}.\\
	
	\textbf{Dispersal}---any movement of organisms or modules with potential consequences for gene flow across space \citep{ronce2007}---is a central life-history trait \citep{bonte2017}. It has extensive eco-evolutionary consequences for spatially structured populations, climate-driven range expansions and biological invasions \citep{bowler2005}. A number of ultimate drivers causing dispersal (opposed to proximate drivers, the individual `triggers') have been identified, such as kin competition, inbreeding avoidance, escaping local unfavourable conditions (e.g. overcrowding), dealing with spatio-temporal variability (e.g. in ephemeral or successional habitats), and unpredictability (e.g. leading to distribution of offspring across conditions: `bet hedging') \citep{bowler2005, matthysen2012}. There are many trade-offs simultaneously affecting the process of dispersal, which make it hard to disentangle these ultimate drivers and consequences in natural populations, which further interact in an eco-evolutionary feedback loop \citep{bonte2012, starrfelt2012}. Traits and related costs and benefits are considered across three main phases of dispersal: departure, transfer, and settlement \citep{clobert2009, bonte2012}. Acknowledging dispersal as this multi-phase life-history trait is crucial to understand each phase's separate and joined consequences and proximate and ultimate drivers \citep{bonte2012}.\\
	
	\textbf{Gene flow} is the transfer of genetic information---compared to organisms or modules for dispersal---across space. Gene flow incorporates dispersal, but is also in itself influenced by many other ecological traits and processes, such as survival, fecundity, effective population sizes, habitat selection and local adaptation \citep{spear2010}. Gene flow and direct movement are typically correlated, but cannot accurately predict each other \citep{bohonak1999, whitlock1999, spear2010}. First, direct movements work on a different spatial and temporal scale than gene flow \citep{anderson2010}. Second, several types of movements are not directly associated to gene flow (which happens between breeding locations), such as foraging, seasonal-dependent movements or migration from and to wintering areas \citep{pilliod2002, friesen2007, cushman2010a}. Also vice versa, gene flow will underestimate crucial movements related to other essential resources that are not reproduction \citep{anderson2010, spear2010}. Dispersal can be more closely related to gene flow, but gene flow also depends on survival and reproduction after settlement, affected by e.g. local adaptation. Gene flow can also take place between population through intermediate populations across multiple generations, even if direct exchange of individuals is not occurring \citep{spear2010}.\\
	
	Linking movement, dispersal and gene flow is essential to understand ecological patterns and to inform conservation. Correlating these three interrelated processes has to be done with the needed nuance and caution considering the relevant scales, traits and processes involved in all three.\\
	
	\section{Fragmented habitat patches and the matrix}
	How suitable habitat is embedded in the landscape---the total amount, number of fragments, spatial configuration, the type of matrix---has an important influence on a species' movement behaviour \citep{knowlton2010}.\\
	
	Many \textbf{types of habitat occur heterogeneously}. Even in large, continuous blocks of a certain habitat type, variation in conditions exist from small to large spatial scales: microstructure and topography, light conditions, soil characteristics, temperature and moisture microclimates, etc. \citep{gonzalez-megias2007, sercu2019}. For any habitat type, many abiotic and biotic filters at several spatial scales exist eventually influencing the species' possible occurrences: the realized niche \citep{pearman2008, colwell2009}. \textbf{Natural fragmentation}---when habitat is naturally structured in discontinuous, isolated patches---occurs in several natural systems, such as island archipelagos, rocky outcrops, pools and lakes... Species occupying such naturally fragmented patches often form dynamic meta-populations and -communities where species' occurrence are governed through dynamics depending on local extinctions and colonisations \citep{hanski1994, hanski1998}.\\
	
	Humans have impacted and are profoundly impacting biodiversity and natural habitats on the local and global level through several types of drivers, such as climate change, pollution and land use change \citep{cardinale2012, haddad2015}. Land use change results in both habitat loss and habitat fragmentation (sensu stricto, i.e. change in spatial configuration; \citeauthor{fahrig2017} \citeyear{fahrig2017}) of continuous habitat, resulting in smaller habitat fragments that become more isolated. \textbf{Anthropogenic habitat fragmentation} (sensu lato, i.e. both habitat loss and change in spatial configuration) affects biodiversity through several, often interacting processes. The primary, most obvious processes are loss of total habitat area and decreased connectivity. According to the well-supported ecological concept of species-area curve, larger areas of habitat support more species and more individuals. The loss of total area will accordingly decrease the maximum number of occurring species and individuals \citep{macarthur1967, fahrig2013}. When habitat patches become more isolated, they are less likely to be (re)colonized because connectivity between patches is decreased \citep{macarthur1967, hanski1998}. Secondarily, habitat fragmentation leads to multiple and variable changes in ecological processes and interactions. This can lead to diminished habitat quality or ecosystem decay, resulting in diversity loss \citep{chase2020}. Fragmentation often leads to an increased edge-core ratio and sharper edges between habitat and matrix, which have different biophysical conditions than the fragment core \citep{didham1999}. These edge effects can penetrate deep into the core habitat, affecting habitat specialists mainly negatively and generalists positively \citep{pfeifer2017}. Trophic interactions can change in habitat fragments and at edges \citep[e.g.][]{dekeukeleire2019}. Processes important at larger spatial scales could also be altered dramatically by fragmentation, such as natural fire regimes in boreal forests \citep{larjavaara2005}, natural wind dynamics in dunes \citep{provoost2011} or hydrological integrity and water cycle in bogs and other marshlands \citep{mackin2017}.\\
	
	Landscape changes by humans can happen rapidly but can also slowly develop across several decades. Conversion types can range from degraded habitat (e.g. managed forest) to extreme modification (e.g. old growth forest converted into urban or agricultural land uses). As these drivers of fragmentation can be variable in identity, intensity and both spatial and temporal scale, \textbf{responses of species to fragmentation} are often species- and context-dependent. Consequently, pre-adaptation to natural fragmentation is not easily extrapolated to the potential to cope with or adapt to anthropogenic fragmentation \citep{cheptou2017}. Specialists, with life-history traits such as low mobility and narrow feeding niche, are generally found to be more strongly affected by anthropogenic fragmentation \citep{ockinger2010, pfeifer2017}.\\
	
	The practical binary division between habitat and matrix has been helpful in many theoretical ecology developments \citep{macarthur1967, hanski1998}. In reality, the landscape between habitat patches should be considered a \textbf{heterogeneous matrix} with various landscape types, especially in human-altered landscapes \citep{driscoll2013, manel2013}. How an organism reacts to unfamiliar landscape types or elements in fragmented landscapes can be variable, depend on the specific landscape context considered and vary between species and between individuals \citep{baguette2007, knowlton2010}. Some species can use human-altered landscape types as part of their niche, but the majority of habitat specialist species will not evolutionary adapt to live in semi-artificial ecosystems, such as agricultural or urban areas \citep{moller2009, mcdonnell2015, kimmig2020}. They will, however, need to react to such landscape types in search for sufficient resources or during dispersal, either by avoiding or crossing them. Behavioural responses can be measured through experimental manipulations of animals in landscapes, studied in the research field of behavioural landscape ecology \citep{knowlton2010} road-crossing after translocation, habitat selection experiments with or without resource manipulation, manipulation of perceived predation risk, etc. Movement paths in different landscapes can also be tracked directly with tracking devices. Measuring gene flow is an indirect way to measure functional connectivity in a heterogenous matrix (see section \ref{measuringmovement}). Linking landscape features directly with gene flow is the focus in landscape genetics research, a field at the crossroads of landscape ecology and population genetics. Landscape genetics methods allow to explicitly quantify the effects of landscape on micro-evolutionary processes such as gene flow \citep{manel2003, manel2013, balkenhol2015}.\\
	
	\section{Spatial patterns of clustering}\label{spatpatt}
	Many \textbf{emergent spatial patterns} arise in ecosystems linked to underlying mechanisms of individual movements and interactions. Spatial self-organised patterns are a fascinating example. For example, regular patterns of mussel clumps arise from density-dependent movement \citep{vandekoppel2008, liu2013}, or regular patterns in arid vegetations and tussock spacing in marshland arise through facilitating processes at short distances and negative feedbacks at larger distances \citep{couteron2001, vandekoppel2006, rietkerk2008}. Such spatial self-organisation patterns are key in understanding ecosystem stability and diversity \citep{rietkerk2004}. Another type of self-organised emergent patterns is related to collective behaviour where individual movement and interaction with neighbouring individuals result in coordinated motions of the group \citep{sumpter2006}. Captivating examples are the many silhouettes a cloud of starlings can produce, the shapes of schools of fish that can transform from balls into vortices and flash expansions or ants that form trails of optimized flow \citep{couzin2003, lopez2012}.\\
	
	Such emergent spatial patterns are the most apparent in homogenous environments, but can also arise in heterogeneous environments (e.g. gradients or patchiness of habitat suitability). It is much harder to then tease apart the relative influence of habitat heterogeneity and internal processes in shaping the patterns. \textbf{Clustering of nests}---in for instance birds or solitary bees and digger wasps---is a good example of a type of emergent spatial pattern where both behavioural mechanisms and habitat heterogeneity can play a role. Many mechanisms---which are not mutually exclusive---could give rise to a clustered nest pattern, such as \textbf{restricted habitat availability, breeding site fidelity, social cue for habitat suitability} or a\textbf{ selfish herd}. The latter two both resulting in conspecific attraction. Firstly, the suitable habitat spots might be restricted, which results in nests constructed in the vicinity of each other, but random within a habitat patch \citep[e.g. in shorebirds][]{swift2017}. Secondly, an individual female might make nests in the vicinity of her other or previous nests \citep{hoi2012, asis2014}. Especially in digger wasps that maintain multiple nests simultaneously---for instance in some \textit{Ammophila} species---this can result in clustering of active nests of one individual \citep{baerends1938}. Thirdly, a social honest cue for habitat suitability can reduce the time invested in individually assessing the quality of the habitat for habitat selection \citep{dall2005, buxton2020}. This results in a pattern of conspecific attraction: the attraction to other individuals of the same species. Conspecific attraction has been found in birds and digger wasps as an important component for nest clustering \citep{brown2000, polidori2008, melles2009, asis2014}. Fourthly, a selfish herd could also give rise to clustering of nests, and can be seen as another specific form of conspecific attraction. In a selfish herd, individuals try to put other conspecifics between themselves and predators or parasites to reduce their individual rate of predation or parasitism \citep{mooring1992}. Such a hypothesis has been put forward for aggregations in the digger wasp species \textit{Crabro cribrellifer} \citep{wcislo1984, larsson1986}, where the individual rate of parasitism per nest decreases with nest density. In such a case, parasitism in absolute numbers often does increase with nest density, but not at the same rate. Consequently, parasitism on average per nest decreases with nest density. The relative importance of these different processes that underlie nest aggregation, which can act simultaneously, remain elusive.\\
	
	\section{Measuring movement and gene flow}\label{measuringmovement}
	\textbf{Directly measuring movement} can be done by tracking individual animals, through methods such as mark-recapture or telemetry. Mark-recapture implies individually marking organisms (with rings, coloured and numbered tags or plates, painted numbers, etc.) and recapturing and marking individuals at several subsequent moments to track their movement \citep[e.g.][]{roland2000, bowne2004}. It however requires intensive effort and rarely detects---and thus underestimates---long-distance dispersal \citep{hassall2012, ugelvig2012, trense2021}. Simulation or modelling studies---such as diffusion models or individual-based models---can elucidate possible movement behaviours explaining the spatial recapture patterns \citep{ovaskainen2008}. Uniquely marked individuals can also be followed directly in the field or in laboratory settings, which can give insights into movement behaviour in short timeframes. For example, butterflies have been tracked with a handheld GPS in the field \citep{fernandez2016}, \textit{Drosophila} flies automatically video-tracked in the lab \citep{branson2009} and ants automatically tracked in the lab with QR-codes glued to their dorsal thoraces \citep{mersch2013, stroeymeyt2018}. Many technological advances in telemetry and tracking devices are adopted into behavioural and movement ecology studies to track animals on larger spatial scales and during extended timeframes \citep{kays2015}. The measured individual movement tracks can be linked with movement mechanisms and population-level patterns and distributions through quantitative methods, simulation models and integration in pattern-oriented modelling \citep{mueller2008}. The use of tracking devices has an extensive history in bird and mammal research, and have also been applied to some larger terrestrial arthropods for the last three decades as these devices have become rapidly smaller and lightweight \citep{kissling2014, portugal2018}. For example, detailed migration paths of individual dragonflies, butterflies and moths have been tracked with radio-transmitters \citep{wikelski2006, knight2019, menz2022}, passive harmonic radar tags have been used to study flight behaviour in bees and moths \citep{riley1998, capaldi2000}, passive radio frequency identification (RFID) transponders have been used to study ant and bee social organisation and foraging behaviour \citep{robinson2009, ohashi2010, vanoystaeyen2013}. Applying these techniques on arthropods has proven fruitful in studies of habitat use, pollination, pest control, social interactions, orientation and conservation \citep{kissling2014}. However, in vertebrate research, researchers have already put forward that tracking devices can and do have impacts on animal behaviour and performance, affecting the reliability of the gathered data \citep{mech2002, godfrey2003, barron2010, mcintyre2015, bodey2018, portugal2018, geen2019}. Such ethical and scientific drawbacks (reliability of the data) have resulted in recent recommendations and guidelines for the application of tracking devices on birds and mammals \citep{wilson2006, casper2009, omara2014, kays2015}. Similar guidelines and critical evaluation are however lacking in the employment of tracking devices on arthropods.\\
	
	\textbf{Directly measuring dispersal events} is hard \citep{duputie2013}, especially in natural settings when dispersal rates are low. In such circumstances, capture-mark-recapture methods and individual tracking rarely detect dispersing individuals, underestimating long-distance dispersal \citep{hassall2012, ugelvig2012, trense2021}. Some systems are more suitable for researching dispersal in natural settings, such as some metapopulations with detectable levels of dispersal \citep[e.g.][]{schtickzelle2006} or when organisms have morphological or behavioural traits directly related to dispersal (`dispersal syndromes') \citep{bonte2003, stevens2013}. The best way to study dispersal in detail is often through laboratory experiments \citep[e.g.][]{dahirel2019} or field experiments \citep[e.g.][]{leitch2021}.\\
	
	In many systems, directly measuring movement or dispersal under natural conditions is impractical or impossible. \textbf{Indirect methods} are then the way forward to study patterns and disentangle ecological (and evolutionary) drivers. One way is through an \textbf{inverse modelling approach}, in which one aims to identify the processes that are able to reproduce a set of observed patterns through simulations of the possible mechanisms \citep{banks2019, curtsdotter2019, stouffer2019}. A form of bottom-up modelling can be applied to study emergent patterns through individual-based or agent-based models (IBMs/ABMs), where the modelling is centred around the behaviour, processes and actions of an individual or agent \citep{deangelis2014}. Another indirect way to study dispersal is by \textbf{quantifying gene flow}. Patterns of gene flow---incorporating realized dispersal and post-settlement survival and reproduction---reflect functional connectivity, especially at large spatial (and temporal) scales \citep{spear2010, kim2013}. Functional connectivity is defined as the connectivity of a landscape from the species' perspective, taking into account behavioural responses to various landscape elements. This is in contrast with structural connectivity, which analyses the landscape structure of contiguous habitat \citep{tischendorf2000}. Looking at functional connectivity through molecular genetic methods allows to look at fluxes of genetic information---which encompasses fluxes of individuals---between populations. Functional genetic connectivity is dependent on the landscape configuration and the species' traits and behaviour, such as dispersal capacity, post-settlement reproduction and subsequent passing of genetic information to the next generations.\\
	
	\section{Dune ecosystems in Belgium}
	An interesting ecosystem, which has become fragmented in Belgium and naturally consists of heterogeneous habitat types, are dune ecosystems. Due to the sandy, often dynamic, hot and dry character of dune habitats, these harbour a specific biodiversity, with especially many arthropod species of conservation interest \citep{maes2006, provoost2011, dero2021}. Particularly the coastal dune nature reserves have been thoroughly monitored and much ecological information is available for the region \citep{provoost2004a, provoost2020a}. Nature management within an ecosystem perspective in the coastal dunes is based on and has been guided by much scientific research, of which a considerable part has focused on the unique arthropod biodiversity \citep[e.g.][]{bonte2002, bonte2005, maes2006a, maes2006, dhondt2008, vandegehuchte2010, dero2021}.\\
	
	There are two main sandy ecoregions in Belgium with (remnants of) dynamic dune habitats: the coastal dunes---a narrow linear system along the Belgian coast---and the Campine region---which covers a large area in North-East Flanders. These sandy regions have gone through extensive---but different---landscape changes and fragmentation during the past decades or centuries. They differ regarding their size, extent, history and nature of fragmentation.\\
	
	\subsection{Coastal dunes}
	The coastal dunes are situated at the transition between the North Sea and the coastal plain, forming a narrow, linear system along the Belgian coast. Coastal dunes in Flanders are \textbf{geological young formations} with \textbf{calcareous} sandy soils \citep{provoost2004, decleer2007}.\\
	
	Coastal dunes harbour \textbf{several ecological specific habitats}, due to several strong gradients from coast to inland (wind, salinity, decalcification), its dynamic character and its interaction with water table and topography. The most dynamic succession stage in dune formation are blond dunes (Natura 2000 habitat 2120) with marram grass (\textit{Calamagrostis arenaria}) as the ecosystem-engineer species. They are situated at the seaside as foredunes or can reach more inland as mobile dunes. When the sand dynamics weaken, pioneer vegetation dominated by mosses, lichens, and pioneer-grasses and -sedges take over: the grey dunes (Natura 2000 habitat 2130). Grey dunes are a xerothermic (hot and dry) vegetation type where soil formation is just starting on the sand substrate \citep{provoost2004}. When the dune gets more fixated and a thin soil layer has formed, dry dune grasslands form which have a much more closed herbaceous layer than grey dunes. When no top-down processes---such as livestock grazing and (secondary) wind dynamics---are present (`unconstrained dune landscape'), dune grasslands get quickly encroached by dominant grasses and shrubs \citep{provoost2004a}. Parallel to this dry succession process (blond dune -- grey dune -- dry dune grassland -- scrub -- woodland), a wet succession is possible when (secondary) blow-outs reach the water table and wet dune slacks (dune wetlands or `pannes') are created that harbour a specific plant diversity, and for instance, provide breeding opportunities for natterjack toads (\textit{Epidalea calamita}).\\
	
	The ecotopes of the different succession stages are present in dunes---with their strong terrain changes---in quick alternation, as a mosaic. Due to this dynamic and heterogenous character, coastal dunes harbour a \textbf{specific arthropod biodiversity}, often particularly related to the open and early-succession vegetation types \citep{maes2006, provoost2011}. Such thermophilous (warmth-loving), psammophilous (sand-loving) and/or early-succession-related species are, for instance, blue-winged grasshopper (\textit{Oedipoda caerulescens}), \textit{Myrmeleotettix maculatus} (NL: knopsprietje), grayling (\textit{Hipparchia semele}), queen of Spain fritillary (\textit{Issoria lathonia}), and northern dune tiger beetle (\textit{Cicindela hybrida}), and many species of solitary bees and wasps---such as \textit{Osmia aurulenta} (NL: gouden slakkenhuisbij) and golden digger wasp (\textit{Sphex funerarius}). These species can also often be found inland, but species that occur almost exclusively at the coast are, for example, great fox-spider (\textit{Alopecosa fabrilis}); many species of ground beetles (Carabidae), such as \textit{Philorhizus notatus} and \textit{Cicindela maritima}; and many species of bees and wasps (Hymenoptera), such as coast leaf-cutter bee (\textit{Megachile maritima}), silvery leafcutter bee (\textit{Megachile leachella}), and \textit{Podalonia luffii} (NL: duinaardrupsendoder).\\
	
	Coastal sandy habitats \textbf{became fragmented} at two scales related to two interdependent processes: urbanisation and scrub encroachment. Urbanisation at the coast is quite a recent phenomenon related to upcoming coastal tourism from the interbellum period (1920's-1930's) onwards. Earlier, dunes were used for marginal domestic and agricultural purposes, such as extensive livestock grazing \citep{provoost2011}. This urbanisation resulted in a physical separation of the larger dune entities and decreased the total area of dunes significantly: the total dune area decreased by half along the Belgian coast \citep{provoost2004a}. In illustration, figure \ref{figI.1} shows a map of the dune landscape at the west coast dating from the end of the 18\textsuperscript{th} century in contrast with a recent aerial photograph: the dunes used to cover a wide area along the coast and formed one integral ecological unit, dominated by wind and sand dynamics: the `dynamic dune landscape' \citep{provoost2004a}. Nowadays, urban areas cover a large part of the landscape (Figure \ref{figI.1}B). Parallel to the urbanisation, open dune vegetations became secondarily fragmented within dune areas. Succession and scrub development increased, caused by the obstruction of sand dynamics due to urbanisation and the loss of agricultural practices, turning the dune landscape in an `unconstrained dune landscape' \citep{provoost2004a, provoost2020}. This further decreased the typical dynamic, early-succession and pioneer dune habitats \citep{provoost2011}. This scrub encroachment can be seen in figure \ref{figI.2}, with aerial pictures from the dune areas at the west coast in 1971, 1989 and the current situation (2021). The urbanisation is already prominently present in 1971 (fig. \ref{figI.2}, A.1 and B.1), and the progressing encroachment is clearly the most conspicuous landscape change during the past decades. Large herbivores have been introduced in many coastal dune reserves during the last three decades to revitalize dune dynamics and have many positive effects on plant and structural diversity \citep{provoost2004, provoost2020a}. However, they have mixed effects on local arthropod species due to intense trampling \citep{bonte2005, bonte2008, vanklink2015}. Nature management with grazing and mowing in coastal dunes does not stop succession, but vegetation and soil development are subject to new processes and continue on a different path: the `stressed dune landscape' \citep{provoost2004a}.\\
	
	
	\renewcommand{\thefigure}{I.\arabic{figure}}
	\begin{figure}[h!]
		\begin{center}
			\includegraphics[width=\textwidth]{figures/FigureI_1.png}
		\end{center}
		\begin{footnotesize}
			\caption{Two overview maps showing the municipalities of De Panne and Koksijde at the Belgian west coast. The left side of the map is the border with France. The upper panel (A) depicts the historical Ferraris map (1771-1778). The dune area between the sea (NW) and the agricultural fields (`polders') is mainly categorized as `dunes' and `grassland'. In the left bottom corner, the nature reserve Cabour, is categorized as `heathland'. The bottom panel (B) depicts the current situation (aerial photograph of 2021). The centre-left urban area is De Panne; the urban area on the right side is the municipality Koksijde (with Sint-Idesbald in the west, Koksijde-Bad east). Source: Agency for Information Flanders (source: geopunt.be) and KBR, Royal Library of Belgium. \label{figI.1}}
		\end{footnotesize}
	\end{figure}


	\begin{sidewaysfigure}[h!]
		\begin{center}
			\includegraphics[width=\textwidth]{figures/FigureI_2.png}
		\end{center}
		\begin{footnotesize}
			\caption{Aerial photographs of dune areas at the Belgian west coast from 1971, 1989 and 2021. The upper panels (A) show a complex of nature reserves next to the French border with the nature reserve Westhoek situated in the western part of the picture, including the large open sand patch in A.1 and A.2. The bottom panels (B) show a complex of nature reserves situated adjacent from panels A to the east, just west from the Yser estuary (not shown). The larger sand patch in the north-east corner of B.1 and B.2 is the nature reserve Ter Yde. Source: Agency for Information Flanders (source: geopunt.be). \label{figI.2}}
		\end{footnotesize}
	\end{sidewaysfigure}
	
	\clearpage
	
	\subsection{Inland dunes of the Campine}
	The inland sandy soils of the Campine (NL: De Kempen) are large postglacial relicts: a top layer of eolic sand was deposited in this region in the last glacial period (Weichselian glaciation, late Pleistocene). The Campine encompasses a large part of the provinces Limburg and Antwerp (east of the city Antwerp), and the most northern part of province Flemish Brabant.\\
	
	Sand blow-outs due to wind dynamics and/or trampling by livestock can give rise to mobile inland dunes, but are currently extremely rare in Belgium. \textbf{Pioneer inland dune vegetations} (Natura 2000 habitats 2310 and 2330) are structurally comparable to grey dunes at the coast, but are present on \textbf{acidic sandy soils} and associated with heathland vegetations \citep{decleer2007}.\\
	
	Acidic, inland dunes harbour, parallel to the coast, many specialist thermophilous and/or psammophilous \textbf{arthropod species}. Both sandy regions have several species in common (see above), but also have species specific for their region. For inland regions, these are for instance the European field cricket (\textit{Gryllus campestris}), ilex hairstreak (\textit{Satyrium ilicis}), silver-spotted skipper (\textit{Hesperia comma}), ladybird spider (\textit{Eresus sandaliatus}), the antlion species \textit{Myrmeleon formicarius} (NL: zwartkopmierenleeuw), and several species of solitary bees and wasps such as the bees heather colletes (\textit{Colletes succinctus}) and \textit{Andrena fuscipes} (NL: heidezandbij).\\
	
	\textbf{Fragmentation} and loss of open dune area started earlier than at the coast, due to heavy afforestation since the 19\textsuperscript{th} century \citep{dekeersmaeker2015}. Later, in the second half of the 20\textsuperscript{th} century---parallel to, but more severe than at the coast---the remaining open habitat patches became further build-up and hence, smaller and more fragmented. Secondary loss and fragmentation of the remaining sand habitat patches inland happened due to acidification and eutrophication, leading to mainly grass encroachment \citep{schneiders2020}. Inland open dunes encroach faster because the lime-poor sand does not buffer eutrophication. This makes eutrophication a more important stressor in inland dunes than at the coast.\\
	
	\subsection{Other sandy or dune remnants}
	There are other sandy remnants (with often a geologically older origin) across Belgium with relicts of, or potential for, sandy pioneer vegetations. They are not as outstretched as the coastal and Campine region, are more fragmented and have less potential for large scale wind dynamics. For instance, there are small heathland areas on sandy soil around Bruges and between Bruges and Ghent. Another example are remnants of river dunes or alluvial dunes along the rivers such as the Scheldt. There are also some sandy areas in southern Belgium (Belgian Lorraine) which are located on sandstone.\\
	\clearpage
	
	\section{Study species}
	\vspace*{1cm}
	``\textit{To identify a system in which to study an interesting problem well, we need to know a lot about the system.}'' \hspace*{\fill}\citet{travis2020}\\
	
	\vspace*{1cm}
	
	\textit{Bembix rostrata} (Linnaeus, 1758) (Hymenoptera, Crabronidae, Bembicini), is a \textbf{univoltine, specialized, solitary, gregariously nesting digger wasp} found in sandy regions in Europe. It is a solitary wasp that makes its nest on its own, although they tend to make their nests in dense clusters or aggregates \citep{larsson1986}. \textbf{They build and maintain only one nest---which contains one larva---at a time} and progressively provision it with flies \citep{nielsen1945, larsson1989, tengo1996}. \textit{Bembix rostrata} can be considered a K-selected species---as most of the digger wasp species---because it lays large eggs, shows low fecundity and displays solitary brood care \citep{larsson1989, evans2007}.\\
	
	\textit{Bembix rostrata} occurs in Europe and Central Asia. Its \textbf{distribution} ranges from the south of Scandinavian countries in the north of Europe to south Europe and reaches from western Europe into the steppes of Mongolia in Central Asia \citep{bitsch1997}. Although it has quite an extended distribution range, it is not a common species. It is restricted to sandy environments, where it can locally reach high nest densities \citep{barbier2007}. In several countries across its European distribution range, the species has declined during the past century \citep{pinkhof1924, thijsse1924, blosch2000, klein2004, barbier2007, bogusch2021}. \textit{Bembix rotstrata} is considered a red list species in several regions in Germany \citep{jacobs2000, barbier2007} and is protected in Wallonia \citep[Belgium;][]{barbier2007}. It however remains mainly unconsidered in most regions, as there is little data for this species group on local, historical and current distributions and population sizes.\\
	
	\subsection{Recognition}
	\textit{Bembix rostrata} is the largest digger wasp (13--25mm) found in Belgium and the only representative of its genus. They have a yellow and black abdomen with curved stripes of which the first (sometimes more) are interrupted. They have striking bright green eyes and distinctive rake spines (`tarsal comb') on the foretarsi \citep{evans2007}. These elongated bristles on their first pair of legs facilitate digging. The epithet in the Latin name \textit{B. rostrata} refers to the elongated rostrum (its `beak'); the Dutch name harkwesp (NL: hark, EN: rake), however, refers to this tarsal comb and their associated digging behaviour \citep{thijsse1901}. Compared to other \textit{Bembix}-species occurring in Europe, they have a dark thorax with only a few lateral yellow patches, posterior edges of the eyes with a yellow and the base of the mandibles is black. Males can be easily recognized by the contrasting complete yellow underside of the antennae (Fig. \ref{figI.3}). The females have no more than a small yellow hint at the tip of the antennae. Males often give an overall brighter yellow impression than females, and have more orange tinted hairs on their head and thorax, while these are mainly white in female. The three last antenna-segments also bend outward and are a bit flattened in males. Males have 7 abdominal segments, females 6.\\
	
	\begin{figure}[hb!]
		\begin{center}
			\includegraphics[width=\textwidth]{figures/FigureI_3.jpg}
		\end{center}
		\begin{footnotesize}
			\caption{a male (left) pursuing and inspecting a digging female (right). Notice the yellow undersides of the antennae in the male. \label{figI.3}}
		\end{footnotesize}
	\end{figure}

	\subsection{Life cycle and behaviour}
	The life cycle of \textit{B. rostrata} is schematically represented in figure \ref{figI.4}. Adults are active throughout summer, showing \textbf{protandry}: males emerge a few days earlier than females \citep{wiklund1977, schone1981, evans2007}. The males show a form of territorial behaviour \citep{asis2006}, guarding the nesting aggregations where females potentially emerge. This behaviour is sometimes described as `sun dances' in digger wasp literature, because the guarding behaviour intensifies when sun radiation increases \citep{rau1918, evans1957, schone1981, evans2007}. The males patrol the area by flying just above the dune surface, in alternating streaming flights, occasionally landing on the soil with their legs extended and inspecting the surface with their antennae. During this sun dance, they quickly launch themselves towards other passing males to chase them away, often---in their eagerness---not discriminating between conspecifics and other passing insects, such as butterflies or beetles. When females start to emerge, males become highly competitive to be the first male to grasp a female individual to copulate. They often start digging in the sand to reach females underground, forming a digging aggregation in which they try to displace each other in the centre. They often continue as a mating ball---with one female in the centre and fighting males around her---rolling over the ground or lifting itself in the air. Females that are not dug up by males are quickly grasped by patrolling males, which pounce on every encountered female on the ground or pursue them in flight \citep{schone1981}. After copulation, females will start digging, but the first few days are dedicated to \textbf{`test diggings'} before digging deeper burrows and building a \textbf{first actual nest} structure \citep{schone1981, larsson1989}. Males can still be seen in the nesting area around females from the digging phase onwards. They inspect or curiously pursue the digging females (Fig. \ref{figI.3}), but often rest, drink nectar and sunbath \citep{schone1981}. The number of active males declines within 2-3 weeks after female emergence and they are rarely seen in the second half of the nesting season \citep{evans2007}.\\
	
	Once the first nest is dug, the female brings the first prey on which she deposits an egg. Once it hatches, she \textbf{progressively provisions} the larva with flies, depending on the consumption rate of the larva: at the start only one or a few a day, and towards the end of the larval development more en masse \citep{tengo1996, asis2004}. The adults themselves feed on nectar, but females are very occasionally seen sucking the body contents of a prey, probably in moments of high need \citep{nielsen1945, evans2007}. Every time the female leaves the nest, she closes the nest entrance with a temporary closure: scraping sand into the nest entrance, often leaving digging traces visible to a trained eye. This is assumed to be an anti-parasite strategy \citep{evans1966, evans2007, polidori2009a}. The nest cycle, including building the nest, is estimated to take on average 12 days \citep{larsson1989}. When finishing the nest cycle---just before the larva turns into a prepupa---one last provisioning is carried out, after which the nest is sealed with a final closure, which includes filling of the nest tube with sand and levelling the surface around the nest entrance \citep{tengo1996}. A completely new nest is built from scratch---often preceding a few test-digging---in which a new larva is provisioned, and the nest cycle starts anew. Activity in aggregations sharply declines towards the end of August, but can stretch into September. By the end of the season, females are sometimes seen aimlessly wandering and test-digging around a nesting aggregation. Estimated from to the time needed for each nest cycle and the activity throughout the season, females \textbf{can make up to around 5 nests} (and consequently as many offspring) in one season \citep{larsson1989, tengo1996}. This is probably a conservative estimate, as this is based on a population from the northern edge of its distribution where the nesting season might be relatively short (south Sweden, \"{O}land island) and other \textit{Bembix} species are known to completely provision their nests much faster \citep{evans2007}. \textit{Bembix rostrata} overwinters as prepupa within a cocoon spun of silk lined with sand grains with a series of pores along the equatorial line, which is spun within 2 days after the final nest closure \citep{nielsen1945, evans2007}. The pupation only takes place before emergence the next summer. Consequently, \textit{B. rostrata} has only one generation each year (it is univoltine).\\
	
	\vspace*{\fill}
	\begin{figure}[ht!]
		\begin{center}
			\hspace*{-1cm}
			\includegraphics[width=1.15\textwidth]{figures/FigureI_4.pdf}
		\end{center}
		\begin{footnotesize}
			\caption{schematic overview of the life cycle of \textit{Bembix rostrata}. \label{figI.4}}
		\end{footnotesize}
	\end{figure}
	\vspace*{\fill}
	\clearpage
	
	\subsection{Nest aggregates}
	\textit{Bembix rostrata} nests are found in aggregates (clusters or nest aggregations), with reported densities of up to 10 or 13 nests/m$^2$ \citep{nielsen1945, larsson1986}. \citet{schone1981} report a maximum of 35 nests/m$^2$ along wheel tracks in sandy soil, which is probably exceptionally dense. In the earlier section \ref{spatpatt} (`Spatial patterns of clustering') I explain some main---mutually non-exclusive---hypotheses that can be put forward for the tendency to cluster nests, such as habitat heterogeneity, conspecific attraction and a selfish herd mechanism.\\
	
	\subsection{Brood parasites}\label{broodpar}
	The broodparasitic fly species \textit{Senotainia albifrons} (Miltogramminae, Sarcophagidae, Fig. \ref{figI.5}) is ubiquitously present in nesting aggregations of coastal populations in Belgium and are known to interfere with the nesting behaviour of \textit{B. rostrata} \citep{peeters2008, gallin2021}. Miltogramminae are a subfamily of brood parasites (or `kleptoparasites') of solitary bees and wasps: the \textbf{brood parasite's larvae feed upon the host's nest provisions} \citep{evans2007}. Data on the ecology and natural history of these Miltogramminae are scarce and much of their life-history and ecology remains elusive \citep{polidori2009a}. For example, how host-specific these species are is unknown, a frequently discussed topic in the context of hidden and cryptic diversity of parasites \citep{smith2008, smit2020, benda2021}. \textit{Senotainia albifrons} seems to be completely associated with \textit{B. rostrata} in Belgium and the Netherlands \citep[personal communication, Liekele Sijstermans;][]{gallin2021}, although its host specificity is uncertain due to previous reports of \textit{S. albifrons} being associated to other digger wasps \citep[e.g.][]{zolda2001}. Miltogramminae are often termed `satellite flies', as they---especially \textit{Senotainia} sp.---pursue \textit{Bembix} individuals arriving in the vicinity of their nest with prey, sometimes forming a little cloud of multiple individuals trailing a prey-carrying \textit{Bembix} female \citep{evans2007, peeters2008, polidori2009a}. The individual \textit{Senotainia} flies can be seen patrolling around nests, ready to fly after an arriving prey-carrying \textit{Bembix} female or waiting to sneak up at a female approaching her nest (Fig. \ref{figI.5}). They deposit their larvae \citep[ovi-larviposition;][]{piwczynski2017} onto the prey of the \textit{Bembix} female when she is opening her nest entrance or when she enters her nest \citep{evans2007}. Several counter-kleptoparasitic behaviours in response to such satellite flies have been described or hypothesized, several of these have been observed in \textit{B. rostrata} as well \citep{larsson1986, spofford1992}. Firstly, prey-carrying \textit{Bembix} individuals can be observed to make evasive flight manoeuvres when followed by satellite flies and by doing so, reducing the number of flies entering the nest \citep{polidori2009a}. Secondly, nest cleaning is a likely anti-parasite behaviour: females sometimes discard old prey fragments from their burrow and deposit these a few centimetres away from the nest entrance \citep{nielsen1945, evans1957, evans2007}. Thirdly, clustering nests in high densities reduces the number of larvi-positions of satellite flies for an individual nest \citep{larsson1986}.\\
	
	\begin{figure}[h!]
		\begin{center}
			\includegraphics[width=\textwidth]{figures/FigureI_5.jpg}
		\end{center}
		\begin{footnotesize}
			\caption{a female \textit{B. rostrata} landing with a prey at her nest entrance, ambushed by a \textit{Senotainia albifrons} fly. \label{figI.5}}
		\end{footnotesize}
	\end{figure}
	
	
	\subsection{Habitat}
	\subsubsection{Nesting habitat}
	\textit{Bembix rostrata} nests in coastal and inland sand dunes with sparse vegetation \citep[Fig. \ref{figI.6};][]{nielsen1945, larsson1986}. Their preferred nesting areas are situated within a gradient from open sand to more vegetated areas. At this ecotype or microhabitat, the sand is a bit more compacted, which is better for the stability of their nests compared to really sand-blown areas, and not much vegetation is present, which can interfere with their digging behaviour \citep{nielsen1945}. The vegetation types that contain their nesting habitat (Fig. \ref{figI.6}) are included in the Annex 1 of the EU habitats Directive: the optimal habitat consist of \textbf{grey dunes} in coastal areas (Natura 2000 habitat 2130) and \textbf{dry sand heaths} and \textbf{inland dunes with open grasslands} in inland areas (habitats 2310 and 2330 respectively). In the south of Belgium, \textit{B. rostrata} can be found in a peculiar extra type of nesting habitat: spoil tips from old coal mines where they make their nests in black sand \citep{barbier2007}.
	
	\begin{figure}[ht!]
		\begin{center}%
			\includegraphics[width=0.5\textwidth]{figures/FigureI_6left.jpg}%
			\includegraphics[width=0.5\textwidth]{figures/FigureI_6right.jpg}%
		\end{center}
		\begin{footnotesize}
			\caption{pictures of (on the left) \textit{B. rostrata} nests in grey dune (Natura 2000 habitat 2130) in the coastal nature reserve Westhoek and (on the right) nests in a sparsely vegetated inland dune (Natura 2000 habitat 2330) in Geel-Bel. Both habitat types are dominated by moss species and pioneer grasses and sedges on bare, nutrient-poor sand, which is at the coast calcareous and inland acidic. \label{figI.6}}
		\end{footnotesize}
	\end{figure}
	
	
	\subsubsection{Grazing management}
	%\enlargethispage{1\baselineskip}
	Nest densities of \textit{B. rostrata} severely \textbf{decline with increased trampling} by cattle and horses or vacationers. The introduction in a human-altered landscape of nature management with large herbivores---as a substitution to natural wind dynamics---can introduce new pressures to species dependent on the dynamic character of dunes \citep{bonte2005}. Although awareness is growing, arthropods are traditionally not high on the priority list when monitoring and evaluating conservation actions \citep{clark2002}. Current management measures---tailored on plants and vertebrates---can and do have negative effects on non-targeted groups, such as insects and other arthropods \citep{vanklink2015, vanklink2018}. As natural wind dynamics have severely declined in coastal dunes in Belgium, suitable nesting habitat for \textit{B. rostrata} will get encroached by grasses and shrub in the long term. Yet, grazing used as a tool to revitalize dune dynamics, have a severe impact on nesting \textit{B. rostrata} on the short term. Consequently, tools and a conceptual framework are required to reconcile the need for local management measures and the conservation of this specialized ground-digging insect.\\
	
	\subsection{The paradox of site fidelity and dynamic habitat}\label{introparadox}
	%\enlargethispage{1\baselineskip}
	In several regards and on different levels, \textit{Bembix rostrata} has been described as being a philopatric species: it is displaying high site fidelity \citep{nielsen1945, blosch2000, bogusch2021}. First, nest aggregates are regularly found \textbf{at the same location for several consecutive years} \citep{nielsen1945}. Second, \textit{B. rostrata} has been put forward as a species that \textbf{does not easily colonize vacant habitat}, even if (newly) vacant habitat is available in proximity of a known nesting location \citep{nielsen1945, blosch2000, bogusch2021}. Third, several \textit{Bembix} and related species show a tendency to \textbf{build their nest in close proximity to their previous nests} \citep{larsson1989, asis2014}, which can be seen as an alternative strategy---next to conspecific attraction---to reduce time invested in assessing the habitat quality \citep{hoi2012}. Summed up, there are indications for site fidelity on the levels of the population or across generations, on the level of (colonizing) individuals and on the level of the nesting female within a season.\\
	
	However, owing to the \textbf{typical dynamic character of the species' habitat} (pioneer dune vegetations), philopatry on the population and dispersing individual level would be highly disadvantageous for the species' persistence in dynamic dune ecosystems \citep{bogusch2021}. High dispersal and colonisation capacity are a selective advantage for insect species living in non-persistent habitats \citep{denno1996, bowler2005}. Other species that also depend on early-successional habitat---such as the grey-backed mining bee (\textit{Andrena vaga})---exhibit good dispersal and colonisation capacities despite a sedentary life style during breeding \citep{exeler2008, cerna2013}. \textit{Bembix rostrata}'s preferred nesting habitat are vegetation types that are naturally occurring heterogeneously or fragmented within dune ecosystems and ephemeral on the temporal scale of up to a few decades (a relatively low turnover rate). In natural dune systems, pioneer vegetations suitable for nesting will eventually get encroached by grasses or shrubs by natural succession, or be blown over by sand due to secondary revitalisation of wind dynamics. The same succession will also ensure the appearance of new pioneer vegetations. Being dependent on such dynamic nesting habitat is in contradiction with the described poor colonisation capacity of \textit{B. rostrata} and seems to be an ecological paradox.\\
	
	\subsection{A flagship species?}
	Most biodiversity measures used to monitor and evaluate nature conservation actions are focused on vertebrates and plants, and if conservation studies take into account insects, they are biased towards butterflies \citep{clark2002}. However, the last couple of years, the decline in abundance and distribution of many insects has raised a widespread public and political awareness on their biological value. Many drivers of this decline---such as habitat loss and fragmentation---have been identified \citep{cardoso2020, harvey2020, didham2020, wagner2021, welti2021}. Native bees have gained much more interest in conservation in the last decade as pollination by wild pollinators is recognized as an indispensable ecosystem service \citep{potts2016}. The majority of wild bees and flower-visiting solitary wasps nest in soils \citep{cope2019, antoine2021}. Especially for this group, nesting resources are often overlooked in management due to a focus on floral resources for pollinators \citep{kimoto2012, buckles2019}. \textit{Bembix rostrata} is an \textbf{emblematic representative of soil-nesting solitary bees and wasps}---and more specifically of digger wasps---a group often neglected in nature conservation. It is an excellent representative for intact grey dunes (Natura 2000 habitat 2130) and the structural equivalent in inland acidic dunes (Natura 2000 habitats 2310 and 2330). These \textbf{habitats are part of the Annex 1 of the European Habitats Directive} (92/43/EEC), which is a list of rare and characteristic habitat types that are targeted for conservation in the Natura 2000 ecological network of protected areas. From my own experience through communication with nature enthusiasts and volunteer or professional nature managers in Belgium, \textbf{people perceived \textit{B. rostrata} as an emblematic and valuable species to have in a nature reserve}. I had the general positive feeling that people were excited if the species was present in their nature reserve, often wanted to advertise its presence and were eager to consider the species in nature management decisions.\\
	
	And finally, why not, when trying to grasp intricate and diverse ecological aspects of insects, start the journey with a mesmerizing digger wasp that caught my attention and gave a spark?\\
	\newpage
	
	\section{Objectives and outline}
	The \textbf{overarching aim} of this dissertation is to better understand the ecological processes behind spatial patterns of nests and functional connectivity with the digger wasp \textit{Bembix rostrata} in fragmented dune areas as a study system. I study ecological patterns in \textit{B. rostrata} at several hierarchical spatial scales related to nest site selection, nest densities, population genetic structure and landscape connectivity (Fig. \ref{figI.7}, \ref{figI.8}). By studying these patterns and processes, I hope to gain insight into the ecological characteristics relevant for the species' distribution and population dynamics. This to eventually understand the ecological conditions relevant to---and formulate recommendations for---their conservation.\\
	
	In the \textbf{introduction}, I point out the importance of movement, dispersal and gene flow to ecological and evolutionary processes. I elaborate on the link of movement with nest spatial clustering and habitat fragmentation and explain how to measure movement directly (through tracking) or indirectly (with inverse modelling and gene flow). I introduce the dune ecosystems in Belgium, where the study species focal to this dissertation occurs: \textit{Bembix rostrata}. I give detailed information regarding the species biology, ecology and behaviour. Finally, I give an overview of the objectives and research questions.\\
	
	To explore the possibility of directly tracking individual movement with a tracking device, I quantify the impact of tags on the behaviour of the study species in \textbf{chapter \ref{chapter1}}. I combine this with a systematic review to evaluate if and how possible impacts or biases were taken into account in studies that use tracking devices on terrestrial arthropods.\\
	
	To investigate the influence and relative importance of both the environment and behavioural mechanisms on clustering of nests, I develop an individual-based model (IBM) in \textbf{chapter \ref{chapter2}} to simulate the spatial organisation of nests. This model integrates environmental quality---estimated with a microhabitat suitability model---and two types of behavioural mechanisms: local site fidelity and conspecific attraction. The spatial point pattern and network pattern from a field study are compared with those from simulations with an inverse-modelling approach, to derive the most likely mechanisms giving rise to these emerging patterns. The nest aggregates are used as nodes in the network analysis, where `within-patch dispersing' individuals can make consecutive nests in different aggregates \citep[i.e. small scale breeding dispersal;][]{ronce2007}. The scale of chapter \ref{chapter2} concerns the spatial scale of a local population or patch (Fig. \ref{figI.8}) with several nest aggregates in a grey dune and the temporal scale of one nesting season or summer.\\
	
	An external pressure, or biotic filter, in nesting areas relevant for \textit{B. rostrata} occurrence, is grazing management. In \textbf{chapter \ref{chapter3}}, I study the impact of sheep grazing as a top-down process on the density of nests in a control site and two sites with different levels of sheep grazing, using a fully-crossed design (before--after control--impact; BACI). Other large grazers (cattle, horses) are known to have a large detrimental effect on nests, acting as a top-down regulating process on potential nest presence. I hypothesized that the impact of sheep trampling is less severe. I put this in a nature management framework to discuss if sheep grazing can be used as a tool to reconcile the need for grazing in dune nature management and the conservation of this emblematic dune insect. The scale of this chapter comprises a larger spatial extent of different populations in areas a few hundred metres from each other (Fig. \ref{figI.8}) and temporally encompasses 3 years.\\
	
	I want to shed light on the population genetic structure of coastal and inland populations in \textbf{chapter \ref{chapter4}} to elucidate on which scale (in or between regions) gene flow could be limited. I investigate the genetic structure within and between coastal and inland regions in Belgium with population genetics methods suitable for haplodiploid species. The spatial extent---the largest in this dissertation---encompasses Belgium and the coast of northern France (Fig. \ref{figI.8}). The population genetics methods reflect realized dispersal across multiple generations.\\
	
	Apart from the quality of the habitat itself, the matrix---in which the habitat is embedded---can have an influence on the exchange of individuals between populations: the functional connectivity. In \textbf{chapter \ref{chapter5}}, I aim to investigate how the human-altered coastal dune landscape influences gene flow between populations with landscape genetics methods. I optimize the relationship between individual genetic distances and landscape resistance distances to deduce resistance values of landscape types. The spatial scale spans a 15 km stretch of human-altered coastal dune landscape at the Belgian west coast (Fig. \ref{figI.8}). The type of genetic distances used in this chapter, based on pairwise individual genetic relatedness, takes into account only the last few generations.\\
	
	This outline can be summarized in five main research questions:
	\begin{itemize}
		\item Q1: Can tracking devices be used to study the movement behaviour of \textit{Bembix rostrata} and how is the potential impact taken into account in other studies on arthropods?
		
		\item Q2: what is the relative influence of the mechanisms---habitat suitability, site fidelity and conspecific attraction---giving rise to the spatial pattern of nest clustering?
		
		\item Q3: how does sheep grazing affect nest densities?
		
		\item Q4: What is the population genetic structure in both coastal and inland populations and is gene flow limited within or between these regions?
		
		\item Q5: how does the heterogenous matrix in between habitat affect gene flow in a human-altered coastal dune landscape?
		
	\end{itemize}
	
	By connecting the different research questions and combining results from across the chapters, I want to gain insight into (1) the \textbf{paradox of high site fidelity} (section \ref{introparadox}), (2) the ecological drivers and constraints to \textbf{population spread} and (3) implications for \textbf{conservation}. Understanding the mechanisms of nest site selection in detail (chapter \ref{chapter2}) combined with patterns of gene flow (chapters \ref{chapter4} and \ref{chapter5}), could provide insights into the first two questions. These insights, combined with results from chapter \ref{chapter3}, can help pinpoint the vulnerabilities relevant for conservation of this emblematic sand wasp and might---with caution and nuance---help to formulate management guidelines.\\

\vspace*{2cm}
	\begin{figure}[hb!]
		\begin{center}
			\includegraphics[width=\textwidth]{figures/FigureI_7.png}
		\end{center}
		\begin{footnotesize}
			\caption{conceptual representation of the five research chapters (Ch), indicating spatial scales and situating the chapters across a schematic landscape. \label{figI.7}}
		\end{footnotesize}
	\end{figure}
\clearpage

	\begin{sidewaysfigure}
	%\begin{figure}[hb!]
		\begin{center}
			\includegraphics[width=0.94\textwidth]{figures/FigureI_8.pdf}
		\end{center}
		\begin{footnotesize}
			\caption{maps situating the (relative) geographical locations of the study areas of each research chapter (Ch). Dots represent sample locations (chapters \ref{chapter4} and \ref{chapter5}) or nests (chapter \ref{chapter2}). Details are in the corresponding research chapters. \label{figI.8}}
		\end{footnotesize}
	\end{sidewaysfigure}



\clearpage
\thispagestyle{plain}
\hbox{}
\clearpage

\CenterWallPaper{1}{pictures/CH1-1.png}
\newpage{\thispagestyle{empty}\clearpage}
\cleardoublepage
\ClearWallPaper

\CenterWallPaper{1}{pictures/CH1-2.png}
\newpage{\thispagestyle{empty}\clearpage}
\hbox{}
\clearpage
\ClearWallPaper

\CenterWallPaper{1}{pictures/CH1-3.png}
\newpage{\thispagestyle{empty}\cleardoublepage}
\ClearWallPaper

%%%%%%%%%%%%%%%%%%%%%%%%%%%%%%%% CHAPTER TWO  %%%%%%%%%%%%%%%%%%%%%%%%%%%%%%%%%%%
\setlength{\thumbwidth}{0.8cm}
\setlength{\thumbheight}{1cm}
\tikzset{
	thumb/.style={
		%   draw=black,
		fill=light-gray,
		text=black,
		minimum height=\thumbheight, %\thumbheight,
		text width=\thumbwidth,
		outer sep=0pt,%   outer sep=10pt,
		font=\sffamily\Large,
	}
}
\pagestyle{mainmatter}
\chapter{The neglected impact of tracking devices on terrestrial arthropods}\label{chapter1}
\chaptermark{Impact of tracking devices}
\lettergroup{\thechapter}	

\begin{flushright} \color{gray}Femke Batsleer\\
	Dries Bonte\\ Daan Dekeukeleiere\\ Steven Goossens\\ Ward Poelmans\\ Eliane Van der Cruyssen\\
 Dirk Maes\\ Martijn L. Vandegehuchte

\vspace*{2cm}
Adapted from: Batsleer et al. (2020) \textit{Methods in Ecology and Evolution},\\ \textbf{11}$($3$)$, 350--361. DOI: 10.1111/2041-210X.13356
\vspace*{\fill}
\end{flushright}


\noindent \color{gray} $\lhd$ European hornet (\textit{Vespa crabro}) equipped with a PIT-tag, photo by Daan Dekeukeleire.\\
\noindent $\lhd\lhd$ A female \textit{Bembix rostrata} flying in a grey dune area.
	
\color{black}


\newpage

\renewcommand\thesection{\arabic{chapter}.\arabic{section}}
\renewcommand{\thefigure}{\arabic{chapter}.\arabic{figure}}
	\section{Abstract}
	\begin{enumerate}
		\item Tracking devices have become small enough to be widely applied to arthropods to study their movement. However, possible side effects of these devices on arthropod performance and behaviour are rarely considered.
		
		\item We performed a systematic review of 173 papers about research in which tracking devices---Radio Frequency Identification (RFID), harmonic radar, and radio telemetry tags---were attached to terrestrial arthropods. The impact of such tags was quantified in only 12$\%$ of the papers, while in 40$\%$ the potential impact was completely disregarded. Often-cited rules of thumb for determining appropriate tag weight had either no empirical basis or were misconstrued.
		
		\item Several properties of a tracking device (e.g. weight, balance, size, drag) can affect different aspects of an arthropod's life history (e.g. energy, movement, foraging, mating). The impact can differ among species and environments. Taken together, these tag effects can influence the reliability of obtained movement data and conclusions drawn from them. We argue that the impact of tracking devices on arthropods should be quantified for each (1) study species, (2) tag type, and (3) environmental context. As an example, we include a low-effort impact study of the effect of an RFID tag on a digger wasp.
		
		\item Technological advancements enable studying the movement of arthropods in unprecedented detail. However, we should adopt a more critical attitude towards the use of tracking devices on terrestrial arthropods. The benefits of tracking devices should be balanced against their potential side effects on arthropods and on the reliability of the resulting data.
		
	\end{enumerate}

\vspace*{\fill}
\noindent \textbf{Keywords:} \textit{Bembix rostrata}, harmonic radar, insect, invertebrate, non-target effect, radio telemetry, RFID, tag
	
\clearpage
	
	\section{Introduction}
	
	Animals need to move to find sufficient food, mating partners, shelter, or to colonise new locations. Movement is thus central to their fitness, and ultimately population and community persistence \citep{jeltsch2013, bonte2017}. Movement ecology has evolved towards a mature sub-discipline of ecology \citep{nathan2013}, in part driven by technological advances in data collection and analysis methods. Especially vertebrate movements are now quantified with an unprecedented precision and frequency across the world \citep{kays2015}.\\
	
	Over the last decades, tracking devices (radio, harmonic, or RFID tags, see Table 1.1) have become small enough to fit terrestrial arthropods with, yielding unparallelled opportunities to study their movement in the natural environment. The use of such devices has proven fruitful in studies of arthropod conservation, pest control, pollination, behaviour, social interactions, and movement including orientation, habitat use, and dispersal \citep{kissling2014}. These techniques have enabled several breakthroughs. For instance, the development of orientation flights in honeybees is now better understood \citep{capaldi2000} and detailed migration routes of dragonflies and butterflies are being tracked \citep{wikelski2006, knight2019}. Radio-tracking techniques also help inform and evaluate conservation measures for endangered species \citep[e.g.][]{hedin2002, stringer2008, kelly2010}.\\
	
	The use of tracking devices has an extensive history in bird and mammal research. Two decades ago, researchers already warned for the potential impact of tracking devices on animal behaviour and performance and hence the reliability of gathered movement data \citep{mech2002, godfrey2003}. For instance, supposedly harmless devices caused a fivefold decrease in foraging efficiency of Magellanic penguins \citep{wilson2004} and heavy collars reduced survival of migratory caribou by 18$\%$ \citep{rasiulis2014}. Recent meta-analyses show that tracking devices have a negative effect on different traits in birds \citep{barron2010, bodey2018, brlik2020}. These effects can be partially mitigated, for instance by the mode of attachment \citep{white2012} and by minimizing tag size \citep{casper2009, bodey2018}. However, as tracking devices are becoming lighter and smaller, they are increasingly used on smaller animals \citep{portugal2018}. As researchers are pushing the boundaries of the size at which species are presumed ``taggable'', the impact of the device on the animal's performance is not necessarily reduced. Furthermore, while the `5$\%$ rule', i.e., the device mass should be no more than 5$\%$ of the animal mass, is often used as a guideline for swimming and flying vertebrates \citep{aldridge1988, kenward2000}, it has a weak empirical basis and is often disregarded \citep{barron2010, omara2014, portugal2018}. These ethical and scientific drawbacks have resulted in recent guidelines and recommendations regarding the use of tracking devices on birds and mammals \citep{wilson2006, casper2009, omara2014, kays2015, mcintyre2015} and a call for more systematic documentation of potential effects \citep{geen2019}.\\
	
	Despite the increased use of tracking devices in arthropod movement ecology, a similar critical assessment of their impact on behaviour and performance is currently lacking. Several studies acknowledge that such devices can influence different aspects of the performance of arthropods \citep{gui2011, lenaour2019}, but little has been done to quantify or mitigate these effects. Some practical rules of thumb exist for certain arthropod groups, but they rarely have an empirical basis. For the honeybee \textit{Apis mellifera}, for example, it is often reasoned that the tag should not weigh more than an average nectar or pollen load \citep{capaldi2000}.\\
	
	To address this knowledge gap, we performed a systematic review \citep{gates2002, berger-tal2019} of 173 published studies in which tracking devices were used on terrestrial arthropods. We evaluated to what extent the potential impact of the devices was considered and whether there was a quantitative impact assessment. Here, we focused on the effect of the tag itself and not on possible additional effects such as those of handling an arthropod or of the attachment method (but see discussion). We argue that for each arthropod tracking study the impact of the specific tracking device on the study species should be evaluated in the light of the research question at hand. We also provide an example of a simple experiment to quantify the impact of a tracking device on the behaviour of the digger wasp \textit{Bembix rostrata} in the field (Box 1.1). Quantifying the impact of tracking devices does not necessarily imply a large effort and should become standard when tracking arthropods.\\
	\clearpage
	\hbox{}
	\vspace*{\fill}
	\begin{table}[h!]
		\begin{center}
			\begin{footnotesize}
				\caption*{\textbf{Table 1.1}: Properties of tracking devices used in arthropod movement ecology.} \label{Tab1.1}
				\begingroup
				\setlength{\tabcolsep}{6pt} % Default value: 6pt
				\renewcommand{\arraystretch}{1.5} % Default value: 1
				\begin{tabular}{p{3cm} p{2.8cm} p{2.8cm} p{2.8cm}}
					
					\toprule
					& \textbf{RFID- or PIT-tags} & \textbf{Harmonic radar} & \textbf{Radio telemetry} \\
					\midrule
					\textbf{Tags with (active) or without (passive) battery} & Passive & Passive & Active \\
					\arrayrulecolor{black!30}\midrule[0.3pt]
					\textbf{Weight} & Lightweight (ca. 0.1 -- 100 mg) & Lightweight (ca. 2 -- 50 mg) & Heavy (ca. 200 mg -- 1 g)\\
					\arrayrulecolor{black!30}\midrule[0.3pt]
					\textbf{Detection range} & a few centimetres & up to 800 m (stationary) or a few metres to maximum 80 m (portable) & a few hundred metres\\
					\arrayrulecolor{black!30}\midrule[0.3pt]
					\textbf{Unique identification possible} & Yes: unique code is read & No & Yes: frequency and pulse interval of radio signal can be chosen when the tag is manufactured\\
					\arrayrulecolor{black!30}\midrule[0.3pt]
					\textbf{Influence of terrain and vegetation} & None & Very high & Often considerable\\
					\textbf{Technique of signalling} & The scanner (or reader) detects the tag when it passes through the magnetic field. & Radar system (stationary or portable) emits a signal that is reflected by the tag (or diode) and received by the antenna at half the frequency. & Tags emit a radio signal at regular time intervals, received by hand-held or stationary antenna.\\
					\arrayrulecolor{black}\bottomrule
				\end{tabular}\endgroup
			\end{footnotesize}
		\end{center}
		%\end{sidewaystable}
	\end{table}
	\vspace*{\fill}
	\clearpage

	\section{Material \& Methods}

We systematically reviewed the published literature on the use of tracking devices on terrestrial arthropods. We searched ISI Web of Science (conducted on 26 August 2019) and scanned reference lists to include additional papers if accessible through the internet or via the authors (Fig. \ref{fig1.1}). Only primary studies about terrestrial arthropods were selected. Non-arthropod terrestrial invertebrates, such as slugs and snails (Gastropoda), were not included as their ecology is highly different from that of terrestrial arthropods. Other kinds of tags or markings \citep[e.g. numbered labels, coloured markings, or biotelemetry; the latter being tags to measure physiological and energetic variables remotely;][]{cooke2004}, were also not considered. Details on the search can be found in supplementary table S1. We selected 173 relevant papers (Fig. \ref{fig1.1}), extracted information about the species, tracking devices, and the assessment of possible effects of these devices. Relevant variables collected for this study are species name, order and family, IUCN status, pest status, tag type, mode of movement, duration of tracking, mass of species and/or tag, the tag-to-body mass ratio, and whether the impact was discussed or quantified (for a full list, see online data and supplementary BibTeX-file). As the tag-to-body mass ratio is principally considered in studies, it is central to this quantitative review, but we additionally discuss other important considerations when assessing tag impact on arthropods.\\

\begin{figure}[hb!]
	\begin{center}
		\includegraphics[width=8.5cm]{figures/Figure1_1_cut.png}
	\end{center}
	\begin{footnotesize}
		\caption{PRISMA flow diagram (\url{http://www.prisma-statement.org/}) of the procedure for selecting studies for a systematic review. \label{fig1.1}}
	\end{footnotesize}
\end{figure}

\clearpage
		
	\section{Results}
\subsection{Taxonomy and species status}
	In total, 94 species of terrestrial arthropods were studied in the 173 reviewed papers, consisting of 212 unique studies of a species-tag combination. Most studies concerned research in which tracking devices were applied to Coleoptera (beetles) and Hymenoptera (bees, wasps, and ants; see Fig. \ref{fig1.2}). The latter can be explained by a disproportionate abundance of studies on economically important pollinators such as the honeybee, \textit{Apis mellifera} (40/212; 18.9$\%$), and bumblebees belonging to the species \textit{Bombus terrestris} (19; 9.0$\%$) or other \textit{Bombus} species (11; 5.2$\%$). These studies together make up 79.5$\%$ of the Hymenoptera studies. Across all studies, 24 species (25.5$\%$) were considered pest or invasive species or potentially damaging to the economy. Most of the species (69; 73.4$\%$) were not evaluated for an IUCN status. 18 species (19.1$\%$) had an IUCN status of Least Concern and one species was listed as Data-Deficient (\textit{Apis mellifera}). Seven species (7.2$\%$) were Near Threatened (NT), Vulnerable (VU), or Critically Endangered (CR; four Coleoptera and three Orthoptera; for details see online data).\\
	
	\begin{figure}[hb!]
		\begin{center}
			\includegraphics[width=10cm]{figures/Figure1_2.png}
		\end{center}
		\begin{footnotesize}
			\caption{The number of times an arthropod from a certain order was studied using tracking devices. If more than one species was studied in a paper, these were counted separately. \label{fig1.2}}
		\end{footnotesize}
	\end{figure}

	\subsection{Mode of movement and duration of tagging}
44 of the species (46.8$\%$) move mainly by flying. 14 species (14.9$\%$) mainly climb or crawl, but are also able to fly. 36 species (38.3$\%$) are not able to fly and are mainly crawling or climbing species.\\

Tracking duration ranged from a few hours to several days, weeks or even an arthropod's lifetime, but was sometimes not (11) or unclearly (25) reported in the papers. Whether the tag was recovered or not was not mentioned in 108 papers (62.4$\%$).\\

	\subsection{Tag-to-body mass ratio}
Half of the reviewed papers did not provide the tag-to-body mass ratio (i.e., ratio of tracking device mass to the mass of the arthropod species or individuals), or enough information to derive this ratio (Fig. \ref{fig1.3}). Of the studies reporting ratios or the information necessary to calculate them (109 out of 212), those using radio tags show the highest tag-to-body mass ratios (Fig. \ref{fig1.4}). For all tag types, about 30$\%$ (32 out of 109) stays below the 5$\%$ rule as is used for flying or swimming vertebrates.\\

\begin{figure}[hb!]
	\begin{center}
		\includegraphics[width=10cm]{figures/Figure1_3.png}
	\end{center}
	\begin{footnotesize}
		\caption{Venn diagram of the information provided by papers regarding the tag mass, species or individuals' mass, and their ratio. 48 papers did not provide any of this information. \label{fig1.3}}
	\end{footnotesize}
\end{figure}

\begin{figure}[hb!]
	\begin{center}
		\includegraphics[width=10cm]{figures/Figure1_4.png}
	\end{center}
	\begin{footnotesize}
		\caption{histogram of the tag-to-body mass ratio for studies providing sufficient information (ratio given or calculable) for RFID, harmonic radar, or radio telemetry tags. If a paper tagged more than one species, these were counted separately. The red line indicates the 5$\%$ threshold used for flying or swimming vertebrates. \label{fig1.4}}
	\end{footnotesize}
\end{figure}
	
	\subsection{How impact was assessed or discussed}
	69 papers (39.9$\%$) did not discuss or mention any possible impact or bias due to the tracking devices. 83 papers (48.0$\%$) only considered the possible impact qualitatively, ranging from shortly mentioning that the tag did not impede normal behaviour to elaborately discussing possible bias. Capture-mark-recapture ratios or post hoc correlations of general measurements were sometimes provided, e.g. the relation between tag-to-body mass ratio and distance travelled. However, as such correlations lack a control group, we did not consider the impact quantified. 21 papers (12.1$\%$) for in total 36 (17.0$\%$) studies on tag-species combinations, quantified the impact of the tags on survival, movement, or primary behavioural aspects. Three of these papers considered the effect of the glue used, one of these solely considered the glue. The papers quantifying the impact used a design with before--after tagging comparisons (4) or control--impact groups (17). Measured responses included walking speed, number of or time until upward flights, and rates and duration of flower visits. 23 out of 36 studies reported no impact (four of these of only the glue). In three cases the impact depended on the tag-to-body mass ratio, and in 10 cases an impact was observed without taking the body mass into account. The impact assessment method and variables measured in these 21 papers varied considerably (Table S3).\\
	
	Papers on beetles in our review often erroneously cite \citet{boiteau2001} to argue that a tag can weigh 23-33$\%$ of an arthropod's body mass \citep[e.g.][]{dubois2008, chiari2013}. \citet{boiteau2001} quantified that to have a minimal impact on the number and quality of upward flights, the tracking device mass should be no more than 23-33$\%$ of the beetle's acceptable extra loading. This `acceptable extra loading' is the maximum mass that can be added to the individual before all flight capacity is lost. However, for the focal species (Colorado potato beetle, \textit{Leptinotarsa decemlineata}), this acceptable extra loading is less than 10$\%$ of its own body mass. Thus, to have minimal impact on number and quality of upward flights, the threshold should actually be between 2.3-3.3$\%$ of the body mass \citep[mention 2.4$\%$]{boiteau2001}. A related study found that for the same species the average walking speed was significantly reduced with a weight of only 2.2$\%$ of the body mass \citep{boiteau2010}. However, the same study reports that for another beetle species, plum curculio (\textit{Conotrachelus nenuphar}), a tag weighing 14.7$\%$ of its body mass did not reduce the mean walking speed.\\
	
	For Hymenoptera, especially the honeybee, another guideline is often blindly adopted: the extra weight of the tag must not exceed an average (or maximum) nectar or pollen load \citep[e.g. in][]{capaldi2000, reynolds2009}. Typical pollen and nectar loads for a honeybee weigh 20$\%$ and 35$\%$ of the body mass, respectively, and maximum 80$\%$ \citep{feuerbacher2003}. Another example is that \textit{Bombus} species are known to carry loads weighing as much as their own body mass in pollen load \citep{hagen2011}. Fitting them with radio tags weighing 50-100$\%$ of their body mass alters their foraging behaviour \citep{hagen2011}. Therefore, the overall performance of these Hymenoptera can be expected to be influenced by tracking devices.\\
	
	\clearpage	
	
\tcbset{box/.style={colframe=black,sharp corners,enhanced jigsaw, colback=white}} % Common settings

	\begin{tcolorbox}[box, breakable, boxrule=1pt,toprule at break=1pt,extras={toprule at break=1pt}]
		\subsection*{Box 1.1: case study of the impact of tracking devices on arthropod behaviour using a BACI design}
		We aimed to use RFID tags to study nest visitation rates in the digger wasp \textit{Bembix rostrata} (Hymenoptera, Crabronidae), and first measured the impact of these tags on general behaviour. To objectively measure nest visitation, wasps need to spend the same amount of time exerting different types of behaviour with and without tag, independent of body size. The feasibility of using these tags was considered high, as harmonic radar tags weighing a similar percentage of the body mass have been used on honeybees \citep{capaldi2000}. The wasps were able to fly with a tag and displayed different types of behaviour. A qualitative assessment, as in most studies, would likely conclude that the tags had no or negligible impact. However, the impact assessment showed that the tags quantitatively altered the behaviour of the wasps, especially in individuals with larger wing loading (ratio of body mass to wing area). The method was therefore considered inappropriate for our purposes and hence abandoned. The data for this impact assessment were collected over 4 days, which shows that such a pilot study does not necessarily require a large effort.\\
		
		\textit{B. rostrata} individuals weigh about 200 mg (measured on two live individuals in the lab with a microbalance). In the field, we used a calliper to measure inter-tegular distance (ITD), maximum width of forewing (FW), and maximum length of forewing (FL). ITD is highly correlated with dry mass in bees and wasps \citep{cane1987, ohl2007} so we used it as a proxy for mass. The area of an ellipse with FW and FL as axes was used to approximate wing area. Wing loading was then calculated as ITD divided by wing area. RFID tags measured 8 mm $\times$ 2 mm and were encapsulated in glass vials to withstand the digging in the sand. The entire tag weighed 55 mg (Eccel Technology LTB, UK). These tags could be detected through a 6.5 cm diameter loop antenna. They were glued on the dorsal thorax (scutum) with cyanoacrylate glue. The tag weighs about 27.5$\%$ of the body, which is well below the maximum possible weight of prey carried by a \textit{Bembix} species \citep{asis2011}. Using a fully crossed (BACI) design \citep{smith2002}, we tested the impact of the tags on the behaviour of the individuals in cubic insect cages with a side length of 30 cm (BugDorm, Taiwan). For 12 minutes the behaviour of the individual was noted every 30 seconds as flying, crawling, resting, washing, digging, or biting. After twelve minutes, half of the wasps were tagged and the other half was similarly manipulated but without attaching the tag to mimic the tagging procedure. After the treatment, the behaviour of the wasp was again recorded for 12 minutes. The test was performed on 94 individuals. Tag effects may differ between sexes and thus ideally should be assessed in both males and females. However, we only sampled males because they emerge earlier in the season, which enabled us to test tag impact before using the technique on the entire population later in the season. Furthermore, \textit{B. rostrata} is an endangered species so we did not want to risk potential negative effects on females as they are more critical to the survival of the population. Data were collected from 18 until 21 June 2018.\\
		
		Percentages of a certain behaviour were analysed using a generalized linear mixed model with a binomial distribution and logit link function. Wasp individual was included as a random effect to account for repeated measurements. Fixed effects were BA (before or after; 2 levels), CI (control or impact; 2 levels), and WL (wing loading; continuous) or ITD (continuous). Our main interest is the tag effect (BA:CI-interaction) and how this tag effect depends on WL or ITD (3-way interaction). We used the function `mixed' from the R-package `afex' \citep{singmann2019}, which uses likelihood ratio tests.\\
		
		We did not have enough records of biting, digging, or washing behaviour for meaningful analysis. In addition to a highly significant before--after effect on the time spent flying, resting, and crawling (see online data), the tag increased the amount of time spent resting (DF=8, p=0.059) and decreased the amount of time spent flying (DF=8, p=0.016). Moreover, both these effects were stronger for individuals with a larger wing loading (DF=8, p=0.059) (Fig. B1, supplementary table S2 and figure S1).\\
		
		%\begin{figure}[H]
			\begin{center}
				\includegraphics[width=11cm]{figures/Figure1_B-1.png}
			\end{center}
			\begin{footnotesize}
				\textbf{Figure B1}: Impact of tags on proportion of time spent flying. Lines are predictions, points (before) and triangles (after) represent raw data. Embedded is a picture of a tagged digger wasp \textit{Bembix rostrata}.
			\end{footnotesize}
		%\end{figure}
	\end{tcolorbox}

\clearpage
	\section{Discussion}
	40$\%$ of papers reporting on the use of tracking devices on terrestrial arthropods provided minimal justification for the chosen method or altogether ignored possible side effects. As side effects of tags raise ethical concerns \citep{drinkwater2019} and may induce---possibly strong---bias in the generated movement data, they should be quantified and mitigated. The weight, size, drag, and balance of the tag as well as the glue used to attach it may affect the performance of the tagged animals (e.g. foraging behaviour, mating, predation pressure, stress; Fig. \ref{fig1.5}). Tag effects could affect different arthropod traits simultaneously, resulting in feedbacks (e.g. energy expenditure affects foraging behaviour and vice versa). These effects on the individuals will affect, in their turn, the generated movement data and conclusions drawn from them in e.g. ecosystem science, pest control, conservation, and fundamental science (Fig. \ref{fig1.5}). The use of these techniques for tracking arthropods is anticipated to further increase in the future, as they provide unprecedented and invaluable data on movement. However, a more critical attitude is needed.\\
	
	We advise to always provide the information needed to calculate the tag-to-body mass ratio in a transparent way to enable inference of the relative magnitude of the added weight. It is clear from our review that adopting a fixed threshold for this tag-to-body mass ratio as a practical guideline risks oversimplifying the study system and underestimating possible impacts \citep{portugal2018}. Tags of the same type can vary greatly in mass depending on signal range, antenna type, coating, etc. Furthermore, the arthropods' body size can vary among and within populations or between seasons \citep[e.g. for honeybees:][]{sauthier2017}, and can be correlated with specific foraging behaviours \citep{peat2005}. The mass of several individuals from the wild or lab population should be provided, or at least a size measure relevant for the arthropod species group (e.g. inter-tegular distance, wing dimensions; Box 1.1). Half of the studies in this review failed to provide enough information to derive this tag-to-body mass ratio.\\
	
	\begin{figure}[hb!]
		\begin{center}
			\includegraphics[width=\textwidth]{figures/Figure1_5.png}
		\end{center}
		\begin{footnotesize}
			\caption{Conceptual diagram of possible effects of tags on arthropods and the bias they can create in research results for different research fields. For details and examples, see main text. \label{fig1.5}}
		\end{footnotesize}
	\end{figure}
	
	Arthropods are often able to lift half to three times their own body weight. This is especially true for bees, as carrying these loads is an integral part of their life history \citep{boiteau2004}. This may nevertheless still affect their flight performance, with potential cascading, complex effects on other aspects of their life history. Generalizing the possible effect of the extra load of a tag is not straightforward. For instance, a honeybee is perfectly able to fly with additional loads of pollen and nectar but this will increase metabolic rate \citep{wolf1989, feuerbacher2003}. \citet{kim2016} found that the take-off propensity of honeybees was negatively affected by the attachment of harmonic radar tags. The tags weighed only 3 mg \citep[about 3$\%$ of their body mass,][]{osborne1997}. Other factors such as the tag's size or effect on balance could have interfered with taking off. Nonetheless, many studies use this tracking technique on honeybees, because the tags are relatively light, but ignore these other potential sources of bias. Dispersal is a life-history trait that may be highly sensitive to such potential cascading effects of tags. Dispersal capacity is frequently measured using radio telemetry without acknowledging possible sources of bias due to the tags \citep[e.g. in][]{kelly2010, chiari2013, thomaes2018}. This may cause underestimation of dispersal capacity.\\
	
	We need to consider different kinds of tag effects (Fig. \ref{fig1.5}). In addition to effects of tag mass, a tag's size and location on the body may affect drag and balance during movement. For example, the tag's drag effect was as large as its mass effect in a small migrant bird, explaining the decreased return rate to the breeding area \citep{bowlin2010}. We also need to consider multiple arthropod performance traits simultaneously (Fig. \ref{fig1.5}). In birds, it has been shown that while a single trait can be affected minimally, the combined impact of multiple affected traits can be substantial \citep{barron2010, bodey2018}.\\
	
	Additionally, effects of tags are context-dependent, as they can differ among seasons, environments, or life-history stages. This could alter the impact of a tag on an individual over time or in space \citep[e.g. in a passerine:][]{snijders2017}. Moreover, tag effects may interact with environmental factors. Individuals carrying tags might respond more strongly to stressors (e.g. pesticides, parasite infection, and displacement) than non-tagged individuals \citep{desouza2018}. In honeybees, effects of stressors on behaviour are often assessed with harmonic radar \citep[e.g.][]{tison2016} or RFID tags \citep[e.g.][]{lach2015}. However, the possible interaction between tag and stressor effect is not yet clearly established, as testing the stressor effect in untagged arthropods is very challenging. Guidelines from the few papers assessing an impact of tags on arthropods are therefore not always readily applicable to studies using the same arthropod species and tracking method in a different environment or focusing on different aspects of the species' biology.\\
	
	The potential tracking device impact should be assessed separately for each species \citep{gui2011}. The relative load that causes minimal impact might vary considerably among different species as shown by the studies on potato beetle and plum curculio \citep{boiteau2001, boiteau2010}. The same reasoning holds for flying or non-flying species (53.2$\%$ of the species in the reviewed papers mainly move by modes other than flight). Papers adopting a misconstrued and therefore arbitrary and untested threshold consequently risk to underestimate the costs of extra weight for these arthropods.\\
	
	Larger individuals within a species or population are often selected to tag and track \citep[e.g.][]{hedin2002, hardersen2007, lenaour2019}. In flying arthropods, wing loading (ratio of body mass to wing area) can scale positively but also negatively with body mass depending on the species \citep{boiteau2001, gilchrist2004, darveau2005}, implying that larger individuals are not necessarily better fliers. Other differences in ecology and behaviour might be related to body size within a species \citep{hillaert2018}. For instance, in a study on the Near-Threatened beetle \textit{Osmoderma eremita} \citep{nieto2010}, it was observed that radio-tagged beetles survived better than beetles marked for classic capture-mark-recapture analysis \citep{legouar2015}. Apparently, this was a side-effect of selecting large individuals for radio-tagging, since larger individuals survive better in general in this species. This possible bias risks overestimating population size and survival. Similarly, ecological or morphological differences between sexes could introduce bias (e.g. an over- or underestimated dispersal rate in a species with sex-biased dispersal).\\
	\clearpage
	
	Not only the tag itself but also other equipment properties and handling procedures can affect the performance of the tagged insects. Such---likely multiplicative---effects need to be accounted for in study designs as well. Glues for instance can be toxic and directly decrease survival \citep{boiteau2009}. The handling required to tag an arthropod can cause stress to an individual as well as to a complete hive \citep{desouza2018}. Cold or CO\textsubscript{2} sedation used during handling can have strong and long-lasting effects on performance \citep{poissonnier2015}. The tag itself might cause entanglement in the vegetation \citep{boiteau2011} or the RFID readers can constrain entry to the hive \citep{desouza2018}. Electromagnetic radiation can directly or indirectly lower performance and survival \citep{cucurachi2013}, but at least for honeybee hives equipped with standard RFID readers mortality rates were unaffected by the radiation \citep{darney2016}. Studies in which no tags are used, such as Capture-Mark-Recapture using simple paint to mark individuals, or studies using glued plate markings, are also documented to influence grooming, aggression, cooperation, and foraging behaviour \citep{packer2005a, desouza2012, switzer2016}.\\
	
	\subsection{Measuring the impact}
	Most studies that quantified the impact of tags found none, and some demonstrate the feasibility of studying tag impact in a quantitative, transparent, and reproducible way. Studies from Boiteau, Colpitts and colleagues are in this sense exemplary in their assessment of the effects of harmonic radar tags and glue on survival, behaviour, movement, and entanglement for three pest insect species \citep[i.a.][for full list see table S3]{boiteau2001, boiteau2009, boiteau2010, boiteau2011}. As a result of this effort, the tag properties required to minimize unwanted side effects and the types of behaviour amenible to measurement are well-established for these species. Studies seemingly lacking power to detect tag effects because of small sample sizes or indicating a trend \citep[e.g.][]{lee2016} should include a power analysis.\\
	
	As effects may be complex as well as context- and species-dependent, we recommend a threefold consideration of tag type, species, and environmental context, tailored to the type of research question. The types of tags to be used should be tested on the focal arthropod species, ideally including intraspecific variation (e.g. mass or wing loading, see Box 1.1), in the relevant environmental context before proceeding with the planned study. The variables measured during the impact assessment should be as relevant as possible for the planned study. For instance, the types of behaviour that should be assessed will depend on whether the tracking method is aimed at locating nests, studying short-term local movements, or quantifying dispersal behaviour. Such a (pilot) impact assessment should be able to integrate several aspects of the tracking devices (weight, drag, balance...), species (mode of movement, metabolism...) and environment (habitat type, temperature, humidity ...) in a simple study design. The benefits of using certain tracking devices should always be balanced against the possible negative effects on the individuals' well-being and reliability of the generated data \citep{barron2010}.\\
	
	We provide an example of such a pilot impact study (Box 1.1) with a fully crossed or BACI (before--after control--impact) design. We need to stress that impact studies such as the one reported in Box 1.1 can lead to abandonment of the envisioned methodology, as in our case. We urge the communication of such `negative' results to the scientific community, especially as we suspect a high publication bias in this topic \citep{cassey2004}. Such efforts not only contribute to openness and transparency in ecology \citep{parker2016}, but also improve research efficiency by preventing others from adopting unsuitable methodologies.\\
	
	\section{Conclusion}
	Recent technological advancements have yielded unprecedented opportunities to study the movement, behaviour, dispersal, and orientation of arthropods using tracking devices, which would not be feasible by any other means \citep{kissling2014}. As a consequence, the use of tracking devices to study arthropods has strongly increased over the past decade. As any novel technology, new tracking devices need to be critically evaluated before being applied in empirical research. For vertebrates, protocols regarding ethical research methodologies are well-adopted and monitored by ethics committees \citep{kays2015, drinkwater2019}. Guidelines for the use and impact assessment of tracking devices for vertebrates are actively being developed and refined \citep{barron2010, mcintyre2015, portugal2018, geen2019}. While the development of devices for tracking arthropods has sprung from technological knowledge gained from vertebrate research, ethical concerns and guidelines about impact assessment were not co-adopted. We should rectify this discrepancy by developing solid guidelines for assessing the impact of arthropod tracking devices and the potential bias this impact creates. Our review highlights that a tag's impact needs to be assessed in relation to the species, intraspecific variation, device type, and environmental context relevant to the proposed research. We should be wary of systematic bias caused by underestimating or neglecting the impact of tracking devices on arthropods. The potential achievement of novel insights using a certain tracking method should be balanced against the quantitatively and objectively assessed impact on the tagged individuals and reliability of the resulting data \citep{barron2010, kays2015}. We caution against adhering to arbitrary thresholds or other rules \citep{portugal2018}. The current generation of tracking devices holds the potential to reveal crucial information about arthropod ecology with important consequences for biodiversity conservation, pest control and ecosystem functioning research. However, a more critical mindset is required if we are to fully exploit this potential.\\
	\vspace*{\fill}
	\clearpage
	
	\subsection*{Acknowledgements}
	FB is supported by Research Foundation Flanders (FWO). We thank Laura Decorte and Celien Van De Velde for help with the case study. We thank two anonymous reviewers for their constructive comments on an earlier version of the manuscript.

	\subsection*{Data Accessibility}
	Data deposited in the Dryad repository:\\ 
	\noindent \url{https://doi.org/10.5061/dryad.x0k6djhfq}
	
	\subsection*{Author contributions}
	FB, DD, DB and MLV conceived the ideas and designed methodology; FB, DD, SG, WP, EVdC, MLV collected data; FB, WP, EVdC, MLV analysed data; FB and MLV led the writing of the manuscript. DB and DM critically revised the manuscript. All authors contributed significantly to the drafts and gave final approval for publication.

	\subsection*{Supplementary material}
	Supplementary files can be found alongside the published article or in the dedicated github repository: \url{http://doi.org/10.1111/2041-210X.13356} or\\
	\url{http://github.com/FemkeBatsleer/SuppPhD}
	


\cleardoublepage
\thispagestyle{plain}
\hbox{}
\clearpage

\CenterWallPaper{1}{pictures/CH2-1.png}
\newpage{\thispagestyle{empty}\clearpage}
\cleardoublepage
\ClearWallPaper

\CenterWallPaper{1}{pictures/CH2-2.png}
\newpage{\thispagestyle{empty}\clearpage}
\hbox{}
\clearpage
\ClearWallPaper

\CenterWallPaper{1}{pictures/CH2-3.jpg}
\newpage{\thispagestyle{empty}\cleardoublepage}
\ClearWallPaper

	%%%%%%%%%%%%%%%%%%%%%%%%%%%%%%%%%%%%% CHAPTER 2  %%%%%%%%%%%%%%%%%%%%%%%%%%%%%%%%%%%%%%%%%%
	%\CenterWallPaper{1}{CH3.jpg}
	\newpage{\thispagestyle{empty}\cleardoublepage}
	%\ClearWallPaper
	\pagestyle{mainmatter}
	\chapter{Behavioral strategies and the spatial pattern formation of nesting} \label{chapter2}
	\chaptermark{Behavior and nest spatial pattern}
	\lettergroup{\thechapter}	

	\begin{flushright} \color{gray} Femke Batsleer\\ Dirk Maes\\ Dries Bonte\\


	\vspace*{2cm}
	Adapted from: Batsleer et al. (2022) \textit{The American Naturalist},\\ \textbf{199}$($1$)$, E15--E27. DOI: 10.1086/717226
	\end{flushright}
\vspace*{\fill}
\noindent \color{gray} $\lhd$ An aggregate of nests of \textit{B. rostrata}.\\
\noindent $\lhd\lhd$ A female \textit{B. rostrata} arriving at her nest.

\color{black}
\newpage
	
	
		\section{Abstract}
		Nesting in dense aggregations is common in central-place foragers, such as group-living birds and insects. Both environmental heterogeneity and behavioral interactions are known to induce clustering of nests, but their relative importance remains unclear. We developed an individual-based model that simulated the spatial organization of nest-building in a gregarious digger wasp \textit{Bembix rostrata}. This process-based model integrates environmental suitability, as derived from a microhabitat model, and relevant behavioral mechanisms related to local site fidelity and conspecific attraction. The drivers behind the nesting were determined by means of inverse modelling in which the emerging spatial and network patterns from simulations were compared to those observed in the field. Models with individual differences in behavior that include the simultaneous effect of a weak environmental cue and strong behavioral mechanisms yielded the best fit to the field data. The nest pattern formation of a central place foraging insect cannot be considered as the sum of environmental and behavioral mechanisms. We demonstrate the use of inverse modelling to understand complex processes that underlie nest aggregation in nature.\\
		
		\vspace*{\fill}
		\noindent \textbf{Keywords:} approximate Bayesian computation, habitat selection, integrated nested Laplace approximation (INLA), spatial self-organization, Crabronidae, Hymenoptera
		
		
\clearpage
	
	\section{Introduction}
	The ideal free distribution predicts that organisms optimally distribute themselves across resource patches to minimize resource competition \citep{kacelnik1992}. This process leads to the sorting of individuals according to their niche \citep{hutchinson1957, holt2009} and induces spatial patterns of high densities where the environment is most suitable \citep[e.g. in shorebirds;][]{swift2017}. Animal aggregations are also widely documented in species inhabiting homogeneous environments, not only in social species but also in non-social central-place foraging wasps, lizards and birds \citep{stamps1988, tarof2004, evans2007}. These inherent spatial patterns can emerge from behavioral and internal dynamics, such as the interplay between positive and negative density dependence, and are an example of spatial self-organization \citep{fortin2005, rietkerk2008, bayard2010, bradbury2014}. The spatial clustering of group-living animals, and more specifically nest clustering, has been explained through several behavioral hypotheses that are intrinsically linked with benefits related to group size \citep{krause2002}. Groups can provide protection against predation or parasitism \citep[for example via a selfish herd; Supplement S1;][]{hamilton1971, larsson1986} or against climatic extremes \citep{gilbert2008} or simply increase foraging efficiency \citep{clark1986}. Both environmental and behavioral mechanisms can result in the spatial clustering of individuals or their nests. However, their relative importance remains elusive, especially for invertebrates. Moreover, it is unclear at which level these processes can vary: can mechanisms vary among individuals or even during an individual's lifetime? The relative strengths of environmental and behavioral mechanisms for spatial clustering are expected to vary among systems \citep[for example, in analogy to bottom-up and top-down regulation of communities;][]{hunter1992}.\\
	
	Information use is central to any decision making, and thus also to settlement. Information can be personal when individuals directly use cues from the environment, or inadvertently social when information is generated by the behavior (e.g., foraging, fighting, mating) of other conspecific individuals \citep{danchin2004, dall2005}. Personal information, in addition to self-assessment of the environment, can consist of a female's experience with previous nesting locations, which results in local site fidelity \citep{hoi2012, asis2014}. Variation in information use among individuals may arise from heterogeneity in these strategies where `producers' rely on personal information and `scroungers' on (inadvertent) social information, or from individuals switching between these sources of information \citep{barnard1981, coolen2007}.\\
	
	The spatial clustering of nests is often regarded as a clearly separated and stepwise process where individuals first collectively select suitable environments at larger spatial scales after which internal dynamics (e.g. competition, attraction, individual and social learning) come into play \citep{melles2009, swift2017}. The prevailing insights are acquired by the analysis of complex spatial patterns and/or from behavioral experiments \citep[e.g.][]{polidori2008, melles2009, asis2014}. Environmental heterogeneity and internal dynamics are anticipated to act simultaneously, or even synergistically. Since the emergent patterns of a complex system cannot be predicted from the sum of the underlying components \citep{bradbury2014}, more integrated approaches are needed. Inverse modelling, which can identify the processes that reproduce a set of observed patterns, has been extremely useful in this perspective \citep{banks2014, curtsdotter2019}.\\
	
	Here, we apply such an inverse approach to understand the contribution of environmental and behavioral mechanisms in nest aggregations of the ground-nesting digger wasp \textit{Bembix rostrata}. We combine a microhabitat suitability model with an Individual-Based Model (IBM) to investigate the processes underlying the spatial dynamics of nest pattern formation as observed in the field. We include three mechanisms in the IBM: i) environmental suitability, ii) local site fidelity and iii) social cues. The direction and strength of these mechanisms can vary at the population-level (uniform for all), between (individually fixed) and within individuals (individually flexible). The simulated spatial point and network patterns are compared with those recorded in the field using Approximate Bayesian Computation (ABC) to select the most likely combination of environmental and behavioral mechanisms that underlie the observed nesting patterns.\\


	\section{Material \& Methods}
	\subsection{Study species}
	\textit{Bembix rostrata} (Linnaeus, 1758) (Hymenoptera, Crabronidae, Bembicinae) is a specialized, highly philopatric, gregariously nesting digger wasp found in sandy regions of Europe. They inhabit sun-exposed sand dunes with sparse vegetation \citep{larsson1986} and are sensitive to trampling \citep{bonte2005}. Adults are active from June to August; females construct one nest burrow at a time in which a single larva is progressively provisioned with flies \citep{nielsen1945, field2005}. A female can make a maximum of five nests each with one offspring \citep{larsson1989}. Several kleptoparasitic fly species (Sarcophagidae) lay their larvae \citep[ovi-larviposition;][]{piwczynski2017} on the prey provided by females of \textit{Bembix} species \citep{nielsen1945, evans2007}. A selfish herd pattern has been observed in \textit{B. rostrata} with regard to such brood-parasites \citep{larsson1986}, where the incidence of brood-parasitism per nest decreased with higher nest densities (we found a similar pattern in our field data between \textit{B. rostrata} and \textit{Senotainia albifrons}: Supplement S1).\\
	
	\subsection{Study site and sampling}
	Field data were collected in the summer of 2016 in the nature reserve De Westhoek in De Panne (51\textdegree 04'38''N, 2\textdegree 33'37''E, Belgium), in a study plot of approximately 40 $\times$ 90 m$^2$. Surveys took place during 30 days of favorable (sunny and warm) weather conditions for \textit{B. rostrata} \citep{schone1991} between 28 June and 15 August. Female wasps were individually tagged with a colored and numbered plastic plate on the thorax (Opalith Zeichenpl\"{a}ttchen) and nests were marked with small, handcrafted flags. We recorded visually when an individual started a nest, entered the nest with a prey, and if the prey was `infected' by kleptoparasitic flies (\textit{Senotainia albifrons}, Miltogramminae, Sarcophagidae), i.e. a fly was able to mount the prey carried by a female at the nest entrance. The study area was covered several times per day, to sample each nest aggregate as equally as possible. The position of each nest was measured with a Trimble GPS (accuracy of 2 cm; Trimble Inc., USA). Remote-sensing imagery was collected using a drone (Rpaswork.com and Didex.be) equipped with a multispectral camera (Red, Green, Blue and Near Infrared bands) at the end of the flight season. These images are processed towards a digital elevation model (DEM; pixel size 2.4 $\times$ 2.4 cm$^2$) and the Normalized Difference Vegetation Index (NDVI; pixel size 1.1 $\times$ 1.1 cm$^2$) \citep{pettorelli2013}. Insolation (as an indicator of the microclimate; pixel size 7.2 $\times$ 7.2 cm$^2$) is calculated from the DEM with the `solar radiation tool' and slope (pixel size 7.2 $\times$ 7.2 cm$^2$) with the ´surface toolset', both extensions of `spatial analyst' in ArcGIS \citep{esri2011}.\\
	
	\subsection{Statistical analyses}
	The workflow of the analyses is shown in Fig. \ref{fig2.1}.\\
	\clearpage\hbox{}\vspace*{\fill}
	\begin{figure}[ht!]
		\begin{center}
			\includegraphics[width=\textwidth]{figures/Fig2_1.pdf}
		\end{center}
		\begin{footnotesize}
			\caption{conceptual figure of the workflow of the analyses. (1) The relation between the nest positions and the environment is investigated by building a microhabitat suitability model with INLA. Because the clustering is much higher than expected based solely on the microhabitat model, (2) the environment and behavioral mechanisms (i.e. local site fidelity and conspecific attraction) are simultaneously modelled with an Individual-Based Model (IBM). The simulations differ in strength of the mechanisms. This is implemented through strength parameters that represent the probability of the mechanisms being present. The presence of the mechanisms can vary on three different levels: the population, inter-individual and intra-individual. (3) The simulations from the IBM are compared with the spatial pattern of the field data using Approximate Bayesian Computation (ABC) to infer which submodels best approach the field data. \label{fig2.1}}
		\end{footnotesize}
	\end{figure}
\vspace*{\fill}
	\clearpage

	\subsubsection{The microhabitat suitability map}
	We used Integrated Nested Laplace Approximation (INLA) \citep{rue2009, lindgren2011, martins2013} to build the microhabitat suitability model. INLA is a Bayesian approach that allows for spatially auto-correlated residuals of the environmental data related to nest location \citep{zuur2017} (Supplement S6). We used a generalized linear mixed model with binomial distribution with logit-link for the response variable and a spatial dependency structure modelled with the Mat\'{e}rn covariance function (see online code). NDVI (vegetation) and insolation (microclimate) were used as (normalized) explanatory variables. Every nest was considered as a presence point and absence/zero data points were generated by selecting an equal number of random points that were at least 1m from any nest within the study plot. As the study plot was searched intensely, we considered the generated points as true absences. Models were compared using Watanabe-Akaike information criterion \citep[WAIC:][]{watanabe2010, gelman2014}, computed within the inla function \citep{rue2005, rue2009}. To confirm our a priori choice of a simple linear model, we considered interactions between both variables and an additional covariate, local slope. WAIC differences for these models with a linear model only including NDVI and insolation were less than 3, so the simple linear model was preferred. The data were split into 70$\%$ training and 30$\%$ evaluation data. As a cross-validation, the final model was run 10 times using different randomly chosen training and evaluation sets each time. To assess the predictive power of the models, AUC (area under the curve) was calculated with the R-package ROCR \citep{sing2005}. Sensitivity (true positive rate; predictive performance of presences), specificity (true negative rate; predictive performance of absences) and balanced accuracy (overall true rate) were calculated with the R-package caret \citep{kuhn2008}. To calculate the latter three performance measures, predictions were transformed into 0/1 using the prevalence criterion \citep{manel2001, liu2005}: predictions that are larger than the prevalence threshold (proportion of presences/absences in the evaluation datasets: $\pm$ 0.5 in our case) are classified as 1 and the other predictions as zero. The plotting of the spatial field, the spatial residuals which INLA corrects for, shows if the degree of clustering was a higher (hot spots) or a lower (cold spots) than expected based on the covariates (NDVI and insolation) in the microhabitat model. Different models that considered different spatial scales were compared (each with cross-validation included): buffers between 0.1 m and 10 m were drawn around each nest at 9 different radii and the mean of each variable was calculated inside those buffers \citep{qgisdevelopmentteam2020}. The models with buffer scales 0.1, 0.2, 0.5, 1 and 2 m had similar WAIC, AUC, sensitivity, specificity, and spatial field plots. We proceeded with the 1 m scale\footnote{This 1 m scale is similar to the 1 m buffer around nests used to choose pseudo-absences. However, if the microhabitat scale here would have proven smaller than 1 m, we would also have reconsidered the the procedure of choosing pseudo-absences, as overfitting might then be a pitfall. In the case as it is, buffers around the presences and absences for the microhabitat model are still overlapping, making sure there is still enough variation taken into account at the environmental transition between presences and absences, avoiding overfitting.}, as the suitability predictions of this buffer scale were detailed but also smooth, to balance overfitting and poorer estimates (Supplement S7). Within each cross-validation, predictions of the final model were projected back onto the field study plot within a grid of 0.5 $\times$ 0.5 m$^2$. A detailed habitat suitability map was created with the average of these predictions (with a 0 to 1 scale of probability of nest presence). This map was used as input for the IBM model (Fig. \ref{fig2.1} and see section \ref{InvMod} `Inverse modelling with IBM and ABC') to be used as the environmental cue for habitat selection. We did not include the uncertainties of the probabilities in the IBM model, as we deem this extra level of stochasticity negligible.\\
	
	\subsubsection{Spatial point pattern and network analyses}\label{SPP}
	A point pattern analysis was carried out with the R package Spatstat \citep{baddeley2015}. Spatial clustering of nests was investigated using Ripley's K at scales between 0 and 40 m, where a higher K than the calculated expected random distribution at a certain scale or radius is indicative of a clustered pattern within that radius, and a lower K of a regular pattern within the radius \citep{baddeley2015}. Ripley's K-values were transformed to represent the relative change compared to complete spatial randomness (CSR) at a scale r with the formula: $K_{rel}(r)=(K(r)-CRS(r))/(CRS(r))$\\
	
	To assign the nests to different nest aggregates, a k-means cluster analysis was implemented. The optimal number of nest aggregates was 11, considering an elbow-plot \citep{kassambara2020}, visualization of the clusters and topography of the area (Supplement S8).\\
	
	A network analysis was carried out with the R package igraph \citep{csardi2006}. Nest aggregates from the k-means cluster analysis were considered network nodes in the network analysis and the consecutive nests of individuals as links (or edges) between network nodes. As such, the network nodes are aggregates that are spatially grouped nests and the individuals moving among (and within) the aggregates to a consecutive nest are the links of the network. Five network metrics were calculated for this directed network, defined according to \citet{farine2015}: i) the number of loops, which is the total number of links or subsequent nests of individuals; ii) the number of internal loops, which is the relative number of links that return to the same node, thus an individual that makes consecutive nest in the same aggregate; iii) transitivity (or clustering coefficient), which quantifies how densely nodes are connected: a high transitivity indicates that triads (trios of nodes) have a high degree of being mutually linked; iv) density (or connectance), which is the number of links divided by the total number of possible links between all clusters; v) reciprocity, which is the relative number of reciprocal links between nodes.\\
	

	\subsubsection{Inverse modelling with IBM and ABC}\label{InvMod}
	We developed an individual-based model (IBM) to simulate and eventually identify the potential drivers behind the species' nesting dynamics using a pattern-oriented approach \citep{grimm2005}. Individual wasps, with their different sets of behaviors, are the entities of simulation within a spatially explicit environment. The ODD (Overview, Design concepts, Details) protocol \citep{grimm2006, grimm2010} is added in Supplement S2, where the detailed explanation, assumptions and parameters are explained. Here we briefly discuss the general set-up.\\
	
	The three mechanisms---environment, local site fidelity and conspecific attrac\-tion---were combined in the model using strengths: the probabilities of the mechanisms being present. Variation in the presence of the mechanisms was possible at three levels as modelled in different submodels or strategies (Fig. \ref{fig2.1}): 1) the population level: the mechanisms used are uniform across all the individuals in the same population; 2) inter-individual: the mechanisms can vary among individuals in a population but are fixed for an individual; 3) intra-individual: the mechanisms can vary within an individual's lifetime and are thus flexible. The null model in which random locations were chosen within the study area was used as the fourth submodel. The flow for an individual \textit{Bembix} female when selecting a nesting site was as follows: first, a random position in the area is sampled; then, the average suitability according to the focal mechanisms is assessed after which that position can be stochastically selected according to the calculated suitability (or probability). When the position is not selected, a new one is sampled according to the same procedure (see ODD protocol Supplement S2).\\
	
	The habitat suitability map serves as a baseline for the environmental cue: the suitability values are used as probabilities for settling. Local site fidelity is implemented as a Gaussian distribution centered around the previous nest, with one parameter $\sigma_{lsf}$ defining the width of the distribution. As such, positions closer to the previous nest have a higher probability of being chosen. Conspecific attraction is coded in two steps: first, the parameter $range_{ca}$ defines the radius of the circle in which the number of other nests are counted. Second, settlement probability is implemented with a sigmoid function \citep{kun2006, broly2016}, with the number of nests counted in the first step as the dependent variable. Two parameters define the sigmoid curve: $mindens_{ca}$ is the intercept and $\sigma_{ca}$ the scale parameter of the function. The Boolean parameter \textit{beh-excl} defines if conspecific attraction and local site fidelity are mutually exclusive: both mechanisms could be strongly present, while not jointly determining an individual's nest site selection. To optimize convergence time and remain within reasonable ecological boundaries, we applied uniform priors in a valid parameter space (Supplement S2, Table S2.1).\\
	
	To initialize the model and to define boundary conditions, the following properties derived from the field study were used: total number of individuals sampled in the study site (432); total number of days to run a simulation (30); distribution of the number of nests initiated each day; distribution of the number of nests per individual; and distribution of time between subsequent nests. The latter three are used as probability distributions when initializing the \textit{Bembix} population (Supplement S1 section 3.1; online code).\\
	
	We verified that the priors were not biasing the analysis towards one of the submodels, by setting them widely for 100,000 simulations (prior predictive check; Supplement S3). Following this analysis, we restricted prior ranges by excluding those ranges where parameters covaried. This step ensures that certain parameter values are not redundant and improves convergence of the actual simulations.\\
	
	Each one of the four submodels was run 250,000 times with parameters randomly sampled in the prior parameter ranges (Supplement S2, table S2.1). Summary statistics for each model simulation were calculated as described in the above section \ref{SPP} `Spatial point pattern and network analyses'.\\
	
	The model was evaluated using Approximate Bayesian Computation \citep{csillery2010, beaumont2010, vandervaart2016}, more specifically rejection-ABC following \citet{vandervaart2015} with the R package abc \citep{csillery2012}. This method is based on the difference between each simulation and the observed field data in the summary statistics of patterns of interest. As this method cannot compare summary statistics that are continuous functions, such as Ripley's K, values of Ripley's K at a discrete set of distances were chosen as part of the summary statistics. The complete set of summary statistics were six Ripley's K values transformed as described in the previous section (at distances between 2 and 40 m) and five network metrics (number of total loops, number of internal loops, transitivity, density, and reciprocity), as described in the previous section. The sum of the differences between normalized summary statistics of field and simulation data was calculated. The summation was obtained from weighing each summary statistic (calculated from all simulations) with the complement of their average Pearson's correlations (1-$\rho$). As such, the summary statistics are corrected for their dependence structure. Minimizing this distance, the 1000 simulations (0.1$\%$) closest to the observed field data were retained. We calculated the percentage of accepted simulations for each of the three submodels. Pairwise Bayes' factors were calculated for each submodel. A Bayes' factor of more than 3 for a model comparison implies that the first model is more substantially supported by the data \citep{kass1995}. Since ABC model selection may be vulnerable to bias \citep{vandervaart2015}, we scrutinized submodel performance by cross-validation of the model selection (Supplement S4).\\
	
	The prior and posterior distributions of the summary statistics and parameters were compared as part of the posterior predictive inspection (Fig. \ref{fig2.3} and \ref{fig2.4}). This allowed us to evaluate how similar the patterns produced by the IBM are to the patterns in the field, estimate parameters and derive which processes are important to reproduce these patterns.\\
	
	We derived data from the same dataset at the three different steps in our ana\-lysis. Generally, using the same dataset for parameterization and model evaluation risks overfitting and overconfidence in the focal model \citep[termed `adaptive overfitting' in machine learning;][]{roelofs2019}. Therefore independent datasets should ideally be used as input and evaluation of the model. We argue that the use of complementary or auxiliary data in the different components of the analysis minimizes the risk of overfitting for the following reasons: 1) As the microhabitat model is adjusted for spatial autocorrelation with the use of INLA, the data used here are independent of the clustering of the nests causing the model predictions to fundamentally represent the effect of environment on nest presence. 2) Initialization of the IBM is carried out using auxiliary data or probability distributions from the field data, to have comparable boundary conditions for the simulations. Parameters in the IBM are not derived from the field data and are implemented using wide priors. 3) The ABC compares simulations with field data based on larger-scale emergent patterns of the point pattern: clustering and network metrics.\\
	\clearpage
	
	\section{Results}
	\subsection{Field study}
	A total of 432 individual digger wasps were tagged and 561 nests were marked with flags. Test-digging holes \citep{nielsen1945} and tagged individuals that were not observed at a nest were excluded. Of those 432 wasps, 330 had one nest, 78 had two, 21 had three and three wasps had 4 nests (Supplement S9). 150 nests (36.5$\%$) were parasitized by the kleptoparasitic fly \textit{Senotainia albifrons} (Supplement S1). The maximum density of nests in a grid of 1 m $\times$ 1 m was 10 nests/m$^2$. Data are deposited in the Dryad Digital Repository: \url{http://doi.org/10.5061/dryad.g79cnp5q8} \citep{batsleer2021}.
	
	\subsection{Microhabitat suitability model}
	The microhabitat suitability model has a high predictive performance: AUC $\pm$ SD of the cross-validated final model was 96.0 $\pm$ 1.3$\%$. The sensitivity $\pm$ SD (true positive rate) was 73.9 $\pm$ 4.1$\%$, the specificity $\pm$ SD (true negative rate) 96.7 $\pm$ 2.1$\%$ and balanced accuracy $\pm$ SD (overall true rate) 85.3 $\pm$ 1.9$\%$. The model therefore performed better in predicting nest absences than nest presences. Nevertheless, predictive performance was overall high. High NDVI-values decreased and high insolation increased nesting suitability. So, sunny sites with a low vegetation cover have a higher probability of containing nests (Supplement S6). For every run, zero was excluded from the 95$\%$ credibility intervals of the effect sizes, indicating that the signs of the effect sizes were clearly determined (this is a Bayesian approach to evaluating statistical significance at a specified level). The predictions (ranging from 0 to 1) for nest suitability in the whole study plot are shown in Fig. \ref{fig2.2}a.\\
	
	Fig. \ref{fig2.2}b shows the spatial random components (spatial field or `residuals' which INLA corrects for in the analysis: Supplement S6) that had clear hot and cold areas, indicating higher and lower clustering respectively than expected based on NDVI and insolation. Such cold and hot spots indicate unmeasured variables that vary in space or other underlying mechanisms that cannot be attributed to the environment, such as behavior.\\
	
	\clearpage
	\hbox{}
	\vspace*{\fill}
	\begin{figure}[h!]
		\begin{center}
			\includegraphics[width=\textwidth]{figures/Fig2_2.pdf}
		\end{center}
		\begin{footnotesize}
			\caption{a) Predictions for microhabitat suitability (0-1) on the field study plot based on vegetation (NDVI) and sun irradiance (insolation), mean of all 10 iterations. b) Posterior mean values of the spatial field, from one of the iterations, others were similar. The spatial field shows where the spatial autocorrelation, corrected for with INLA, deviated from zero indicating higher clustering (hot spots; dark orange) or lower clustering (cold spots; dark purple) than expected based on NDVI and insolation. This indicates that other mechanisms must be involved in the clustering of the nests. Black dots are nest locations. Pixel size is 50 $\times$ 50 cm$^2$ on ground. Cartesian coordinate reference system used is Belgian Lambert 72, epsg:31370. \label{fig2.2}}
		\end{footnotesize}
	\end{figure}
	\vspace*{\fill}
	\clearpage
	
	\subsection{Spatial point pattern and network analyses}
	Clustering of nests was present up to 10 m (blue dots in Fig. \ref{fig2.3}). 60$\%$ of consecutive nests were made in the same aggregate (internal loops, blue line \ref{fig2.3}). The network was not densely connected internally (transitivity, blue line \ref{fig2.3}); had a low ratio of possible links present (density, blue line \ref{fig2.3}); and had a low level of reciprocal connections (reciprocity, blue line \ref{fig2.3}). The distances among consecutive nests had a median of 4.30 m, a mean of 11.01 m and a maximum of 81.53 m (see histogram in Supplement S9).\\
	
	\begin{figure}[h!]
		\begin{center}
			\includegraphics[width=\textwidth]{figures/Fig2_3.pdf}
		\end{center}
		\begin{footnotesize}
			\caption{Violin plots of the distribution of prior summary statistics of all 1,000,000 simulations (grey), the distribution of the posterior summary statistics of the 1000 (0.1$\%$) best models (green), and field data (blue dots or lines). Yellow distributions are from the null model (submodel random). Summary statistics are of two types: spatial clustering (Ripley's K, relative change compared to complete spatial randomness CSR) and network metrics (all and internal loops, transitivity, density and reciprocity). \label{fig2.3}}
		\end{footnotesize}
		\end{figure}
	\clearpage
	
	\subsection{Inverse modelling with IBM and ABC}
	The prior predictive check made it possible to restrict two parameters' prior ranges for the actual simulations (Supplement S3). Data from simulations are deposited in the Dryad Digital Repository: \url{http://doi.org/10.5061/dryad.g79cnp5q8} \citep{batsleer2021}.\\
	
	The submodel with fixed (inter-individual) strategies was substantially better supported than the uniform strategy. ABC analysis showed to a lesser extent the superior performance for the flexible strategy (intra-individual) compared to the uniform one. The fixed and flexible (inter- and intra-individual) submodels perform equally well based on the cross-validation of model selection (Supplement S4) and Bayes' factors (Table \ref{Tab2.1}). Similar results were obtained with the 10,000 (1$\%$), 500 (0.05$\%$) and 100 (0.01$\%$) best simulations (Supplement S5), indicating that the ABC-analysis converged for the number of simulations run.\\
	
	Both Ripley's K values and network metrics calculated from the selected simulations matched the field data well (Fig. \ref{fig2.3}), despite Ripley's K having a large range at small distances.\\
	
	Strengths for conspecific attraction and local site fidelity were on average 0.739 (median=0.760, Q\textsubscript{1}=0.617, Q\textsubscript{3}=0.895) and 0.674 (median=0.716, Q\textsubscript{1}=0.535, Q\textsubscript{3}=0.853; both skewed distributions to 1; fig. \ref{fig2.4}a), which means that these behavioral mechanisms are strongly present in the population. The strength for environment was on average 0.209 (median=0.174, Q\textsubscript{1}=0.079, Q\textsubscript{3}=0.305; distribution skewed toward 0; fig \ref{fig2.4}a), which means the environmental cue is less strongly present: on average 3.5 and 3.2 times weaker than conspecific attraction and local site fidelity, respectively.\\
	
	The estimated range of the conspecific attraction was on average 2.29 m (median=1.98 m, Q\textsubscript{1}=1.29 m, Q\textsubscript{3}=3.05 m; fig. \ref{fig2.4}a). The two parameters for the sigmoid response function for conspecific attraction were both low ($mindens_{ca}$: mean=-8.13, median=-8.29, Q\textsubscript{1}=-9.25, Q\textsubscript{3}=-7.27; $\sigma_{ca}$: mean=3.16, median=1.74, Q\textsubscript{1}=0.69, Q\textsubscript{3}=4.28; see Supplement 2 for detailed information about parameter ranges), indicating that the response function had an intercept very close to zero and a steep slope (fig. \ref{fig2.4}b): the probability of nest site selection becomes large at low densities of conspecific nests. The parameter $\sigma_{lsf}$ for local site fidelity tended towards a narrow Gaussian distribution, with a scale up to 10 m (fig. \ref{fig2.4}b). 76.5$\%$ of the accepted simulations included the parameter \textit{beh-excl}, indicating that the two behavioral mechanisms are mutually exclusive.\\
	
	
	\begin{table}[h!]
		\begin{center}
			\begin{threeparttable}
				
				\begin{footnotesize}
					\caption{Bayes' factors and proportions of accepted models for model selection with ABC-analysis.}  \label{Tab2.1}
					
					\begingroup
					\setlength{\tabcolsep}{6pt} % Default value: 6pt
					\renewcommand{\arraystretch}{1.5} % Default value: 1
					\begin{tabular}{p{2cm} p{1.5cm} p{1.5cm} p{1.5cm} p{1.5cm} p{1.8cm}}
						
						\toprule
						& \textbf{Random} & \textbf{Population} & \textbf{Inter\-individual} & \textbf{Intra\-individual} & \textbf{$\%$ accepted simulations}\\
						\midrule
						Random & -- & 0.00 & 0.00 & 0.00& 0\\
						Population & $\infty$ & 1.00 & 0.31 & 0.36 & 14.3$\%$\\
						Inter-individual& $\infty$ & 3.20 & 1.00 & 1.14 & 45.7$\%$\\
						Intra-individual & $\infty$ & 2.80 & 0.88 & 1.00 & 40.0$\%$\\
						\bottomrule
					\end{tabular}
					\begin{tablenotes}
						\small
						\item Note: The ABC-analysis retained the 1000 best simulations of 1,000,000 (0.1$\%$). The submodels represent at which level the mechanisms can vary: population, inter-individual, intra-individual. Bayes' factors (BF) are the ratios of the posterior probabilities of two models, indicating the strength of evidence for model M\textsubscript{1} (rows) relative to model M\textsubscript{0} (columns), given the data. Evidence categories according to \citep{kass1995} are: BF $<$ 1 more evidence for M\textsubscript{0} than M\textsubscript{1}; 1 $<$ BF $<$ 3 weak evidence for M\textsubscript{1} compared to M\textsubscript{0}; 3 $<$ BF $<$ 10 substantial evidence for M\textsubscript{1} compared to M\textsubscript{0}.
					\end{tablenotes}\endgroup
				\end{footnotesize}
			\end{threeparttable}
		\end{center}
	\end{table}
	%\end{sidewaystable}
	
	\begin{figure}[h!]
		\begin{center}
			\includegraphics[width=\textwidth]{figures/Fig2_4.pdf}
		\end{center}
		\begin{footnotesize}
			\caption{a) priors (transparent red) and posteriors (dark grey) of the strengths (of the mechanisms: ENV environment; LSF local site fidelity; CA conspecific attraction) and parameters of the behavioral mechanisms. Parameter \textit{beh-excl} defines if conspecific attraction and local site fidelity are mutually exclusive (1) or not (0). b) Effect of the posteriors on the response functions, defined by the bottom 4 parameters, plotted for median (50$\%$ quantile), 20$\%$ and 80$\%$ quantiles of the corresponding parameter. Other parameters than the focal are held constant at the median of the posterior distributions. Local site fidelity (first graph in b) is implemented with a Gaussian curve with the center at the previous nest. Conspecific attraction is implemented with a sigmoid curve (3 graphs on the right of b), with the dependent variable the density of nests (number of nests within $range_{ca}$). See main text and Supplement S2 for further details on parameter definitions and ranges. \label{fig2.4}}
		\end{footnotesize}
	\end{figure}
	

\clearpage	
	
	\section{Discussion}
	We used an inverse modelling approach to study the processes underlying the aggregative nest pattern formation in the digger wasp \textit{Bembix rostrata}. The observed patterns in nature were best predicted by simultaneously considering effects of the environment, conspecific attraction and local site fidelity. We found that nest pattern formation cannot be decomposed into a stepwise process of environmental filtering and behavioral effects. Rather, it represents a complex system with varying nest choice strategies that rely on the simultaneous integration of environmental and behavioral mechanisms with differing strengths. The spatial patterns of nesting are primarily explained by models with individual differences in behavior, including that an individual uses either personal information or inadvertent social information.\\
	
	Conspecific attraction is widespread in digger wasps \citep{evans2007}. Individuals can be attracted to conspecifics as their presence provides an honest cue for habitat suitability. In such situations, the use of social information may have adaptive payoffs by reducing the investment of time and energy in sampling of the environment \citep{dall2005}. Conspecific attraction strongly affects nest site selection in the studied population \citep{buxton2020}, while environmental cues appear to have a weaker effect. While the use of social information is best known in vertebrates and social insects, non-social insects also possess individual and even social learning abilities that eventually contribute to higher fitness \citep{coolen2005}. \textit{Bembix rostrata} is known to perform test-digging behavior, in which individuals seem to sample the environment by digging shallow burrows in the sand across the nesting area before starting to dig an actual nest \citep{nielsen1945}. The use of social cues for habitat suitability is therefore likely adaptive as it reduces the time and energy spent on this behavior. The large contribution of social attraction in predicting nest patterning likely explains the high levels of philopatry in \textit{B. rostrata} \citep{nielsen1945, larsson1986, blosch2000}, i.e. their tendency to remain in the same nesting area for several consecutive generations.\\
	
	The selfish herd hypothesis states that individuals within a population attempt to reduce their predation or parasitism rate by putting other conspecifics between themselves and predators or parasites \citep{mooring1992}. This theory has been invoked to explain the aggregation of \textit{B. rostrata} \citep{larsson1986} and another closely related digger wasp, \textit{Crabro cribrellifer} \citep{wcislo1984}. These studies, along with our data (Supplement S1), show that the incidence of parasitism per nest decreases with nest density. In our system, the presence of \textit{Senotainia albifrons} brood parasites is anticipated to convey information to the wasps and to induce conspecific attraction. However, how much the actual presence of these brood parasites contributes to the overall conspecific attraction is uncertain. Since we lack more information on the dynamics of these parasites, such mechanisms were not directly incorporated in the model, but were instead included as part of the primary process of attraction. Several nests nevertheless occurred at quite low densities, where the parasitism rate, and especially its variation, is higher (Supplement S1 Fig. S1.2). Potential reasons for this more risky behavior could include, non-exhaustively, spatial bet hedging \citep{philippi1989}, imperfect information of parasitism risk \citep{koops1998} or avoidance of perceived intraspecific competition \citep{polidori2008}. Regarding the latter mechanism, conspecific kleptoparasitism (wasps that steal prey from neighboring individuals) has been observed in five other \textit{Bembix}-species, but not (yet) in \textit{B. rostrata} \citep{evans2007}.\\
	 
	The most likely nest selection strategy identified by our model is one with individually consistent but mutually exclusive behaviors (parameter \textit{beh-excl}): when local site fidelity is used, conspecific attraction is not used simultaneously for nest site selection. Consistent individual variation in movement behavior, with individuals relying on either personal or social information, can be responsible for the emergence of ecological patterns at larger spatial scales \citep{spiegel2017}. Such heterogeneity in behavior due to individual specialization, may be especially relevant in populations experiencing high levels of intraspecific competition \citep{araujo2011}. The second most probable, but slightly less supported model, considered individual behavioral flexibility during an individual's lifetime. Shifts in individual behavior have been found across taxa in, for example, foraging in heterogeneous environments \citep{newlands2004, webber2020}, mating \citep{perrill1982}, migration \citep{eggeman2016} and seasonal aggregation \citep{bonar2020}. These shifts arise from plasticity in response to environmental and demographic changes (e.g. density, competition, predation). We modeled behavioral shifts as a stochastic process since any information on the potential conditionality of such shifts was lacking. Explicitly considering thresholds that underlie movement changes is nevertheless important to explain larger-scale patterns \citep{morales2002, newlands2004, goossens2020}. In our study, a flexible strategy is not clearly distinguishable from a consistent one in explaining the spatial nest pattern. Therefore, our results show there are clear behavioral differences between individuals, but it is not conclusive if these behaviors vary over time. Moreover, in some species (e.g. caribou, \textit{Rangifer tarandus}), both consistent and flexible strategies can be present in a population depending on the socio-spatial position of individuals \citep{bonar2020}. The relative importance of different strategies can change within and among populations, seasons or with different levels of parasitic pressure \citep{spiegel2017}. More fine-scaled, individual studies or experiments are therefore required to explore the importance of consistent and flexible strategies.\\
	
	The joint contributions of the environment and internal dynamics to predict spatial nest patterns in our system suggest synergism among multiple processes underlying spatial pattern formation. These mechanisms are often studied in isolation, as behavioral and landscape ecologists traditionally work on very different scales and units of research \citep{lima1996}. These research fields have accordingly developed their specific analytical methods that can be regarded as separate approaches to explain spatial pattern formation of nests in a stepwise manner \citep{melles2009}. Such combined approaches have been applied in insect \citep{polidori2008, asis2014} and bird-oriented research \citep{brown2000, perry2003, melles2009}, most often from a conservation perspective \citep{etterson2003, ward2004, bayard2010, swift2017}. Spatial pattern analysis of point data from homogeneous landscapes allows inference of putative feedbacks that eventually lead to spatial self-organization \citep{rietkerk2008}. The assumption of environmental homogeneity is not always valid. We likewise first decoupled environment from behavior by building the microhabitat model with INLA, but integrated both types of mechanisms again in the IBM. IBMs are ideal for bridging and integrating these seemingly separate processes at different scales and enable quantifying their relative importance and synergism. This inverse approach has much to offer for understanding behavioral mechanisms underlying spatial organization, including identifying its own methodological limits. Spatial organization processes are inherently stochastic and replications are hard to obtain as the strength or shape of processes are likely to change with space and time \citep{wagner2005}. The general approach we illustrate here could thus be applicable in teasing apart coexisting, context-dependent processes, which are pervasive in ecological systems.\\
	
	\clearpage
	\subsection*{Acknowledgements}
	FB is supported by Research Foundation --- Flanders (FWO). DB is supported by Research Foundation --- Flanders (FWO) project INVADED G018017N. The computational resources (Stevin Supercomputer Infrastructure) and services used in this work were provided by the VSC (Flemish Supercomputer Center), funded by Ghent University, FWO and the Flemish Government --- department EWI. We thank Richard Sibly and his research group at the University of Reading, for advice and workshop on the ABC analysis, and Johan Lamaire (ANB) for help and permission to access the field study area in nature reserve `De Westhoek'. We are grateful for logistical help during fieldwork from Wouter Van Gompel, Daan Dekeukeleire, Steven Goossens, Nadine De Schrijver and Marc Batsleer. We thank Sam Provoost for feedback on this study in an early stage. We thank Martijn L. Vandegehuchte for comments on an earlier version of the manuscript and Frederik Mortier for feedback on Python code. We thank two anonymous reviewers and associate editor Dr. Benjamin M. Bolker for the detailed and in-depth comments which improved the manuscript greatly.
	
	\subsection*{Data Accessibility}
	Data deposited in the Dryad Digital Repository: \url{http://doi.org/10.5061/dryad.g79cnp5q8} and code deposited in Zenodo: \url{http://doi.org/10.5281/zenodo.5212680}
	
	\subsection*{Author contributions}
	All three authors conceived the ideas and designed methodology; FB collected and analysed the data, developed the model and led the writing of the manuscript. All authors contributed critically to the drafts and gave final approval for publication.
	
	\subsection*{Supplementary material}
	Supplementary files can be found alongside the published article or in the dedicated github repository: \url{http://doi.org/10.1086/717226} or \url{http://github.com/FemkeBatsleer/SuppPhD}.
	


%\cleardoublepage
%\thispagestyle{empty} % empty 
%\hbox{}
\clearpage

\CenterWallPaper{1}{pictures/CH3-1.png}
\newpage{\thispagestyle{empty}\clearpage}
\cleardoublepage

\ClearWallPaper

\CenterWallPaper{1}{pictures/CH3-2.png}
\newpage{\thispagestyle{empty}\clearpage}
\hbox{}
\clearpage
\ClearWallPaper

\CenterWallPaper{1}{pictures/CH3-3.jpg}
\newpage{\thispagestyle{empty}\cleardoublepage}
\ClearWallPaper



%%%%%%%%%%%%%%%%%%%%%%%%%%%%%%%%%%%%% Chapter 3 %%%%%%%%%%%%%%%%%%%%%%%%%%%%%%%%%%%%%%%%%	

	\pagestyle{mainmatter}
	\chapter{Rapid conservation evidence for the impact of sheep grazing on a threatened digger wasp} \label{chapter3}
	\chaptermark{Nest density and sheep grazing}
	\lettergroup{\thechapter}	

		\begin{flushright} \color{gray}Femke Batsleer\\ Jan Van Uytvanck\\ Johan Lamaire\\ Dirk Maes\\ Dries Bonte\\
	
	 
	\vspace*{2cm}
	Adapted from: Batsleer et al. (2022) \textit{Insect Conservation and Diversity},\\ \textbf{15}$($1$)$, 149– 156. DOI: 10.1111/icad.12532
	\vspace*{\fill}
\end{flushright}
\noindent \color{gray} $\lhd$ \textit{B. rostrata} female in flight.\\
\noindent  $\lhd\lhd$ The rams that grazed `site 2' in this study.	

	\color{black}
	\newpage
	
	
	\section{Abstract}
	\begin{enumerate}
		\item Insect populations show strong temporal fluctuations in abundance. This renders classical monitoring studies extremely difficult to provide insights into specific management actions. For rare species of conservation concern, it is not an option to develop large scale experiments to assess and steer landscape-level actions such as grazing management.
		
		\item \textit{Bembix rostrata} (Linnaeus, 1758) is a threatened digger wasp from coastal dunes and inland sandy regions occurring in a limited number of populations in NW Europe. Since coastal dunes are rapidly being encroached by bushes, grazing management (cattle, sheep, and horses) has been implemented to keep this biotope open.
		
		\item In order to provide insights for local evidence based conservation, a BACI (before/after and control/impact) experiment was set up to assess the impact of sheep grazing on \textit{B. rostrata}. We quantified the number of nests during 3 years at two grazed sites and a control-site excluded from grazing. We additionally assessed grazing pressure.
		
		\item The BACI design allowed us to directly adjust the current grazing management. The implemented sheep grazing reduced densities of \textit{B. rostrata}, but did not lead to its local extinction. We discuss these findings in relation to the biology of the species.
		
		\item Our efficient and effective experimental design allowed a fast assessment of the current grazing management and showed that spatially heterogeneous sheep grazing could be used as a management tool to ensure the conservation of the emblematic digger wasp \textit{B. rostrata}.
		
	\end{enumerate}

\vspace*{\fill}		
\noindent \textbf{Keywords:} BACI, coastal dunes, Crabronidae, evidence-based conservation, GPScollar, grazing management, grey dunes, Hymenoptera, trampling
	
\clearpage
	
	\section{Introduction}
	Biodiversity measures used to monitor and evaluate conservation actions are mainly focused on vertebrates and plants \citep{clark2002}, assuming that other species groups will automatically piggy-back on these actions. Current management measures, aiming at the recovery or maintenance of priority habitat types, can have local short-term negative effects on non-targeted or non-monitored groups, such as insects and other arthropods \citep{vanklink2018}. Grazing management is a frequently used tool in open- or semi-open landscape types but, only recently, possible contradictory effects on plant and arthropod diversity have gained more attention \citep{mysterud2005, vanklink2015}. In order to promote heterogeneity and to allow semi-natural herbivore dynamics, this measure is usually implemented at the landscape scale. Grazing has positive effects on the diversity of certain arthropod groups when they directly provide resources, such as dung for dung beetles \citep{verdu2007}. Other groups benefit from the indirect changes in vegetation structure or abiotic conditions, the local disturbance of soil development and secondary blow-outs (reactivation of wind-eroded depressions with bare sand) \citep[e.g.][]{bonte2000, wallisdevries2001, maes2006, maes2006a}. However, grazing can also have negative effects on arthropods through direct effects (unintentional predation and disturbance) and indirect effects (changes in vegetation composition and structure, resources and abiotic conditions \citep{maes2006, poyry2006, vanklink2015, broder2019}. Moreover, mammalian herbivores can cause trophic cascades and modulate arthropod food webs in complex ways \citep{vandegehuchte2017a}.\\
	
	Natural wind dynamics create and sustain dynamic, calcareous dune habitats, including blond and grey dunes \citep{provoost2004, provoost2011}, harbouring specific arthropod diversity adapted to overblowing events \citep{bonte2002, bonte2005}. These dynamics are largely lacking in the highly fragmented and urbanised Belgian coastal dunes. The introduction of large grazers is a local measure to remove biomass and stimulate blow-outs. Large grazers can maintain a diverse vegetation structure and increase vegetation diversity, but can rarely stop succession and soil development. Therefore, it remains unknown whether such local management actions may benefit rare and specialized species that are adapted to large-scale sand dynamics. Consequently, the rewilding of nature reserves with large grazers does not automatically benefit all specialized dune arthropods \citep{vanklink2018}. Managers are, therefore, often conflicted as the use of grazers appears to be the most applicable measure to stop encroachment at locations that harbour relict populations of threatened insects.\\
	
	An evidence based conservation approach should integrate arthropods explicitly into adaptive conservation planning \citep{sutherland2004, ferraro2006, samways2020}. Therefore, adaptive management needs to be guided by empirical evaluation and sound local impact assessment \citep{sutherland2020}. However, insect population densities can fluctuate strongly among years, either due to meteorological variation or more intrinsic demographic processes \citep{didham2020, welti2021}. These fluctuations render it extremely difficult to interpret short-term changes in population sizes in light of conservation actions. Consequently, it is difficult to perform large and robust experiments for rare species, replicated over several landscape contexts, to study the general management impact in great detail, an approach needed to improve the accuracy of estimates of biodiversity responses to management \citep{christie2019}. We here show the value of a Before--After Control--Impact (BACI) design for fast impact studies of rare insect populations in a single location in the context of adaptive management. BACI designs use differences between a control and impact treatment before the actual impact event as a null hypothesis to test later differences between both \citep{thiault2017}. This way, between-year fluctuations can be better accounted for.\\
	
	We here use such a BACI design to test the local impact of sheep grazing on the nest density in aggregations of the threatened ground-nesting digger wasp \textit{Bembix rostrata}. Earlier research already revealed that trampling by cattle and vacationers is very detrimental for the presence of digger wasp nests \citep{bonte2005}. Conversely, areas with a management practice involving sheep grazing seem more likely to have \textit{B. rostrata} present (unpublished data of inventory along the Belgian coast), although any direct quantification of the impact of sheep grazing is missing. Since grazing management in larger areas is spatiotemporally heterogeneous \citep{vanklink2015}, we additionally GPS-collared one sheep in each location during the first year of grazing, to quantify sheep density at the research plots. We put the results in a management context to inform local wardens about appropriate conservation actions for this specialised, emblematic wasp.\\
	\clearpage
	\section{Material \& Methods}
	\subsection{Study species}\label{3studyspecies}
	\textit{Bembix rostrata} (Linnaeus, 1758) (Hymenoptera) is a specialized, highly philopatric, gregariously nesting digger wasp found in sandy regions of Europe. This species nests in coastal and inland sand dunes with sparse vegetation \citep{larsson1986, klein2004}. These usually moss-dominated dunes are included in the Annex 1 of the EU Habitats Directive (Natura 2000 habitat 2130, CORINE biotope 16.22). Adults are active in summer, and females construct one nest burrow at a time in which a single larva is progressively provisioned with flies \citep{nielsen1945, field2005}. Such brood-care behaviour indicates the need to maintain and locate a single nest for several days. During foraging, females temporarily close the nest entrance. Such digging traces are apparent around active and recently-active nests and are easy recognisable for a trained observer \citep{evans2007}.\\
	
	\subsection{Impact of grazing}
	\subsubsection{Study area and field data}
	The study was performed in the dune complex Westhoek--Cabour at the Belgian west coast near the French border (Fig. \ref{fig3.1}). We chose a BACI design at the site-level scale since it was not possible to have replicated or paired grazed and control plots due to the overall small area of the nature reserves, the species' local rarity and specific nesting-habitat requirements. Given the landscape-scale impact of grazing on insect densities \citep{bonte2008, didham2020}, small control plots within larger grazed areas would not be reliable estimators of the absence of grazing management. Plots were selected at three sites and systematically surveyed during three years. None of the sites had been grazed for at least 5 years before the experiment and none was grazed the first year of the experiment (2018). The control site had no grazing management for all three years. Sheep were introduced to the other two sites from the second year in May 2019 until after the third summer, but excluded in early spring 2020. Site 1 had 12 sheep on 38.49 ha (0.31 sheep/ha; `low-intensity') and site 2 had 8 sheep on 2.17 ha (3.69 sheep/ha; `high-intensity'). The control site had an area of 0.37 ha. Study plots of 3 $\times$ 3 m$^2$ with similar moderate-to-high sheep densities were selected the first year within the three sites where \textit{B. rostrata} aggregates were present. As such, five plots were established in the control and site 2 and three in site 1. Such plots were ideal to monitor densities for a few years, as \textit{B. rostrata} is known to remain in the same nesting area for several consecutive years \citep{nielsen1945, larsson1986, blosch2000, bogusch2021}.\\

	In 2018, 2019 and 2020, nests of \textit{B. rostrata} were counted in each plot once a week during favourable weather conditions from the end of June until mid-August (the main flight season). Females are active mainly during sunny periods with sufficiently high temperatures ($>$21\textdegree C) \citep{nielsen1945, schone1992, evans2007}. During every census, the number of closed (active and old) nests was counted and marked with sticks to avoid double counting. These nests are considered a proxy for reproduction in a plot (hereafter `breeding density') as these are nests with traces of active provisioning (see section \ref{3studyspecies} `study species'). For all plots, censuses were performed six times in 2018 and 2020, and seven times in 2019. This resulted in 244 censuses in total for 13 plots. As traces of nesting activity are less visible during and after periods of rain, censuses sometimes had to be postponed which resulted into a maximum of two weeks between two censuses.\\

	\subsubsection{Statistical analysis}
	We applied a fully crossed or BACI (before--after control--impact) design \citep{smith2002} as we had two types of controls: a treatment--control with no grazing (control--impact) and a temporal control as none of the sites were grazed the first year (before--after). The number of nests was analysed using a generalized linear mixed model with a negative binomial distribution in R v3.5 \citep{rcoreteam2020} using the function `glmer.nb' from the package lme4 \citep{bates2015}. The variable `plot' was included as a random effect (13 plots in total) to account for repeated measurements and `timepoint' (the serial number of the census during one season) to account for changing activity during the season. Fixed effects were year (3 levels: 2018, 2019 and 2020) and site (3 levels: control, site1 and site2). Our main interest is the interaction between `year' and `site', or the increase or decrease in nests relative to the first year and the control site. We cannot generalize the effect of sheep grazing, as we do not have replicates at the treatment-level (sheep grazing and control sites). However, with this design, we can provide evidence for the local and current impact, to rapidly adapt management, if needed.\\
	
	\subsubsection{GPS collar data}
	To monitor the location of the sheep herd in the grazed sites, we used the GPS-tracking system of Digitanimal, an IoT (Internet of Things) based system. As sheep forage in herds, the temporal location of the collared sheep is considered representative for the whole herd. Two sheep (one in each grazed site) were equipped with GPS-collars (type AI762) in late spring 2019 when grazing started until the battery died in early winter. GPS-positions were recorded every 30 minutes. Location errors are smaller than 5 meters. Data were collected from 28-05-2019 until 27-01-2020 (245 days) for site 1 (low intensity grazing) and from 7-05-2019 until 3-01-2020 (242 days) for site 2 (high intensity grazing). Point density raster maps of the GPS-positions were made using the Spatial Analyst extension in ArcGIS v10.2 \citep{esri2011} with rectangular cells of 5 $\times$ 5 m$^2$. This resolution provides the best trade-off between detail and uncertainty of the GPS-fixes and position of the sheep within the herd. These density maps give the expected number of positions/m$^2$ for the period of GPS-tracking and were plotted in a histogram for each grazing area. The values of expected positions for the \textit{B. rostrata} study plots were extracted using the package `raster' in R \citep{hijmans2017, rcoreteam2020} and summarised using a weighted average of pixel area (for a rectangular buffer of 1 m around 3 $\times$ 3 m$^2$ plots). The values of sheep position densities for each \textit{B. rostrata} study plot were subsequently used for qualitative discussion, as this method was only available for the first year of grazing.\\
		
	\begin{figure}[h!]
		\begin{center}
			\includegraphics[width=\textwidth]{figures/Fig3_1.pdf}
		\end{center}
		\begin{footnotesize}
			\caption{maps of the study area. a) The three study sites, b) overview map to situate study area at the border of Belgium with France (in light red) and details of c) the high-intensity grazed and control area d) the low-intensity grazed area. Maps made with QGIS v3.10 \citep{qgisdevelopmentteam2020}. Aerial photographs (summer 2018) source: Agency for Information Flanders (geopunt.be). \label{fig3.1}}
		\end{footnotesize}
	\end{figure}
	
	\clearpage
	

	\section{Results}
	\subsection{Impact of grazing}
	The number of \textit{Bembix rostrata} nests (breeding density) increased in the control--site from 2018 to 2020 (Fig. \ref{fig3.2}a; Table \ref{Tab3.1}). Relative to 2018 (before--after) and the control site (control--impact), the breeding density decreased significantly only in year 2 for site 1 with low intensity grazing (Fig. \ref{fig3.2}b; Table \ref{Tab3.1}) and was significantly lower in both years with grazing for site 2 with high intensity grazing (Fig \ref{fig3.2}b; Table \ref{Tab3.1}). Thus, relative to the control-site, the breeding density decreased significantly under sheep grazing (Fig. \ref{fig3.2}b). Nevertheless, nests stayed present in the plots of all three sites (Fig \ref{fig3.2}a), which remained the principal nesting areas throughout the study period.\\
	
	\subsection{GPS collar data}
	Density maps and corresponding histograms of sheep GPS-fixes are shown in Fig. \ref{fig3.3}, including the values for the study plots. These values are for the first year of sheep grazing (2019). Local grazing pressure differed among the sites. In the plots of site 1 (low-density grazing), the number of sheep positions was between 0 and 0.049 sheep positions/m$^2$ (Fig \ref{fig3.3}c). In the plots of site 2 (high-density grazing), between 0.183 and 0.404 positions/m$^2$ were observed (Fig. \ref{fig3.3}d). These data thus confirm the relative grazing pressure for the two sites within the study plots. The grazing densities in the studied plots are all within the lower half of the range of possible densities, which suggests that the nesting locations of \textit{B. rostrata} are not the preferred locations for sheep.\\
	
		\begin{table}[h!]
		\begin{center}
			\begin{footnotesize}
				\caption{Statistical output of the BACI-analysis, a GLMM with plot and timepoint as random effects. Site 1 had low-intensity grazing and site 2 high-intensity grazing. Interaction terms give the change relative to the control plot and year 2018 (without sheep grazing).}  \label{Tab3.1}
				
				\begingroup
				\setlength{\tabcolsep}{6pt} % Default value: 6pt
				\renewcommand{\arraystretch}{1.5} % Default value: 1
				\begin{tabular}{p{2.5cm} l r r r r @{\hspace{3pt}} l}
					
					\toprule
					\textbf{Response} & \textbf{Term} & \textbf{Estimate} & \textbf{z-value} & \textbf{df} & \multicolumn{2}{c}{\textbf{p-value}} \\
					\midrule
					\multirow{8}{*}{\parbox{2.5cm}{Number of nests (breeding density)}} & Year2019 & 0.702 & 5.268 & 2 & $<$0.001 & $\ast\!\ast\!\ast$ \\
					 & Year2020 & 0.832 & 6.187 & 2 & $<$0.001  & $\ast\!\ast\!\ast$ \\
					 & Site1 & -0.037 & -0.089 & 2 & 0.929 & \\
					 & Site2 & -1.078 & -3.010 & 2 & 0.003 & $\ast\!\ast\color{white}*\color{black}$ \\
					 & Year2019:Site1 & -0.222 & -1.020 & 4 & 0.308 & \\
					 & Year2020:Site1 & -0.665 & -2.921 & 4 & 0.003 & $\ast\!\ast\color{white}*\color{black}$\\
					 & Year2019:Site2 & -1.763 & -7.811 & 4 & $<$0.001 & $\ast\!\ast\!\ast$\\
					 & Year2020:Site2 & -1.493 & -6.740 & 4 & $<$0.001 & $\ast\!\ast\!\ast$\\
					 
					
					\bottomrule
				\end{tabular}\endgroup
			\end{footnotesize}
		\end{center}
		%\end{sidewaystable}
	\end{table}

	\clearpage
	\hbox{}\vspace*{\fill}
	\begin{figure}[h!]
		\begin{center}
			\includegraphics[width=\textwidth]{figures/Fig3_2.pdf}
		\end{center}
		\begin{footnotesize}
			\caption{a) average number of nests (breeding density) of \textit{B. rostrata} in the control-site (light-blue), site 1 (mid-blue) and site 2 (dark blue). In year 2018 no grazing was present at any site. The control site remained ungrazed in 2019 and 2020, while site 1 was grazed at 0.31 sheep/ha and site 2 at 3.69 sheep/ha (see methods for further details). b) Proportion of nests compared to control in each year. Significance levels are shown for changes between 2018 and the two consecutive years with grazing and are from the interaction terms between `year' and `site' in Table \ref{Tab3.1} (p $<$ 0.001 $\ast \ast \ast$; 0.001 $<$ p $<$ 0.01 $\ast \ast$; 0.01 $<$ p $<$ 0.05 $\ast$; p $>$ 0.05 NS). Error bars are standard errors. \label{fig3.2}}
		\end{footnotesize}
	\end{figure}
\vspace*{\fill}\clearpage
	\hbox{}\vspace*{\fill}
		\begin{figure}[h!]
		\begin{center}
			\includegraphics[width=\textwidth]{figures/Fig3_3.pdf}
		\end{center}
		\begin{footnotesize}
			\caption{density visualisations (a, b) and histograms (c, d) of GPS-fixes for one sheep per herd per grazed area in late spring until winter 2019. Site 1: low-density grazing (a, c) and site 2: high-density grazing (b, d). The mean density value extracted at the \textit{B. rostrata} study plots are indicated on the histogram with black dots and the corresponding number is labelled on the maps. The expected number of sheep positions/m$^2$ are for the period GPS-fixes were made (245 and 242 days respectively). Zero density is visualised as transparent and a linear amount of transparency is added up to a value of 0.175. Maps were made with QGIS v3.10 \citep{qgisdevelopmentteam2020}. Aerial photographs (summer 2018) source: Agency for Information Flanders (geopunt.be). \label{fig3.3}}
		\end{footnotesize}
	\end{figure}
	\vspace*{\fill}
	\clearpage

	\section{Discussion}
	Using a BACI (Before--After Control--Impact) design, we showed that sheep grazing decreases nesting densities of the threatened digger wasp \textit{B. rostrata} in the focal study area. Given the rarity of the species and the small size of the nature reserves, only site-level effects could be estimated. Any other experimental design in this particular landscape, for instance installing multiple enclosures within a grazed site, could not provide reliable evidence of such landscape-level conservation actions. Under these circumstances, smaller control plots would be equally influenced by the site-level grazing and either follow the same impact (site-level dilution of the densities) or show opposite effects \citep[concentration effects in the few excluded plots][]{didham2020}. We here show that conservation evidence, often rapidly needed in a focal nature reserve, can be collected with minimal investments. We argue that such designed experiments and surveys at a site-level are needed to guide local adaptive management of threatened insect species inhabiting landscapes that are currently restored by site-level management using larger grazers.\\
	
	Grazing is introduced in many coastal dune reserves to reverse processes of shrub and grass encroachment, and hence to revitalise dynamic dunes \citep{provoost2004}. Trampling by grazers loosens the top soil thereby creating open sand patches with sharp edges. These open sand patches can quickly develop into suitable breeding areas---with transitional micro-gradients from open sand to more vegetated dune grassland---for many dune-specific arthropods, for instance ground-nesting bees and wasps, spiders and butterflies \citep{wallisdevries2001, vanklink2015}. For ground nesting bees and wasps in particular, such nesting resources are often overlooked in management due to the focus on floral resources for pollinators \citep{kimoto2012, buckles2019}. In the short term, however, and especially in the early years of introduction of large herbivores, the few available open habitats can be overused by large grazers, and thereby destroy the few relict populations of these arthropods in encroached dunes \citep{bonte2005, vanklink2015}. The protection of these relict populations is therefore essential for the conservation of the entire metapopulation \citep{bonte2008} and needs to be carefully surveyed, but also steered by adaptive management. Our results show that nests of \textit{B. rostrata} stay present with sheep grazing, which will be beneficial for population persistence of this philopatric species at a location that does not need to be completely recolonized anew \citep{blosch2000, bogusch2021}. When some activity of nesting persists in a grazed site, attraction to conspecifics during nest site selection \citep[chapter \ref{chapter2}][]{batsleer2022} can reinforce the establishment of a larger population. However, several years are probably needed to rebuild aggregations, as an individual female wasp is estimated to produce a maximum of 5 offspring in one summer \citep{larsson1989}.\\
	
	Effects of grazing are known to differ among arthropods with different suites of life-history traits and habitat characteristic requirements \citep{vannoordwijk2012, vanklink2015}. However, the specific differences in grazing impact linked to life-history traits remain understudied \citep{iida2016}, especially for ground-nesting bees and wasps. In general, all types of grazers can have two main direct impacts on ground nesting arthropods: (a) trampling of nests causing a higher mortality of larvae and/or pupae and (b) disturbance of nesting behaviour. The use of shallow nests and progressive provisioning renders \textit{B. rostrata} probably more sensitive to trampling effects than other ground-nesting bees and wasps. Cattle grazing in open vegetation and grey dunes mainly occurs during winter and spring and at a constant, low rate throughout the year as they use grey dunes as corridors between grassy foraging sites \citep{lamoot2005, bonte2005}. Large grazers such as cattle and horses probably have a larger impact than small grazers such as sheep \citep{adams1975, metera2010, pakeman2020} on the survival of pupae overwintering at a depth of 8 to 15 cm belowground \citep{nielsen1945, blosch2000}. The low trampling density of sheep in our plots with nests of \textit{B. rostrata} confirms that grey dunes are also not the preferred habitat type of sheep (Fig. \ref{fig3.3}). As \textit{B. rostrata} is a progressive provisioner, feeding its larva gradually during development \citep{nielsen1945, field2005}, it will need an intact nest for several days to two weeks to nourish the larva. Hence, disturbance of nesting behaviour due to trampling during summer months can destroy unfinished nests and prevent nest completion. Even if nests are not completely destroyed, grazers can interfere with homing behaviour when orientation marks surrounding a nest are trampled \citep{tinbergen1932a, schone1993}. Grazing can also interfere with time allocation between nest maintenance and larva provisioning when nest entrances collapse, with possible adverse effects on the fitness rate of the offspring. In the first grazed site, where sheep were at a lower density (0.31 sheep/ha), the sheep are very rarely expected at the nesting sites (Fig. \ref{fig3.3}c), consequently disturbance during summer months will be negligible. A combination of both effects, disturbance of nesting behaviour and trampling throughout the year, is most likely responsible for the decrease in nests in the second grazed site, where sheep density was higher (3.69 sheep/ha). An option which might be beneficial to minimize the trade-off between the direct impact of sheep grazing and desired vegetation management outcome, but which we did not include in our set-up, is to exclude sheep grazing during summer months when \textit{B. rostrata} adults are active and let sheep graze at a higher density during the winter months when fewer negative effects are expected.\\
	
	We suggest reconciling the short-term negative and long-term positive effects within a dynamic landscape management approach. A sufficiently high level of variation in grazing intensity and grazer type at a landscape and temporal level is needed, integrated with site specific adaptive management based on monitoring. Regional heterogeneity in grazing together with rotational grazing are suggested to maintain the needed habitat heterogeneity for arthropods, either to allow life cycle completion \citep{vannoordwijk2012} or the provisioning of essential resources \citep{maes2006a}. Such a heterogeneity approach for grazing can be complemented with a mosaic pattern of small-scale mechanical removal of vegetation and creation or (re)activation of blow-outs \citep{vanboxel1997}. Small-scale bare sand patches, embedded within a variety of other resources, are known to promote other threatened bee and wasp species \citep{heneberg2013, heneberg2014}.\\
	
	Overall, we show that sheep grazing does not severely impact the presence of \textit{B. rostrata}, although relative densities were lower in two grazed sites than in an ungrazed site. These insights, achieved by using a simple BACI design, show that sheep grazing could be used as a tool in an adaptive vegetation management approach to reconcile the need for local management measures and the conservation of a specialized ground-digging insect in dune areas.\\
	
	\vspace*{\fill}
	\subsection*{Acknowledgements}
	We thank ANB (Agency for Nature and Forests) for permission to access the nature reserves, the sheep owners for permission to apply the GPS-collars and Hans Matheve for ArcGIS assistance. FB is supported by Research Foundation --- Flanders (FWO). We thank the two reviewers, editor and associate editor for the constructive feedback on the manuscript.
	
	\subsection*{Data Accessibility}
	Data and code is deposited in Zenodo: \url{http://doi.org/10.5281/zenodo.5220852}
	
	\subsection*{Author contributions}
	FB, DB and JL conceived ideas; FB, DM and DB designed methodology; FB and JVU collected data; FB and JVU analysed the data; FB led the writing of the manuscript. All authors contributed critically to the drafts and gave final approval for publication.


%\cleardoublepage
%\thispagestyle{empty} % empty 
%\hbox{}
\clearpage

\CenterWallPaper{1}{pictures/CH4-1.jpg}
\newpage{\thispagestyle{empty}\clearpage}
\cleardoublepage

\ClearWallPaper

\CenterWallPaper{1}{pictures/CH4-2.jpg}
\newpage{\thispagestyle{empty}\clearpage}
\hbox{}
\clearpage
\ClearWallPaper

\CenterWallPaper{1}{pictures/CH4-3.jpg}
\newpage{\thispagestyle{empty}\cleardoublepage}
\ClearWallPaper

	%%%%%%%%%%%%%%%%%%%%%%%%%%%%%%%%%%%%% Chapter 4 %%%%%%%%%%%%%%%%%%%%%%%%%%%%%%%%%%%%%%%%%	
\pagestyle{mainmatter}
\chapter{Asymmetrical gene flow between coastal and inland dunes in a threatened digger wasp} \label{chapter4}
\chaptermark{Population genetics}
\lettergroup{\thechapter}	

\begin{flushright} \color{gray}Femke Batsleer\\ Matthieu Gallin\\ Moyra Delafonteyne\\ Daan Dekeukeleire\\ Filiep T'Jollyn\\ Pieter Vantieghem\\ An Vanden Broeck\\ Joachim Mergeay\\ Dirk Maes\\ Dries Bonte\\
	
	
	\vspace{2cm}
	
	This chapter is available as a preprint on bioRxiv:\\ doi.org\slash 10.1101/2022.09.20.508247
	\vspace*{\fill}
	
	\end{flushright}
\noindent \color{gray} $\lhd\lhd\lhd$ Female \textit{B. rostrata} digging at her nest (from left to right) in Bernissart (inland Wallonia), De Panne (coastal Flanders), Geel-Bel (inland Flanders).\\
\color{black}
\newpage
	\section{Abstract}
	Connectivity is a species- and landscape-specific measure that is key to species conservation in fragmented landscapes. However, information on connectivity is often lacking, especially for insects which are known to be severely declining. Patterns of gene flow constitute an indirect measure of functional landscape connectivity. We studied the population genetic structure of the rare digger wasp \textit{Bembix rostrata} in coastal and inland regions in and near Belgium. The species is restricted to sandy pioneer vegetations for nesting and is well known for its philopatry as it does not easily colonize vacant habitat. It has markedly declined in the last century, especially in the inland region where open sand habitat has decreased in area and became highly fragmented. To assess within and between region connectivity, we used mating system independent population genetic methods suitable for haplodiploid species. We found more pronounced genetic structure in the small and isolated inland populations as compared to the well-connected coastal region. We also found a pattern of asymmetrical gene flow from coast to inland, including a few rare dispersal distances up to 200 to 300 km based on assignment tests. We point to demography, wind and difference in dispersal capacities as possible underlying factors that can explain the discrepancy in connectivity and asymmetrical gene flow between the different regions. Despite \textit{B. rostrata} being a poor colonizer, gene flow between existing populations appeared not highly restricted, especially at the coast. Therefore, to improve the conservation status of \textit{B. rostrata}, the primary focus should be to preserve and create sufficient habitat for this species to increase the number and quality of (meta)populations, rather than focusing on landscape connectivity itself.\\
	
	\vspace*{\fill}		
	\noindent \textbf{Keywords:} haplodiploid, microsatellites, insect conservation, dunes, coastal, sandy habitats, Hymenoptera, Crabronidae, \textit{Bembix rostrata}
	\clearpage
	
	\section{Introduction}
	The decline in abundance and distribution of many insects has raised widespread public and political awareness on their biological value \citep{harvey2020, didham2020, wagner2021, welti2021}. As habitat loss and fragmentation have been identified as major drivers of this insect decline, a focus on connectivity conservation for (meta-)population persistence is essential and justified \citep{hanski1996, hanski2002, haddad2015, cardoso2020}. Connectivity is a biological concept, in which fluxes of individuals between patches in heterogenous landscapes are determined by both landscape configuration and the species' dispersal capacity. Patterns of gene flow reflect realized dispersal across multiple generations and can shed light on the functional landscape connectivity especially at large spatial scales \citep{kim2013, hodgson2022, maes2022}. Responses to habitat loss and fragmentation are species- and context-dependent, because drivers of fragmentation can be diverse in identity, scale and intensity \citep{cheptou2017}. Consequently, every landscape and its regional context is unique for each species and habitat connectivity remains difficult to generalize, even across different regions for the same species. \\
	
	Dune habitats harbor a specific insect biodiversity with typical species of conservation interest \citep{maes2006, provoost2011, dero2021}. These sandy habitats in Belgium, both at the coast and inland, have gone through extensive---but different---landscape changes and fragmentation during the past decades or centuries. Coastal and inland dunes differ regarding their size, extent, history and nature of fragmentation, even though general levels of natural habitat fragmentation in Flanders (Belgium) are among the highest in Europe \citep{europeanenvironmentagency2011}. Firstly, coastal dunes in Flanders are calcareous and form a narrow, linear system along the coast \citep{provoost2004, decleer2007}. Coastal sandy habitats became fragmented at two scales. Primarily, urbanization from the interbellum period onwards decreased the total area of dunes significantly and resulted in a physical separation of the larger dune entities. In parallel, loss of low-intensity agricultural practices (including grazing of livestock) and obstruction of sand dynamics due to urbanization stimulated the succession and shrub development, at the cost of the open, early-succession or pioneer dune habitats \citep{provoost2011}. Large herbivores have been introduced in many coastal dune reserves during the last three decades to revitalize dune dynamics \citep{provoost2004}, but might have mixed effects on local arthropod species due to intense trampling \citep[chapter \ref{chapter3}]{bonte2008, vanklink2015, batsleer2022a}. Second and contrastingly, inland sandy soils in Flanders are acidic \citep{decleer2007}. In this region, large open heathland and land dune systems were heavily afforested since the 19\textsuperscript{th} century \citep{dekeersmaeker2015}. Later, in the second half of the 20\textsuperscript{th} century---parallel to, but more severe than at the coast---the remaining open habitat patches became further build-up and hence, smaller and more fragmented. Secondary loss and fragmentation of the remaining sandy habitat patches took place, here due to acidification and eutrophication leading to a continuous grass encroachment of sparsely vegetated sand areas on lime-poor soils \citep{schneiders2020}.\\
	
	The digger wasp \textit{Bembix rostrata} is a univoltine habitat-specialist associated with dynamic sandy habitats with early-successional vegetation: grey dunes in coastal areas (EU Habitats Directive habitat 2130) and dry sandy heaths and inland dunes with open grasslands in inland areas (habitats 2310 and 2330 respectively). The species occurs in Europe and Central Asia, with a northern limit reaching south Scandinavia \citep{bitsch1997}. In several European countries, \textit{B. rostrata} has declined during the 20\textsuperscript{th} century and is considered a Red List species in several regions in Germany and is protected in Wallonia, Belgium \citep{blosch2000, jacobs2000, klein2004, barbier2007, bogusch2021}. In Belgium and the Netherlands, mostly inland populations were lost, resulting in a distribution with local strongholds at the coast and more fragmented or isolated populations present inland \citep{klein2004}. \textit{Bembix rostrata} is labeled as a philopatric species that does not easily colonize vacant habitat and aggregates stay present at the same location for many consecutive years \citep{nielsen1945, larsson1986, bogusch2021}. This presumed philopatry is likely linked to the species gregarious life-style where females base their nest choice on the presence of conspecifics \citep[chapter \ref{chapter2}]{batsleer2022}. Given the typical dynamic character of the species' habitat (pioneer dune or other sandy vegetations), philopatry should be highly disadvantageous and eventually put the species' at risk if new early-successional sites cannot be colonized \citep{bogusch2021}. Hence, as for other species from early-successional habitat, good dispersal capacities would be expected despite the overall sedentary life style during breeding \citep[e.g. \textit{Andrena vaga}:][]{cerna2013, exeler2008}. Correct information on the species' dispersal capacity is, therefore, essential to guide future conservation strategies.\\
	
	Given the species' conservation flagship status in Belgium as emblematic ground-nesting hymenopteran \citep{batsleer2021b}, we studied connectivity through gene flow between \textit{B. rostrata} populations within and between coastal and inland fragmented sandy habitats in Belgium and bordering areas. Our main question is whether and how different populations and regions are genetically connected to each other. We amplified 21 microsatellite markers and used a non-lethal sampling method with wing clips \citep{chaline2004} to minimize the impact of sampling on the often small and/or geographically isolated populations. As the species has a haplodiploid sex-determination system, we used population genetic analyses that are mating system independent and do not incorporate a diploid population genetic model.\\
	\clearpage
	
	\section{Material \& Methods}
	
	\subsection{Study species}
	\textit{Bembix rostrata} is a univoltine specialized, gregariously nesting digger wasp from sandy habitats with sparse vegetation \citep{larsson1986, klein2004}. Adults are active in summer, showing protandry: females are directly mated when emerging by the guarding males, who emerge one to five days earlier \citep{wiklund1977, schone1981, evans2007}. Females show brood care: one individual constructs one nest burrow at a time in which it progressively provisions a single larva with flies \citep{nielsen1945, field2020}. An estimated of up to 5 nests are produced, each with one offspring \citep{larsson1989}. There are no overlapping generations, as the species overwinters as prepupa.\\
	
	\subsection{Study sites and sampling}
	Sampling took place during the summers of 2018, 2020 and 2021. Samples were taken across 49 Belgian populations, three French populations and one Dutch population \ref{fig4.1}. Detailed information about each sampling site---region, name, coordinates, sample year(s), number of samples---can be found in supplementary material S1. A large part of the Belgian coastal populations and two inland populations were sampled both in 2018 and 2020 and are used to check if samples from different years can be pooled in the subsequent analyses (see \ref{mmamova} `hierarchical Analysis of Molecular Variance'; AMOVA). Only females were sampled, to solely use diploid individuals in the population genetic analyses for this haplodiploid mating system.\\
	
	The populations were a priori divided into four regions based on geographical configuration and sampling design. In Flanders (north part of Belgium) all populations at the coast or inland known to exist at the time of sampling were covered. In Wallonia (southern part of Belgium) the two main known large populations were covered. At the French coast bordering coastal Flanders, three extra populations were sampled opportunistically, to be able to consider genetic links further along the French coast, although intermediate populations certainly exist. We considered coastal France as an a priori separate region (Fig. \ref{fig4.1}) to avoid biased interpretation of gene flow patterns due to this incomplete sampling coverage.\\
	
	As it is estimated that \textit{B. rostrata} has 5 nests (or equivalent number of offspring) per female \citep{larsson1989}, population sizes grow slowly, especially in isolated fragments. Therefore, to minimize the impact of sampling on the populations, we used a non-lethal sampling method with wing clips, a method shown to give good-quality DNA for microsatellite PCR amplification \citep{chaline2004}. Tips of both forewings from live digger wasp individuals were cut. Both wingtips were stored in absolute ethanol, stored in a refrigerator at 4\textdegree C after sampling and transferred to a freezer (-18\textdegree C) for long-term storage. For each sample, individual coordinates of the capture position (most often a nest) were noted.\\
	
	\begin{figure}[h!]
		\begin{center}
			\includegraphics[width=\textwidth]{figures/Figure4_1.png}
		\end{center}
		\begin{footnotesize}
			\caption{overview map with locations of sampling. Populations 1-3 in coastal France (dark brown dots), 4-40 in coastal Flanders (light brown dots), 41-51 in inland Flanders (light green dots), 52-53 in inland Wallonia (dark green dots). Neighboring countries and the three administrative regions of Belgium are indicated: FR (France), NL (Netherlands), D (Germany), L (Luxembourg); Flanders (Flemish region), Wallonia (Walloon region) and Brussels (Brussels-capital region).. \label{fig4.1}}
		\end{footnotesize}
	\end{figure}

	\subsection{DNA extraction and PCR amplification}
	Genomic DNA was extracted from the wing tips with a protocol based on Chelex (Biorad; details in supplementary material S2). The development of species-specific microsatellites was outsourced to AllGenetics\textregistered  (A Coru\~{n}a, Spain), who provided 500 non-tested microsatellite primers and tested 72 of those biologically with 11 individuals. Of those, we selected 33 polymorphic microsatellites (based on polymorphism, size range, and length of repeat motif) and rearranged them with the program multiplex manager \citep{holleley2009} in 3 pairs of primer-multiplexes for PCR and amplified these for each sample. Details on the PCR-conditions and multiplexes are in supplementary material S2; characteristics of the primers are in supplementary material S3. PCR products were run on an ABI 3500 analyzer with the GeneScan-600 LIZ size standard and the electropherograms were scored using Geneious Prime (Biomatters). Samples from three recaptured individuals were blindly and randomly added to the workflow. They popped up as duplicate genotypes in the first part of the data analysis, so we were confident the scoring error was minimal.\\
	
	\subsection{Genetic data analysis}
	\subsubsection{Hardy-Weinberg, linkage disequilibrium and null alleles}
	To exclude microsatellites that are uninformative or have artefacts, the assumptions of Hardy-Weinberg (HW) equilibrium and null alleles at individual loci \citep[non-amplified alleles;][]{chapuis2007}, and of no linkage disequilibrium (LD) across pairs of loci were examined before subsequent genetic analysis \citep{waples2015}. HW deviations, null alleles and LD deviations were calculated and examined with the R-packages pegas \citep{paradis2010}, PopGenReport \citep{adamack2014} and poppr \citep{kamvar2014} respectively, for populations that had at least 10 samples. Assumptions testing followed the general reasoning and multiple testing from \citet{waples2015}, see supplementary material S5.\\
	
	\subsubsection{Hierarchical AMOVA to validate pooled samples from different years}\label{mmamova}
	To test if samples from populations of different years (2018, 2020 and 2021) could be pooled, a hierarchical Analysis of Molecular Variance (AMOVA) was carried out using the package poppr \citep{kamvar2014}. To test the null hypothesis of no population structure between years, 23 populations that were sampled twice (2018 and 2020; table S1.1) were used in this test and year was hierarchically nested within population.\\
	
	\subsubsection{Population-level statistics}
	For each population, we calculated the number of private alleles (NP) with the R-package poppr \citep{kamvar2014}. Rarefied allelic richness (AR), expected heterozygosity (H\textsubscript{e}), observed heterozygosity (H\textsubscript{o}) and inbreeding coefficient (F\textsubscript{IS}) were calculated with the R-package hierfstat \citep{goudet2020}. These measures should only be interpreted relatively within the studied haplodiploid system.\\
	
	To check the robustness of the population-level statistics in light of skewed sample sizes per population (supplementary material S1), we repeated these population-level statistics on a subsampled dataset. For this dataset, we randomly selected 10 samples if a population has more than 10 samples and omitted the two populations of sample size of 5.\\
	
	\subsubsection{Genetic distance}
	Genetic distance between populations was quantified using Nei's standard genetic distance D\textsubscript{S} \citep{nei1972} calculated with the package hierfstat \citep{takezaki1996, goudet2020}. To check which populations have on average the highest genetic distance to other populations, the mean D\textsubscript{S} per population was calculated. Similar calculations for genetic differentiation indices F\textsubscript{ST} and Jost's D \citep{weir1984, jost2008, keenan2013} can be found in supplementary material S4, but are considered less suitable for haplodiploids as they are by definition based on expected diploid allele frequencies under Hardy-Weinberg equilibrium.\\
	
	To check robustness of this analysis against skewed sample sizes (see previous section), we repeated the analysis on a subsampled dataset (10 samples are randomly selected if a population has more than 10 samples and the two populations of sample size of 5 were omitted).\\
	
	
	\subsubsection{Isolation-by-distance}
	We compared patterns of IBD between coastal and inland populations. Only samples from Flanders were used, where sampling was spatially covering all populations known to exist at that time. This way, a balanced distribution of geographic distances across the range of possible distances. IBD was examined with the multivariate approach distance-based redundancy analysis (dbRDA) \citep{diniz-filho2013} because the suitability of Mantel tests to examine IBD is highly debated \citep{meirmans2015}. First, a Principal Coordinate Analysis (PCoA) to the Nei's genetic distance D\textsubscript{S} matrix was applied. The resulting PCoA-axes were then used as a response matrix in a Redundancy Analysis (RDA) to correlate them with the geographic coordinates. The adjusted coefficient of determination R\textsuperscript{2} of the RDA's (the proportion explained by the constrained axes) was used to compare the strength of IBD for coastal and inland populations. Permutations tests (n = 999) were used to check if the R\textsuperscript{2} significantly differed from zero. Nei's genetic distance D\textsubscript{S} between populations was plotted against the pairwise geographical distance for the two regions. To check if the intercept and/or slope differ between the two regions, a permutation test was applied with the R package lmPerm with a maximum of 10,000 iterations \citep{wheeler2016} with the formula $D_S \sim distance + region + distance:region$. In general, realized gene flow is dependent on population sizes and dispersal capacity ($\sim N\cdot m$), with spatial configuration of populations a confounding factor. A different intercept in IBD (for the same species) for different regions would be mainly related to differences in population sizes (as $N$ decreases, all else being equal, differentiation will increase due to total number of migrants) and a differing slope to different dispersal capacities.\\
	
	\subsubsection{Discriminant Analysis of Principal Components}
	A Discriminant Analysis of Principal Components (DAPC) was performed with the R-package adegenet to explore between-population structure and differentiation \citep{jombart2008, jombart2010}. DAPC is a multivariate statistical approach wherein data on individual allele frequencies is first transformed using a principal component analysis (PCA) and subsequently a discriminant analysis (DA) is performed. Genetic variation is partitioned into a between-group and a within-group component, maximizing discrimination between groups (i.e. populations in this case). DAPC does not assume a population genetic model, which make it more suitable for haplodiploid mating systems than Bayesian clustering algorithms to analyze between-population structure \citep{jombart2010, grunwald2011}. We performed a DAPC for all populations together and the coastal and inland populations separately. Populations are used as the a priori groups (no K-means clustering is run). A cross-validation with 1,000 replicates was performed for the three sub-analyses to retain an optimal number of PC-axes with the function xvalDapc from the R-package adegenet \citep{jombart2008, kamvar2015}. We use scatterplots of the first four principal components of the DAPC analyses to visualize within and between population variation in the study area.\\
	
	\subsubsection{Assignment tests}
	To identify the most likely population of origin for all individuals based on the genetic profiles of these individuals and the populations, we performed individual assignment with \textsc{GENECLASS2} tests to identify immigrant individuals or individuals that have recent immigrant ancestry \citep{rannala1997, piry2004}. The most standard method currently is to perform first-generation migrant detection \citep{paetkau2004}. However, as this model is explicitly based on the sampling of gametes from haploid or diploid populations, we considered this method inappropriate for a haplodiploid mating system. Therefore, we used a Bayesian criterium to estimate likelihoods for each individual to originate from any of the given populations based on allele frequencies, combined with the probability computation from the same method using 10,000 Monte Carlo simulations and the expected type \Romannum{1} error rate (alpha) set to 0.01 \citep{rannala1997}. These probability computations are based on random drawing of alleles using allele frequencies directly estimated from the reference population samples and are thus mating system independent. With this method, we can identify individuals that are immigrant or have recent immigrant ancestry. However, interpretation should be done with care, as these assignment or exclusion methods are (compared to first-generation migrant detection) known to be prone to over-rejection of resident individuals and thus might overestimate gene flow \citep{paetkau2004, piry2004}. All individuals were considered and all populations were included as possible source populations. We made a flow chart (spatial directed network graph) in QGIS \citep{qgisdevelopmentteam2020} representing the links between sampled populations and putative origin population according to the assignment tests. Arrows in such a graph start at the putative source population according to the assignment tests and end in the sampled population.\\
	
	To check the robustness of the assignment analyses against skewed sample sizes, we repeated the assignment tests for a subsampled dataset (10 samples are randomly selected if a population has more than 10 samples and the two populations of sample size of 5 were omitted).\\
	
	\section{Results}
	In total, wing tips of 867 individuals from 53 populations were genotyped. Five microsatellite loci showed a lot of stutter in the amplification profiles which were hard to score and were therefore discarded from further analysis (AGBro486, -329, -196, -437, -298). Hardy-Weinberg (HW), linkage disequilibrium (LD) and null alleles assumption testing identified a further 7 microsatellites that were left out of the analysis (supplementary material S5): AGBro35, -57, -419 (HW and null alleles), AGBro111 (HW and LD), AGBro20, -16 (null alleles) and AGBro138 (HW, LD, null alleles). This resulted in a total of 21 microsatellite loci for further population genetics analyses. If an individual had more than 8 loci with missing data, it was discarded from the analysis beforehand (10.5$\%$; 102 out of 969). A total of 133 alleles with an average of 6.3 alleles per locus (ranging from 3 to 14) were observed across the 21 microsatellite loci. The resulting dataset had an overall 3.48$\%$ of missing data for the 867 individuals.\\
	
	Hierarchical AMOVA comparing genetic variation between populations and between years (for a dataset of 522 samples (of 867) from 23 populations sampled both in 2018 and 2020), showed that sampling year explained 0.19$\%$ of the variation (supplementary material S6). Thus, it was decided to pool the different years for the subsequent population genetics analyses (supplementary material S1).\\
	
	Subsampling performed on several of the analyses (as sample sizes per population ranged between 5 and 35; supplementary material S1) showed that our results are robust for skewed sample sizes per population (details below).
	
	\subsection{Population level statistics}
	The complete table with number of private alleles (NP), rarefied allelic richness (AR), expected (H\textsubscript{e}) and observed (H\textsubscript{o}) heterozygosity and inbreeding (F\textsubscript{IS}) can be found in supplementary material S7. Table \ref{Tab4.1} gives a summary of these population-level statistics for coast and inland separately. Allelic richness, and expected and observed heterozygosity were in general lower in the inland populations (table \ref{Tab4.1}). Inbreeding was in general high and very variable across all populations (table \ref{Tab4.1}), which is expected for a haplodiploid system \citep{zayed2004}. From the 10 populations with the highest F\textsubscript{IS}, six were from the mid- and eastern part of the coast and four from inland populations, including the two Walloon populations (populations 52-53). The repeated analysis for the subsampled dataset yielded very similar results (supplementary material S7).\\

	
		\begin{table}[h!]
			\begin{center}
				\begin{footnotesize}
					\caption{summary table for the population-level statistics (Statistic): rarefied allelic richness (AR), expected (H\textsubscript{e}) and observed (H\textsubscript{o}) heterozygosity, and inbreeding coefficient (F\textsubscript{IS}). Summary calculations are for two regions: Coast (coastal France and coastal Flanders combined) and Inland (inland Flanders and inland Wallonia combined). For each statistic and region, the mean, standard deviation (SD), minimum (min) and maximum (max) of the range are given. To check the difference of a statistic between regions, a two-sided t-test was performed and t-value (t), degrees of freedom (df) and p-value are given.}  \label{Tab4.1}
					
					\begingroup
					\setlength{\tabcolsep}{6pt} % Default value: 6pt
					\renewcommand{\arraystretch}{1.5} % Default value: 1
					\begin{tabular}{p{1.5cm} l r r r r p{1cm} p{1cm} p{1.2cm}} %@{\hspace{3pt}} l}
					
					\toprule
					\textbf{Statistic} & \textbf{Region} & \textbf{mean} & \textbf{SD} & \textbf{min} & \textbf{max} & \multicolumn{1}{r}{\textbf{t}} & \multicolumn{1}{r}{\textbf{df}} & \multicolumn{1}{r}{\textbf{p-value}} \\
					\midrule
					\multirow{2}{*}{AR} & Coast& 2.61 &0.09 &2.41 & 2.84 &
					\multicolumn{1}{r}{\multirow{2}{*}{6.21}} &  \multicolumn{1}{r}{\multirow{2}{*}{14.84}} & \multicolumn{1}{r}{\multirow{2}{*}{$<$0.001}}\\
					& Inland& 2.33 &0.15 &2.11 &2.61 &&&\\
					\arrayrulecolor{black!30}\midrule[0.3pt]
					\multirow{2}{*}{H\textsubscript{e}} & Coast& 0.59& 0.03 & 0.53& 0.64& 
					\multicolumn{1}{r}{\multirow{2}{*}{5.82}} &  \multicolumn{1}{r}{\multirow{2}{*}{14.41}} & \multicolumn{1}{r}{\multirow{2}{*}{$<$0.001}}\\
					& Inland& 0.51 &0.05 & 0.43 & 0.58 &&&\\
					\arrayrulecolor{black!30}\midrule[0.3pt]
					\multirow{2}{*}{H\textsubscript{o}} & Coast& 0.52 & 0.06 & 0.36 & 0.62& 
					\multicolumn{1}{r}{\multirow{2}{*}{3.5}} &  \multicolumn{1}{r}{\multirow{2}{*}{19.38}} & \multicolumn{1}{r}{\multirow{2}{*}{0.002}}\\
					& Inland& 0.45 &0.07 & 0.33 & 0.54 &&&\\
					\arrayrulecolor{black!30}\midrule[0.3pt]
					\multirow{2}{*}{F\textsubscript{IS}} & Coast& 0.11 & 0.09 & -0.02 & 0.36 & 
					\multicolumn{1}{r}{\multirow{2}{*}{-0.18}} &  \multicolumn{1}{r}{\multirow{2}{*}{20.60}} & \multicolumn{1}{r}{\multirow{2}{*}{0.86}}\\
					& Inland& 0.12 &0.09 & -0.03 & 0.27 &&&\\
					
					
					\arrayrulecolor{black}\bottomrule
				\end{tabular}\endgroup
			\end{footnotesize}
		\end{center}
		%\end{sidewaystable}
		\end{table}
	
	
	\subsection{Genetic distance}
	Figure \ref{fig4.2} shows the pairwise D\textsubscript{S} (Nei's standardized genetic distance) between all populations. The 10 populations with the highest mean D\textsubscript{S}, which have the highest average differentiation from all other populations, were inland populations, including the two Walloon populations (table S4.1). Similar figures for pairwise differentiation measures F\textsubscript{ST} and Jost's D can be found in supplementary material S4, giving similar results. Genetic distances were overall large and variable among inland populations (right upper corners Fig. \ref{fig4.2}) and small among coastal sites (left lower corner Fig. \ref{fig4.2} and y-axis in figure \ref{fig4.3}). The genetic distances between coastal and inland regions are medium to large. The repeated analysis for the subsampled dataset yielded similar results (supplementary material S4). Extra hierarchical AMOVA's for coastal and inland regions separately also confirmed there is more differentiation between populations inland than at the coast (supplementary material S6).\\
	
	\clearpage
	\hbox{}
	\vspace*{\fill}
	\begin{figure}[h!]
		\begin{center}
			\includegraphics[width=\textwidth]{figures/Figure4_2.png}
		\end{center}
		\begin{footnotesize}
			\caption{graphical matrix representation of Nei's standardized genetic distance (D\textsubscript{S}): blue are low, white are mid, and red are high genetic distance values between pairwise populations. The x- and y-axes represent the population ID, subdivided in the four different regions. Genetic distances are symmetrical and consequently the matrix is mirrored along the diagonal. There is overall large genetic distances within the inland regions (right upper corner, populations 41-53) and small genetic distances within the coastal regions (left lower corner, populations 1-40). The genetic distances between coastal and inland regions (y-values 41-53 with x-values 1-40, or vice versa) are medium to large. \label{fig4.2}}
		\end{footnotesize}
	\end{figure}
	\vspace*{\fill}
	\clearpage
	
	\subsection{Isolation-by-distance}
	Isolation-by-distance (IBD) is only calculated for coastal and inland Flanders as they have the most complete spatial coverage in sampling. An RDA performed with Nei's genetic distance D\textsubscript{S} and the pairwise geographical Euclidian distances indicate there is spatial genetic structure, and most strongly for the Flanders inland region: proportion explained by the constrained axes (or R\textsuperscript{2}) is 22$\%$ for all populations together, 23$\%$ for the coastal region and 61$\%$ for the inland region. All explained variance is larger than zero (p=0.001; Df=2). Adjusted R\textsuperscript{2} is 18$\%$, 18$\%$ and 51$\%$ respectively. The relationship between D\textsubscript{S} and geographical distance is shown in Figure \ref{fig4.3}. Both the intercept (region: p $<$ 0.001; SS=0.342 (Type \Romannum{3}); Df=1) and slope (distance-region interaction: p $<$ 0.001; SS=0.088 \Romannum{3}; Df=1) differed significantly between regions according to the permutation test (adjusted R\textsuperscript{2} of complete model was 58$\%$). Some coastal datapoints (Fig. \ref{fig4.3}, in brown) are situated on the inland trendline. These are populations from the mid- and eastern part of the coast (populations 26-38; Fig. \ref{fig4.1}).\\
	
	\subsection{Discriminant Analysis of Principal Components}
	Figure \ref{fig4.4} gives scatterplots for the Discriminant Analysis of Principal Components (DAPC analysis) for the complete dataset. Scatterplots for the coastal and inland regions separately are given in supplementary material S8. The coastal Flanders populations (light brown in Fig. \ref{fig4.4}) clearly clump together, with coastal France (1-3, dark brown in Fig. \ref{fig4.4}) partially overlapping. A similar pattern can be seen in the DAPC for the coastal regions separately (Fig. S8.1A) and are in line with a pattern of isolation-by-distance. Inland populations show more between-populations structure (greens in Fig. \ref{fig4.4}; Fig. S8.1B). This is also confirmed by the separate analyses for both regions: for the coast, the first two principal components together explain 27.2$\%$ of the variation, while for inland this is 54.6$\%$.\\
	
	\begin{figure}[h!]
		\begin{center}
			\includegraphics[width=\textwidth]{figures/Figure4_3.png}
		\end{center}
		\begin{footnotesize}
			\caption{Isolation-by-distance (IBD) graph with Nei's genetic distance (D\textsubscript{S}) plotted against Euclidian geographical distance (in km), separately for coastal and inland Flanders. Lines are smoothers plotted for indicating the trend, not regressions: statistical tests are done through RDA and a permutation test (see main text). These showed that the spatial genetic structure is higher for the inland regions and that both intercept and slope differ between the regions. \label{fig4.3}}
		\end{footnotesize}
	\end{figure}
	
	\begin{figure}[h!]
		\begin{center}
			\includegraphics[width=\textwidth]{figures/Figure4_4.png}
		\end{center}
		\begin{footnotesize}
			\caption{scatterplot for DAPC results on complete dataset for A) the first two and B) the third and fourth principle components. Labels indicate population ID and point and label colors refer to the regions from figure \ref{fig4.1}: coastal France (dark brown), coastal Flanders (light brown), inland Flanders (light green), inland Wallonia (dark green). Coastal populations cluster together genetically in a large point cloud while inland regions show more genetic differentiation. \label{fig4.4}}
		\end{footnotesize}
	\end{figure}
\clearpage
	
	\subsection{Assignment tests}
	Figure \ref{fig4.5} summarizes results of all the assignment tests per region and figure \ref{fig4.6} depicts derived flow charts for genetic links between regions and within the inland region (not for within the coastal region as these are very numerous; Fig. \ref{fig4.5}). The assignments depict immigrants or individuals with recent immigration ancestry, probably up to two generations \citep{rannala1997}, i.e. not pure first generation migrants as in \citet{paetkau2004}.\\
	
	Three main patterns can be deduced from the assignment tests: high genetic connectivity at the coast, restricted genetic links within inland, and asymmetrical gene flow from coast to inland. Within coastal Flanders, genetic connectivity among populations is substantial (Fig. \ref{fig4.5}): a relative low number of individuals are assigned to their original sampled population (47$\%$) but almost all are assigned within the region (97$\%$). Within the coastal Flanders populations, populations from the west coast are the largest source of gene flow to populations at the mid- and east coast. Individuals from coastal France are mainly assigned to the region itself, but there is connectivity with coastal Flanders in both directions (Fig. \ref{fig4.5}). Inland regions have relatively high numbers of individuals assigned to the original sampled populations (79$\%$ and 90$\%$) compared to coastal Flanders (47$\%$). However, they also have a higher number of individuals assigned to another region (13$\%$ and 10$\%$ compared to 3$\%$ in coastal Flanders), mainly coming from the western part of coastal Flanders and France (Fig. \ref{fig4.5} and \ref{fig4.6}). The only genetic connectivity present within the inland populations is a cluster in east Flanders (populations 44-52; Fig. \ref{fig4.5}B), wherein population 46 (Geel-Bel) seems to be a central source population for surrounding population. Thus, inland populations are genetically more isolated from each other, but there is a genetic influx from the coast, which seems to happen unidirectional from coast to inland (Fig. \ref{fig4.5} and \ref{fig4.6}). The results of the assignment tests with the subsampled dataset are similar but have some minor differences (supplementary material S9). Nevertheless, the main conclusions---gene flow high within coast, restricted inland, asymmetrical from coast to inland---remain robustly present.\\
	
	The pairwise geographical distances between sampled and assigned populations show that 75$\%$ of the distances between sampled and putative source populations lie below 20 km, with the largest distance being 320 km from coastal France to the south of Belgium (supplementary material S9, Fig. S9.3). The largest distance within Flanders between a putative source coastal population and sampled inland population is 201 km.\\
	
	\begin{figure}[h!]
		\begin{center}
			\includegraphics[width=\textwidth]{figures/Figure4_5.png}
		\end{center}
		\begin{footnotesize}
			\caption{barplot of summarized results of the assignment tests for all populations from each sampled region (x-axis). If an individual was assigned to its original population where it was sampled, the barplot-area is filled with grey. If an individual was assigned to another population within the same region or to another region, the barplot-area is colored by region (dark brown: assigned to coastal France, light brown: assigned to coastal Flanders, light green: assigned to inland Flanders). Total number of samples per region (n) is indicated above each barplot. Apart from samples assigned to their original population, there were no other individuals assigned back to inland Wallonia (would have been dark green colored in barplot). Coastal Flanders has the least samples assigned back to their original populations. However, most were assigned within the region, indicating high genetic connectivity within coastal Flanders. If samples were assigned to another region, they were always from coastal regions (brown colours; Fig. \ref{fig4.6}). \label{fig4.5}}
		\end{footnotesize}
	\end{figure}

	\clearpage
	\thispagestyle{empty}
	%\newgeometry{margin=1cm}
	\begin{figure}[h!]
		\vspace*{-1cm}
		\makebox[\linewidth]{
			\includegraphics[width=\linewidth]{figures/Figure4_6.png}
		}
		\begin{footnotesize}
			\vspace*{-4mm}
			\caption{flow chart of genetic links (A) between regions and (B) within the inland region for \textit{B. rostrata} according to assignment to putative source populations. Genetic links within the coastal region are not depicted as these were too numerous (Fig. \ref{fig4.5}). The links represent the number of individuals assigned to a putative origin population (start of the arrow) that were caught in the sampled population (end of the arrow). Brown arrows are links starting from the coast, green arrows start from inland populations. The thicker the end of the arrow, the higher the number of individuals assigned to the putative source. Genetic links are present from coast to inland, but not from inland to coast (A; fig. \ref{fig4.5}). Within the inland region, there are only genetic links within the cluster of populations 44 to 50 (B). The main source within inland Flanders is population 46 (Geel-Bel), which is the largest and oldest known inland population in Flanders. Individuals from other inland populations (41-43; 52, 53) are either assigned to their sampled population or are assigned to a coastal population. These source populations are predominantly from the west of coastal Flanders (A).  \label{fig4.6}}
		\end{footnotesize}
	\end{figure}
	\clearpage
	

	\section{Discussion}
	\textit{Bembix rostrata} populations from inland sandy regions exhibited low levels of gene flow, low genetic diversity, and high genetic differentiation, contrary to the coastal region, which has an overall high level of genetic connectivity. Asymmetrical gene flow from the coast to inland demonstrate that the species is---contrary to expectations based on its behavior and poor colonization capacity---capable of dispersing to existing populations at a distance of 200 to 300 km.\\
	
	\textit{Bembix rostrata} has always been considered to be a philopatric species, not able to easily colonize vacant habitat \citep{nielsen1945, larsson1986}. The retrieved pattern of genetic structure and gene flow within and across sandy regions of Belgium clearly demonstrates this does not prevent gene flow between already existing populations. The species is known to be gregarious, with on the one hand local nest choice behavior showing positive density dependence because of conspecific attraction, and on the other hand individuals making consecutive nests close to one another \citep[chapter \ref{chapter2}]{larsson1986, batsleer2022}. This intragenerational individual site fidelity combined with the reported low colonization capacity \citep{nielsen1945, bogusch2021}, made this species a presumed poor disperser. Our results show that dispersal between existing populations is not highly restricted, especially in a well-connected, stepping stone landscape, such as in the Belgian coastal dunes. Likely, female colonization capacities are not restricted by the species' movement capacity, but mainly by the settlement phase of dispersal---with conspecific attraction of crucial importance in \textit{B. rostrata} \citep[chapter \ref{chapter2}]{batsleer2022}. When conspecifics are already present in existing populations, the settlement phase is less restricted, which can explain the pattern of gene flow between existing populations.\\
	
	Alternatively, as colonization capacity in \textit{B. rostrata} is clearly disconnected from gene flow, the latter may be equally or largely driven by male dispersal. Male-biased dispersal has indeed been found to be most common in other bees and wasps \citep{johnstone2012}. Given the protandry of the species, such dispersal may be common in the period prior to female emergence, as a strategy to avoid strong (kin) competition \citep{bonte2012, baguette2013}. As we only sampled females, we cannot quantify a biased dispersal strategy, with for instance genetic spatial autocorrelation analyses \citep{banks2012}. Because of the haplodiploid mating system, indirect analyses by the comparison of nuclear and mitochondrial markers are neither suitable because nuclear introgression is reduced relative to mitochondrial introgression \citep{patten2015}.\\
	
	Effective dispersal rates, resulting in establishment, depend on the species' (i.e. female) capacity to move and settle, but also on the size of the source population. Colonization at short distance of vacant, newly emerging pioneer habitats remains overall much more likely than distant colonization. In addition, during periods of exceptionally suitable environmental conditions, e.g. warm summers or years with high resource abundance (nectar, prey), any local overshooting of carrying capacities may further increase the magnitude of gene flow, both in terms of extent \citep[the threshold in density dependent dispersal;][]{kun2006, best2007} and spatial scale \citep[the fatness of the dispersal kernel][]{bitume2013}. We hence hypothesize that the establishment of new populations may only succeed when Allee-effects are overcome by the simultaneous settlement of multiple females into a single cluster. Such demographic contributions are often overlooked in the dynamics of spatially structured populations in both population genetics and connectivity studies \citep{lowe2017, drake2022}. With our genetic data, we could not detect if a population was recently established or not, but at least for one population (Averbode, population 44 in Fig. \ref{fig4.1}) we know the area was only recently made suitable (ca. 2010). For this population, but also other nearby populations, assignments tests indicate that genetic connections mainly originate from large (meta)populations at the west coast and within inland from one nearby population inland (Geel-Bel, population 46, Fig. \ref{fig4.6}B). Dunes from the west coast (populations 1-25) are known to hold the highest number of old and large (meta)populations of \textit{B. rostrata} in Belgium \citep[unpubl. monitoring data;][confirmed by citizen science data from observation.org, and confirmed to a certain degree by F\textsubscript{IS} results, supplementary material S7]{klein2004}, while Geel-Bel is the single largest and oldest known population in the inland region. This indirectly confirms the very often neglected role of source population size compared to absolute dispersal potential (the dispersal kernel) for gene flow in metapopulations and their dynamics. In larger populations, the absolute number of dispersers will be larger, even if per-capita dispersal rate is constant for different population sizes. The important role of demographics could also explain the observations of \citet{bogusch2021}, who observed highly restricted local and regional colonization abilities of two small and isolated populations in the Czech Republic.\\
	
	The strong impact of the size of source populations on gene flow likely also underlies the general asymmetrical gene flow from coastal to inland populations. In the well-connected, stepping stone landscape at the (west) coast, short-distance dispersal---dispersal related to routine movements of resource exploitation \citep{vandyck2005}---results in a pattern of weak isolation-by-distance. As the genetic population structure is dominated by such short-distance dispersers, the proportionally low numbers of long-distance dispersers do not leave a detectable genetic signal within the coastal network. However, some long-distance dispersers---males or females---from the coast appear to reach inland populations and leave a proportionally larger signal of gene flow in these smaller populations. In addition, also within regions, such demographic signals are picked-up. First, populations from the west coast are the largest source of gene flow to populations at the mid- and east coast, known to hold fewer and smaller populations (unpubl. monitoring data; confirmed by F\textsubscript{IS} values, supplementary material S7). Second, as mentioned previously, the largest and oldest known inland population (Geel-Bel, population 46) leaves the strongest genetic signal in the surrounding inland populations (Fig. \ref{fig4.6}B).\\
	\clearpage
	
	In addition to the demographic causes, other mutually non-exclusive mechanisms may underlie the retrieved pattern of asymmetrical gene flow. Although we consider it less likely in our case, wind has been put forward as a dominant factor for long-distance flight behavior and migration in insects \citep{alerstam2011, knight2019, leitch2021}. In the focal study area, the predominant wind-direction is from coast to inland (SW and WSW), which could reinforce the demographic process at the coast. Alternatively, dispersal capacity itself could be more restricted in inland regions as well. Indeed, our isolation-by-distance results suggest that apart from the intercept, the slope differs between the regions as well. The intercept is related to population sizes ($N$): if $N$ decreases, all else being equal, differentiation will increase due to genetic drift and a lower total number of migrants. The steeper slope could be---apart from the influence of spatial habitat configuration \citep{vanstrien2015}---due to dispersal capacity being more restricted in the inland region. A difference in dispersal capacity within a species can result from evolved dispersal reductions in highly fragmented landscapes \citep{cheptou2017} where costs of dispersal are highest or where spatio-temporal patch turnover is lowest \citep{bowler2005, bonte2012, duputie2013}. Such conditions may indeed be more prominent for the more fragmented and locally stable populations from the inland sandy regions. Only a combination of behavioral experiments and/or quantifying physiological differences in flight metabolic performance may shed light on the likelihood of such processes \citep{hanski2004}.\\
	
	When not all possible source populations are sampled and included, assignment tests might give rise to misleading results \citep{rannala1997, cornuet1999}. In our sampling design, sampling in Flanders covered all known populations at the time and the main known large populations in Wallonia. Nevertheless, unsampled potential source populations might be present across the border in the Netherlands and France. Consequently, the populations from northern inland Flanders (populations 42, 43, 49-50) could be connected to Dutch populations and not be as isolated as our results suggest. Especially for the connections from coastal France to inland (Fig. \ref{fig4.6}), intermediate populations might be present \citep{bitsch1997, barbier2007}. Therefore, the detected connections between coastal France and inland regions might not be from directly dispersing individuals, but through an indirect connection of an unsampled French population. If this is the case, the maximum distance from a direct connection within Flanders would be 201 km instead of 320 km. A second bias that can arise with the specific assignment method we used---a method suitable for haplodiploids---has been detected with a simulation study: the possibility over-rejection of resident individuals, ultimately overestimating gene flow \citep{paetkau2004, piry2004}. Considering our results, the absolute number of genetic links might be lower and the maximum dispersal distance an overestimation. However, as the absolute number of genetic links is also dependent on sample sizes and number of genetic markers, our interpretations are essentially relative and will still hold: more restricted gene flow in inland populations than at the coast and asymmetrical gene flow from coast to inland.\\
	
	Pollinator conservation, and more specifically that of wild bees, is currently a major topic of interest to policy and science \citep{potts2016}. A major inherent factor complicating the interpretation of population genetics results of hymenopteran species in a classical conservation genetics framework is the haplodiploid mating system. Males are haploid (unfertilized) and females are diploid, which results in non-symmetrical inheritance of genes across generations. In such a mating system, inbreeding coefficients will be inherently high and effective population sizes low due to, for instance, purging effects on deleterious alleles in haploid males \citep{zayed2004, zayed2009}. These specific attributes render metrics based on assumptions of HW-equilibrium and population genetics models difficult to apply and interpret, as different genetic processes will predominate in a haplodiploid conservation genetics framework \citep{zayed2009}. While classical population genetic analyses may be used if not overinterpreted \citep{cerna2013, sanllorente2015}, we decided to report mating system independent analyses, such as a multivariate approach DAPC and classical assignment tests. The descriptive statistics provided should be interpreted with care and only be considered relatively within the study system. In our opinion, future (modelling) studies should further elucidate the potential biases for haplodiploid systems when using classical population genetics studies that are based on diploid mating systems and related assumptions. In general, integration of haplodiploids in the conservation genetics framework is lacking, although about 15$\%$ of all animal species are haplodiploid \citep{evans2004, lohse2015}.\\
	
	Connectivity remains difficult to generalize among species and even for different landscapes within a single species. In Flanders, the genetic connectivity of the grayling butterfly (\textit{Hipparchia semele})---also occurring in sandy habitats---was in general much more restricted and gene flow slightly higher in the inland region compared to the coast \citep{dero2021}. Differences in life history traits and niche are the most likely reason for the contrasting results with \textit{B. rostrata}. These diverging findings for the focal region between two species from sandy habitat stress that connectivity is a trait of both species and population configuration combined---and should as such be considered in conservation policies. Additionally, for \textit{B. rostrata} itself, the asymmetrical gene flow and discrepancy in connectivity between different regions were important to discuss the disconnection of gene flow from colonization capacity and consider the plausible role of demographics for the observed genetic connectivity. As such, it is crucial to combine and compare results from different regions for a single species to fully understand possible mechanisms of gene flow. It remains to be tested whether our insights on the species metapopulation structure can be scaled up towards the species' full range. The general insight that the species' low colonization capacity does not imply low levels of gene flow are likely to hold across other well-connected, healthy and large (meta)populations. Nature management implications discussed below are potentially helpful for \textit{B. rostrata} populations across Europe, depending on the local and regional context.\\
	
	
	\subsection{Conservation implications}
	Our findings have direct implications for nature management and conservation of the flagship insect species \textit{B. rostrata} at both local and landscape scale. At the coast, a well-connected metapopulation occurs, while inland populations show restricted gene flow in a fragmented sandy habitat landscape. Moreover, there is an asymmetrical genetic influx from coast to inland, which we mainly interpret as being linked to the larger population sizes at the coast. The species' poor colonization capacity, resulting in a low establishment probability, should be considered disconnected from gene flow between existing populations, as the latter seems much less restricted. To maintain the well-connected, large coastal populations, conservation should focus on local management and internal processes to ensure a constant amount of suitable habitat through time. Ideally, dunes are revitalized with aeolian (wind) dynamics at the landscape level \citep{provoost2011}. However, in a fragmented and urbanized landscape, the current management framework focuses on grazing used as a tool to locally revitalize sand dynamics \citep{provoost2004}. It is recommended to use a heterogenous approach for grazing and grazer type in space and time to reconcile short-term negative (trampling) and long-term positive (open dune landscape) effects of grazing on \textit{B. rostrata} \citep[chapter \ref{chapter3}]{bonte2005, batsleer2022a}. The genetically well-connected landscape and large metapopulation context ensures population recovery and persisting connectivity when implementing such a dynamic management approach.\\
	
	Contrastingly, more isolated, small populations such as in the inland region, need a more cautious approach and management should consequently focus on the protection of individual populations or clusters of nearby populations. Creating extra stepping stones to increase landscape connectivity, which may already be partially present but vacant, might be less effective in the current context, as potential gene flow is not highly restricted in this species. We suggest that the primary focus should be on enlarging (source) population sizes by improving the quality of the local and directly surrounding habitat. This can be achieved by maintaining or creating open, pioneer sand dune habitat. Preventing or removing encroachment preferably happens manually, as grazers should be used with caution in small, isolated populations \citep[chapter \ref{chapter3}]{batsleer2022a}. Apart from nesting resources, sufficient neighboring floral resources for both nectar and prey hunting may also be important to sustain large populations  of \textit{B. rostrata} \citep{kimoto2012, buckles2019}.
	
\clearpage
	\subsection*{Acknowledgements}
	We thank the following persons and instances for permission and access to nature reserves: Johan Lamaire, Guy Vileyn, Koen Maertens, Evy Dewulf and Klaar Meulebrouck from ANB (Agency for Nature and Forests --- Flemish government); Bruno Nicolas from Eden 62 (France); Thierry Paternoster from DEMNA (D\'{e}p. de l'\'{E}tude du Milieu Naturel et Agricole --- Service public de Wallonie); Rika Driessens from IWVA/Aquaduin; Rudi Delvaux from Grenspark Kalmthoutse Heide; Griet Limet from Kempens Landschap. We thank Pieter Vanormelingen from Natuurpunt to update us on the latest observations in new areas of \textit{B. rostrata}. We also thank Maarten Jacobs (Sjacky) and all other friendly volunteers from Averbode Bos- en Heide (Natuurpunt) for help with searching and sampling \textit{B. rostrata} in and around their nature reserve. We also than following people for assistance during sampling: Margaux Boeraeve, Ward Tamsyn, Marc Batsleer, Nadine De Schrijver, Pepijn Boeraeve and Ward Langeraert. We thank Viki Vandomme for assistance during lab work.\\
	We thank 2 anonymous reviewers for their detailed and constructive comments that greatly improved the manuscript. We also thank Jan Van Uytvanck, Laurence Cousseau, Viktoriia Radchuk and Josep D. Asís for additional comments.\\
	Permit numbers for collection of genetic materials and entrance of nature reserves in the study area: TREL2022990S/383 (Republique France --- Ministère de la transition \'{e}cologique; according to Nagoya protocol); DNF/DNEV/JPB/SLA/2020-RS-22 (Service public Wallonie - D\'{e}partement de la Nature et des For\^{e}ts); N69.21232 (La D\'{e}fense – Direction G\'{e}n\'{e}rale Material Resources; access to military domain in Wallonia). Permission to enter governmental nature reserves in Flanders was granted by ANB (Agency for Nature and Forests).\\
	F.B. was supported by Research Foundation --- Flanders (FWO).
	
	\subsection*{Data Accessibility}
	Scripts and data are made available on github:\\
	\url{https://github.com/FemkeBatsleer/PopGenBembix.git}
	
	\subsection*{Author contributions}
	FB, DD, JM, AVB, DM, DB contributed to the study conception and design. Data collection was performed by FB, MG, MD, PV, FT; laboratory work and data analyses were performed by FB, MG, MD. The first draft of the manuscript was written by FB and all authors commented on previous versions of the manuscript. All authors read and approved the final manuscript.
	
	\subsection*{Supplementary material}
	Supplementary files can be found in the dedicated github repository: \url{http://github.com/FemkeBatsleer/SuppPhD}.

\clearpage
\thispagestyle{plain}
\hbox{}
\clearpage

\CenterWallPaper{1}{pictures/CH5-1.png}
\newpage{\thispagestyle{empty}\clearpage}
\cleardoublepage

\ClearWallPaper

\CenterWallPaper{1}{pictures/CH5-2.png}
\newpage{\thispagestyle{empty}\clearpage}
\hbox{}
\clearpage
\ClearWallPaper

\CenterWallPaper{1}{pictures/CH5-3.jpg}
\newpage{\thispagestyle{empty}\cleardoublepage}
\ClearWallPaper
%%%%%%%%%%%%%%%%%%%%%%%%%%%%%%%%%%%%% Chapter 5 %%%%%%%%%%%%%%%%%%%%%%%%%%%%%%%%%%%%%%%%%	
\thispagestyle{plain} % empty
%\CenterWallPaper{1}{CH5.jpg}
\newpage{\thispagestyle{empty}\cleardoublepage}
%\ClearWallPaper
\pagestyle{mainmatter}
\chapter{Strong gene flow across an urbanized coastal landscape in a dune specialist digger wasp} \label{chapter5}
\chaptermark{Landscape genetics}
\lettergroup{\thechapter}

\begin{flushright} \color{gray}Femke Batsleer\\Fabien Duez\\Dirk Maes\\Dries Bonte\\
	
	\vspace{2cm}
	
	This chapter is available as a preprint on bioRxiv:\\ doi.org\slash 10.1101/2023.04.15.537020
	\vspace*{\fill}	
\end{flushright}
\noindent \color{gray} $\lhd$ A male \textit{B. rostrata} drinking nectar on sea holly (\textit{Eryngium maritimum}).\\
\noindent $\lhd\lhd$ Aerial photograph (summer 2018) of the urbanized dune landscape at the west coast (source: geopunt.be). The municipalities visible are: in the west Koksijde, in the middle Oostduinkerke, and in the east Nieuwpoort.

	\color{black}
	\newpage
	\section{Abstract}
	Genetic connectivity is often disrupted by anthropogenic habitat fragmentation, and therefore often a focus in landscape-scale conservation. Landscape genetics methods allow for studying functional connectivity in heterogenous landscapes in detail to inform conservation measures for a species' regional persistence. Yet, for insects, functional connectivity through landscape genetics remains largely unexplored. We studied the functional connectivity in the dune-specialist digger wasp \textit{Bembix rostrata} in a human-altered coastal region in Belgium based on landscape genetics methods. We used an optimization approach to correlate individual genetic distances with landscape resistance distances to deduce the conductance of natural and anthropogenic landscape categories to gene flow. Overall, the populations of this dune-specialist insect are genetically well-connected. Through multi-model inference we could detect---on top of the prominent background process of isolation-by-distance---a weak but consistent signal of urban features facilitating gene flow. However, because urbanisation leads to larger scale fragmentation, its impact on the distribution of populations in the landscape and related effective regional gene flow remains substantial. We discuss the results in the context of movement behaviour and conservation. As this species depends on early-succession dune vegetations, restoring and increasing sand dynamics at the local and landscape scale should be the focus of conservation aimed at the regional species' persistence. This would be more effective for \textit{B. rostrata} than trying to increase habitat connectivity at the landscape scale in the focal human-altered dune ecosystem.\\
	
	\vspace*{\fill}
	\noindent \textbf{Keywords:} landscape genetics, landscape resistance modelling, \textsc{ResistanceGA}, resistance optimization, dispersal, Hymenoptera, Crabronidae
	
	\clearpage
	\section{Introduction}
	Movement of individuals through landscapes is an inherent component of dispersal and can lead to exchange between populations, resulting in gene flow and connectivity. Connectivity can ultimately support a species' regional persistence, as it maintains or supports a wide range of natural processes, such as resource use, demographic rescue, genetic diversity, range expansion, or metapopulation dynamics \citep{hanski1998, crooks2006, mcrae2008, hodgson2022}. The environment in between habitat patches (i.e. the matrix) can impede or facilitate the movement or dispersal of an organism and is in reality more heterogenous than a practical binary simplification of habitat versus homogeneous matrix, especially in human-altered landscapes \citep{hein2003, manel2013}. How organisms react to unfamiliar or inhospitable habitat in fragmented landscapes can be manifold and depends on the specific landscape context and varies between species and individuals \citep{baguette2007, knowlton2010}. More than the structural connectivity---the landscape structure itself, independent from any organism---or distances between populations, such behavioural responses will determine the realized connectivity between populations. The functional connectivity---the connectivity from the species' perspective \citep{tischendorf2000}---is therefore crucial to understand the dynamics of spatially structured populations (or metapopulations) and to inform effective conservation measures \citep{mcrae2006}.\\
	
	A general view on functional connectivity is that many organisms will avoid unfamiliar habitat altogether---they exhibit high reluctance to cross gaps and favour instead considerable detours---making unfamiliar habitat a barrier to movement \citep{develey2008, knowlton2010, driscoll2013}. This is the general assumption in for instance least cost mapping \citep{adriaensen2003}. Unfamiliar habitat can, however, also act as a facilitator to movements when search behaviour of individuals in unfamiliar habitat is more directional---and thus faster---than in favoured habitat \citep{vandyck2005, schtickzelle2006, kuefler2010, knowlton2010}. Or conversely, movements in more suitable habitats can be impeded because of the availability of resources, leading to less directed---and thus slower---exploratory movements than in unfamiliar habitat \citep{vandyck2005, mcrae2008, keller2012}. Independent of the exact mechanism behind this potential (relative) increase in movement through the matrix, the eventual functional connectivity will remain dependent on the distance between habitats and thus the scale of the fragmentation. The same matrix typology or landscape category can therefore potentially facilitate or constrain connectivity in different landscapes, depending on the eventual distance-related costs in terms of survival and energy-use \citep{segelbacher2010, spear2010, bonte2012, richardson2016, haran2017}.\\
	
	How a landscape influences the mobility of an organism has implications for considering regional conservation measures in human-altered landscapes. If animals show high reluctance towards crossing habitat boundaries, creating corridors might be an effective structural connectivity measure to increase functional connectivity between subpopulations. Such regional landscape-scale conservation measures, including corridors and stepping-stones, which are based on logical and sound scientific principles, have been widely embraced by conservation practitioners and policy-makers \citep{crooks2006, watts2016}. However, there is much debate about the effectiveness of landscape-scale conservation and the emphasis on increasing connectivity for species' regional persistence in conservation decisions \citep{hodgson2009, humphrey2015, watts2016}. The emphasis of practitioners and policy-makers on landscape-scale conservation is probably biased by a focus towards larger terrestrial vertebrates, for which structural connectivity---such as corridors---seems effective \citep{humphrey2015}. Empirical evidence for many other groups is more limited and equivocal \citep{ovaskainen2008, watts2016}. For instance, it is expected that for many insect species other processes might dominate the regional persistence of spatially structured populations, such as demographic processes or effective population sizes \citep{richardson2016, drake2022}.\\
	
	In such cases, site-based conservation focusing on improving habitat area and quality could be more effective than increasing structural connectivity on the landscape-scale. To effectively direct the limited resources available for conservation measures towards implementations with the highest success rate or biodiversity gain, implementations should be founded in empirical evidence \citep{sutherland2004, ferraro2006, samways2020}. Especially for insects---whose decline and biological value get widespread attention \citep{didham2020, wagner2021}---scientific evidence on the effects of landscapes on functional connectivity and gene flow is still limited \citep{keyghobadi1999, wilcock2007, keller2012, perez-espona2012, watts2016, haran2017, trense2021}.\\
	
	An indirect way of measuring functional connectivity is through molecular genetic data. With landscape genetics methods, one can quantify the effects of landscape composition between habitat patches on gene flow between populations \citep{manel2003, manel2013, balkenhol2015}. Patterns of gene flow reflect realized or successful dispersal and depend on post-dispersal survival and reproduction \citep{spear2010}. Landscape genetics methods allow for inferences regarding connectivity without the need to collect individual movement data. Collecting such field data---for example from telemetry or mark-recapture methods---requires intensive effort or is even unfeasible for many organisms, such as the majority of insects \citep[chapter \ref{chapter1}]{spear2010, batsleer2020}. Only some larger insect species have been employed with tracking devices, and even for these, the biases on the resulting measurements remain insufficiently considered \citep[chapter \ref{chapter1}]{batsleer2020}. Both mark-recapture and telemetry rarely detect, and thus underestimate, long-distance dispersal \citep{ugelvig2012, trense2021, dero2021}. The advancing tools in molecular and landscape genetics can help us understand connectivity and landscape use in species for which direct estimates of habitat use and movement are difficult to obtain with more classical field methods \citep{spear2010}.\\
	
	An interesting ecosystem to thoroughly study functional connectivity and its consequences for nature management is coastal dunes. These dune habitats harbour a specific insect biodiversity typical for early-succession and pioneer vegetations subject to very dry and hot conditions, nourished by the natural processes of wind and sand dynamics \citep{maes2006}. Coastal habitats in northwest Europe, and more specifically Flanders (Belgium), have gone through extensive landscape changes and fragmentation during the past century \citep{provoost2011}. Coastal dunes in Flanders are geological young formations with calcareous sandy soils forming a narrow, linear ecosystem along the coast \citep{provoost2004, decleer2007}. The fragmentation of coastal sandy habitats had two main impacts: loss of total area and decrease in habitat quality of the remaining patches. Urbanization from the interbellum period onwards decreased the total dune area by half during the previous century and separated the major dune entities physically \citep{provoost2004a}. The resulting obstruction of wind and sand dynamics, combined with the loss of agricultural practices (such as grazing of livestock), stimulated the succession and scrub development, decreasing the amount of open dune habitats \citep{provoost2011}. Large herbivores have been introduced in many coastal dune reserves to revitalize dune dynamics as a substitute for natural wind dynamics, but have---due to trampling---a variable effect on local arthropod species \citep[chapter \ref{chapter3}]{bonte2008, vanklink2015, batsleer2022a}. Altogether, coastal dunes in Flanders are a human-altered landscape with complex nature management considerations to take into account in both landscape scale and site-based nature management. In such a landscape, understanding functional connectivity---with its many specialist dune species---is vital to make informed conservation decisions.\\
	
	Here, we study the genetic (functional) connectivity of a dune-habitat specialist digger wasp \textit{Bembix rostrata} in a human-altered coastal landscape in Belgium using landscape genetics methods. Although a population genetics study shows that the subregion is genetically well connected \citep[chapter \ref{chapter4}]{batsleer2022b}, the influence of the surrounding landscape on genetic relatedness between individuals remains unexplored. We use a resistance raster approach to quantify resistance values of the distinctive landscape types in our study area. The resistance values of each landscape type are established through an optimization algorithm, which maximizes correlation between the landscape resistance distances and observed genetic distances between all pairwise individuals. All possible combinations of landscape categories were optimized to be able to perform multi-model inference. This allows us to deduce possible barriers or facilitators in the landscape for this dune-specialist digger wasp. Based on this estimate of functional connectivity, we can make recommendations on the balance between landscape- and site-based conservation.\\
	\clearpage

	\section{Material \& Methods}
		\subsection{Study species}
		\textit{Bembix rostrata} (Linnaeus, 1785; Hymenoptera, Crabronidae, Bembicinae) is a univoltine dune-specialist, gregariously nesting digger wasp from sandy habitats with sparse vegetation, sensitive to grazing management \citep[chapter \ref{chapter3}]{larsson1986, klein2004, bonte2005, batsleer2022a}. Adults are active in summer, showing protandry: females are immediately mated upon emergence by the guarding males, who emerge one to five days earlier \citep{wiklund1977, schone1981, evans2007}. Females show brood care: one individual constructs one nest burrow at a time in which it progressively provisions a single larva with flies \citep{nielsen1945, field2020}. An estimated of up to 5 nests (in northern regions) are produced, each with one offspring \citep{larsson1989}. There are no overlapping generations, as the species overwinters as prepupa and there is only one generation per year. \textit{Bembix rostrata} is often considered to be a philopatric species, as they do not easily colonize vacant habitat and exhibit site fidelity throughout the nesting season \citep[chapter \ref{chapter2}]{nielsen1945, larsson1986, bogusch2021, batsleer2022}. However, a population genetics study across Belgium shows that this does not preclude gene flow between existing populations, including dispersal to distant, more isolated populations \citep[chapter \ref{chapter4}]{batsleer2022b}.\\
		
		\subsection{Genetic sampling}
		Sampling took place in the summer of 2018 in the dunes along the Belgian west coast (Fig. \ref{fig5.1}), from the French-Belgian border (De Panne) to the Yser estuary (Nieuwpoort), stretching 14 km along the coast and reaching maximum 4 km inland. As nests are clustered together at nesting aggregates in grey dunes (EU Habitats Directive habitat 2130), we tried to spatially cover all known nesting locations in the landscape and locally took samples spread within and across nesting areas. The coordinates of each individual sample were saved. Only females were sampled, to solely use diploid individuals in the genetic analyses for this haplodiploid mating system. To minimize the impact of sampling, we used a non-lethal sampling procedure with wing clips, a method shown to produce good-quality DNA for microsatellite PCR amplification \citep{chaline2004}. Forewing tips from live digger wasps were cut on both sides and stored in absolute ethanol, kept at 4\textdegree C after each day of sampling and transferred to a freezer (-18\textdegree C) for longer term storage.\\
		
		DNA extraction and PCR amplification was performed as described in \citep[chapter \ref{chapter4}]{batsleer2022b}, and summarized hereafter. DNA is extracted from the wing tips with a protocol based on Chelex (Biorad). 33 polymorphic microsatellites are PCR-amplified for each sample and electropherograms are scored using Geneious Prime (Biomatters). Microsatellites that were excluded after assumption testing (Hardy-Weinberg, null alleles and linkage disequilibrium) in the population genetics study in \citet{batsleer2022b}, chapter \ref{chapter4} were also excluded here, resulting in 21 microsatellites for the subsequent genetic analysis. A total of 539 samples were successfully amplified.\\
		
		\subsection{Land cover categories}
		We combined two types of land cover datasets available for the study area. For landscape categories within dune areas, we used detailed vegetation maps (resolution 0.4 m) available for the region derived from remote sensing data with a classification machine learning technique \citep{bonte2021}. For landscape categories outside dune areas, we used the ground cover map 2018 provided by the Flemish government (`Bodembedekkingskaart' BBK 1 m resolution, source: geopunt.be). We aligned, reclassified and combined the raster datasets (see online code; supplementary S1 for detailed information on reclassificaitons) to have one raster file with resolution of 1 m of the study area with 7 simplified landscape categories: urbanized (\textit{urban}), \textit{water}, agricultural (\textit{agric}), \textit{beach}, \textit{scrub}, \textit{trees} and open dune (\textit{opend}) vegetation (Fig. \ref{fig5.1}). Category \textit{beach} was added by rasterizing a manually drawn shapefile based on aerial maps and the high water line. For further analyses, we resampled the rasters to work with resolutions of 20 m and 50 m. We deem these resolutions (spatial grains) adequate to capture the turnover in heterogeneity of the focal landscape and to be linked with the ecological processes investigated \citep{mcrae2008, cushman2010}.\\
		
		\begin{figure}[h!]
			\begin{center}
				\includegraphics[width=\textwidth]{figures/Figure5_1.png}
			\end{center}
			\begin{footnotesize}
				%\vspace*{-3mm}
				\caption{overview map of study area: west coast in Belgium from the French-Belgian border (west; dashed red line) to the Yser estuary (east). Sample locations of each individual (blue dots) are shown. and four dune area clusters (blue dashed lines) are shown. Four dune area clusters (blue dashed lines) are ecological clusters deemed relevant to nature management; 1) Westhoek 2) Cabour 3) Doornpanne 4) Ter Yde. The landscape raster categories are shown with a 20m $\times$ 20m resolution (cell size). Neighboring countries of Belgium (BE): France (FR), Luxembourg (LU), Germany (DE), The Netherlands (NL), United Kingdom (GB). Map made with QGIS v3.22 \citep{qgisdevelopmentteam2020}. Background aerial photographs (summer 2018) from Agency for Information Flanders (source: geopunt.be).  \label{fig5.1}}
			\end{footnotesize}
		\end{figure}
	
	
		\subsection{Genetic distance measure}
		To quantify the genetic distance between pairwise individuals, we used a metric based on multiple axes of principle components analysis (PCA). Such PCA-based individual genetic distances perform better in landscape genetics analyses, especially when sample size and genetic structure are low \citep{shirk2017, kimmig2020}. Principal components (PC) were calculated from a multiple contingency table (0, 1, 2 values for each allele for each individual), for which the Euclidean distance among the first 16 principle components was calculate to create a distance matrix \citep{shirk2017}.\\
		
		\subsection{Resistance distance measure}
		\enlargethispage{1\baselineskip}
		To quantify the resistance distance between pairwise individuals (i.e. distance based on the landscape), we used the commute resistance distance for individuals in the \textsc{ResistanceGA} R package \citep{vanetten2017, peterman2018}. The commute distance is the expected time of a random walker between a pair of points (sampled individuals) along a circuit-based resistance surface. It is a measure directly related to circuit-theory-based resistance distances \citep{mcrae2008, peterman2018}.\\
		
		\textsc{ResistanceGA} uses a linear mixed effects model with maximum likelihood population effects (MLPE) parameterization to fit pairwise resistance distances (predictor) to pairwise genetic distances (response). MLPE parameterization accounts for the non-independence among pairwise data \citep{clarke2002, bates2015}.\\
		
		
		\subsection{Landscape resistance optimization}
		To determine the resistance values of each landscape type with \textsc{ResistanceGA} \citep{peterman2018}, we ran an optimization procedure. The pairwise resistance distances are recalculated at each iteration based on the resistance values of the landscape types during the previous iteration. The optimization procedure iteratively searches many possible resistance value combinations to maximize the MLPE model fit (based on log-likelihood), resulting in the best fit between resistance distances and genetic distances \citep{peterman2018}. This optimization procedure is based on a genetic algorithm \citep{scrucca2013}. For each raster, we used the SS\textunderscore OPTIM function in \textsc{ResistanceGA} to optimize the resistance values (see online code). Parameters were set to default, only the maximum number of iterations (max.iter) was set to 300. There are no a priori expectations for the resistance values of the landscape categories, as no information is present regarding habitat selection and movement behaviour at the landscape scale for this species \citep{knowlton2010}.\\
		
		\noindent \textbf{Multicategorical rasters.} We optimized multi-categorical rasters, which take into account several landscape categories at once. Although more complex to parameterize, multi-categorical raster optimization adds more biological realism to the analysis than considering only one single category at a time, which assumes an organism's perception of a landscape to be binary \citep{spear2010}. We ran optimizations for each possible combination of all 7 categories. A raster then consists of one to seven focal landscape categories and all else or `the rest' (for example when categories \textit{urban} and \textit{scrub} are considered, all five other categories are lumped into a third category). The total number of all possible combinations of 7 categories (\textit{urban}, \textit{water}, \textit{agric}, \textit{beach}, \textit{scrub}, \textit{trees} and \textit{opend} vegetation) is 127: $ \sum_{k=0}^{7}{7 \choose k}$. Ascii-files needed as input for \textsc{ResistanceGA} were created for each possible combination. Thus, optimizations are run for each possible combination for the 7 categories.\\
		
		\noindent \textbf{Bootstrap analysis.} To compare the performance of each of the 127 optimized rasters---inlcuding isolation-by-distance (IBD) as null model---we performed a bootstrap analysis. With the RESIST.BOOT function, 75$\%$ of the individuals are subsampled and the MPLE model refitted for each of the 127 optimized rasters and IBD, this for 1000 iterations. Relative support for each model is then compared using the difference in corrected Akaike information criterion (AICc, correction for small sample sizes), number of times the model was the top model across all bootstrap samples, and the average Akaike weight across bootstrap samples. Average marginal R\textsuperscript{2} and log-likelihood across bootstrap samples are also given.\\
		
		\noindent \textbf{Independent optimizations runs.} As the optimization with \textsc{ResistanceGA} is a stochastic process, optimized values can differ from run to run, and it is advised to run all optimizations at least twice to confirm convergence and relative relationships among resistance values \citep{peterman2018}. To check this convergence and account for a possible bias due to the starting cell values of a category in multi-categorical rasters (i.e. the initial order: urban, water, agricultural, beach, scrub, trees, open dune) \citep{peterman2018, kimmig2020}, we ran the multi-categorical raster optimizations and bootstrap analysis a second time with the order of categories inverted (starting with \textit{opend}, ending with \textit{urban}). To confirm convergence more conclusively, we ran optimizations and bootstrap analyses of the seven single category rasters (that consider only one category) four times each (supplementary S2). This is computationally less intensive then running all possible combinations, but gives anyway a broad idea of the robustness of the optimization.\\
		
		\noindent \textbf{Multi-model inference.} As all possible combinations of categories were optimized to obtain resistance values, we were able to use a multi-model inference approach to obtain robust estimates of average weight for each category and average relative resistance value. Multi-model inference is a procedure to account for the uncertainty of parameter estimates related to model selection and is especially useful when no single model is overwhelmingly supported by the data \citep{johnson2004}. We used Akaike weights from the bootstrap analysis considering all the models (i.e. optimized rasters) for each area type. From these, we calculated the summed Akaike weight per category by taking into account a model if the focal category was included. These summed Akaike weights give an indication for the relative importance of each category to explain the pairwise genetic distances. We also calculated relative resistance values of each category. As optimized resistance values depend on the other categories included in a multi-categorical model, interpretation for a focal category should be done relatively to these other categories. To be able to do this for each landscape category, resistance values were rescaled per optimized raster by subtracting the resistance value of the focal category from the other resistance values that were simultaneously optimized. As such, the value of the focal category becomes zero and is considered facilitating to gene flow compared to categories with values larger than zero and a barrier to gene flow compared to categories with values smaller than zero. All models and the relative resistance values per focal category were then summarized by plotting boxplots for each focal category.\\
		
		To get an idea of the possible small scale variation in landscape effects on gene flow we repeated the analyses on four dune area clusters within our study area. These dune areas that are seen as ecological clusters relevant for nature management (Fig. \ref{fig5.1}). These optimizations for the dune area clusters are run with a resolution of 20 m (in contrast to 50 m for the complete study area). Details and results can be found in supplementary S4.\\
		
		The total computational time (runs for all possible combinations of categories) was very high and could not be run on a single desktop. Therefore, we made use of the high performance computing infrastructure of Ghent University (HPC-UGent) to run the optimizations and bootstrapping analyses. By doing this, we could run all possible combinations in the multi-categorical raster optimizations, which is the most ideal for robust statistical inference and allows for model averaging to assess the importance of the different landscape categories. Previous studies reported computational constraints for optimizations with \textsc{ResistanceGA} considering all combinations and used alternative approaches for model comparison based on model selection \citep[e.g.][]{lourenco2019, kimmig2020}.\\
		
		The best-supported models (i.e. optimized rasters) for the complete study area were used to visualize the putative gene flow currents using Circuitscape \citep{mcrae2013, kimmig2020}. Currents were inferred between all pairs of sample locations and the resulting conductance (reverse of resistance) of each grid cell are summed for the final map.\\
		
		The main analyses were performed in RStudio V1.4 \citep{rstudioteam2022} using R v4.1.3 \citep{rcoreteam2020}, see online code. Except for calculations for the current map with the GUI of Circuitscape v.4.0.5 \citep{mcrae2013} and cartographic visualizations in QGIS V3.22 \citep{qgisdevelopmentteam2020}.\\
		
		
	\section{Results}
	
	Results of the top five best supported models with optimizations of multi-categorical models (all possible combinations of landscape categories) after bootstrapping can be found in table \ref{Tab5.1} (full table in supplementary S3). The summed Akaike weights per landscape category are in table \ref{Tab5.2} and the relative resistance values are shown in figure \ref{fig5.2}.\\
	
	In general, there is no overwhelming support for one or a few models, as isolation-by-distance (null model) is one of the best supported models (Distance; \textDelta AICc=1.20 run 1 and 1.42 run 2; table \ref{Tab5.1}) and is a competing model considering \textDelta AICc $<$ 2 with \textit{urban} (the best supported model). In general, R\textsuperscript{2}m's are low (table \ref{Tab5.1}; overall $<$ 1$\%$), Akaike weights are small (avg.weight in table \ref{Tab5.1}) and all \textDelta AICc's are relatively small (table \ref{Tab5.1}; maximum \textDelta AICc was 14.58; supplementary S3); and models with \textDelta AICc ranging between 2--7 have some support and should not be dismissed \citep{burnham2011}. When no single model is overwhelmingly supported by the data, multi-model inference is ideal to deduce average weights and relative resistance values of categories \citep{johnson2004}. From such a multi-model inference based on the optimization of all possible combinations, we can deduce more general patterns from the data (table \ref{Tab5.1}, Fig. \ref{fig5.2}). \textit{Urban} is an important explanatory category to discriminate pairwise genetic distances (table \ref{Tab5.2}), which acts mainly as facilitator to gene flow compared to other landscape categories (table \ref{Tab5.1}, Fig. \ref{fig5.2}). Natural landscape categories (\textit{beach}, \textit{scrub}, \textit{trees}, \textit{opend} ; bottom row in Fig. \ref{fig5.2}) are in general more resistant to gene flow than anthropogenic categories (\textit{urban}, \textit{agric}, \textit{water}; top row in Fig. \ref{fig5.2}) in the focal landscape. As most categories, apart from \textit{urban}, are of little (and inconclusive) importance to discriminate pairwise genetic distances (table \ref{Tab5.2}), the uncertainty of relative resistance values between these categories (especially natural categories, Fig. \ref{fig5.2} second row) should be considered high. The second independent optimization run (included in table \ref{Tab5.1} and \ref{Tab5.2}; Fig. \ref{fig5.2}) confirms these patterns and shows that category \textit{urban} is consistently present in the top best supported models (table \ref{Tab5.1}) and the most important (and only conclusive) category to discriminate pairwise genetic distances (table \ref{Tab5.2}). This is also confirmed by the four independent runs of the single categories, which have consistently urban as the first supported model and IBD as the second best competing model (supplementary S2).\\
	
	Circuitscape current maps (Fig. \ref{fig5.3}) were made for the three best supported models (table \ref{Tab5.1}). As all \textDelta AICc's are relatively small (supplementary material S3), there are many more models worth considering (as done in the multi-model inference, Fig. \ref{fig5.2} and table \ref{Tab5.2}). However, these Circuitscape current maps give good visualisations of how the landscape likely influences genetic connectivity within the study area.\\
	
	The analysis repeated on the dune area clusters (Fig. \ref{fig5.1}) show isolation-by-distance (IBD) is the overall best supported model on this smaller scale (supplementary material S4 and S5). Some specific patterns arise depending on the focal dune area cluster, based on the best supported models and the multi-model inference (supplementary material S4). These in general confirm the overall pattern from the complete study area: anthropogenic categories (\textit{urban}, \textit{agric}, \textit{water}) are more facilitating to gene flow than natural landscape categories. There are two exceptions: for the dune area Doornpanne (dune cluster 3 in Fig. \ref{fig5.1}), \textit{trees} is a facilitating landscape type to gene flow, and for Cabour (dune cluster 2 in Fig. \ref{fig5.1}), \textit{agriculture} is a barrier to gene flow. This is probably related to the specific habitat composition and landscape context for these dune areas: Doornpanne has relatively much encroached area (trees and shrubs), while Cabour is the dune area most embedded in agricultural land (Fig. \ref{fig5.1}).\\
	
			\clearpage
			\newgeometry{margin=2cm}
			\begin{sidewaystable}[h!]
				\begin{center}
					\begin{threeparttable}
						
						\begin{footnotesize}
							\caption{Top 5 bootstrap results for the multi-categorical model optimizations for the complete study area for the two independent optimization runs.}  \label{Tab5.1}
							
							\begingroup
							\setlength{\tabcolsep}{6pt} % Default value: 6pt
							\renewcommand{\arraystretch}{1.5} % Default value: 1
							\begin{tabular}{p{1.2cm} l l r r r r r r r >{\raggedleft\arraybackslash}p{0.8cm} >{\raggedleft\arraybackslash} p{0.8cm} >{\raggedleft\arraybackslash}p{0.8cm}} %@{\hspace{3pt}} l}
							
							\toprule
							\textbf{Indepen\-dent run} & \textbf{Model} & \textbf{k} & \textbf{avg.AICc} & \textbf{\textDelta AICc} & \textbf{avg.weight} & \textbf{avg.rank} & \textbf{avg.R\textsuperscript{2}m} & \textbf{avg.LL} & \textbf{n.top} & \textbf{Res. value 1\textsuperscript{st}} & \textbf{Res. value 2\textsuperscript{nd}} & \textbf{Res. value other}\\
							\midrule
							\multirow{6}{*}{Run 1} & urban & 3 & 239704.83 & 0.00 & 0.10 & 4.50 & 0.00010 & -119849.38 & 326 & 1 & - & 2500\\
							& urban.water & 4 & 239705.86 & 1.03 & 0.06 & 7.47 & 0.00015 & -119848.88 & 94 & 8 & 1 & 2500\\
							&Distance & 2 & 239706.03 & 1.20 & 0.05 & 6.75 & 0.00001 & -119851.00 & 180 & - & - & -\\
							& beach.scrub & 4 & 239706.67 & 1.84 & 0.05 & 11.92 & 0.00005 & -119849.29 & 136 & 1 & 2500 & 776\\
							& urban.beach & 4 & 239706.87 & 2.04 & 0.04 & 12.29 & 0.00010 & -119849.38 & 0 & 1 & 2500 & 2438\\
							\arrayrulecolor{black!30}\midrule[0.3pt]
							\multirow{6}{*}{Run 2} & urban & 3 & 239650.74 & 0 & 0.10 & 4.83 & 0.00011 & -119822.34 & 343 & 1 & - & 2500\\
							& water.urban & 4 & 239651.83 & 1.09 & 0.06 & 8.52 & 0.00016 & -119821.87 & 94 & 1 & 2.5 & 2500\\
							& Distance & 2 & 239652.16 & 1.42 & 0.05 & 8.43 & 0.00001 & -119824.07 & 180 & - & - & -\\
							& beach.urban & 4 & 239652.78 & 2.04 & 0.04 & 13.16 & 0.00011 & -119822.34 & 0 & 2500 & 1 & 2488.5\\
							& opend.urban & 4 & 239652.89 & 2.14 & 0.03 & 13.79 & 0.00011 & -119822.39 & 0 & 1884 & 1 & 2500\\
							\arrayrulecolor{black}\bottomrule
						\end{tabular}
						\begin{tablenotes}
							\footnotesize
							\item Notes: combinations of 7 landscape categories compared in bootstrap analysis after optimization of multi-categorical models. The two independent optimization runs (Independent run) have an inverted order of categories, to check convergence and account for a possible bias due to the starting cell values. Urbanized (\textit{urban}), \textit{beach}, \textit{water}, \textit{trees}, agriculture (\textit{agric}), \textit{scrub}, open dune (\textit{opend}). Distance is the isolation-by-distance null model: increasing genetic distance with increasing Euclidean geographic distance. Predictor, landscape category or univariate model; k, number of parameters; avg.AICc, average AICc across all bootstrap iterations; \textDelta AICc, difference in avg.AICc compared to the lowest avg.AICc (the best supported model); avg.weight, average Akaike weight across iterations; avg.rank, average rank across iterations; avg.R\textsuperscript{2}m, average marginal R\textsuperscript{2} across iterations; avg.LL, average log-likelihood across iterations; n.top, number of times the model was the top model across iterations (does not sum op to 1000, as many more models were considered, see supplementary 3); Res.value 1\textsuperscript{st}, the optimized resistance value for the first mentioned landscape category (e.g. \textit{urban} in the second row); Res.value 2\textsuperscript{nd}, the optimized resistance value for the second mentioned landscape category (e.g. water in the second row); notice the inverted order of categories in run 2 compared to run 1; Res.value other, optimized resistance value for all else (combined into one landscape variable).
						\end{tablenotes}
						\endgroup
						
					\end{footnotesize}
				\end{threeparttable}
			\end{center}
			%\end{sidewaystable}
		\end{sidewaystable}
	\clearpage

\newgeometry{margin=2.3cm}
	
	\hbox{}
	\vspace*{\fill}
	\begin{table}[h!]
	\begin{center}
		\begin{threeparttable}
			
			\begin{footnotesize}
				\caption{Summed Akaike weights per category for multi-categorical model optimizations, indicating the importance of a category in all possible multi-categorical optimized rasters to explain the pairwise genetic distances.}  \label{Tab5.2}
				
				\begingroup
				\setlength{\tabcolsep}{6pt} % Default value: 6pt
				\renewcommand{\arraystretch}{1.5} % Default value: 1
				\begin{tabular}{p{1.5cm} >{\raggedleft\arraybackslash}p{2cm} >{\raggedleft\arraybackslash}p{2cm}}
					
					\toprule
					\textbf{Category} & \textbf{Sum weights run 1} & \textbf{Sum weights run 2}\\
					\midrule
					urban & 0.496 & 0.501\\
					agric & 0.357 & 0.335\\
					beach & 0.356 & 0.376\\
					trees & 0.338 & 0.307\\
					water & 0.297 & 0.315\\
					scrub & 0.286 & 0.259\\
					opend & 0.246 & 0.258\\
					
					\bottomrule
				\end{tabular}
				\begin{tablenotes}
					\small
					\item Note: the Akaike weight for an optimized model was added to the summation (Sum weights) if the focal category (Category) was in it.
				\end{tablenotes}\endgroup
			\end{footnotesize}
		\end{threeparttable}
	\end{center}
	\end{table}
	\vspace*{\fill}
	\clearpage
	\hbox{}\vspace*{\fill}
		\begin{figure}[h!]
			\begin{center}
				\includegraphics[width=\textwidth]{figures/Figure5_2.png}
			\end{center}
			\begin{footnotesize}
				%\vspace*{-3mm}
				\caption{relative resistance values from the optimization of multi-categorical models for the complete study area. Anthropogenic landscape categories (\textit{urban}, \textit{water}, \textit{agric}=agriculture) are in the top row, natural landscape categories (\textit{beach}, \textit{scrub}, \textit{trees}, \textit{opend}=open dune) in the bottom row. The two independent runs are combined as paired boxplots for each category in each panel (left run 1, right run 2). As optimized resistance values depend on the other categories included in a model, interpretation for a focal category (panel) should be done while holding it constant and rescale the resistance values for all other categories in each optimized raster (resistance values -- resistance value of focal category). Each panel should then be interpreted by checking what lies above zero (focal category more facilitating) and below zero (focal category more resisting). The red dashed line is zero, the boundary of equal resistance values between the focal category (panel) and other categories. For example, in the \textit{urban} panel (left upper corner), the medians of \textit{beach}, \textit{scrub}, \textit{trees} and \textit{open dune} are situated above the red line, consequently, \textit{urban} is in general more facilitating than those four categories. As most categories, apart from \textit{urban}, prove to be of little importance to discriminate pairwise genetic distances (table 5.2), the uncertainty on the relative resistance values of these categories should be considered high.  \label{fig5.2}}
			\end{footnotesize}
		\end{figure}
	\vspace*{\fill}\clearpage
	
		\begin{figure}[h!]
			\begin{center}
				\includegraphics[width=\textwidth]{figures/Figure5_3.png}
			\end{center}
			\begin{footnotesize}
				%\vspace*{-3mm}
				\caption{Circuitscape current maps for the complete study area based on the three best supported models. Current maps of A) best supported model with \textit{urban} as facilitator B) 2\textsuperscript{nd} best supported model \textit{urban} and \textit{water} combined, both facilitators C) 3\textsuperscript{rd} best supported model isolation-by-distance (resistance values for all cells was set to 1). The map visualizes conductance (high conductance---or low resistance---dark red, average conductance or resistance blue, low conductance---or high resistance---white-transparent) to gene flow based on the summed currents between pairwise sampled individuals (small black dots). Map made with QGIS v3.22 \citep{qgisdevelopmentteam2020}. Background aerial photographs (summer 2018) from Agency for Information Flanders (source: geopunt.be). \label{fig5.3}}
			\end{footnotesize}
		\end{figure}
	\clearpage
	
	\section{Discussion}
	We studied functional connectivity based on landscape genetics methods of the digger wasp \textit{Bembix rostrata} in a human-altered coastal region. Overall, the populations of this dune-specialist insect are genetically well-connected in this region \citep[chapter \ref{chapter4}]{batsleer2022b}. Despite this overall genetic admixture, we could detect a consistent signal of urban features facilitating gene flow, in addition to the prominent isolation-by-distance (IBD). In general, anthropogenic landscape categories (urban, agriculture, water) seem more facilitating than natural habitat types in this human-altered dune landscape context.\\
	
	Isolation-by-distance (IBD) was a prominent pattern in our results: genetic distances increase with geographic distance between individuals or populations \citep[chapter \ref{chapter4}]{batsleer2022b}. An IBD pattern emerges naturally when dispersal is limited at a certain spatial scale, i.e. dispersal is more likely to nearby populations than distant ones, independent of the landscape \citep{kimura1964}. Consequently, IBD is seen as a biological realistic null model (opposed to panmictic gene flow) in landscape genetics when considering the influence of the landscape on gene flow, i.e. isolation-by-resistance (IBR) (Wagner and Fortin 2015). Our results showed that IBD was consistently a competing model and one of the best supported models (table \ref{Tab5.1}, considering \textDelta AICc $<$ 2 with \textit{urban}). Because overall support for models was low, as models with \textDelta AICc ranging between 2--7 have some support and should not be dismissed \citep{burnham2011}, we applied multi-model inference to deduce average weights and relative resistance values of categories \citep{johnson2004}. This confirmed the importance of \textit{urban} for discriminating pairwise genetic distances and deduce that generally anthropogenic landscapes seem more facilitating than natural habitat types. Therefore, considering all possible combinations for landscape resistance optimization can be powerful when landscape resistance signals are weak and hard to detect.\\
	
	The current maps in figure \ref{fig5.3} visualize the potential realized conductance of all sample locations according to three best supported and competing models (\textit{urban}, \textit{urban}-\textit{water}, IBD; table \ref{Tab5.1}). We can see that these models should be considered as complementary rather than opposing. First, waterways often run parallel to roads and waterbodies are embedded along or within urban areas (Fig. \ref{fig5.1}). Such water---which is also a scarce landscape type---adds only a small extra effect on top of the urban areas for the resulting current map (Fig. \ref{fig5.3}B). Consequently, we cannot easily disentangle the effect of water from that of urban within the given landscape \citep{angelone2011, keller2012}. The Akaike weights from multi-model inference (table \ref{Tab5.2}) indicate that the urban category is overall the most important to discriminate pairwise genetic distances. Second, the IBD pattern (Fig. \ref{fig5.3}C) is also present within the dune areas in the current map for the \textit{urban} and \textit{urban}-\textit{water} models (Fig. \ref{fig5.3}A, \ref{fig5.3}B). In the latter models, the natural habitat categories (\textit{open dune}, \textit{scrub}, \textit{trees}, \textit{beach}) all have the same resistant values (relatively higher than \textit{urban} and/or \textit{water}; table \ref{Tab5.1}), resulting in a homogenous resistance landscape within dunes, resulting in an average IBD pattern. The importance of IBD on this scale is also confirmed by the analyses repeated on the dune area clusters within our study area (supplementary material S4). This interpretation is also in agreement with the multi-model inference, as natural habitat categories proved to be of little importance to discriminate pairwise genetic distances (table \ref{Tab5.2}) and the uncertainty on the relative resistance values of these categories was high (Fig. \ref{fig5.2}). Consequently, IBD can be considered as the main background process, wherein gene flow only depends on the pairwise distances. The extra signal from anthropogenic features is then at play between dune areas where especially urban areas give a weak but consistent extra signal of conductance.\\
	
	It is counterintuitive that gene flow of a specialized digger wasp nesting in grey dune habitat can be facilitated by urban features in a dune landscape. Importantly, patterns of gene flow only reflect a certain component of movement in a species and interpretation of results in a larger ecological framework should consider other crucial components of a species survival and resource requirements \citep{spear2010, cushman2010a}. As \textit{B. rostrata} is a dune-specialist insect, urban areas should be considered as inhospitable environments. As the deduced resistance values per category are relative values (Fig. \ref{fig5.2}), the pattern can be explained by both slower, confined movement in dune habitat and/or faster, compensatory movement in inhospitable areas. Indeed, search behaviour in unfamiliar environments can be more directional and faster than in natural, familiar habitats, where movements are more exploratory \citep{vandyck2005, schtickzelle2006, knowlton2010}. Suitable habitat intervening the landscape might hinder gene flow between more distant populations, as individuals are likely to settle at encountered habitat during dispersal \citep{adriaensen2003, mcrae2008, keller2012}. On the other hand, compensatory movement through inhospitable habitats can result from a lack of resources or can be a strategy to reduce mortality risk \citep{schtickzelle2006, peterman2014}. Both can be counterbalanced by moving faster in inhospitable environments, resulting in increased genetic connectivity between fragmented habitats \citep{schtickzelle2006}. As isolation-by-distance is a persistent underlying pattern in our results, the behavioral responses to anthropogenic or natural landscape types are only relevant on top of the fundamental isolation-by-distance process.\\
	
	The deduced resistance values are values per grid cell and relative compared to the other considered landscape types. Consequently, the deduced resistance values of a category will depend on the habitat composition and the spatial configuration of how populations are embedded within the landscape \citep{richardson2016, haran2017}. In our study, urbanisation induces large-scale fragmentation compared to more natural processes like shrub and woodland encroachment. Hence, a low resistance does not conflict with the retrieved low gene flow caused by an increased isolation by distance from this urban infrastructure. Differences in a species' ecology (and evolutionary history) across regions could further complicate and diversify possible responses to a certain landscape type in different landscapes, especially at larger scales and across regions \citep{segelbacher2010, spear2010}. Therefore, results regarding the influence of anthropogenic landscape types on the gene flow of \textit{B. rostrata} cannot easily be extrapolated or generalized. Nevertheless, our results for this species will reasonably remain valid for similar landscapes at comparable scales and with similar compositions, such as several human-altered coastal landscapes of northwest Europe. But extrapolation to inland sandy regions, where the species occurs as well, would be invalid.\\
	
	We find that the connectivity of \textit{B. rostrata} does not seem limited by the matrix composition. Even on the contrary, on top of the general process of IBD, inhospitable urban features seem to facilitate gene flow in this landscape for a given distance between different habitats. This indicates that for \textit{B. rostrata}'s persistence in this study area, landscape-based conservation measures aimed at increasing connectivity between dune areas (e.g. with habitat corridors) would not be an effective management measure unless they are able to give rise to the establishment of new populations. It is probably more critical to increase or maintain large population sizes, by increasing habitat area and habitat quality \citep{richardson2016, watts2016}. The effectiveness of increasing connectivity in this landscape might be worth more consideration for cursorial species experiencing high mortality costs when crossing urban infrastructure \citep[e.g., natterjack toad,][]{cox2017}. However, benefits gained from landscape-scale and site-based conservation should always be balanced to maximize biodiversity benefits with the limited resources at hand \citep{watts2016}. For the conservation of \textit{B. rostrata}, other processes linked with dune habitat fragmentation are anyway more fundamental to consider than landscape connectivity, such as the decrease in habitat area itself and shifts in ecological processes \citep[e.g. edge effects and `ecosystem decay' such as inhibition of sand dynamics in dunes;][]{pfeifer2017, chase2020}. This fragmentation due to urbanization decreases the overall potentially suitable locations, increases isolation and affects regional levels of exchange among populations in a spatially altered population network \citep{cheptou2017}. As urban areas are not forming a barrier to direct gene flow, these more indirect effects of fragmentation should be the focus to secure persistence of \textit{B. rostrata} in the human-altered dune landscape. In the contemporary context of dune nature management, a site-based conservation approach to maintain and improve open dune habitat is recommended, for which grazing with large herbivores (cattle, horses) is a crucial management tool. However, a reconciliation is needed between short-term negative effects of trampling by grazers on nest densities of \textit{B. rostrata} and long-term positive effects of maintaining open dune vegetations within dune areas. Therefore, a local adaptive management approach with sheep grazing is recommended \citep[chapter \ref{chapter3}]{bonte2005, batsleer2022a}. Additionally, improving habitat quality of open dune vegetations through landscape-scale conservation aimed at restoring and increasing sand dynamics---rather than connectivity---would be a larger scale, sustainable measure for the persistence of \textit{B. rostrata} in the coastal dunes in Belgium.\\
	
	
	\clearpage
	\subsection*{Acknowledgements}
	We thank the following persons and instances for permission and access to nature reserves: Johan Lamaire, Guy Vileyn, Koen Maertens, Evy Dewulf and Klaar Meulebrouck from ANB (Agency for Nature and Forests --- Flemish government); Rika Driessens from IWVA/Aquaduin.\\
	F.B. was supported by Research Foundation --- Flanders (FWO).\\
	The computational resources (Stevin Supercomputer Infrastructure) and services used in this work were provided by the VSC (Flemish Supercomputer Center), funded by Ghent University, FWO and the Flemish Government --- department EWI.
	
	\subsection*{Data Accessibility}
	Data and code are made available on github:\\
	\url{https://github.com/FemkeBatsleer/LandGenBembix}
	
	\subsection*{Author contributions}
	All authors contributed to the study conception and design. Data collection and laboratory work were performed by FB. Data analyses were done by FB and FD. The first draft of the manuscript was written by FB and all authors commented on previous versions of the manuscript. All authors read and approved the final manuscript.
	
	\subsection*{Supplementary material}
	Supplementary files can be found in the dedicated github repository: \url{http://github.com/FemkeBatsleer/SuppPhD}.

\cleardoublepage
\thispagestyle{empty}
\hbox{}
\clearpage

\CenterWallPaper{1}{pictures/DISC-1.png}
\newpage{\thispagestyle{empty}\clearpage}
\cleardoublepage

\ClearWallPaper

\CenterWallPaper{1}{pictures/DISC-2.png}
\newpage{\thispagestyle{empty}\clearpage}
\hbox{}
\clearpage
\ClearWallPaper

\CenterWallPaper{1}{pictures/DISC-3.jpg}
\newpage{\thispagestyle{empty}\cleardoublepage}
\ClearWallPaper

\setlength{\thumbwidth}{0.8cm}
\setlength{\thumbheight}{1cm}
\tikzset{
	thumb/.style={
		%draw=black,
		fill=gray,%white
		text=gray,
		minimum height=\thumbheight, %0cm, %\thumbheight,
		text width=\thumbwidth, %0cm,
		outer sep=0pt,%   outer sep=10pt,
		font=\sffamily\Large,
	}
}
	%%%%%%%%%%%%%%%%%%%%%%%%%%%%%%%%%%%%% General discussion %%%%%%%%%%%%%%%%%%%%%%%%%%%%%%%%%%%%%%%%%	
 % empty 
 \csname @openrightfalse\endcsname	
 \clearpage
 \thispagestyle{plain} % empty 
 %\CenterWallPaper{1.1}{CH3.jpg}
 \backmatter
% \newpage{\thispagestyle{empty}\cleardoublepage}
 %\ClearWallPaper

\chapter{General discussion}
\label{discussion}
\chaptermark{General discussion}
\lettergroupID{\thechapter}

\begin{flushright} \color{gray}Femke Batsleer
	\color{black}\end{flushright}

	\vspace*{\fill}
\noindent \color{gray} $\lhd$ A female holding a prey and ready to dig and open her nest entrance.\\
\noindent $\lhd\lhd$ A male \textit{B. rostrata} (left) pursuing and inspecting a digging female (right). Males have antennae with yellow undersides.

\color{black}
\newpage
\clearpage


\setcounter{section}{0}
\setcounter{figure}{0}
\renewcommand*{\thesection}{D.\arabic{section}}
\renewcommand*{\thefigure}{D.\arabic{figure}}

	\noindent In this dissertation I study ecological processes underlying spatial patterns of nests and functional connectivity with the digger wasp \textit{Bembix rostrata} in fragmented dune areas as a study system. Movements---including dispersal---are a fundamental component for various ecological and evolutionary processes giving rise to spatial patterns, such as point patterns within populations, species distributions, and metapopulation dynamics \citep{hanski1998, bonte2004, nathan2008}. Dispersal is an important component of gene flow: the transfer or spread of genes across populations and space.\\
	
	In \textbf{chapter 1}, I assessed if tracking movement directly would be feasible, by attaching tracking devices to individual wasps. \textbf{A lightweight tag quantitatively altered the behaviour} of the wasps, especially in individuals with larger wing loading (ratio of body mass to wing area). I therefore abandoned the method as these tags could bias the generated data considerably and would not provide a representative image of their movement behaviour. Therefore, chapter \ref{chapter1} is largely devoted to a systematic literature overview of studies using tracking devices on terrestrial arthropods and how possible impacts and biases have been measured or were taken into account. I found that such \textbf{side-effects are to a great extent neglected in existing literature}. Mostly, literature is too optimistic about the employment of tags on arthropods, and often guided by rules of thumb with no empirical basis or that were misconstrued. Therefore, I used indirect methods for the other chapters to study ecological patterns at hierarchical spatial scales arising from different types of movement.\\
	
	In \textbf{chapter \ref{chapter2}}, I look at the relative importance of both the environment and behavioural mechanisms on clustering of nests. By following up on individual wasps in a field study throughout the nesting season, I can detect `mini-dispersing' individuals that make consecutive nests in different aggregates in the same patch. The spatial point pattern of nests and the network pattern from the field study are compared with those from simulations of an individual-based model (IBM) of nest site selection. With this inverse approach I found that the \textbf{simultaneous effect of a weak environmental cue and strong behavioural mechanisms}---local site fidelity and conspecific attraction---could best explain the emerging spatial pattern of nesting. This included individual differences in behaviour, where the two behavioural mechanisms are not used simultaneously. In this study, I integrated seemingly separate processes of nest clustering to quantify their relative importance and possible synergism.\\
	
	In chapter \ref{chapter3}, I test the impact of sheep grazing as a biotic filter on nest density with a before--after control--impact (BACI) design in a coastal nature reserve. Compared to grazing with large grazers---which have a large detrimental impact on nest densities \citep{bonte2005}---\textbf{sheep grazing also decreased nest densities, but to a lesser extent}. Nests stayed present in both the extensive and intensive sheep grazed sites, which---combined with the conspecific attraction from chapter \ref{chapter2}---is probably beneficial for the persistence of \textit{B. rostrata} in the sites. As this experiment was performed in three sites and not replicated on the landscape level, these results cannot easily be extrapolated to other regions. Nevertheless, it provides a strong study design that can easily be copied to monitor management effects on \textit{B. rostrata} in a framework of adaptive management.\\
	
	In \textbf{chapter \ref{chapter4}}, I look into genetic connectivity at a larger spatial scale: between and within coastal and inland regions. Although \textit{B. rostrata} is a poor colonizer, \textbf{gene flow between existing populations is not highly restricted}. Nevertheless, the small and isolated inland populations show more pronounced genetic structure than the well-connected coastal region, including a pattern of \textbf{asymmetrical gene flow} from coast to inland. These genetic structures are most likely linked to demographic factors within different landscape contexts: the coastal dunes holds more and larger populations than the isolated inland patches.\\
	
	In \textbf{chapter \ref{chapter5}}, I investigate the influence of the heterogeneous matrix between nesting locations on the functional connectivity in a human-altered coastal dune landscape with a landscape genetics approach. I optimized resistance values of different landscape types to derive if urbanized or natural dune landscape types are facilitating gene flow. Results show that, \textbf{apart from a general strong pattern of isolation-by-distance, urban features are not forming a barrier---rather the opposite}---to the functional connectivity of \textit{B. rostrata} in the focal landscape. Such an effect can be explained by compensatory, i.e. faster and more-directed, movement in an inhospitable landscape type and/or slower, exploratory movement in suitable habitat.\\
	
	Below, I discuss the links between the results from the different research chapters. I link the results from the different research chapters to get insights into the relevant ecological processes and characteristics to better understand the species' behaviour, population exchange and colonization capacity. More specifically, I first discuss conspecific attraction and population exchange and their trade-off. Second, I discuss colonisation probabilities and---related to this---the paradox of site fidelity and dynamic habitat. Third, I discuss considerations in light of the haplodiploid mating system. I further discuss implications for conservation and elaborate on open questions for future research.\\
	\clearpage
	
	\section{Conspecific attraction}\label{conspattr}
	Group living can provide several benefits to individuals, such as protection against predation or parasitism \citep{hamilton1971, mooring1992}, protection against climatic extremes \citep{gilbert2008}, or increase in foraging efficiency \citep{clark1986}. Many direct and indirect benefits and costs are at play, which are mainly considered in research on social and eusocial animals, such as in many mammal and bird species, and colony-forming insect species \citep{ross1995, kingma2018, crisp2021}. The system focal to this dissertation deals with a solitary wasp that individually raises offspring with prolonged brood care and clusters together in nesting aggregations. Such a system would sometimes be labelled as subsocial \citep{field2020}. Thus, \textit{B. rostrata}'s nesting strategy is a sort of intermediate stage between a strictly solitary hymenopteran species with mass provisioning and an eusocial, collective brood-rearing, colony-forming species. Studying the diverse levels of group living and parental care strategies in Hymenoptera is informative to better understand the mechanisms---which still remain elusive and debated---that led to the evolution of eusociality \citep{foster2006, polidori2006, hughes2008, nowak2010, field2020}. Prolonged association between parents and offspring---such as the progressive provisioning in \textit{B. rostrata}---may be a precursor to or share a close common ancestral state with eusociality \citep{field2020}. Group formation---including its persistence and how it arises---is proposed as a major primary condition for the evolution of eusociality \citep{nowak2010}. Understanding how nest aggregates arise in a clustering solitary wasp is consequently non-trivial and relevant within this larger evolutionary framework.\\
	
	Conspecific attraction---the attraction to other individuals of the same species---proved an important and strong mechanism in the formation of clustered nests in chapter \ref{chapter2}. In that chapter, I included a probability for \textbf{nest site selection with positive density-dependence probability} in the model: the more nests are present around a spot, the higher the chance an individual will select that spot for making a nest (left side of Fig. \ref{figD.1}). I pointed to a selfish herd pattern present in the field data as a possible driving mechanism: the probability of brood parasitism by brood parasitic flies (section \ref{broodpar}) is for an individual nest lower when the nest density is higher. This was already previously described and termed a selfish herd pattern for \textit{B. rostrata} and another digger wasp, \textit{Crabro cribrellifer} \citep{wcislo1984, larsson1986}. The selfish herd hypothesis states that individuals within a population attempt to reduce their predation or parasitism rate by putting other conspecifics between themselves and predators or parasites. This can be achieved either through moving to the center of a group---comparable to what happens in a school of fish---or by being closer to other individuals, resulting in increased local densities \citep{hamilton1971, mooring1992}. The latter is the resulting pattern which was found in chapter \ref{chapter2} for \textit{B. rostrata} and in \citet{larsson1986} and \citet{wcislo1984}. As the center of an aggregate always has higher nest densities than the edge, other type of edge effects are hard to disentangle from density-effects. The selfish herd pattern does not preclude other mechanisms that could be responsible for conspecific attraction, such as \textbf{a social cue for habitat suitability}. This inadvertent social cue can reduce the time invested into assessing the quality of the habitat by an individual, compared to a personal cue with, for instance, test digging behaviour \citep{dall2005, buxton2020}. As the selfish herd pattern mentioned in chapter \ref{chapter2} is a purely correlative relation, we cannot know for sure what causes the conspecific attraction. Is it a direct response to the presence of brood parasites? Or is it rather a result of the social cue, a more inherent behaviour with a beneficial side-effect of individual parasite dilution? A \textbf{dilution effect} or `safety in numbers' exists when larger groups decrease (or not increase) the individual detection probability by predators or parasites \citep{mooring1992}. The main difference between selfish herd and dilution effect is the underlying assumption of an active behaviour in a selfish herd. And even though I follow the definition from Larsson and Wcislo and termed the pattern a selfish herd pattern, we cannot know for sure if it is an active behaviour or passive effect. Gregariously nesting is common in digger wasps (Crabronidae and Sphecidae) and is almost omnipresent in sand wasps (subfamily Bembicinae), which points towards a more inherent clustering behaviour \citep{evans1957, evans2007}. In that case, dilution of parasites can be a beneficial side-effect. On the other hand, the presence and direct detection of brood parasites might reinforce an even denser clustering, which is a selfish herd behaviour. This is not unlikely, as many anti-parasite behaviours are described for \textit{Bembix} species \citep{nielsen1945, evans1966, evans2007, polidori2009a}. The precise mechanisms and evolutionary history of this conspecific attraction behaviour could only be enlightened through a combination of behavioural experiments and phylogenetic research.\\
	
	
	\begin{figure}[h!]
		\begin{center}
			\includegraphics[width=\textwidth]{figures/FigureD_1.png}
		\end{center}
		\begin{footnotesize}
			\vspace*{-4mm}
			\caption{conceptual graph for conspecific attraction. Left side of the panel: a clustering species (full brown line) will be attracted to select a site for building a nest with increasing nest density (P(settlement)=probability of settlement). This is positive density-dependent nest site selection. A (theoretical) strict solitary species (dashed light-brown line) will show negative density-dependent nest site selection: it will avoid places which already have high nest densities. Right side of the panel: at a certain nest density, competition will start playing a role and conspecific attraction will turn into repulsion (negative density dependence). \label{figD.1}}
		\end{footnotesize}
	\end{figure}
	
	On the level of the individual, the results from chapter \ref{chapter2} point towards mutually exclusive behaviours: when local site fidelity is used by an individual, conspecific attraction is not used simultaneously for nest site selection, and vice versa. This could be a form of \textbf{individual specialisation}. Such heterogeneity in behaviour has been shown to give rise to larger spatial patterns, such as in caribou, \textit{Rangifer tarandus} \citep{spiegel2017, webber2020, bonar2020}. The model from chapter \ref{chapter2} was not conclusive on whether such individual specialisation was consistent or flexible during an individual's lifetime. The model probably needs more data on consecutive nests of the same individuals (which was in the current dataset 102 individuals with 2 or more nests) to converge for this variable. For the same reason were other \textbf{temporal aspects} not explicitly taken into account. For instance, possible temporal changes in clustering throughout the season were not looked into as the model-data combination did not have enough power to split up the data in separate periods \citep[e.g. as in][]{asis2014}. The model in chapter \ref{chapter2} consequently considers mechanisms with parameters which are fixed throughout the season, fitted to the cumulative spatial and network patterns. It could however be that the mechanisms for nest site selection and their relative strengths change throughout the season (e.g. depending on changing brood parasite densities, competition for nesting habitat, decisions might change depending on previous experiences with failed and succeeded nests...). As I solely looked at the overall emerging spatial and network patterns, these individual behavioural subtleties throughout the season are not detectable. To detect such temporal changes of mechanisms, behavioural experiments or more detailed information on individual traits and reproductive success should be incorporated \citep[e.g. in][]{bonar2020}. Most likely, the relative importance of mechanisms of nest clustering in \textit{B. rostrata} changes between years and is context-dependent. As such, Fig. \ref{figD.1} is a conceptual diagram that can be dynamic, and the shape of the curve and the position of the outset of competition might be context-dependent (due to e.g. climate, weather and resource variability between years, presence of parasites, etc.).\\
	
	Aggregating in groups generates a cost for individuals, which have to be balanced to the benefits of clustering. Intraspecific \textbf{competition for resources}---such as \textbf{suitable nesting area}---will start to play a role at certain nest densities in nesting areas (Fig. \ref{figD.1}) \citep{polidori2006, dann2006}. I did not look into this in any of my research chapters, because other types of field studies or behavioural experiments would be necessary to decently detect intraspecific competition. For instance, manipulating nest densities in the lab and study subsequent behavioural interactions or nest site selection \citep[with e.g. the hypothesis of density-dependent fighting;][]{lin1963}. I did, however, anecdotally observe a few times individual females fighting each other when their nests were in very close proximity (a few centimetres) and their digging behaviour interfered \citep[also reported by][]{evans1957}. Thus reasonably, based on basic ecological principles, at a certain nest density, the positive density-dependence will bend into negative density-dependence. Too high nesting densities become unattractive for settling (right side of Fig. \ref{figD.1}), due to limited space to dig a nest in the sand without interfering with other nests.\\
	
	Responses to intraspecific nest area competition at the patch level can be manifold, such as territorial behaviour or \textbf{nest spreading} or overdispersion of nests. For instance, nests of certain species of stingless bees are uniformly dispersed due to aggressive behaviour \citep{hubbell1977}, ant nests can form an overdispersed pattern \citep{ryti1986}, the congeneric \textit{B. oculata} is known to make either solitary nests or diffuse colonies \citep{asis1992}, and another congeneric\textit{ B. pallidipicta} (=\textit{B. pruinosa}) makes its nests at least 20 cm apart for conspecifics \citep{evans1957, rubink1982}. Such nest dispersion is the default in a strict solitary species, which shows negative density-dependent nest site selection (Fig. \ref{figD.1}). Small scale nest dispersion (at a centimetre-scale) might be present in \textit{B. rostrata}, but was not be detectable with the resolution of the data in chapter \ref{chapter2} (accuracy of 2 cm). Another possible reaction to intraspecific competition due to overcrowding can be \textbf{`within-patch dispersal'} to other aggregates or to a solitary location, i.e. a relocation at the patch-level. This can be seen as a type of small-scale breeding dispersal, which is the movement between two reproduction events \citep{ronce2007}. Next to avoidance of intraspecific competition, this interpatch relocation could also be a form of \textbf{spatial bet hedging} in a variable or unpredictable environment \citep{philippi1989, bowler2005}. Such a strategy distributes offspring over different conditions to increase variance in expected fitness \citep{matthysen2012}. The probability to get parasitized by brood parasites at a solitary nest is higher, but more variable (chapter \ref{chapter2}): in such a case, spatial bet hedging with some risky behaviour might pay off.\\
	
	Another resource relevant to intraspecific competition---in addition to nest site availability---is food availability. This resource is probably not important on the level of the aggregate, as with nest resources in the previous paragraphs. Individual \textit{B. rostrata}'s can likely fly up to 3 to 4 km to forage \citep{nielsen1945} and has homing capabilities of the same order of magnitude \citep{tengo1990, schone1991}. Food depletion is therefore mostly important on the larger spatial scale of the landscape or dune area, relevant for several patches combined or the (meta)population. Other \textit{Bembix} species have been observed stealing prey from conspecifics---i.e. intraspecific kleptoparasitism \citep{peckham1905, evans2007}---which might be a sign of competition for food resources (or high costs of obtaining food oneself). Like with any kind of limited resource, it sets a limit to the number of individuals of a population in a particular area and the population size will fluctuate around this carrying capacity.\clearpage
	
	\section{Population exchange}
	When conspecific attraction is such an important process in this study system, how come we still find quite a \textbf{high level of gene flow}? Indeed, the coastal populations are well connected (chapters \ref{chapter4} and \ref{chapter5}), there is also some limited gene flow within a part of the Campine region, and there is gene flow detected from coast to inland (chapter \ref{chapter4}). A prerequisite for gene flow, is that individuals are induced to disperse (Fig. \ref{figD.2}).\\
	
	In chapters \ref{chapter4} and \ref{chapter5} I measured gene flow---the transfer of genetic information---which incorporates dispersal, but is also influenced by many other ecological traits, such as survival probability, fecundity, habitat selection and local adaptation. Even though I did not directly measure dispersal---which is particularly difficult in the natural settings of the focal study system---this biological process has a fundamental role in the process of gene flow and consequently deserves some discussion here. I build upon scientific literature on dispersal literature to link the different research chapters with each other, to connect the observed conspecific attraction and high levels of gene flow. I discuss the theoretical framework of the benefits and costs of dispersal and mainly elaborate on the first and second phase of dispersal: emigration (departure, or propensity) and transfer phase (cf. dispersal distance and kernel). However, the speculations and hypotheses I give---to explain the gene flow found in chapters \ref{chapter4} and \ref{chapter5}---could be simultaneously happening in the study system (they are mutually non-exclusive) and their relative importance---including if they actually have an influence at all---is unknown. I take into account the biology of \textit{B. rostrata} and the specific landscape context. The populations at the coast are a known historical stronghold of \textit{B. rostrata} distribution in Belgium and surrounding regions \citep[www.observations.be;][]{klein2004}. It is qualitatively know that the coastal region holds more and larger (sub)populations than the inland region. This can also be seen from the sampling scheme in chapter \ref{chapter4} (Fig. \ref{fig4.1}), which covered all reported populations in Flanders at that time: 36 populations are sampled in coastal Flanders, while 13 in inland Belgium. I take this into account when highlighting the processes that are possibly relevant to explain the results from the research chapters.\\
	
	\begin{figure}[ht!]
		\begin{center}
			\includegraphics[width=\textwidth]{figures/FigureD_2.png}
		\end{center}
		\begin{footnotesize}
			\caption{conceptual graph for conspecific attraction (as Fig. D.1) and hypothetical, density-dependent dispersal propensity (departure). The probability of settlement [P(settlement)] of a clustering species shows conspecific attraction (full brown) and the probability of departure [P(departure)] shows the chance of dispersal (full green). Competition or over-crowding (dashed blue line) can be an important cue for emigration. The earlier onset of dispersal---than which is caused by intraspecific competition---can be solely due to the avoidance of kin competition, the competition among relatives (hatched blue area). At the onset of competition, both kin and intraspecific competition can play a role in emigration. The optimal group size (grey) is the intersection of P(settlement) and P(departure), when there are as many individuals attracted to the locality as there are leaving. \label{figD.2}}
		\end{footnotesize}
	\end{figure}
	
	\subsection{Causes of dispersal}\label{causesdispersal}
	The benefits related to the ubiquitous conspecific attraction in \textit{B. rostrata} will sometimes---depending on the local and momentary conditions---not outbalance the costs. This results in other strategies outperforming conspecific attraction, such as \textbf{context-dependent emigration} (Fig. \ref{figD.2}). An individual's departure can be informed or triggered by several proximate causes. Relevant for the focal study system, these can be related to within patch-density (Fig. \ref{figD.2}), carrying capacity (amount of local resources), body condition, phenotype or life-history traits \citep[i.e. dispersal syndromes,][]{stevens2013}, expected local fitness, density of predators or parasites... \citep{duputie2013}. The evolutionary forces shaping dispersal strategies (i.e. ultimate causes), are widely recognized to be related with the benefits and costs of kin competition, inbreeding avoidance and spatio-temporally changing environments \citep{gandon1999, ronce2000, friedenberg2003, poethke2007, bonte2010, bonte2012, matthysen2012, bitume2013, duputie2013}. \textbf{Kin competition}---the competition among relatives---can be alleviated when related individuals disperse, provided that a mechanism for recognition of kin is present and reliable \citep{bowler2005, hovestadt2010}. Even if the dispersing relatives end up in patches which are crowded, they will experience competition with non-relatives and relax competition for their relatives in the natal patch. This ultimately increases their inclusive fitness \citep{hamilton1977}. Kin competition has been found to affect both emigration (i.e. departure or dispersal propensity) and the transfer phase of dispersal \citep{lena1998, legalliard2003, bitume2013}. Kin competition can cause emigration independently from intraspecific competition (dashed area in Fig. \ref{figD.2}), but these two mechanisms can also simultaneously affect emigration \citep{lena1998, bitume2013}. \textbf{Inbreeding avoidance} is sometimes hard to disentangle from kin competition in experiments \citep{bitume2013}. By definition, however, inbreeding avoidance is mainly related with reducing the chance to mate with relatives, which ultimately prevents inbreeding depression \citep{duputie2013}. \textbf{Spatio-temporally changing environments} as a cause for dispersal include both variability and unpredictability of the environment in space and time. Because environmental conditions can differ between habitat patches and can change between years within a patch or across the landscape, essential resources (such as nesting area, nectar and prey availability) will fluctuate in space and time. This variability is not always predictable due to stochastic events (such as trampling of patches or weather conditions throughout the year). This environmental variability---and especially \textbf{environmental stochasticity}---seem important to the emergence of dispersal in theoretical evolutionary models \citep{mcpeek1992, olivieri1995}. Such stochasticity can give rise to an unconditional, \textbf{fixed rate of dispersal} \citep{duputie2013}. Such a dispersal strategy could be present in \textit{B. rostrata}, as it is dependent on successional habitat, which can act as an evolutionary selection pressure on dispersal. Such a fixed, background dispersal probability is not mutually exclusive from a density-dependent dispersal strategy as in figure \ref{figD.2}, and both dispersal strategies could be simultaneously present in \textit{B. rostrata}. This background dispersal probability is not contradictory to the results as it can explain the asymmetrical gene flow: the larger populations at the coast result in a larger total number of dispersers, even when the relative number of dispersers is the same in both large and small populations. Such demographic factors contributing to population exchanges are often neglected in connectivity research and conservation \citep{drake2022}, but crucial to explain for instance invasions \citep{neubert2000}. Environmental stochasticity can also favour a strategy with risk spreading (i.e. \textbf{spatial bet hedging}), where dispersal increases the variance in expected fitness when offspring is distributed across different environments \citep{matthysen2012}. This bet hedging can happen at the scale of the patch (see previous section \ref{conspattr}) or happen between patches at, for instance, the coastal region and be one of the reasons for the high levels of gene flow and subsequent pattern of isolation-by-distance in the coastal region (chapters \ref{chapter4} and \ref{chapter5}). In variable and unpredictable environments, also conditional dispersal---next to a fixed background dispersal rate or spatial bet hedging---can help an organism to escape unfavourable conditions, such as intraspecific competition due to overcrowding (Fig. \ref{figD.2}) or trampling by grazers (chapter \ref{chapter3}). \textbf{Demographic stochasticity} (independent of environmental stochasticity) can result in variable population sizes, also leading to moments of higher intraspecific (and kin) competition, which favours dispersal \citep[Fig. \ref{figD.2};][]{parvinen2003, cadet2003}. As there are more and larger populations at the coast, this stochastic overshooting could happen more at the coast than inland (with more isolated and smaller populations). This could be another reason for the asymmetrical gene flow (chapter \ref{chapter4}).\\
	
	\subsection{Costs of dispersal}
	Even when dispersal is beneficial through many different processes, \textbf{dispersal is}---inherently and related to the landscape---\textbf{costly} \citep{bonte2010, bonte2012}. It takes time and energy, which could otherwise be invested in feeding or reproduction. It is also risky: there might be increased mortality during dispersal (due to e.g. predation or accumulation of damage) and there is a probability that no suitable habitat patch is found. This leads to trade-offs: balancing costs and benefits. This is especially true for species with positive density dependence, as in the focal case: increased density can both induce emigration (Fig. \ref{figD.2}, full green line) or induce philopatry (Fig. \ref{figD.2}, full brown line) \citep{demeester2010, baguette2011, bitume2013}. In a certain aggregate or habitat patch, when there are as many individuals leaving as settling (Fig. \ref{figD.2}, intersection of full brown and full green line), the \textbf{optimal group size} or density is reached \citep{markham2015}. When densities are higher, more individuals are departing than are settling; and when densities are lower, more individuals are attracted than are leaving. This might be the case for the populations in the coastal regions, where an overall high level of gene flow was detected (chapters \ref{chapter4} and \ref{chapter5}) and which send out more dispersers to the inland region (chapter \ref{chapter4}). Of course, this optimal density is not always reached and \textbf{densities might remain low} in certain populations. Philopatry is then the default state (Fig. \ref{figD.2}, left to the onset of dispersal). This is especially relevant for a species that shows high levels of brood care, low fecundity and  that has slowly growing population sizes \citep[K-selected species;][]{larsson1989, evans2007}. Other biological or environmental factors might also cause brood loss, such as parasitism and predation or trampling and pupa-survival during winter \citep{casiraghi2003, bonte2005}, further decreasing population growth rate. In chapter \ref{chapter4}, there are populations with no outgoing genetic connections, mainly small and isolated populations inland (Fig. \ref{fig4.6}). This does not mean there are no dispersers from these populations, the sampling and genetic method just might not have detected them. Nevertheless, it might be the case in those populations for which high densities---where kin and intraspecific competition start to play a role---are almost never reached.\\
	
	\subsection{Spatial scales}
	The trade-offs between the benefits of dispersal (related to kin competition, inbreeding avoidance and spatio-temporal variable environments) and costs (related to time, energy and risk) can play at \textbf{several spatial scales}. At the \textbf{aggregate level}, a form of `mini-dispersal' can happen between aggregates or between nearby patches, which could be a form of spatial bet hedging or spreading risk (see previous paragraph on conspecific attraction). At the \textbf{patch or population level}, both kin competition and intraspecific competition can induce dispersal to other patches and populations (Fig. \ref{figD.2}). Suitable habitat patches in the intervening landscape can reduce dispersal distances, especially in organisms with active dispersal and when conspecifics are already present \citep{adriaensen2003, ims2005, mcrae2008, keller2012, bitume2013}. Related to the isolation-by-distance pattern (chapter \ref{chapter4}), these intervening suitable habitat patches with conspecifics might be important at the coastal region. When individuals disperse in the coastal dune landscape, they have a high chance to find a suitable habitat patch with conspecific present to settle quickly and at a short distance. Isolation-by-distance then arises through a stepping stone landscape: each generation, gene flow happens between nearby patches indirectly linking the more distant patches \citep{kimura1964}. Such short-distance dispersal cannot connect the more isolated and smaller populations inland, where gene flow has to rely on long-distance dispersal. Long-distance dispersal is typically more rare: the number of dispersers decrease with distance. This distribution of dispersal distances---the dispersal kernel---is bell-shaped but often with a fat-tail \citep{hovestadt2001, petrovskii2009, nathan2012, fronhofer2013}. In a more fragmented landscape, the distance between patches is often situated in the tail of the curve, automatically reducing the number of dispersers between patches. There is often a distinction in dispersal behaviour between short- and long-distance dispersal: the first can be related more to routine movements of resource exploitation, while the latter often involves specific, fast and directed movements \citep{vandyck2005}. I think in \textit{B. rostrata} this distinction is a bit harder to make because, although it is a useful framework. \textit{Bembix rostrata} is a central place forager and consequently will not be easily triggered to settle during foraging movements. While butterflies---on which this framework is mainly based---can move through a landscape and both forage and deposit eggs. Nevertheless, long-distance dispersal could be triggered differently compared to short-distance dispersal in \textit{B. rostrata}, e.g. through kin competition \citep{starrfelt2010, bitume2013}. On the \textbf{between-region level}, long-distance dispersal is the main factor important for gene flow. The asymmetrical gene flow from coast to inland and stronger isolation-by-distance inland (chapter \ref{chapter4}) are most likely rooted in the differences in spatial habitat configuration of two main sandy regions in Belgium \citep{vanstrien2015}. These differences in landscape configuration are the basis for possible processes leading differences in dispersal rate discussed in the previous paragraphs: 1) total number of dispersers, even if the relative number of dispersers is constant, and 2) the position on the density-dependent dispersal curve, whether or not stochastic (Fig. \ref{figD.2}), 3) the combination of dispersal kernel shape and between-patch distance distribution. Apart from these ecological processes, the correlative nature of the gene flow patterns in chapters \ref{chapter4} and \ref{chapter5} cannot eliminate the possibility of an evolutionary selection pressure against dispersal in more fragmented landscapes \citep{schtickzelle2006, bonte2010, cheptou2017}. However, I expect such evolutionary processes to be less important than the ecological ones for \textit{B. rostrata} in the focal study area. Such landscape-level selection pressures on dispersal are expected to be strong in passive dispersers (e.g. ballooning spiders, wind-dispersed plant seeds...), and are less significant in active dispersing organisms with certain habitat detection abilities, such as the solid flyer and able navigator \textit{B. rostrata} \citep{tengo1990, schone1991, lima1996, bonte2010}. Only a combination of behavioural experiments or quantifying flight metabolic performance, preferably from more landscapes (i.e. more replicates on the landscape level), can help to elucidate on possible evolutionary processes \citep{hanski2004, schtickzelle2006}.\\
	
	\section{Colonisation and population spread}
	The previous paragraph mainly dealt with the first two phases of dispersal: emigration and the transfer phase. Exchange between populations is less constrained by the settlement phase, as the conspecific attraction in this species reinforces density-dependent establishment in already existing populations (Fig. \ref{figD.1} and \ref{figD.2}). However, for colonisation of vacant habitat and population spread, the consideration of the \textbf{settlement phase} is crucial. It is important to consider conspecific attraction again, and especially what the possibilities are around a density of zero (Fig. \ref{figD.3}).\\
	
	\textit{Bembix rostrata} has been described as being a philopatric species because it shows high site fidelity on several scales: individuals build their nest in close proximity to their previous nests \citep[chapter \ref{chapter2}; ][]{larsson1989}, they do not easily colonize (nearby) vacant habitat \citep{nielsen1945, blosch2000, bogusch2021}, and aggregates are found at the same locations for several consecutive years \citep{nielsen1945}. Consequently, \textbf{colonisation probability is likely very low at density zero} (Fig. \ref{figD.3}). On top of that, due to the absence of several of the benefits of high densities (parasite dilution, availability of mates, social cues...), there might be an \textbf{Allee effect} present at low densities (Fig. \ref{figD.3}). A population with a small size or low density might suffer from a reduced mean individual fitness, leading to a weaker (or even negative) population growth rate \citep{stephens1999a, stephens1999}. Such Allee effects can lead to an increased extinction risk and hence a reduced probability of durable establishment.\\
	
	\begin{figure}[ht!]
		\begin{center}
			\includegraphics[width=\textwidth]{figures/FigureD_3.png}
		\end{center}
		\begin{footnotesize}
			\caption{the same as fig. \ref{figD.2}, but with a focused box on the settlement probability at zero and low densities. Colonisation probability is likely very low at density zero and could be a fixed or conditional rate (double arrow and question mark). An Allee effect could be present (i.e. a low density population can suffer from a reduced mean individual fitness, leading to a weaker (or even negative) population growth rate), increasing the extinction risk of newly colonized areas. \label{figD.3}}
		\end{footnotesize}
	\end{figure}	
	
	 However, there were populations that were \textbf{only recently discovered} after the summer of 2017 and, hence, some establishment probably did happen. Of course, I cannot be completely certain that these populations were not already present, as these areas were not consistently surveyed for \textit{B. rostrata} presences or absences. In general, real absences are hard to detect, especially when data is opportunistically sampled with citizen science data \citep{barbet-massin2012}. This can lead to biased data due to preferential sampling because observers tend to look for a specific species in areas where they expect to find it \citep{pennino2019}. Nevertheless, Flanders is an area which is quite well covered by many active observers on the platform observations.be (observations.be) and \textit{B. rostrata} is quite a conspicuous species and unmistakably distinctive from other (digger) wasps species in this area of its distribution, as it is the only species of the genus \textit{Bembix} (or closely related, morphologically similar genera, such as \textit{Stizus}). The species does seem to have increased in the last few years: numbers of nests are reported to have increased since around 2016 at the coast (personal communication, Dries Bonte) and a recently established heath with open sand in an inland area---formerly a pine plantation---has been colonized by \textit{B. rostrata} (Averbode Bos en Heide; population 44 in chapter \ref{chapter4}). Thus, despite the possible observation bias, there are a number of indications that \textit{B. rostrata} has indeed shown a population spread in Flanders the last couple of years.\\
	
	How could this population boost and spread arise? Some of the last few years were exceptionally warm summers (2017, 2018, 2019, 2022) and exceptionally dry (2018, 2022) or warmer (2020) or dryer (2017, 2019, 2020) than average since 1981 (source: KMI). As the activity of \textit{B. rostrata} is highly dependent on temperature and especially insolation \citep{nielsen1945, schone1992}, a combination of more successful nest completions and higher survival rate of both pupae and adults can have \textbf{boosted population sizes} and densities, which increased dispersal propensity (of which the possible mechanisms are elaborated on in the previous section \ref{causesdispersal}). Colonisation probability could both be---parallel to emigration---a fixed rate or conditional dependent (double arrow with question mark in Fig. \ref{figD.3}). If colonisation probability is a \textbf{fixed rate}, the higher total number of dispersers also increases the total number of colonisations, which is a pure demographic effect \citep{drake2022}. Also, when there is an overall higher number of dispersers, Allee effects can be more easily counteracted with a higher influx of individuals. However, it can also be that long-distance dispersers are better colonizers than philopatric or short-distance dispersers, as dispersers are often no random sample of a population \citep{bonte2009, cote2010, kisdi2012}. Phenotypes of dispersers---which can both be behavioural or morphological---can \textbf{depend on the mechanisms inducing dispersal}. For example, dispersal induced by kin competition can increase colonisation success \citep{cote2007}. Following this reasoning, a sudden boost in population size could increase kin competition, inducing more distant dispersal \citep{bitume2013} with better colonizers \citep{cote2007}.\\
	
	An alternative explanation for the high gene flow and low colonisation probability, is \textbf{sex-biased dispersal}. Males might disperse more often than females in the period prior to female emergence, i.e. natal dispersal (due to kin competition, see further section \ref{haplodiploidy} on haplodiploidy) \citep{ronce2007}. However, solitary dispersing males cannot establish a population or colonize vacant habitat as they cannot reproduce any offspring, for which females are the only possible vector. In the case of male-biased dispersal, the previous discussion still holds, but colonisation capacity might be less restricted in dispersing females, which are themselves just more rare. Male-biased dispersal has been found to be common in other bees and wasps \citep{johnstone2012}. As I only sampled females---and other complications due to the haplodiploid mating system---I cannot quantify if there is a biased dispersal strategy (see further section \ref{haplodiploidy} on haplodiploidy).\\
	
	\section{The paradox of site fidelity}\label{discparadox}
	In the previous sections I have discussed why and how conspecific attraction is a major component of the ecology of \textit{Bembix rostrata} (chapter \ref{chapter2}). I also discussed how---through multiple causes of dispersal---there is still a high level of gene flow possible (chapters \ref{chapter4} and \ref{chapter5}), and how restricted colonisation and population spread can be possible. Here I bring together several aspects from these discussions to elucidate and elaborate on the paradox of site fidelity.\\
	
	In the introduction and previous paragraphs, I mention that there are indications for \textbf{site fidelity on several levels} for \textit{B. rostrata}: female nesting behaviour within a season (chapter \ref{chapter2}), across generations and they poorly colonize new habitat \citep{nielsen1945, blosch2000, bogusch2021}. They are however \textbf{dependent on non-persistent, sparsely vegetated pioneer dune vegetations} (Fig. \ref{figI.6}). It is however expected that successional and ephemeral habitats---due to their spatio-temporal variability---act as a selective advantage on the dispersal capacity of a species dependent on such habitat \citep{mcpeek1992, olivieri1995, bowler2005, matthysen2012} (see previous section \ref{causesdispersal} `causes of dispersal'). For example, the grey-backed mining bee (\textit{Andrena vaga}) exhibits good dispersal and colonisation capacities despite a sedentary life style while nesting in early-successional habitats \citep{exeler2008, cerna2013}. As another example, dispersal capability is both intra- and interspecifically related to ephemeral habitat in planthoppers \citep{denno1991, denno1996}. Consequently, philopatry and poor colonisation capacity would be highly disadvantageous for \textit{B. rostrata}'s persistence in dynamic dune ecosystems. However, exactly those characteristics related to site fidelity are reported in literature \citep{nielsen1945, blosch2000, bogusch2021}.\\
	
	A reduction of \textit{B. rostrata} populations has already been reported in the Netherlands in the 1920's \citep{pinkhof1924, thijsse1924}, resulting in a contraction of its distribution with a reduced stronghold persisting in coastal dunes, making the species a rarity, especially inland \citep{klein2004}. Its distribution has also decreased in the Czech Republic, with reported poor colonisation ability in the remaining small, isolated inland populations \citep{bogusch2021}. However, I did report on some newly established populations the last couple of years in Flanders and linked this to boosted population sizes due to several consecutive warm and dry summers (see previous section). This probably increased \textbf{propagule pressure} of colonizers \citep{lockwood2005, chase2022}. Thus, it seems that under certain conditions, this species can colonize new areas. Apart from demographic factors and the trade-offs between conspecific attraction and dispersal propensity discussed in the previous sections, I here elaborate on two more factors important for explaining the mismatch between site fidelity and the dynamic nature the habitat in \textit{B. rostrata}: the slow turnover rate of the habitat and fragmented landscapes.\\
	
	Temporal habitats are expected to favour dispersal, because a species needs to be able to track a moving resource. The strength of the selective pressure does however depend on the relative ratio of the species' generation time and the \textbf{turnover rate of such temporal habitat} \citep{travis1999}. South-facing calcareous grey dunes can likely sustain themselves for up to 15 or 20 years, after which soil formation accumulates nutrients and shrubs and/or grasses start encroaching the sandy moss dunes \citep{provoost2004, provoost2004a}. The pioneer inland vegetations on acidic sand will encroach much faster, as it is much less buffered from nutrient enrichment than calcareous sand \citep{provoost2004, schneiders2021}. In a modelling study, habitat persistence of between 10 and 20 generations was shown to already cause a lower dispersal rate \citep{travis1999}. Thus, it could be that the natural successional development in dune landscapes is too slow to induce a sufficiently strong selective pressure on \textit{B. rostrata} to cause high dispersal and/or colonisation capacity. Especially when in the case of \textit{B. rostrata}, the strong conspecific attraction---which has more direct beneficial pay-offs in reproductive success---could counter-balance a weak selective pressure which operates at a larger temporal scale.\\
	
	\textbf{Habitat availability and spatial configuration} in the landscape also play an important role for dispersal strategies \citep{travis1999}. The dune landscapes in Belgium have gone through extensive landscapes changes and fragmentation during the past decades and centuries and got more encroached (see introduction, Fig. \ref{figI.1}, \ref{figI.2}). Even when the turnover rate of the dynamic nesting habitat is too low to increase dispersal, when habitat is widely available, a low---fixed or context-dependent---dispersal rate could still maintain (meta)population persistence. Dune ecosystems are naturally large dynamic entities. In such systems, the optimal nesting habitat for \textit{B. rostrata} would be available in different stages of its succession gradient throughout the landscape: from pioneer grey dune with much open sand to more moss-dominated grey dune which is gradually gets more grass- and shrub encroached. When habitat availability is high, a low colonisation rate can be enough to ensure a species' persistence in the landscape, even if habitat persistence only exists on a scale of one or two decades. When the total available nesting area or number of suitable habitat patches decreases, population sizes decrease, which reduces the total number of dispersers and colonizers. A low colonisation rate can further reinforce in a feedback loop a decline in population numbers and sizes. Consequently, fragmentation of such a successional vegetation type puts pressure on a species that is adapted to large areas of dynamic sand dunes. This can possibly push the parameters or boundary conditions of population dynamics towards values at which regional extinction is more probable.\\
	
	Summarized, I think the reported poor colonisation capacity of \textit{B. rostrata} \citep{nielsen1945, blosch2000, bogusch2021} is strongly related with the following interacting factors: 1) the \textbf{low availability of suitable habitat} in fragmented coastal and inland dune landscapes, 2) an \textbf{inherent low rate of colonisation} due to the strong conspecific attraction cue, and 3) \textbf{demography}---populations sizes grow slowly and small populations have less dispersers.\\
	
	One could argue that \textit{B. rostrata}'s natural habitat (successional grey dune vegetations) occurs fragmented or patchily within dune ecosystems. But the response of a species to anthropogenic fragmentation cannot be predicted from natural fragmentation as the drivers are different and diverse in identity, scale and intensity (urbanisation, encroachment, afforestation...) \citep{cheptou2017}. In fragmented landscapes, dispersal can both be selected for or against, as many trade-offs simultaneously affect the process \citep{bonte2012}. As mentioned previously, my study design does not allow to deduce anything regarding possible evolutionary changes in dispersal capacity. However, as I discussed in the previous paragraph, colonisation can---purely due to ecological processes---decrease with habitat loss and fragmentation. There is also a risk that this is reinforced by a selection pressure against dispersal. When costs of dispersal are high---fragments are highly isolated and the chance to reach suitable habitat is low---and the temporal turnover rate of habitat is too slow to provide any (weak) counteracting selection pressure, a reduction in dispersal and/or colonisation capacity could evolve when fragmentation is high. Especially in temporally variable habitat, this can lead to evolutionary suicide, i.e. selection drives a viable population to extinction \citep{gyllenberg2002, parvinen2003, poethke2003}. I think this is unlikely in the focal current landscape because \textit{B. rostrata} is an actively dispersing species \citep{lima1996, bonte2010, cheptou2017}, but might become relevant at fragmentation levels as described in \citet{bogusch2021}.\\
	
	\section{Haplodiploidy}\label{haplodiploidy}
	The haplodiploid sex determination in Hymenoptera causes several unusual genetic properties of relevance to population genetics, their ecology and conservation. However, these properties are often overlook in population genetics models and theory. For this reason, we had to carefully select methods that are mating system independent and do not incorporate a diploid population genetic model (which many standard population genetics methods do). In general, the characteristics of haplodiploid mating systems---which I describe in detail below---should be better integrated in population genetic methods \citep{zayed2004, zayed2009}.\\
	
	In haplodiploid organisms, females are diploid---they have two sets of genes---and males are haploid---they have one set of genes (Fig. \ref{figD.4}). What physiological mechanisms can result in haplodiploid sex determination and how the needed developmental pathways are activated, are still being unravelled \citep{heimpel2008}. There are several possible (complex) modes of sex determination that can result in haplodiploidy, but the one best understood is `arrhenotoky': fertilized eggs develop as females and unfertilized eggs develop as males.\\
	
	\subsection{Asymmetrical relatedness}
	The haplodiploid sex determination generates \textbf{asymmetrical relatedness among siblings and parents-offspring} (Fig. \ref{figD.4}). For instance, fathers share their whole genome (relatedness r=1) with their daughters, while the daughters share half of their genome with their father and mother (r=0.5). Full sisters share 0.75 of their genome with each other, while only 0.5 with their mother or own daughters. This high relatedness of sisters is considered a key determinant to the evolutionary development of altruism and eusociality through kin selection \citep{foster2006}. The important feature to \textbf{kin selection} is inclusive fitness: an actor's action has an effect on the actor's own fitness (or costs, c) and the fitness of the recipient weighted by the relatedness between actor and recipient (R$\cdot$b). Altruism or cooperation is favoured when the total sum of inclusive fitness is larger dan zero (R$\cdot$b -- c $>$ 0), which is equal to the ratio of costs against benefits is smaller than the relatedness (R $>$ c/b). This is generally known as Hamilton's rule \citep{hamilton1964}. If sisters have a higher relatedness among each other than with their own offspring, altruistic individuals could evolve more frequently in haplodiploids than in diploid mating systems and form a basis for the evolution of eusociality in Hymenoptera \citep{wilson1975}. This longstanding standard model for the evolution of eusociality has relatively recently been supplemented by an alternative mechanism where group formation is proposed as a major binding force in eusocial evolution \citep{wilson2005, nowak2010} (see section \ref{conspattr} on conspecific attraction). However, the importance of either kin selection or group formation as driving forces for the evolution of eusociality is debated \citep{wilson2005a, foster2006, nowak2010}.\\
	
	\begin{figure}[ht!]
		\begin{center}
			\includegraphics[width=\textwidth]{figures/FigureD_4.png}
		\end{center}
		\begin{footnotesize}
			\caption{asymmetrical relatedness in haplodiploids. Two loci or genes are depicted, with males having one set of genes and females two. Dashed arrows indicate relatedness values, with the number closest to an individual indicating how much that individual shares with the individual the arrow is pointing to. In haplodiploids, the haploid father contributes his whole genome to its diploid daughters, the diploid mother contributes 50$\%$ of its genome to its diploid daughters and 50$\%$ to its haploid sons. Because males only have one set of genes (n), they share their whole genome with their mother or daughter (relatedness r=1). However, as females have two set of genes (2n), they only share half of their genome with their father or son (r=0.5). As such, relatedness among parents-offspring and siblings is not reciprocal, as it would be in diploid organisms. In diploids, the amount of shared alleles between parent-offspring and siblings is always 50$\%$ (symmetrical r=0.5). As such, females have r=0.75 with their (full) sisters and r=0.25 with their brothers. Males, on the other hand, have r=0.5 with their brothers and sisters. \label{figD.4}}
		\end{footnotesize}
	\end{figure}
	
	The asymmetrical relatedness patterns also have eco-evolutionary consequences related to dispersal. In the previous section, I already mentioned that \textbf{male-biased dispersal} could be responsible for the high gene flow observed in chapters \ref{chapter4} and \ref{chapter5}. The low colonisation capacity is then related to the mainly rare dispersing females. Such male-biased dispersal is common in other bees and wasps and might also be an extra factor related to kin selection important in eusocial evolution \citep{johnstone2012}. Sex-biased dispersal can arise when males and females experience dissimilar selective forces and is often dependent on the mating system \citep{trochet2016}. Several life-history traits and ecological processes could promote male-biased---and to a lesser extent female-biased dispersal---in \textit{B. rostrata}. First, parental care favours philopatry in the caring sex (here female) as it has less time and energy to allocate to dispersal, making dispersal more costly \citep{bonte2012, trochet2016}. Second, there is protandry---males emerge earlier than females---and males possibly experience more competition (cf. `sun dance' in introduction) as they show a form of territorial behaviour \citep{asis2006}. Territoriality by males promotes male-biased dispersal mainly when they are polygamous \citep{trochet2016}. Also related with protandry, is that females are almost immediately mated when they emerge. Consequently, inbreeding avoidance---if relevant---should favour male-biased dispersal for several reasons. Firstly, males can actively avoid mating with their sisters when dispersing. Secondly---due to the asymmetrical relatedness (Fig. \ref{figD.4})---inbreeding avoidance is expected to be higher for brothers towards their sisters (r=0.5) than the other way around (r=0.25). Third, as males show high competition for females, kin competition among males might also be strong, inducing male dispersal. However, as sisters share 0.75 of their genome (Fig. \ref{figD.4}), it would be expected that females experience higher kin competition among each other than brothers (r=0.5), but this is likely largely counteracted by conspecific attraction (chapter \ref{chapter2}). Interactions with local conditions and body size are other factors that should be taken into account in balancing possible mechanisms of life-history traits, inbreeding avoidance and kin competition involved in either male-or female-biased dispersal \citep{moore2006}. Due to the factors regarding life-history traits, inbreeding avoidance and kin competition discussed here, I hypothesize that male-biased dispersal is present in \textit{B. rostrata}.\\
	
	To not further complicate comparisons in the population genetic analyses (see next subsection)---and because adult males are only present for a short period of time---I only sampled diploid females. A method that would in theory still be possible to detect sex-biased dispersal is by comparing the differentiation in nuclear and mitochondrial DNA \citep[e.g. in][]{hardy2008}. As mitochondrial DNA is only inherited through females, male-biased dispersal would be supported by less differentiation in nuclear DNA and female-biased dispersal by less differentiation in mitochondrial DNA. However, such a method is not suitable for haplodiploids as---due to the unique inheritance genetics (see next section \ref{asymminh})---nuclear introgression is reduced relative to mitochondrial introgression \citep{patten2015}. If males would have been sampled, males and females could have been compared with genetic spatial autocorrelation analysis \citep{banks2012}. However, due to the asymmetrical relatedness (Fig. \ref{figD.4}) and asymmetrical inheritance (see next section \ref{asymminh}), interpretation would in my opinion not be straightforward. Molecular data to detect sex-biased dispersal in haplodiploids would probably need to be supplemented with simulations (e.g. individual-biased model) to correctly assign, quantify and interpret the detected differences in dispersal between males and females.\\
	\clearpage
	
	\subsection{Asymmetrical inheritance}\label{asymminh}
	The asymmetrical relatedness also trickles down in consecutive generations---resulting in \textbf{more variable gene inheritance than in diploids}---which has a few implications for population genetics. To illustrate this, I made schematic diagrams of inheritance of migrant alleles in a resident population for diploids (Fig. \ref{figD.5}) and haplodiploids (Fig. \ref{figD.6}), adapted from a study on hybridisation between haplodiploid species \citep{patten2015}. In this figure, migrant alleles are blue, resident alleles are white, and mating always happens with a resident male or female. Because of \textit{B. rostrata}'s biology and life cycle, I expect that dispersing males will mate in a new population, while females are probably already mated. Consequently, the female's sons and daughters will be the `migrants' crossing back with the resident. In diploids (fig. \ref{figD.5}), the number of migrant alleles are diluted each generation with 50$\%$ due to recombination during meiosis and subsequent mating. By the fourth generation (F4), each descendant---male or female---will have on average 6.25$\%$ of migrant alleles in its genome. In haplodiploids however (fig. \ref{figD.6}), males transmit their complete genome to their daughters and females transmit 50$\%$ of their genome to their offspring. Sons consequently inherit only maternally derived alleles. Due to this skewed inheritance, the number of migrant alleles in each descendant genome depends on the history of the maternal and paternal lines (fig. \ref{figD.6}). A descendant of the fourth generation will have between 6.25$\%$ and 25$\%$ of migrant alleles in its genome. Consequently, there are on average \textbf{more migrant genes present after a few generations} in a descendant and there is \textbf{more variation} in the proportion of migrant alleles between same-generation descendants in haplodiploids compared to diploids.\\
	
	This asymmetrical inheritance of migrant alleles has some important consequences for population genetics patterns and methods. First, when there is gene flow between populations of haplodiploids, \textbf{homogenisation of allele frequencies} and introgression of migrant alleles into the population takes much longer than in diploids. As a consequence, population genetic structures most likely remain more pronounced and thus easier to detect. Second, if you sample offspring from a migrant, it will be possible for several consecutive generations to \textbf{link a descendant to the original population}. Third, however, as there is more variation possible on proportion of migrant alleles for descendants of the same generation, the variance and \textbf{noise on allele frequency data} from random samples will be higher. In my own results and data in chapter \ref{chapter4}, I did find pronounced population structure, especially in the inland populations. Nonetheless, the assignment tests in chapter \ref{chapter4} also detected many individuals that were not assigned to the population they were sampled in. Thus, several individuals have a recent immigration ancestry, mainly originating from the coastal populations. I ascribe this combination of finding considerable population genetic structure but also quite a strong signal of immigrants, to this haplodiploid asymmetrical inheritance (Fig. \ref{figD.6}). When I initiated the population genetics research on \textit{B. rostrata}, I knew the haplodiploid mating system would complicate sampling design and analyses. Hence, I pushed through to have many microsatellite markers and many samples, which was probably useful regarding the higher expected variance on allele frequency data in haplodiploids. In retrospect, I was also lucky that both structure and signals of gene flow are particularly well detectable because of the inheritance linked to this mating system.\\
	
	\textbf{Standard population genetics methods were not directly applicable} to my data, especially because many assumptions are violated due to the asymmetrical inheritance. I did test for deviations from Hardy-Weinberg (HW) equilibrium, for linkage disequilibrium (LD) and for the presence of null alleles, mainly to exclude misbehaving microsatellites which had artefacts (chapter \ref{chapter4}). However, as such tests for deviations are a necessary but not sufficient step in establishing Mendelian inheritance, it does not automatically allow to use molecular analyses that have stringent assumptions regarding random mating or are based on HW allele frequencies \citep{waples2015}. With the asymmetrical inheritance for haplodiploids, it is clear that such assumptions are violated. Therefore, I could not use Bayesian clustering algorithms, such as STRUCTURE \citep{pritchard2000} or BAPS \citep{corander2008}, that assume a standard population genetic model (i.e. explicit random mating of diploids modelled). These methods were used in two master thesis topics \citep{gallin2021, delafonteyne2022}, which was very useful as data exploration. Instead, I used methods that compare allelic frequencies without assumptions on inheritance, which are often multivariate types of analyses such as discriminant analysis of principal components \citep[DAPC;][]{jombart2010}. Such population genetic model-free methods are already used for (partially) clonal organisms, such as pathogens \citep{grunwald2011}. For the assignment tests performed in chapter \ref{chapter4}, the current most standard model also uses a population genetic model and explicitly samples gametes from populations \citep{paetkau2004}. I therefore used an older method that is purely based on allele frequencies \citep{rannala1997}, but can be prone to overestimation of gene flow \citep{paetkau2004, piry2004}. Results from this type of assignment tests should not be interpreted as direct first-generation migrants \citep{paetkau2004}, but as the accumulated signals of gene flow from the last few generations. In other words, these results should be interpreted with the most basic definition of assignment test: the most probable origin population based on the individual's allelic profile and populations' allele frequencies. Individual based relatedness measures used in the landscape genetics chapter \ref{chapter5} are based on allelic frequencies as well, and do not have an underlying population genetic model. This discussion illustrates that population genetics methods that are standard practice for diploid organisms should not be applied directly to haplodiploid organisms before scrutinization of the underlying assumptions or population genetic model. One should then also take care not to overinterpret results (such as the assignment tests) and be cautious to interpret the results relative within the studied system and not extrapolate or compare values (e.g. inbreeding coefficients or effective population sizes) to those from other studies.\\
	
	\begin{figure}[ht!]
		\begin{center}
			\includegraphics[width=\textwidth]{figures/FigureD_5.png}
		\end{center}
		\begin{footnotesize}
			\caption{a schematic diagram of inheritance for a diploid migrant. A total of 8 loci are depicted (8 pairs of boxes) with migrant alleles in blue and resident alleles in white. Diploids have two sets of genes (2 rows of 8 boxes). Mating or crossing happens each time with a resident (white) male or female, resulting in hybrid offspring. Gametes formed in both males and females combine, due to recombination, on average 50$\%$ of both sets of genes. This results each generation (F1, F2,...) in a reduction by 50$\%$ of the migrant alleles. By the fourth generation (F4), a descendant from a migrant individual will have around 1/16 (6.25$\%$) of migrant alleles in its genome. Adapted from \citet{patten2015}. \label{figD.5}}
		\end{footnotesize}
	\end{figure}

\clearpage
	\begin{sidewaysfigure}
	%\begin{figure}[ht!]
		\begin{center}
			\includegraphics[width=\textwidth]{figures/FigureD_6.png}
		\end{center}
		\begin{footnotesize}
			\caption{a schematic diagram of inheritance for a migrant with a haplodiploid mating system. Inheritance in haplodiploids deviates from that in diploids: hybrids with 25$\%$ of migrant alleles can still occur in the third (F3) and fourth generation (F4), while this is limited to the second generation (F2) in diploids (Fig. \ref{figD.5}). A total of 8 loci are depicted (8 pairs of boxes) with migrant alleles in blue and resident alleles in white. A female has two sets of genes (2 rows of 8 boxes), males have one set of genes (1 row of 8 boxes). Mating or crossing happens each time with a resident (white) male or female, resulting in hybrid offspring. Males always transmit their complete genome to their daughters. Females transmit 50$\%$ of their genome to their offspring and their sons only inherit maternally derived set of genes, while daughters are the result of mating with a resident male. Due to this skewed inheritance of diploid females and haploid males, by the fourth generation, a descendant from a migrant individual will have between 1/16 (6.25$\%$) and 1/4 (25$\%$) of migrant alleles in its genome. Adapted from \citet{patten2015}. \label{figD.6}}
		\end{footnotesize}
		\end{sidewaysfigure}
	%\end{figure}
	\clearpage

	\subsection{Haplodiploidy and conservation genetics}
	Other haplodiploid-specific processes on top of the asymmetrical inheritance further complicate conservation genetic interpretations. The arrhenotoky---fertilized eggs develop into females and unfertilized eggs into males---mostly involves complementary sex determination (CSD) as a physiological mechanism to induce the correct developmental pathways. In CSD, a single locus (or multiple loci) regulates sex determination and females are heterozygous (2 different alleles) and males hemizygous (1 allele) \citep{beye2003, evans2004, heimpel2008}. Homozygous individuals (2 identical alleles)---which mainly happens with high inbreeding---result in diploid males that can have different levels of sterility or viability \citep{paxton2000, ayabe2004, liebert2005, heimpel2008, zayed2009}. The production of diploid males anyway represent a high genetic cost and can result in a specific kind of inbreeding depression with relevant conservation implications \citep{zayed2009}. On the other hand, recessive deleterious alleles are constantly exposed to selection in haploid males and are more likely to be purged for non-sex-related traits, resulting in a lower genetic load in a population \citep{tien2015}. Consequently, the more classical inbreeding depression through dominance effects (the higher chance of expression of recessive alleles through homozygotes when inbreeding is high) is expected to have less detrimental effects in haplodiploids than in diploid mating systems \citep{liautard2005, holzman2009}. This purging together with the asymmetrical inheritance (previous section) makes standard inbreeding coefficients inherently high and variable and effective population sizes low \citep{zayed2009}. As such, interpretation of these coefficients and related values should be done with care and not over-interpreted. The production of diploid males, purging of deleterious alleles, and the asymmetrical inheritance from the previous section, are examples of processes specific for haplodiploid mating systems that are relevant for population genetics and conservation. Apart from complementary sex determination (CSD), many more possible modes of sex determination exist \citep[e.g. pattern genome elimination (PGE) or thelytoky;][]{heimpel2008}, which further complicate these haplodiploid-specific processes. Therefore, nuances need to be made for inbreeding and other genetic effects in haplodiploids \citep{zayed2004, zayed2009, heimpel2008}. These examples illustrate that the standard ideas and concepts of conservation genetics---which are based mainly on diploids---are not directly applicable to haplodiploids \citep{zayed2009}. Integration of haplodiploids in the conservation genetics framework is lacking, although about 15$\%$ of all animal species are haplodiploid and pollinator conservation---which includes many Hymenoptera---is currently a major topic of interest to policy and science \citep{evans2004, lohse2015, potts2016}.\\
	\newpage
	
	\section{Conservation implications}
	Both the impact of sheep grazing on the number of nests (chapter \ref{chapter3}) as the genetic connectivity results (chapters \ref{chapter4} and \ref{chapter5}) are relevant for conservation of \textit{Bembix rostrata}. The species is an emblematic representative of arthropods in grey dunes and more specifically, of soil-nesting solitary bees and wasps in sandy habitats. Also the previous discussion on conspecific attraction, population exchange and colonisation capacity are important to take into account in discussing conservation and nature management.\\
	
	My research has implications for the debate to prioritize either landscape-scale versus site-based conservation measures. Both the results from landscape genetics study (chapter \ref{chapter5}) as the population genetics study (chapter \ref{chapter4}) indicate that there is \textbf{considerable population exchange} present. Although inland populations are more isolated, there still is genetic influx from the coast. At the coast, anthropogenic landscape types do not impede gene flow between the populations (chapter \ref{chapter5}). Thus, exchange between populations does not seem highly restricted---especially when populations are large---but colonisation appears more limited, as it is related to the settlement phase of dispersal. Consequently, landscape-scale conservation measures aimed at corridors or stepping stones to increase connectivity between existing populations, would have no or limited efficacy for the persistence of \textit{B. rostrata} in a human-altered dune landscape. Such landscape-scale conservation measures aimed at increasing connectivity have been widely embraced by conservation practitioners and policy-makers \citep{crooks2006}. Although, the effectiveness of increasing connectivity has been debated, the general positive perception is probably steered by a focus towards vertebrates, while empirical evidence for others groups such as invertebrates is limited and equivocal \citep{ovaskainen2008, humphrey2015, watts2016}. Probably, on a larger spatial scale, also invertebrate species communities could benefit from an integrated landscape approach aimed at increasing connectivity for climate-driven range expansion, but scientific studies to base recommendations on are still scarce \citep{maes2022, hodgson2022}. However, it is important to consider that there are also species in the focal dune landscapes that could benefit from increased connectivity on the landscape scale. These are mainly non-flying species, such as for instance carabid beetles (Carabidae) or the natterjack toad (\textit{Epidalea calamita}) \citep{cox2017}, or passively dispersing species, such as spiders \citep{bonte2003}. More evidence is needed to maximize biodiversity benefits with the limited resources at hand for landscape-scale or site-based conservation \citep{watts2016}.\\
	
	My results and discussion for \textit{B. rostrata} indicate that for this species, increasing population sizes---by increasing habitat area and habitat quality---is likely more critical than increasing functional connectivity \citep{richardson2016, watts2016}. More habitat with increased quality for \textit{B. rostrata} could also be obtained through both site-based and \textbf{landscape-scale measures aimed at restoring and increasing sand dynamics}, rather than connectivity. There are some projects at the coast that aim to create some more sand dynamics, e.g. Interreg V-project VEDETTE with `Save the Sahara of De Panne' (NL: `Red de Sahara van De Panne'), Life+ Nature project FLANDRE for ecological restoration across the French-Belgian border, or Life DUNIAS aimed at restoration of dunes by removing invasive species. Natural sand dynamics provide early-succession vegetations such as grey dunes with a steady gradient from open sandy patches into more vegetated areas, the ideal nesting habitat for \textit{B. rostrata}. However, such landscape-scale natural wind dynamics remain in general hard to restore in highly fragmented and urbanized dune areas, which have known a general increase in encroachment the last decades \citep{provoost2011, provoost2020}. Large-scale ecological restoration is often not realistic goal in small and fragmented areas in the current landscape context in Belgium. The way forward for nature management to conserve the specific biodiversity in dune ecosystems, which are mainly associated with the open and early-succession vegetation types, is with \textbf{site-based conservation} with management measures aimed at species and habitats. To maintain a dynamic heterogeneity in dunes in space and time, large herbivores have been introduced in many coastal dunes from the 1990's onwards, which have many positive effects on botanical and structural diversity \citep{provoost2004}. However, such grazers have mixed effects on local arthropod species due to intense trampling, and especially \textit{B. rostrata} experiences detrimental effects \citep{bonte2005, maes2006, bonte2008, vanklink2015}. This large detrimental effect is also illustrated in figure \ref{figD.7}, a picture with \textit{B. rostrata} nests in an intact grey dune next to a fence with on the other side a grazed area year-round by large grazers. However, in an fragmented, unconstrained dune landscape lacking natural sand dynamics, such grey dune spots will eventually get encroached by grasses and scrubs. An extra difficulty for \textit{B. rostrata} is that it is a bad colonizer (see section \ref{discparadox} on the paradox of site fidelity), so it does not easily recolonize areas when grazing has ceased. Thus, there are several aspects to take into account when aiming to tailor local nature management to conserve \textit{B. rostrata} populations.\\
	
	\begin{figure}[ht!]
		\begin{center}
			\includegraphics[width=\textwidth]{figures/FigureD_7.JPG}
		\end{center}
		\begin{footnotesize}
			\caption{the impact of large grazers on grey dunes in nature reserve `De Westhoek' at the coast. On the right of the fence, there is year-round (extensive) grazing by Konik horses and Highland cattle. On the left of the fence is an intact grey dune with nests of \textit{B. rostrata} (bare open sand spots). The grazed habitat is unsuitable for nesting habitat, but the suitable habitat on the left would get encroached on the long-term. \label{figD.7}}
		\end{footnotesize}
	\end{figure}

	\enlargethispage{1\baselineskip}
	As mentioned in chapter \ref{chapter3}, sheep grazing could be a tool to reconcile the short-term negative effect of trampling and long-term positive effects of revitalizing dune dynamics by grazers. Especially since I showed in chapter \ref{chapter3}, that in those particular sites, nests did not completely disappear. The strong conspecific attraction (chapter \ref{chapter2}) can then potentially help reinforce the establishment of a larger population when sheep grazing has ceased. Sheep grazing in function of \textit{B. rostrata} conservation should be implemented in a \textbf{dynamic landscape management framework} with sufficiently high levels of variation in grazing intensity and grazer type at a spatial and temporal level in a landscape (Fig. \ref{figD.8}). Landscape-scale heterogeneity in grazing is combined with rotational grazing to maintain needed habitat succession stages and can be supplemented with small-scale mechanical removal of vegetation and the creation of blowouts \citep{vanboxel1997}. Such a framework should integrate an \textbf{adaptive management approach} \citep{sutherland2020}, to steer local and quick changes in local management measures by monitoring nests densities (as for instance with the design from chapter \ref{chapter3}) and surveying the impact of grazing.\\
	
	To give more specific and detailed recommendations to tailor nature management to \textit{B. rostrata}, I made a box with \textbf{a management perspective for \textit{Bembix rostrata} (Box D.1)}, adapted from an article written in Dutch \citep{batsleer2021b}. Of course, nature management is making choices and compromises, and these recommendations would have to be integrated in more general nature management goals for a specific area (if it is desired to take \textit{B. rostrata} into account at all). As explained in the introduction, \textit{B. rostrata} is an emblematic representative for many soil-nesting solitary bees and wasps. Taking this specific species into account---which has quite demanding ecological preferences with many balanced trade-offs---is likely beneficial for many other arthropods associated with intact, non-trampled grey dunes and the structural equivalent vegetations from inland dunes. Especially other threatened ground-nesting bees and wasps that depend on small-scale bare sand patches, embedded within a variety of other resources, could benefit from nature management aimed at \textit{B. rostrata} \citep{heneberg2013}. The habitat types optimal for \textit{B. rostrata} nests encompass three protected habitats from the European Habitats Directive, which is aimed at protecting natural habitats (Annex I) and wild fauna and flora (Annex II, IV, V) in the EU. Unfortunately, insects (and other invertebrates) are barely represented as protected species, even though these are good indicators of high structural heterogeneity in habitats \citep{vanklink2015}. In my opinion, \textit{B. rostrata} could be a suitable indicator species for intact grey dunes and inland pioneer dune vegetations (Natura 2000 habitats 2130, 2310 and 2330).\\
	
	The recommendations from box D.1 should not be seen as something fixed (these should be approached in an adaptive way themselves), but part of our \textbf{ongoing progressive management insights for this species}. They are based on my research, but also largely on my own experience in the field and conversations with managers and conservationists. During my PhD, I tried to talk with many managers that are directly involved in the nature management applications in the field, to learn about feasibility of nature management approaches and how \textit{B. rostrata} could fit in. These conversations together with the insights from my research and own experience, has led to Box D.1 and the framework in this section. However, I think nature management regarding \textit{B. rostrata} (and also in general) should be a constant feedback loop of learning between science and practice. I therefore hope we get more insights in how to take \textit{B. rostrata} into account for nature management in the future.\\
	
	\enlargethispage{1\baselineskip}
	The study of chapter \ref{chapter3} was performed in three sites at the west coast and \textbf{not replicated on the landscape level}. Therefore, these results cannot easily be extrapolated to other regions and we cannot know for sure what the effect of sheep grazing would be in other contexts. The west coast is the region with the most and largest population of \textit{B. rostrata} in Belgium. The results of sheep having limited adverse effect should therefore be interpreted as the `best case scenario'. Sheep grazing could still be more detrimental elsewhere, in other landscape contexts. Nevertheless, chapter \ref{chapter3} provides a strong study design that can easily be applied to monitor management effects on \textit{B. rostrata} in a framework of adaptive management (Box D.1). The possibly more adverse effects of sheep grazing on \textit{B. rostrata} should especially be taken into account in the smaller and \textbf{inland populations}, where the habitat also consists of different dune vegetation types because of the acidic sandy soil (in contrast to calcareous sandy soil at the coast). Local and/or temporal enclosures to safeguard nest aggregates when grazing is desired could help the slowly growing \textit{B. rostrata} populations (see previous discussion on Allee effects). In my opinion, many of the small open sandy patches in the inland region in Flanders where \textit{B. rostrata} is present have persisted due to some form of intermediate disturbance, often by recreationists. Similarly, at the coast, nests could sometimes be found at the edges of paths or trails. I noticed in the largest population in inland Flanders (Geel-Bel), small posts were set up to guide hikers around nest clusters of \textit{B. rostrata} and prevent trampling. At some spots however, this had stimulated encroachment which had also pushed away the nesting aggregate. Consequently, establishing posts or enclosures to guide recreationists should also be done with an adaptive management approach to find the right balance for an optimal level of disturbance.\clearpage
	
	\begin{figure}[h!]
		\begin{center}
			\includegraphics[width=\textwidth]{figures/FigureD_8.png}
		\end{center}
		\begin{footnotesize}
			\caption{conceptual figure depicting the proposed dynamic landscape management aimed at conservation of \textit{B. rostrata}. This figure is closely connected to Box D.1. \textit{Bembix rostrata} needs pioneer vegetation, which will get encroached if not enough sand dynamic is present, but too high disturbance due to trampling by grazers is detrimental. Wind dynamics on large spatial and temporal scale is the optimal aim in which the optimal nesting habitat will have the most chances to develop. If this is not possible, sufficient heterogeneity in a landscape mosaic is needed with several terrain types: source populations (yellow) and buffer zone (dark and light blue). Due to succession, rotation will be needed between the source populations (1) and buffer zone (2+3) on a mid-long-term, where grazing is done with preferably sheep (2) if areas get too encroached. It is important to leave the created open patches unmanaged for long time periods (3). to increase the chance of re-colonisations or reinforcement of existing aggregates from source populations. An adaptive approach is needed to monitor the several terrain types, encroachment levels, grazing effects on vegetation and \textit{B. rostrata} densities. \label{figD.8}}
		\end{footnotesize}
	\end{figure}

\clearpage
	\tcbset{box/.style={colframe=black,sharp corners,enhanced jigsaw, colback=white}} % Common settings
	
	\begin{tcolorbox}[box, breakable, boxrule=1pt,toprule at break=1pt,extras={toprule at break=1pt}]
		\subsection*{\textbf{Box D.1:} management perspective for \textit{Bembix rostrata}}
		\textbf{Think in a dynamic landscape mosaic}: \textit{B. rostrata} needs pioneer vegetation, which will get encroached if not enough sand dynamic is present, but too high disturbance due to trampling by grazers is detrimental. Wind dynamics on large spatial and temporal scale is the optimal aim in which the optimal nesting habitat will have the most chances to develop. If this is not possible, sufficient heterogeneity in a landscape mosaic is needed with several terrain types (Fig. \ref{figD.8}).\\
		\begin{enumerate}
			\item \textbf{Source populations} (1 in Fig. \ref{figD.8}): long-term non-managed locations with strong source populations. Enclosures are often recommended and encroachment can locally be pushed back with mechanical removal. These source populations are critical for the larger metapopulation for inducing dispersal (importance of demography for dispersal).
			
			\item In a \textbf{buffer zone} around these source populations, grazing with preferably sheep is possible if areas got too encroached (2 in Fig. \ref{figD.8}), combined with other management techniques to create open sandy areas. It is important to leave the created open patches unmanaged for long time periods (3 in Fig. \ref{figD.8}) to increase the chance of re-colonisations or reinforcement of existing aggregates from source populations. An adaptive approach is needed to monitor both encroachment and \textit{B. rostrata} densities.
			
			\item In the \textbf{larger landscape context}, in areas designated to recreation or that have large grazers, small patches that have a low level of disturbance could be present with some nests of \textit{B. rostrata}. For instance, when grazing in summer is discontinued.
			
		\end{enumerate}
	
	Due to succession, rotation will be needed between the source populations and buffer zone on a mid-long-term (Fig. \ref{figD.8}). New source populations can be found in the larger landscape context. Open sandy patches should be created pro-actively, as (re)colonisation can probably take several years and source populations have to persist in the landscape to recolonize open patches. An adaptive management approach is needed to follow-up on the source populations and open sandy patches for (re)colonisation or population increase.\\
	\clearpage
	
	Extra tips:
	\begin{enumerate}
		\item \textbf{Avoid grazing during nesting period} (June-August) in good source locations. Large nest clusters could locally and/or temporally safeguarded with enclosures.
		
		\item Recently colonized or \textbf{isolated populations can be extra vulnerable}, for instance, the inland populations. Populations grow slowly because of the low fecundity of \textit{B. rostrata}. Extra safeguarding with enclosures could be helpful.
		\item An \textbf{adaptive grazing management} with monitoring is needed because many context-dependent and temporal variabilities will influence outcomes. Due to local differences in soil and vegetation development, grazing should be flexibly (dis)continued based on monitoring and follow-up.
		
		\item Disturbance by \textbf{hikers and recreationists} seems less detrimental than large grazers, as long as the disturbance is not too high. Here as well, a healthy balance is crucial: hikers can prevent grass encroachment along paths, of which the edges often provide suitable nesting spots for \textit{B. rostrata}. If the pressure by recreationists becomes too high, low stumps or posts (with wire) can guide hikers more. Here as well, adaptive management is needed, as such posts can prevent too much disturbance, causing encroachment.
		
		\item Small-scale \textbf{blow-outs in dense dune grasslands} (a vegetation type which have much biodiversity value on its own) could be (re)activation to increase heterogeneity. They induce local sand dynamics and can be reinstated by directed excavation during nature restoration works. The edges are ideal nesting spots for \textit{B. rostrata}.
		
	\end{enumerate}
	
		
	\end{tcolorbox}
\clearpage
	
	\section{Open questions and future perspectives}
	In this section, I discuss some detailed species-specific and broader ecological research questions and applications, inspired by the research chapters or general discussion.\\
	
	\textbf{Grazing} might not only have an effect on nest densities, but might also change \textit{B. rostrata}'s behaviour. Does grazing cause a higher (breeding) dispersal propensity? How does it affect conspecific attraction, as nests might be less visible in trampled areas? Is their more bet hedging with higher grazing disturbance? Do the individuals that make their nests in grazed area flexibly adjust to trampling by for instance making deeper nests or nest in slightly more vegetated areas?\\
	
	I only took the \textbf{brood parasites \textit{Senotainia albifrons}} indirectly into account in chapter \ref{chapter2} as a possible cause of conspecific attraction. However, the presence of brood parasites might form a direct cue to induce nest clustering. Would a difference in \textit{S. albifrons} density change the behaviour of \textit{Bembix rostrata}? Depending on the stage of the nest, the brood parasites might have a different effect on both the nesting behaviour of the adult female and the survival probability of the larva \citep{evans1957, polidori2009a}. However, what is the exact impact of \textit{S. albifrons} on the \textit{Bembix} larva? How do dispersal and colonisation capacity and dynamics work in \textit{S. albifrons}, compared and in relation to those from the host? How host-specific is \textit{S. albifrons}?\\
	
	Movement---and more specifically dispersal---behaviour remains hard to study in small insect species in natural and field settings. For \textit{B. rostrata} (and many other insect species), \textbf{field behavioural experiments} could enlighten a bit more how they react to the surrounding environment or boundaries between landscape types \citep[i.e. behavioural landscape ecology experiments;][]{knowlton2010}. For example, translocations to study homing behaviour \citep{tengo1990, schone1991} in different types of landscapes. Or manipulation or alteration of landscapes in a large artificial arena. However, long-distance dispersal is often linked to \textbf{special movements} \citep{vandyck2005} and those will remain hard to detect even in large field experimental settings. Many questions remain regarding the limitations to and processes important for long-distance dispersal in insects \citep{leitch2021}. How important are special movements in \textit{B. rostrata} (and more general, in other flying insect species) for dispersal and large-scale connectivity patterns? And how different are these movements (triggers, mode of movement) and can we relate this to specific phenotypes \citep[i.e. dispersal syndromes; larger, better flyers, metabolic differences;][]{stevens2013}? How do landscape types differently affect such special movements, if at all?\\
	
	An important research question to shed light on the mechanisms involved in population exchange and colonization, is to what extent\textbf{ dispersal is sex-biased}? As explained in section \ref{haplodiploidy} on haplodiploidy this is most likely male-biased in \textit{B. rostrata}. Direct measurement, through capture-mark-recapture of males and females, often underestimates dispersal. Especially when dispersal would happen right after emergence in either of the sexes, sex-biased dispersal would be underestimated. There are two main molecular methods used for sex-biased dispersal: analysing the difference between nuclear and mitochondrial DNA \citep[with the latter only maternally inherited;][]{hardy2008} and comparing genetic spatial autocorrelation curves between sampled males and females \citep{banks2012}. However, in haplodiploids, the biased introgression of mitochondrial genes \citep{patten2015} and asymmetrical inheritance obscure and complicate molecular detection of sex-biased dispersal. Simulation studies taking haplodiploid inheritance into account should elucidate if and at which differences in dispersal between males and females detection of sex-biased dispersal is possible. Probably, some correction factors should be taken into account when quantifying sex-biased dispersal with such molecular methods. Such simulations studies could also help enlighten how much standard population genetics methods are biased when using them on haplodiploids. As explained in the haplodiploid section, the \textbf{assumptions of several classic populations genetics methods} are violated due to the skewed inheritance. Especially when such methods are based on standard population genetic models with explicit random mating of diploids integrated. Simulations could help test if and under which limitations such classical methods can be used for haplodiploids. It could be that methods should be adjusted to be valid for haplodiploid inheritance.\\
	
	I point to \textbf{demography} as a crucial component for dispersal and colonization in the general discussion. Looking more explicitly and detailed into population dynamics of \textit{B. rostrata} could help enlighten the precise demographic limitations to colonization (dispersal propensity, overshooting, Allee effects) of this K-selected species with brood care. Demography is known to be a crucial component for invasions of insects \citep{lockwood2005, chase2022}. How essential is demography in general for insects their distribution patterns, (re)colonisations and population connectivity? Is overshooting rare and how is this related to climate change, climate extremes and environmental stochasticity?\\
	
	\enlargethispage{1\baselineskip}
	I used an \textbf{inverse modelling approach} in chapter \ref{chapter2} to look at the mechanisms behind spatial pattern formation. A major question remaining is how variable the relative importance of these mechanisms is. Do they change across years and are they context-dependent, e.g. depending on local population sizes, environmental conditions, or resource availabilities? How much intraspecific variation in behaviour is really present and how variable is this? Inverse modelling can in general be useful in ecology on several spatial and temporal scales to mechanistically link patterns with possible processes: e.g. population exchange and connectivity, food-web ecology, evolution of dispersal in spatially structured population, speciation on a macro-ecological scale, etc. Although pattern-oriented modelling can help to build robust individual-based models \citep{grimm2005}, a persisting challenge in inverse modelling is to formulate appropriate alternative models and not overfit the data to an (unconsciously) predisposed hypothesis. It is a fundamental problem to appropriately incorporate all relevant processes in models representing the inherently complex systems studied in ecology.\clearpage

 	\section{General conclusion}
 	In this dissertation, I study ecological processes involved in nest spatial patterns and functional connectivity in \textit{Bembix rostrata}. At the scale of a grey dune patch, conspecific attraction and local site fidelity are important mechanisms to explain the nest spatial patterns in \textit{B. rostrata}. At a slightly larger scale, sheep grazing did not have high adverse effects on nest densities and nests remained present at two levels of sheep grazing. On a larger landscape and regional scale, functional connectivity through gene flow is not highly restricted for this species, even though it is considered philopatric. Even urban areas did not impede gene flow between nest locations at a coastal fragmented dune landscape. Smaller and more isolated populations inland are genetically less connected by gene flow and there is asymmetrical gene flow from coast to inland. I discuss the mechanisms of conspecific attraction and causes of dispersal in a broad ecological framework, and link this to the poor colonisation capacity of \textit{B. rostrata}. The low colonisation rate is likely reinforced by a slow population growth rate (demographic factors) and a low availability of suitable habitat in the currently fragmented dune landscapes. Sheep grazing can likely be used as a tool in an adaptive management approach to reconcile the need for local site-based management to prevent encroaching of open dune vegetations and the protection of \textit{B. rostrata} which is sensitive to trampling. I put the results and discussion in a conservation framework and give detailed recommendations for an adaptive management approach in a dynamic landscape framework.\\
 	
 	\vspace*{\fill}
 	\noindent My research on \textit{Bembix rostrata} has made me realize how little we know about this historically relative well-documented species. And there are still so many insect species of which we known much less or have barely any ecological information on. And each of them is worth our unbridled fascination, on the same level as I got captivated (even obsessed) by \textit{Bembix rostrata}'s rich behaviour and all its intriguing characteristics.\\
 	
 	\noindent ``The earth is a planet so richly endowed with life that we shall never run out of problems worthy of study.''\\
 	\hspace*{\fill} H.E. Evans in `Life on a little-known planet'
 	
 	\clearpage
%\cleardoublepage
%\thispagestyle{empty}
%\hbox{}
%%%%%%%%%%%%%%%%  SUMMARY   %%%%%%%%%%%%%%%%%%%%%%%
\setlength{\thumbwidth}{0cm}
\setlength{\thumbheight}{0cm}
\tikzset{
		thumb/.style={
				%draw=black,
				fill=white,
				text=white,
				minimum height=0cm, %\thumbheight,
				text width=0cm,
				outer sep=0pt,%   outer sep=10pt,
				font=\sffamily\Large,
			}
	}



\thispagestyle{empty}		
\vspace*{8.00cm}
\begin{small}
	\noindent \begin{center}``Field studies of \textit{Bembix} are often arduous and frustrating, but never dull.''\\
		\end{center}
	\vspace*{0.2cm}
	\hspace*{\fill}\citet{evans1957}
\end{small}
\newpage



\thispagestyle{mainmatter} % empty
\chapter*{Summary}
\chaptermark{Summary}
\addcontentsline{toc}{chapter}{Summary}
%\thispagestyle{empty}
Movement is a fundamental component of various ecological and evolutionary processes across spatial and temporal scales. Many types of movement exist, such as foraging, searching for a nest location or dispersing. Dispersal is a specific type of movement with potential consequences for gene flow, i.e. the transfer of genetic information across space. It is a central life-history trait with extensive eco-evolutionary consequences. Ultimately, these different types of movement shape the distribution of genes, individuals and species across space and time.\\

Individual movements and interactions underly the emergence of many spatial patterns in ecosystems, such as clustering of nests. A species' movement behaviour is influenced by how suitable habitat is embedded in landscapes, which have been intensely altered by humans across the world. Both coastal and inland dune ecosystems have been altered profoundly---especially in Belgium---by urbanization, fragmentation, and historical afforestation, resulting in the loss of habitat and of their typical dynamic character naturally steered by wind. Dunes got more encroached by shrubs at the expense of the open and early-succession vegetations, while these vegetations harbour the most specific species and valuable habitats. \textit{Bembix rostrata}, the focal species of this dissertation, nests in such early-succession vegetations in dunes (grey dunes, inland open dunes). It is a solitary digger wasp which makes its nests in clusters, showing brood care for one nest at a time, which contains one larva. The species' is known for its high site fidelity: nests are found in the same area for many consecutive years, and they do not easily (re)colonize new suitable habitat. This is rather surprising, as high dispersal and colonization capacities would be advantageous when relying on dynamic pioneer dune vegetation as typical habitat.\\

The overarching aim of this dissertation is to better understand the ecological processes involved in nest spatial patterns and functional connectivity in \textit{B. rostrata} in fragmented dune areas in Belgium. I study ecological patterns in \textit{B. rostrata} at several hierarchical spatial scales related to nest site selection, nest densities, landscape connectivity and population genetic structure. By studying these ecological patterns and processes, I want to gain insight into the ecological characteristics relevant for the species' distribution and population dynamics and exchange. This to eventually understand the ecological conditions relevant to---and formulate recommendations for---their conservation.\\

In chapter \ref{chapter1}, I wanted to assess if tracking movement directly would be feasible. First, I quantified the impact of tracking devices on individual wasps. Second, I performed a systematic literature overview of studies using tracking devices on terrestrial arthropods and checked how possible impacts and biases have been measured or were considered. I found that a lightweight tag quantitatively alters the behaviour of the wasps, especially in individuals with larger wing loading (ratio of body mass to wing area). I therefore abandoned the method as these tags could bias the generated data considerably and would not provide a representative image of their movement behaviour. In the systematic review I found that such side-effects are apparently to a great extent neglected in existing literature. The majority of studies is too optimistic about the employment of tags on arthropods, often guided by rules of thumb with no empirical basis or that were misconstrued. Therefore, I use indirect methods for the other chapters to study ecological patterns at hierarchical spatial scales arising from different types of movement.\\

In chapter \ref{chapter2}, I investigate the relative importance and possible synergism of both the environment and behavioural mechanisms on clustering of nests. By following up on individual wasps in a field study throughout the entire nesting season, I can detect `within-patch dispersing' individuals that make consecutive nests in different aggregates in the same patch. The emergent spatial point pattern of nests from the field study are compared with those from simulations of nest site selection. I found that the simultaneous effect of a weak environmental cue and strong behavioural mechanisms---local site fidelity and conspecific attraction---can best explain the emerging spatial pattern of nesting. This includes individual differences in behaviour, where the two behavioural mechanisms are not used simultaneously by an individual. The strong conspecific attraction cue from this chapter is a crucial process for the larger framework relevant for the ecology and conservation of \textit{B. rostrata}.\\

In chapter \ref{chapter3}, I test the impact of sheep grazing as a biotic filter on nest density in a coastal nature reserve. Compared to grazing with large grazers---which have a large detrimental impact on nest densities---sheep grazing also decreases nest densities, but to a lesser extent. Nests stayed present in both the extensive and intensive sheep grazed sites, which---combined with the conspecific attraction from chapter \ref{chapter2}---is probably beneficial for the persistence of \textit{B. rostrata} in these sites. As this experiment was performed in three sites and not replicated on the landscape level, these results cannot easily be extrapolated to other regions. Nevertheless, it provides a strong study design that can easily be applied to monitor management effects on \textit{B. rostrata} in a framework of adaptive management.\\

In chapter \ref{chapter4}, I explore genetic connectivity at a larger spatial scale: between and within coastal and inland regions. Despite begin a poor colonizer, gene flow between existing populations is substantial. Nevertheless, the small and isolated inland populations show more pronounced genetic structure than the well-connected coastal region, including a pattern of asymmetrical gene flow from coast to inland. These genetic structures are most likely caused by demographic factors within different landscape contexts: the coastal dunes holds more and larger populations than the isolated inland patches.\\

In chapter \ref{chapter5}, I use a landscape genetics approach to investigate the influence of the heterogeneous matrix between nesting locations on the functional connectivity in a human-altered coastal dune landscape. I analyse the resistance to gene flow of different landscape types, to derive if urbanized or natural dune landscape types are facilitating or forming a barrier to gene flow. Results show that---apart from a general strong pattern of isolation-by-distance---urbanized landscapes are not forming a barrier---rather the opposite---to the genetic connectivity of \textit{B. rostrata}. Such an effect can be explained by faster and more directed movements in an inhospitable landscape type compared to slower and exploratory movements in suitable habitat.\\

I discuss the connections between the different research chapters in a broad ecological framework and elucidate what these insights can teach us regarding the focal species' ecology, behaviour and conservation. Although \textit{B. rostrata} is known as a poor colonizer, it shows quite high levels of gene flow. Due to the strong conspecific attraction cue, population exchange is not limited but colonization of empty patches likely is. The poor colonization capacity is likely reinforced by a slow population growth rate (demographic factors) and a low availability of suitable habitat in the currently fragmented dune landscapes. I put the results and discussion in a conservation framework and give detailed recommendations for an adaptive management approach in a dynamic landscape framework. Movement of \textit{B. rostrata} at several spatial scales has a fundamental influence on spatial patterns within populations and how population are connected within a landscape and between regions.\\

\clearpage
\thispagestyle{empty}		
\vspace*{8.00cm}
\begin{small}
	\noindent ``Nu zal ik niet beweren, dat mijn woord Harkwesp zoo schilderachtig of welluidend is, integendeel; maar ik had toch geen lust, om \textit{Bembex rostrata} in te leiden onder den naam van Snavelwesp,... Want die snavel, hoe merkwaardig ook, haalt nog lang niet bij de harken, waarmee zijn voorpoten gewapend zijn.''\\
	\hspace*{\fill} J.P. \citet{thijsse1901} \textit{De Levende Natuur}
\end{small}
\newpage

\chapter*{Nederlandse samenvatting}
\chaptermark{Nederlandse samenvatting}
\addcontentsline{toc}{chapter}{Nederlandse samenvatting}
%\thispagestyle{empty}
  Beweging is een fundamenteel onderdeel van diverse ecologische en evolutionaire processen op verschillende ruimtelijke- en tijdsschalen. Dieren bewegen om verschillende redenen zoals om te foerageren, een nestplaats te zoeken of om te disperseren. Individuele bewegingen en interacties bepalen het ontstaan van verschillende ruimtelijke patronen in ecosystemen, zoals het samen clusteren van nesten Dispersie is een specifiek type beweging met gevolgen voor genetische uitwisseling in de ruimte en is een centraal levenskenmerk met mogelijks eco-evolutionaire gevolgen. De verschillende types beweging bepalen uiteindelijk de verspreiding van soorten, individuen en genen in ruimte en tijd. Het bewegingsgedrag van een soort wordt beïnvloed door de manier waarop geschikt habitat in het landschap is ingebed. e Landschappen over de hele wereld zijn sterk veranderd door de mens. Zowel aan de kust als in het binnenland zijn duinecosystemen ingrijpend veranderd door verstedelijking, fragmentatie en historische bebossing, zeker in Belgi\"{e}. Hierdoor verdween niet alleen veel duinhabitat, maar ook de winddynamiek ging verloren waardoor verstruweling toenam, ten koste van de meer open pioniersvegetaties. Deze pioniersvegetaties herbergen net de meest bijzondere, duinspecifieke soorten. De harkwesp (\textit{Bembix rostrata}), de centrale soort van deze thesis, maakt nesten in zulke pioniersvegetaties in de duinen (grijze duinen, open duinen in het binnenland). Het is een solitaire graafwesp die haar nesten in clusters of aggregaten maakt. De soort toont broedzorg waarbij ze één nest, waarin één larve zit, bevoorraadt met prooien (vooral vliegen). De harkwesp staat erom bekend plaatstrouw te zijn: nesten worden in hetzelfde gebied aangetroffen voor meerdere opeenvolgende jaren en ze koloniseren niet gemakkelijk nieuwe geschikte nestplaatsen of (opnieuw) beschikbaar leefgebied. Desondanks zou een hoge dispersie- en kolonisatiecapaciteit voor een soort voordelig zijn als die afhankelijk is van dynamisch duinpioniersvegetaties.\\
  
  De algemene doelstelling van deze thesis is om de ecologische processen, die aan de basis liggen van ruimtelijke patronen en de functionele connectiviteit, bij de harkwesp beter te begrijpen. Ik bestudeer ecologische patronen van de harkwesp in gefragmenteerde duingebieden in België op verschillende, opeenvolgende ruimtelijke schalen. Ik kijk naar nestplaatskeuze gerelateerd aan nestclustering, nestdensiteit in functie van schapenbegrazing, landschapsconnectiviteit en populatie-genetische structuren en uitwisseling. Door deze ecologische patronen en processen te bestuderen, wil ik inzicht krijgen in de ecologische kenmerken van de harkwesp die relevant zijn voor de verspreiding van en uitwisseling tussen populaties. Dit om uiteindelijk inzicht te krijgen in de beperkingen van---en aanbevelingen te formuleren voor---het behoud van de soort.\\
  
  In hoofdstuk 1 bekijk ik of ik rechtstreeks bewegingen kan opvolgen gebruik makende van automatische tags. Eerst kwantificeerde ik de impact van tags op individuele wespen. Daarna voerde ik een systematisch literatuurzicht uit over studies die gebruik maken van tags en trackingapparatuur bij terrestrische geleedpotigen en onderzocht ik hoe mogelijke negatieve invloeden werden onderzocht en systematische afwijkingen in rekening werden gebracht. Ik vond dat een lichtgewicht tag het gedrag van de harkwespen te hard verandert, vooral bij individuen met een grotere verhouding totale gewicht op vleugeloppervlakte. Ik ben afgestapt van deze methode omdat die geen representatief beeld geeft van hun bewegingsgedrag. Ik vond in de systematische review dat dergelijke neveneffecten grotendeels verwaarloosd worden in de bestaande literatuur, die te optimistisch omgaat met het gebruik van tags op geleedpotigen. Er worden vaak vuistregels gebruikt die geen empirische basis hebben of verkeerd zijn opgevat. Bijgevolg gebruik ik voor de andere hoofdstukken indirecte methoden om ecologische patronen, die het gevolg zijn van beweging, te bestuderen op verschillende ruimtelijke schalen.\\
  
  In hoofdstuk 2 onderzoek ik waarom Harkwespen in clusters nesten. Ik bekijk het relatieve belang van, en het mogelijke synergisme tussen, de omgeving en gedrag. Door individuele harkwespen gedurende het hele broedseizoen op te volgen in een veldstudie, kan ik   `mini-verspreidende' individuen detecteren die een `mini dispersie-beweging' maken om opeenvolgende nesten te maken in verschillende aggregaten in hetzelfde gebied. Het ruimtelijke puntpatroon van nesten dat in de veldstudie gevonden werd vergeleek ik vervolgens met de puntpatronen van simulaties waarin verschillende types gedrag verwerkt zitten. Het gelijktijdig effect van de omgeving met een zwakke invloed en sterke gedragsmechanismen (plaatgetrouwheid en aantrekkingskracht door soortgenoten) gaf de beste verklaring voor het ruimtelijk patroon van clusterende nesten. De resultaten tonen ook aan dat individuen verschillen in gedrag, waarbij de twee gedragsmechanismen niet gelijktijdig door een individu worden gebruikt. De aantrekking tot soortgenoten is een cruciaal proces voor de grotere omkadering omtrent ecologie en behoud van de harkwesp.\\
  
  In hoofdstuk 3 test ik het effect van begrazing door schapen op de nestdensiteit in een natuurgebied aan de kust. Eerdere studies toonden aan dat grote grazers zoals runderen en paarden zeer nadelig zijn voor de harkwesp. Ik toon aan dat schapen een veel minder groot effect op de nestdensiteit van de harkwesp hebben. In zowel de extensief als intensief begraasde stukken bleven nesten aanwezig, wat in combinatie met de aantrekking tot soortgenoten uit hoofdstuk 2, waarschijnlijk gunstig is voor de lokale instandhouding van harkwespaggregaten en continuïteit van nestplaatsen. Aangezien dit experiment werd uitgevoerd op drie locaties en niet werd gerepliceerd op landschapsniveau, kunnen deze resultaten niet gemakkelijk worden geëxtrapoleerd naar andere gebieden. Niettemin biedt het een sterke studieopzet die eenvoudig kan worden gekopieerd om het effect van lokaal beheer op de harkwesp---of andere soorten---te monitoren in het kader van adaptief beheer.\\
  
  In hoofdstuk 4 bekijk ik de genetische connectiviteit op een grotere ruimtelijke schaal: binnen en tussen duingebieden aan de kust en in het binnenland. Hoewel de harkwesp bekend staat als een slechte kolonisator, is er vrij veel genetische uitwisseling tussen bestaande populaties. De kleinere en meer geïsoleerde binnenlandse populaties vertonen meer uitgesproken genetische differentiatie dan de goed geconnecteerde kustpopulaties. Bovendien is er asymmetrische genetische uitwisseling, waarbij er meer genetische uitwisseling van de kust naar het binnenland verloopt dan omgekeerd. Deze genetische patronen worden waarschijnlijk het verklaart door demografische factoren in beide landschapscontexten: de duinen aan de kust herbergen meer en grotere populaties dan de geïsoleerde gebiedjes in het binnenland.\\
  
  In hoofdstuk 5 bestudeer ik, aan de hand van een landschapsgenetisch onderzoek, de invloed van een heterogeen, sterk gewijzigd kustlandschap op de functionele connectiviteit tussen nestlocaties. Ik onderzoek of verstedelijkte of natuurlijke duinlandschapstypes de genetische uitwisseling vergemakkelijken of juist belemmeren. De resultaten tonen aan dat---naast een algemeen sterk patroon van isolatie door afstand---verstedelijkte landschappen geen barrière vormen voor de genetische uitwisseling van de harkwesp. Een dergelijk effect kan waarschijnlijk verklaard worden doordat individuen in een onherbergzaam landschap vermoedelijk sneller en rechtlijniger vliegen dan in een vertrouwd landschap, waar beweging trager en meer verkenning is.\\
  
  In de algemene discussie bespreek ik de verbanden tussen de resultaten van de verschillende onderzoekshoofdstukken in een breed ecologisch kader en licht toe wat deze inzichten ons kunnen leren over de ecologie, het gedrag en het behoud van de harkwesp. Hoewel de soort erom bekend staat een slechte kolonisator te zijn, vond ik vrij veel genetische uitwisseling. Door de sterke aantrekking tot soortgenoten is de uitwisseling tussen populaties niet beperkt, maar de kolonisatie van leeg leefgebied waarschijnlijk wel. De geringe kolonisatiecapaciteit wordt wellicht versterkt door een trage populatiegroei (demografische factoren) en een kleine beschikbaarheid van habitat in de huidige, gefragmenteerde landschapscontext. Ik plaats de resultaten en discussie in een groter conservatiekader en geef gedetailleerde aanbevelingen voor adaptief beheer in een dynamisch landschapsmozaïek. Beweging van de harkwesp op verschillende ruimtelijke schalen heeft een fundamentele invloed op ruimtelijke patronen binnen populaties en hoe populaties in verbinding staan in een landschap en tussen regio's.
  
                       
\clearpage
%\thispagestyle{plain}
%\hbox{}
%\clearpage

%\includepdf{pictures/PolCosmo_Charlotte.pdf}
\CenterWallPaper{1}{pictures/PolCosmo_Charlotte.pdf}
\newpage{\thispagestyle{empty}\cleardoublepage}
\ClearWallPaper
%%%%% DANKWOORD %%%%
	
\chapter*{Acknowledgments --- Dankwoord}
\pagestyle{mainmatter}
\chaptermark{Acknowledgments --- Dankwoord}
\addcontentsline{toc}{chapter}{Acknowledgments --- Dankwoord}
\label{Acknowledgments}
	

\begin{flushright}
Femke Batsleer\\
April 2023
\end{flushright}

\vspace*{\fill}
\noindent \color{gray} $\lhd$ Artist impressions of \textit{Bembix rostrata} by Pol Cosmo (bottom) and Charlotte Taelman (top).\\
\color{black}
\clearpage
\begin{small}
\noindent Of course, I would not have made it through this PhD without a network of colleagues, friends en family that supported me and took care of me. Their presence, friendship and several layers of essential care, make me a contented and balanced human being. Only under these optimal conditions, and with a good portion of luck, was I able to tackle this enormous amount of work. Through these acknowledgments, I want to let the people around me know how much I appreciate them and their support for me. Disclaimer: `I would have written a shorter letter, but I did not have the time'.\\

First and foremost, \textbf{Dries}. I don't exaggerate when stating this PhD is here because of you. From the beginning you gave me every possible opportunity, along with the freedom to develop my own skills, pursue pop-up ideas and trust that I would and will do a great job. Your support and enthusiasm were crucial for every step that lead to this work and I don't think I would be as proud of what I accomplished if it wasn't accomplished together with you. You are a very involved supervisor and generous with your time: you always make sure we can quickly have a meeting when I ask, because I got stuck again or entangled in my own thinking. I am not sure if you are aware of the big influence our meetings can have on my self-esteem and mental health: when I'm struggling with my research and my imposter syndrome starts haunting me again, I know I need a meeting with you to get confident again and back on track. Those meetings are always filled with enthusiasm and energy, quickly you start to draw and think aloud. With an enlightened mind, confidence and a clear vision of how to proceed I turn back to work. Although sometimes I get quickly entangled again when going through my notes and your sketches (`what?!... how was this crystal clear to me just seconds ago when Dries explained it?'), but the motivation and confidence always linger. You push and pull me to a high standard for my scientific output, supporting me to reach my full potential and more. Often, I don't like too much external pressure or striving to meet people's expectations (I already have my own standards set), but I always feel motivated by your challenges, which you combine with a crucial portion of freedom and trust. It makes me confident and brings out the best in me. Summarized, you were the best possible match for me as a supervisor and I am so grateful for all the opportunities you gave and are still giving me.\\
To me, you are the embodiment of a `complete ecologist'. You have roots in natural history and as a nature enthusiast with JNM. You are also a bright thinker with deep theoretical knowledge and perceptive insights. A combination of those makes you come up with brilliant and refreshing ideas for both fieldwork \'{a}nd lab experiments to gain deep understanding of ecological and evolutionary processes. It is a rare but very valuable combination which I admire and deeply respect. On top of that, you are more recently also expanding your work into more applied research (`building with dunes'). This really makes the picture complete: naturalist, theoretician \'{a}nd applied researcher...\\
I am thrilled our collaboration is not over yet and I'm looking forward to the next challenges you will throw in my direction!\\

\textbf{Dirk}, thank you for all the support as co-supervisor and the various opportunities you provided. You were always super quick to read and give feedback on my manuscripts. I was really lucky with not one but two very approachable and reachable supervisors! I always felt at ease with you to speak my mind and you made me feel confident about my work. Through you I could network at INBO and get the needed expert input. You provided help and extra assistance with fieldwork, let me follow a course at INBO that proved pivotal to my work, arranged to get a podcast recorded, let me present my work in Nijmegen, pushed me for the natuurfocus-article (and got \textit{Bembix rostrata} as a cover girl!) and invited me to several nice `field work' days to catch butterflies for research (and I hope you will think of me in the future as well)! Thank you so much for facilitating so many aspects of my research and providing so many little extras that give a finishing touch to this work!\\

Thank you to the four jury members and chair of my PhD thesis, I am honored to have had such a balanced group of experts with diverse backgrounds to review my manuscript. Your comments improved the thesis and especially the last two chapters as you provided me with pre-submission peer-reviews. \textbf{Josep As\'{i}s Pardo}, it was an honor to have you as a real (former) \textit{Bembix} researcher on my committee. I really appreciate your knowledge and experience with these species, it's a very rare skill! \textbf{Viktoriia Radchuk}, with your modelling background and roots in insect-research, I knew the technical and methodological soundness of my work would be inspected thoroughly. I am much impressed (and admitted, also a bit intimidated) by your work! \textbf{Jan Van Uytvanck}, thank you for your conservationist point-of-view, it's mind-broadening. I'm looking forward to learn so much more from your vast knowledge of plants, nature management and ecological insights during the next series of excursions for the students! \textbf{Laurence Cousseau}, with a background in non-insect behavioral ecology and population genetics, you were a perfect match for my committee. Thank you for the critical and honest comments. You also have been a colleague of me since I started at the Terec, always there with a friendly, welcoming smile and ready for a chat. I have always admired the thoroughness of your work and how you combine your scientific career with your passion for music! \textbf{Mieke Verbeken}, thank you for being the chair of my PhD committee and also for your passionate and enthusiastic teaching, it was one of the factors in kindling the flame of my passion for biology when I was still studying physics.\\

\textbf{Martijn}, thank you for being such a great colleague, friend and extra mentor. You were there form the beginning, already helping me with great input for my FWO proposal. It was always great to have you around in the lab for quick chats, often turning into much longer conversations. The excursions together were always a delight and with you taking the lead, it always were enjoyable and exciting trips. Your help was crucial for my first paper, which was actually a project that completely got out of hand after some impact-tests of tags on digger wasps. I remember you said, while we were in the field and talking about the subject, `Maybe you should write something about it, even if it's a small paper, you never know'. Sitting together working on the paper, you showed and taught me how to revise, knead and reformulate scientific writing. It's a never-ending developing skill, and you often quoted `Hell is sitting on a rock reading your own scientific publications' (it's not true for that first paper though, I hear your voice echoing when I reread it, hehe...). I do miss your presence, but luckily (thank you!) you got me involved in your strawberry project (your enthusiasm was contagious), so we will have research to work on together in the near future!\\

A big thank you to many people who helped me in the field: friends, family and colleagues, but I want to thank two persons specifically: \textit{\textbf{Filiep}, bedankt voor je vele grapjes, gelach en getetter in het veld, ik had die luchtigheid en (zelf)relativering nodig om veel van het veldwerk uit te houden. Het zou maar een héél saaie boel geweest zijn zonder jou af en toe mee te hebben! \textbf{Johan}, bedankt voor je gigantisch vriendelijke ontvangst en vertrouwen om je tuin en schuur voor mij beschikbaar te stellen tijdens het veldwerk. En vooral bedankt voor de `pootjes-op-de-grond' of `wat-moet-een-beheerder-nu' babbels over de harkwesp, het is een (misschien iets te) complex verhaal geworden, maar jouw aanmoedigingen deden me inzien hoe belangrijk het is om als wetenschapper m'n verhaal aan de man/beheerder te proberen brengen!}\\

There were many people from INBO that have helped me with several practical and theoretical aspects during my PhD.  \textit{\textbf{Wouter Van Gompel}, bedankt voor de hulp en gebruik van de trimble gps en het was fijn af en toe een vriendelijk en rustig gezicht tegen te komen in de duinen! \textbf{An Vanden Broeck}, bedankt voor je raad om mijn genetisch labowerk op te starten en je constante strenge bedenkingen rond haplodiploidy in populatie genetica, ze hebben me het uiteindelijke werk hoger doen hijsen. \textbf{Joachim Mergeay}, bedankt voor je raad van bij het begin van het project. Verschillende van die raadgevingen hebben me behoed van moeilijkheden, want vaak had ik pas veel later door wat je redenering was geweest voor die keuzes. Je theoretische inzichten in populatie-genetische processen zijn bewonderenswaardig. \textbf{Sam Provoost}, bedankt voor je inzichten in hoe processen en beheer werken in duinen, en de monitoring ervan. Zonder jouw eerder werk had ik niet tot dezelfde inzichten kunnen komen over harkwespen hun habitat en noden. \textbf{Petra}, wat was het fijn om met jou een podcast op te nemen! Je diepe interesse en gedrevenheid werkten motiverend en stelden me helemaal op mijn gemak. Jij doet wetenschapscommunicatie toch zo simpel lijken... Maar zonder jou is dat zwoegen!}\\

There are so many \textbf{professionals and nature-enthusiasts} I met during my PhD that showed direct interest and fascination for my study species and directly offered their support in arranging field work and communication regarding my research. \textit{Heel wat mensen van ANB, bedankt voor jullie enthousiasme en hulp: de boswachters van aan de kust en andere plekken, zoals Johan, Guy Vileyn, Jeremy Demey, Dirk Raes en Koen Maertens; de regiobeheerders Evy Dewulf en Klaar Meulebrouck. Merci Thierry Paternoster de DEMNA pour votre enthousiasme et votre aide dans la belle population d'Hensies. Bedankt Rika Driessens van Aquaduin (voormalige IWVA), en je opvolger, Thomas Rogier: bedankt om me die workshop te laten geven in de Doornpanne en wat is het fijn om een oud-studiegenoot tegen te komen die helpt de prachtige natuur in de duinen in stand te houden! Bedankt ook aan de gidsen in de Doornpanne voor jullie enthousiasme en interesse, om mij een dik uur te laten babbelen en ratelen over mijn passie voor de harkwesp! Bedankt aan heel wat enthousiastelingen bij natuurpunt, zoals Pieter Vanormelingen en de vrijwilligers van Averbode Bos en Heide. Bedankt Rudi Delvaux van het Grenspark Kalmthoutse heide, bedankt medewerkers van Kempens Landschap voor de aandacht voor de populatie in Geel-Bel. Wim, bedankt om Pol Cosmo te creëren! Jij weet op zo'n leuke en (schijnbaar) eenvoudige manier mensen de schoonheid van insecten te laten zien. Zulke artiesten kunnen entomologen wel vaker gebruiken!}\\

What a wonderful place I could be part of during my PhD: the \textbf{TEREC} (and the adventure is luckily not over!): we are not only a group of colleagues, several of you have become my friend. The support and care for each other is amazing. It's an incredible thing to be part of: to have lunch together (sometimes going for yummy pizza or noodles and in spring outside in the botanical garden), drink coffees, indulge in many chats and conversations in the corridors, go for afterwork drinks... You are inspiring to me as a scientist and social being. Your presence and support was crucial to accomplish my work, especially the last few months, to keep me afloat mentally, to relativize this finalizing a PhD and keep my levels of stress under control. I must also say that we as PhD students, by having the 11\textsuperscript{th} floor to ourselves, it really makes us a cohesive group of island-offices. There is an atmosphere of `we're in this together', ready to help each other out, give each other pep-talks, indulge each other in complaining and have some relaxing social activities.\\

\textbf{Luc} Lens, thank you for, together with Dries, creating such a supportive research group. Thank you as well for being such a passionate and eloquent lecturer, your clear and enthusiastic teaching aided me in quickly turning my interests towards ecology. Hans, Pieter, Viki, Angelica: You are the four pillars of the TEREC keeping everything afloat and running! \textit{\textbf{Hans}, bedankt voor alle babbels en je cynische zelf, aanstekelijke lach en je scherpe stekken onder water, die meer dan eens de broodnodige zelfrelativering in mij aanwakkeren. \textbf{Pieter}, bedankt voor al je hulp en babbels, je kalme zelf en je pure en diepe natuurstudie-interesse. Dank je voor of en toe wat opluchtend gemopper, zoals over aardbeiplantjes. Jouw praktische kijk, wetenschappelijke input en ervaring zijn van onschatbare waarde. \textbf{Viki}, bedankt voor je nuchtere hulp, administratief en in het labo. Je sobere `okay's via mail, chat of in real-life doen me maar al te vaak beseffen dat ik iets te veel wolkjes schrijf en babbel als ik iets nodig heb. Bedankt ook om de terec letterlijk een heel cosy plek te maken. Met jou erbij op café is het altijd gezellig, en wat heerlijk om dan nog iets harder je sappig Gents te horen doorklinken. \textbf{Angelica}, zonder jouw strak geteugelde sturing zouden we hier op de TEREC allemaal als een kieken zonder kop rondlopen. We zouden praktisch en logistiek als een kaartenhuisje ineen vallen. Bedankt voor al je hulp, verwelkomende lach en koffie. Jij bent een groot onderdeel van die warme en ondersteunende werkplek die de TEREC is.} I would also like the thank the vital and helpful cleaning staff and people from the sandwich bar downstairs at the Ledeganck. Your provide a level of essential care that enable us to work optimally.\\

Due to corona, I witnessed a clear separation in a pre-corona and post-corona squad of TEREC-colleagues. However, they were and are both a group of warm, supportive, positive-minded colleagues and friends.\\

\textit{\textbf{Jasmijn}, bedankt om me vanaf de eerste dag meteen welkom en thuis te doen voelen op onze bureau. Je maakte me meteen wegwijs en jouw gebabbel, gelach en sfeerverzorging brachten heel wat gezelligheid, het contrast was groot toen je weg ging!} \textbf{Jiao}, I'm so happy I got to know you better during and after we went to a conference together in Birmingham. Your often (but not always!) silent presence, sincere smiles and now and then some giggling brightened up our office. I was amazed to hear you say on several occasions `guichelheil' better than anyone from West-Flanders. Thank you for being my friend, it is a comforting feeling to know I have someone on the other side of the planet who cares about me and supports me. \textbf{Pei}, thank you as well for the support and friendship during the conference in Birmingham. I always admired your optimism and the gratefulness you showed constantly to everyone around you. It was truly interesting and refreshing to talk openly and with mutual respect about culture, habits and society. It's a wonderful notion to have more overlapping core values with someone at the other side of the world than with many people right here in this tiny country. \textbf{Stefano}, thank you for always having a dry and/or sarcastic remark ready for every situation. \textbf{Alejandro}, thank you for keeping discussions at the TEREC lively and sharp, debate club was quickly initiated at lunches and social gatherings when you were around. \textbf{Bram S.}, I really admire your activism and engagement in many aspects of your live. Thank you for co-organizing the coding club in the old days. \textbf{Dries van de Loock}, thank you for your dry humor, welcoming Daan and me in Zambia and being in my mind the embodiment of a true field biologist: feet firmly on the ground, attentive ears and watchful eyes, with binoculars grown fixed around the neck. \textbf{Irene}, you were a great colleague and friend to have around, talking to you was always easy. Your all-round interest in nature and roots in NJN quickly made it clear to me we are very like-minded people. \textbf{Lionel}, thank you for the coding clubs you started, I realized only much later how much they helped in starting to develop my own coding skills. You a great and skillful scientist and you are also a very considerate listener in conversation, thank you for that! \textbf{Lore}, we started at the same moment as a PhD student and getting to know you changed my first impression of you drastically: from a shy but friendly colleague into an amazingly strong, experienced, talented and creative woman! I really admire your courage, for instance for deciding to leave your job as a PhD student, but also for renovating a house all on your own. All the best to you, Lionel and Alix in France. We'll visit! \textbf{Svana}, thank you for still being such a close and supportive friend. Thank you for the covid-walks and your literal support during one of those, when I strained my ankle on a very treacherous hill. I admire you for being simultaneously an amazingly talented scientist and one of the most humble and modest persons I know. Thank you for all your warmth, support and friendship. Our online talks always lighten up my day. We'll soon visit you and Vincente to meet that little swallow Enara in person! \textbf{Steven}, thank you for all the lively conversations about bumblebees, radio-tracking, societal issues and popular science in general. Thank you as well for the nice board game evenings! I admire you as great critical thinker and scientist and an enormously engaged educator. \textit{\textbf{Bram D'Hondt}, jij was een lange tijd een extra add-on van ons team tijdens de excursies. Bedankt voor het delen van al je kennis en inzicht over planten en hun ecologie (en ook vooral hoe dit educatief over te brengen bij de studenten) tijdens de voorbereidingen van de excursies. Jouw rustige, ervaringsdeskundige en respectvolle manier van ons klaar te stomen als begeleiders deed me altijd uitkijken naar een nieuwe reeks voorbereidingen.}\\

\textbf{Ruben}, I will thank you here, as you were a solid bridge I could always count on across the pre and post corona squads. Our bond really grew more strongly over the years. It intensified when the two of us were one of the few PhD students left during Corona, soon Garben joined and our digital coffee breaks were crucial for me to feel supported and understood, as colleagues and as friends, during corona-times. The apotheosis happened the last few months: we were both finalizing our PhD, you a bit ahead of me. With a simple description of any of my feelings during this rollercoaster, a simple smile or laugh from you and `been there, I know exactly how you feel', could relieve so much stress and frustration and I felt understood and supported. Thank you for that! I will really miss you in the lab, as a friend, a colleague, someone to joke and laugh with and especially someone to talk to about my social insecurities and to ask advice whenever I felt socially awkward. You are a chameleon, you can adjust to any kind of social group or situation, respectful and considerate towards any person! And I'm sure we will find time for excursions, a drink or a coffee together in the future! \textbf{Garben}, I knew quite soon we would get along very well. You are a big nature enthusiast and I'm sure you often underestimate the knowledge and expertise you have. You are immensely thoughtful and you do your utmost to make me and other friends feel comfortable. That also makes it easy to both laugh together and have serious conversations. I learn a lot about myself and other people when we talk and share our perspectives and observations of the (social) world. I really admire your empathy and courage. I look forward to many more excursions together and sharing, in all modesty, the shear joy of seeing a cool species! Dune-\textbf{Charlotte} (T.), I first met you as a student on the biological excursions and then for your bachelor thesis. You directly made a deep impression on me: your enthusiasm to tackle everything I throwed at you and the ease with which you (seemed to) teach yourself the needed skills. I could also read your master thesis, and I was more than impressed by your learning curve! When you become our colleague not that long ago, I immediately fell we would get along, you're sustainably minded and considerate towards people around you. I also find it very easy to talk to you in person and we can quickly engage in personal conversation and it's very easy for me to open up about my inner thoughts and reflections. I'm looking forward to more time together as colleagues and friends! \textbf{Marina}, thank you for your broad smile every day. You are always very approachable and easy to talk to. I really admire that you possess both gentleness and fierceness, especially as a female scientist and birder. You always stay very modest, sometimes too modest! \textbf{Cesare}, thank you for all the pleasant and attentive talks. You're a very considerate and caring person for people around. Thanks for indulging together in playing (and showing-off) racket in the office the past few weeks to relieve some stress, or an intense game of kicker, commenced through a few simple glances towards Marina and you. \textbf{Silvija}, your presence at our office always lightens op my mood thanks to your cynical humor, your mocking twinkling eyes and smirking grins. You're the much-needed balance for us arduous naturalist nerds at the TEREC to pull us back from our self-centered little world. Thank you for you advice to always choose evil, it's a life-changing attitude. \textbf{Felipe}, thank you for the many mathematical-philosophical discussions and reflections we so often engage in when at the office. Talking to you often stirred the nerdy physicist in me that had faded in the background but does sincerely love mathematical formulas and ideas. You always ask in all seriousness how I was really doing and that is priceless for feeling understood and appreciated. \textbf{Bram} Catfolis, thank you for being my loyal kicker-buddy, I much appreciate you rather get kicked in the ass together with me than letting your competitive soul (which is strong!) take over. I must admit I wasn't sure how to behave towards you until a late game of `koehandel' during the writing week together with Ruben. Poker faces, (double)bluffs, heavy negotiations and desperate moves easily broke the ice. \textbf{Katrien}, you are an involved colleague and friend. I admire your engagement as a volunteer, your nature enthusiasm (in which we especially share a passion for bees), and your hands-on mentality to get things done. You are a strong woman and scientist with much to show to people and a very large heart. \textbf{Ferehiwot}, I admire your courage and perseverance and wish you all the best! \textbf{Karen}, you are an always friendly and considerate face at the office. You are an amazingly talented scientist which radiates discrete confidence and respect. \textbf{Jonathan}, it's so easy to talk to you and you are always very engaged and interested in other people's life in conversation. I admire your ambitious mindset and your vigorous and exorbitant defense of your definition of proper Italian cuisine. \textbf{Frederik}, thank you as well for welcoming me to our then shared office. You were generous with advice and helping me with any small practical or work-related problem. I remember fondly our conversations on more serious and topical issues and how we reinforced each other's frustrations (especially directed at one certain philosopher). Just recently you made me sentimental by walking absent-mindedly in our office, your feet bringing you back to our office on auto-pilot stating `you can take a postdoc out of the 11\textsuperscript{th}, but you can't take the 11\textsuperscript{th} out of the postdoc'. I am already getting nostalgic by the thought of that nearing fate. \textbf{Diederik}, thank you for your dry and unparalleled sense of humor. I am still puzzled by your skill to, with the speed of light, share the most relevant and applicable memes to any digital conversation. \textbf{Thomas}, I really admire your dedication and passion for your amazingly diverse and interesting study system, myrmecophiles. You are an amazing scientist. You have been around for a very long time, and it always feels comfortable to talk to you and have you around. \textbf{Nicky}, you are an amazingly talented and intelligent researcher and you add an extra dimension of type of research to the TEREC. There are \textbf{so many more people} at the TEREC which I much appreciate and want to thank to be part of this amazing group of colleagues: Claudia, Laurence, Maxime, Bram VT, e-DNA-Charlotte (VD), Aliz\'{e}e, Camille,...\\

Thank you to all the students I have (co-)supervised during their bachelor- or masterthesis, internship and terrestrial field course: Moyra, Matthieu, Fabien, Phaedra, Jens, Charlotte T., Ward L., Robbe, Brecht, Elke Lycke, Maya, Ward P., Eliane, Laura, Celien. I gained a lot of insights during your projects and the learning really went both ways. I feel a deep respect for every single one of you for the individual learning paths I could witness from the sideline and in interaction. \textit{\textbf{Moyra}, met jou babbelen voelde van in het begin heel vertrouwd en aangenaam en jouw moed om door te zetten in moeilijke situaties werkt inspirerend! \textbf{Matthieu}, ik had telkens bewondering voor jouw toewijding en enthousiasme om de verschillende onderdelen van je thesis aan te pakken. Je zette op je eentje, zelfstandig gigantische stappen in je vorderingen, jij kan véél aan (maar je mag ook hulp vragen!)} \textbf{Fabien}, you were such a motivated, driven and mature intern who was enthusiastic to tackle very complex analyses. Tutoring you was the opposite of work, because I got so much insights and output back from you. \textit{\textbf{Robbe}, bedankt voor je hulp met veldwerk als jobstudent, en een super vriendelijke, altijd goedlachse JNM'er te zijn! Bedankt ook om zo eerlijk te zijn om over QGIS heel veel te zagen tijdens je bachelorproef, om dan later toe te geven dat je extra cursussen GIS volgde. Het is de perfecte anekdote om haat-liefde verhouding tegenover GIS of R aan iemand uit te leggen. \textbf{Phaedra}, bedankt om een van mijn eerste, overenthousiaste bachelorstudenten te zijn die heel mild voor me was, want ik had zelf nog veel te leren als begeleider! Ik bewonder je moed om als tuinier aan de slag te gaan in de plantentuin nadat je als bioloog was afgestudeerd. Het is fijn om jouw volle, warme glimlach zo nog af en toe nog eens tegen te komen! \textbf{Ward} L., tijdens je bachelorproef had ik meteen door hoe intelligent en gedreven je bent. Je passie voor slakken was je geleider om fantastische dingen te doen.}\\

\textit{Ik wil ook nog heel wat extra \textbf{vrienden} bedanken: de NWG-bende/Ouwezakken/\-Pintjes-groep: Hans, Sander, Pepijn, Proesmans, Dora, Neeltje, Bartje, Jefke, Michieltje, Ward en Margaux (en Daan uiteraard). Op excursie gaan met jullie is altijd plezant en leerrijk. Wat is het fijn om zo'n natuurstudie-minded vriendengroep te hebben overgehouden aan mijn JNM-tijd, waarbij ieder gewoon een beetje gek en heel nerdy mag zijn. Hans, bedankt om zo attent te zijn, jij blijft altijd ons knuffelcontact. Ward en Margaux, we missen jullie! Maar jullie zitten daar ook goed tussen de elanden en otters in Zweden. Femke, Maaike, Ana\"{i}s (mijn `Liefdes van ons leven'), bedankt om samen goed te kunnen lachen, elkaar te steunen, te babbelen en te reflecteren over wat er echt toe doet in het leven. Femke, bedankt voor je activisme en de duurzame innerlijke Batje in mij levend te houden! Sofie De Schryver, wat bewonder ik je voor je doorzettingsvermogen en positivisme! Je doet me altijd op een andere, inclusieve en hoopvolle manier naar de wereld kijken. Paulien, bedankt voor alle steun en tijdens onze studies mijn dichtste maatje te zijn. Het blokken samen deed altijd veel deugd! Limburg is eigenlijk niet zo heel ver, ik moet maar eens tijd maken om die kant op te komen! Ook aan alle vrienden die ik niet bij naam noem of waarvan het contact wat verwaterd is (ik ben echt niet goed in contact blijven houden op afstand): bedankt!}\\
	
\textit{\textbf{Mama en papa}, bedankt om me altijd met raad en daad bij te staan en voor jullie warme opvoeding, met een rode draad van respect voor medemens en natuur. Jullie lieten me mijn eigen weg zoeken, met onvoorwaardelijke steun en respect. Bedankt voor alle begrip en geduld voor een puber die eerst verkondigde dat ze een `echte wetenschap' wou gaan studeren om zo'n twee jaar later schoorvoetend toe te geven dat die enkele biologische keuzevakken toch wel deden hunkeren naar meer... De essenti\"{e}le zorg die jullie me gaven---thuis komen in een warm nest, voeten onder tafel schuiven, altijd en overal klaar staan met een luisterend oor of me ergens naartoe te voeren---was van onschatbare waarde om onbezorgd te kunnen studeren, te ontdekken en mezelf te ontwikkelen. Ik besef pas nu hoeveel jullie toen voor me deden. Een welgemeende, diepe dank jullie voor alles.}\\

\textit{\textbf{Joke}, bedankt om zo'n ondersteunende zus te zijn en zoveel momenten en dingen te delen terwijl we opgroeiden. Ik heb altijd heel hard naar je opgekeken, voor je doorzettingsvermogen, creativiteit en empathie. Wat ben jij een sterke en fantastische mama. De toewijding en liefde die jij en \textbf{Tom} tonen voor jullie gezinnetje is bewonderenswaardig. Bedankt alle twee voor jullie steun! \textbf{Sam}, jouw jeugdig enthousiasme voor diertjes werkt aanstekelijk. Het is verfrissend om de wereld te ontdekken doorheen jouw ogen. Ik ga graag bij jou in de leer (of was het omgekeerd?), zoals 2 zomers geleden, jij als 3-jarige die ik niet kreeg weggesleurd van bij enkele gravende bijenwolven: `Kij-kij-kij-kijk, kijk... graafwesp'. Wellicht was het mijn eigen schuld. \textbf{Ella} en \textbf{Lotte}, jullie doen me even alles en de wereld vergeten telkens jullie lachen en kirren. \textbf{Opa}, je trots en je lach als je zegt dat ik `de primus' ben doen me altijd warm voelen. Bedankt voor de vele uitstappen vroeger en de kleurrijke en gelukkige herinneringen.}\\

\textit{\textbf{Meme}, jij bent de enige waarvoor ik dit werk zo'n anderhalf jaar eerder had willen klaar hebben. Je onvoorwaardelijke steun, liefde en trots blijven hangen, ook de spiegel die je me voorhield: bedankt voor je doorzettingsvermogen, plichtsbesef en kracht. Ook voor de herkenning van de negatieve spiralen waarop ik soms (de laatste tijd was het vaker) bots. Ik weet niet altijd wat ermee aan te vangen, jij wist dat ook niet, maar het is een deeltje van jou in mij, zo kan ik het wat gemakkelijker dragen. Soms beeld ik me in hoe het zou zijn jou mijn werk te tonen. Je zou elke pagina omdraaien en bekijken, ook al verstond je geen Engels, telkens met een grote zucht of `oh' vol van bewondering. Af en toe een opmerking dat het toch maar vieze beesten zijn. Je zou opkijken en naar papa turen. 't Is toch een hoaze hé. 't Zal wel zijn. Je zou zo breed glimlachen. Die lach, die staat op mijn net- en trommelvlies gebrand. Je zou één van de weinige zijn waarvan ik de trots en welgemeende proficiats ook écht diep van binnen zou aanvaarden en kunnen internaliseren. We zouden een heel warm moment delen, jij vol stralende trots, ik gevuld van dankbare warmte. Zo zou het zijn geweest. Zo is het.}\\

\textit{\textbf{An} en \textbf{Jos}, bedankt voor jullie onvoorwaardelijke steun en om me al meer dan 7 jaar geleden zo vlug en gemakkelijk op te nemen in jullie warme en zorgzame nest. Ik bewonder jullie hard voor jullie engagement, jullie betrokkenheid, en jullie inzet en toewijding naar vrienden en familie. Ik word altijd met open armen ontvangen, ik heb een extra thuis gekregen en jullie staan altijd klaar voor Daan en mij. \textbf{Thijs} en \textbf{Adel}, jullie zijn een onderdeel van die extra warme thuis. Jullie zijn altijd geïnteresseerd en betrokken, en ik kan mezelf bij jullie zijn. Bedankt voor al jullie steun!}\\

\textit{\textbf{Daan}, mijn liefde. Waar begin ik? Je betekent zoveel voor me in zoveel verschillende aspecten. Eerst was je een vriend die ik bewonderde voor al zijn natuurkennis en respectvolle omgang met iedereen rond jou. Die vriend ben je nog steeds. Je werd mijn lief en aftastend maar met volle overgave leerden we elkaar beter en intiemer kennen. Vanaf het begin voelde ik je onvoorwaardelijke steun en werd ik minder onzeker (al zie jij me vaak veel liever dan ik mezelf graag zie). Ik kan zijn wie ik ben bij jou, inclusief onnozel doen, lachen en gekke woorden te verzinnen. Je was ook mijn collega die me meteen hielp wegwijs maken op het labo en ook van bij het begin een wetenschappelijke mentor. Als mensen dachten dat ik heel vlug en vol zelfvertrouwen leek aan te voelen wat je als PhD moet aanvangen in die ecologisch, academische wereld, dan was het omdat jij me bij elke stap uitleg gaf, je inzichten deelde en me steunde. Zeker toen we samen nog aan ons doctoraat bezig waren babbelden en discussieerden we uitvoerig (soms iets te veel dan gezond was) tot laat 's avonds over ons onderzoek. Nu nog steeds kan ik bij jou terecht om mijn ideeën en onzekerheden te pitchen, jij bent mijn eerste klankbord. Meer dan eens gaf jij mee een aanzet voor een project of duwde je me in gang omdat je waarde zag in een onzeker idee van mij. Jij was ook mijn labo-begeleider, door jouw ervaring met je thesis hielp je me mijn genetisch labowerk op te zetten, je legde uit hoe jij te werk ging en je wees me op onvoorzichtigheden. Zonder jou zou het allemaal een boeltje geweest zijn. Je bent de eerste tegen wie ik mijn hart lucht over frustraties en onzekerheden. En je bent dan ook de eerste die merkt als het niet goed met me gaat, en ook diegene die de eerste ladingen van negativiteit over zich krijgt. Bedankt om altijd aan mijn zijde te blijven staan en me ook actief uit mijn negatieve spiralen te proberen sleuren. Het is niet echt een dankbare taak, maar wat een geluk heb ik dat jij mijn balans wil zijn. Bedankt om de laatste maanden zo fantastisch goed voor mij te zorgen. Door jou liep ik met gewassen kleren rond, at ik nog gezond en leefde ik niet tussen het vuil. Ik hoop dat ik dit vlug weer wat meer in evenwicht kan brengen! Ik bewonder je enorme kennis, interesse, nerdiness, ecologische inzichten, je engagement voor de natuur en jouw inzet voor vrienden. Je bent zoveel en alles voor me: mijn beste vriend, klankbord, raadgever, liefde, bondgenoot, inspiratie-bron, steun... Ik kijk hard uit naar samen nog heel vaak de natuur in te trekken (met vosjes, witbuikzandhoenen, forellen en eekhoorntjes) en elkaar te kunnen blijven vasthouden onderweg. Ik hou van je.}

\end{small}


\clearpage
\thispagestyle{plain}
\hbox{}
\clearpage
%%%%% List of publications %%%%
\newpage
\pagestyle{mainmatter}
\chapter*{List of publications}
\addcontentsline{toc}{chapter}{List of publications}
\chaptermark{List of publications}

\begin{large}\textbf{Articles in international peer-reviewed journals}\end{large}\\

	\begin{enumerate}
		\item \textbf{Batsleer, F.}, Portelli, E., Borg, J. J., Kiefer, A., Veith, M., \& Dekeukeleire, D. (2019). Maltese bats show phylogeographic affiliation with North-Africa: Implications for conservation. \textit{Hystrix, the Italian Journal of Mammalogy}, 30(2), 172–177. \url{https://doi.org/10.4404/hystrix-00237-2019}
		
		\item \textbf{Batsleer, F.}, Bonte, D., Dekeukeleire, D., Goossens, S., Poelmans, W., Van der Cruyssen, E., Maes, D., \& Vandegehuchte, M. L. (2020). The neglected impact of tracking devices on terrestrial arthropods. \textit{Methods in Ecology and Evolution}, 11(3), 350–361. \url{https://doi.org/10.1111/2041-210X.13356}
		
		\item Dekeukeleire, D., Janssen, R., Delbroek, R., Raymaekers, S., \textbf{Batsleer, F.}, Belien, T., \& J. Vesterinen, E. (2020). First molecular evidence of an invasive agricultural pest, \textit{Drosophila suzukii}, in the diet of a common bat, \textit{Pipistrellus pipistrellus}, in Belgian orchards. \textit{Barbastella}, 13(1), 109–115. \url{https://doi.org/10.14709/BarbJ.13.1.2020.18}
		
		\item Bonte, D., \textbf{Batsleer, F.}, Provoost, S., Reijers, V., Vandegehuchte, M. L., Van De Walle, R., Dan, S., Matheve, H., Rauwoens, P., Strypsteen, G., Suzuki, T., Verwaest, T., \& Hillaert, J. (2021). Biomorphogenic Feedbacks and the Spatial Organization of a Dominant Grass Steer Dune Development. \textit{Frontiers in Ecology and Evolution}, 9, 761336. \url{https://doi.org/10.3389/fevo.2021.761336}
		
		\item \textbf{Batsleer, F.}, Maes, D., \& Bonte, D. (2022). Behavioral Strategies and the Spatial Pattern Formation of Nesting. \textit{The American Naturalist}, 199(1), E15–E27. \url{https://doi.org/10.1086/717226}
		
		\item \textbf{Batsleer, F.}, Van Uytvanck, J., Lamaire, J., Maes, D., \& Bonte, D. (2022). Rapid conservation evidence for the impact of sheep grazing on a threatened digger wasp. \textit{Insect Conservation and Diversity}, 15(1), 149–156. \url{ https://doi.org/10.1111/icad.12532}
		
		\item De-la-Cruz, I. M., \textbf{Batsleer, F.}, Bonte, D., Diller, C., Hytönen, T., Muola, A., Osorio, S., Posé, D., Vandegehuchte, M. L., \& Stenberg, J. A. (2022). Evolutionary Ecology of Plant-Arthropod Interactions in Light of the ``Omics'' Sciences: A Broad Guide. \textit{Frontiers in Plant Science}, 13, 808427.\\ \url{https://doi.org/10.3389/fpls.2022.808427}
		
	\end{enumerate}

\vspace*{1cm}
\noindent \begin{large}\textbf{Articles in other peer-reviewed journals}\end{large}\\

\begin{enumerate}
	\item \textbf{Batsleer, F.}, Dekeukeleire, D., Batsleer, M., \& Verbelen, D. (2019). Kleur\-afwijkende vuursalamander in België. \textit{RAVON}, 73, 20–22.
	
	\item 	Boeraeve, M., \textbf{Batsleer, F.}, Vermeiren, H., Thomaes, A., Opstaele, B., \& Dekeukeleire, D. (2019). Winterverblijfplaatsen voor vleermuizen---Het belang van bunkergordels, ijskelders en forten in Oost-Vlaanderen. \textit{Natuurfocus}, 18(4), 136–144.
	
	\item Cool, R., \textbf{Batsleer, F.}, \& Bonte, D. (2020). In het spoor van een nieuwe leefomgeving---Biodiversiteit langs spoorwegen: De Kiezelsprinkhaan als case\-studie. \textit{Natuurfocus}, 20(3), 18–25.
	
	\item \textbf{Batsleer, F.}, Maes, D., Uytvanck, J. V., Provoost, S., Lamaire, J., \& Bonte, D. (2021). De moeilijke balans tussen duinbeheer en bescherming van de Harkwesp---Valt begrazing in de duinen te verzoenen met het behoud van ongewervelden? \textit{Natuurfocus}, 20(3), 100–108.
	

\vspace*{1cm}	
\end{enumerate}

%\cleardoublepage
%\thispagestyle{empty}
%\hbox{}
\clearpage

\CenterWallPaper{1}{pictures/Biblio-1withtext.png}
\newpage{\thispagestyle{plain}\clearpage}
%\vspace*{\fill}
%\begin{footnotesize}\begin{flushright}
%	{\color{darkgray} Specimens of \textit{B. rostrata} at the collection of the Royal Belgian Institute of Natural Sciences (RBINS).}
%	\end{flushright}
%\end{footnotesize}
\cleardoublepage
\ClearWallPaper

	%%%%%%%%%%%%%%%%%%%%%%%%%%%%%%%%%%%%%  Bibliography  %%%%%%%%%%%%%%%%%%%%%%%%%%%%%%%%%%%%%%%%%%%%%%%%
%	\thispagestyle{empty}

	%\chapter*{Bibliography}
	
		\bibliographystyle{customPhDEcology}
		\begin{footnotesize}
			\bibliography{ExportBib08042023}
		\color{black}
		\setlength{\thumbwidth}{0cm}
		\setlength{\thumbheight}{0cm}
		\setlength{\bibsep}{0.0pt}

	\end{footnotesize}

%\includepdf{../cover/Backcover.pdf}	
\end{document}