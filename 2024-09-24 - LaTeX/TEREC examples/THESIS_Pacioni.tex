%%%%%%%%%%%%%%%% PRESETTINGS %%%%%%%%%%%%%%%%%%%%%%%

\documentclass[10pt, twoside]{book} %

\usepackage[paperheight=24cm,paperwidth=17cm,margin=2.3cm]{geometry}
%\usepackage[margin=2.3cm,b5paper]{geometry} %lmargin=2.5cm,rmargin=4cm,tmargin=3cm,bmargin=3cm
%\pdfpagewidth 17cm
%\pdfpageheight 24cm

\usepackage[table]{xcolor}
\usepackage{hanging}

\usepackage{amsmath,amssymb,mathtools,graphicx,textcomp,booktabs,url,setspace,xcolor,soul,eurosym}
\usepackage{multirow,listings,setspace,gnuplottex,latexsym,keyval,ifthen,moreverb,lscape,forest}
\usepackage[pagewise]{lineno}
\usepackage{adjustbox}

\usepackage{wallpaper}
\usepackage{fixltx2e}
\usepackage[T1]{fontenc}

\graphicspath{{fig/}} %This line specifies the path where the graphics files (images) for the document are located.

\usepackage[figuresright]{rotating}

\usepackage[]{microtype} %activate={true,nocompatibility},final,kerning=true,spacing=true,factor=1100,stretch=10,shrink=10
\usepackage{booktabs}
\usepackage{romannum}
\usepackage{textgreek}

\usepackage{lmodern}
\usepackage{tabularx}
\usepackage[most]{tcolorbox}

\usepackage{tikz,pgfplots}

\usetikzlibrary{calc,shapes,arrows,intersections,shadows}
\usepackage{textcomp}

\usepackage[most]{tcolorbox}
\usepackage{longtable}

\usepackage[]{threeparttable}
\usepackage{array}
%\usepackage[all]{nowidow} %avoid hanging lines


\usepackage[indention=0.5cm,labelsep=colon,font={sf,small},labelfont={bf,sf}]{caption}
\usepackage[indention=0.5cm,font={sf,small},labelfont={bf,sf}]{subcaption}

\definecolor{lightgray}{gray}{0.90} 
\definecolor{darkgray}{gray}{0.3}







\usepackage{fancyhdr}  
\pagestyle{fancy} 
\renewcommand{\chaptermark}[1]{\markboth{#1}{}}
\fancypagestyle{frontmatter}{%    
	\fancyhf{} 	
	\fancyfoot[RO,LE]{\textsf{\thepage}} %Page 
	\fancypagestyle{plain}{
		\renewcommand{\headrulewidth}{0pt}
		\fancyhead{}
		\fancyfoot[RO,LE]{\textsf{\thepage}} %Page
	} 
}
\fancypagestyle{mainmatter}{%    
	\fancyhf{} 	
	\fancyhead[RO,LE]{\nouppercase{\small \textsf{\leftmark}}}
	\fancyfoot[RO,LE]{\textsf{\thepage}} %Page 
}
\setlength{\headheight}{10pt}

\usepackage{pdfpages}

\newcounter{letternum}
\newcounter{lettersum}
\setcounter{lettersum}{13}
\newlength{\thumbtopmargin}
\setlength{\thumbtopmargin}{1cm}
\newlength{\thumbbottommargin}
\setlength{\thumbbottommargin}{3cm}
\newlength{\thumbheight}
\pgfmathsetlength{\thumbheight}{%
	(\paperheight-\thumbtopmargin-\thumbbottommargin)/\value{lettersum}}

\newlength{\thumbwidth}
\setlength{\thumbwidth}{1.2cm}
\setlength{\thumbheight}{1cm}


\tikzset{
	thumb/.style={
		%   draw=black,
		fill=light-gray,
		text=black,
		minimum height=\thumbheight, %\thumbheight,
		text width=\thumbwidth,
		outer sep=0pt,%   outer sep=10pt,
		font=\sffamily\Large,
	}
}
\newcommand{\oddthumb}[1]{%
	\begin{tikzpicture}[remember picture, overlay]
		\node [thumb,text centered,anchor=north east,] at ($%
		(current page.north east)-%
		(0,\thumbtopmargin+\value{letternum}*\thumbheight)%
		$) {#1};
	\end{tikzpicture}
}
\newcommand{\eventhumb}[1]{%
	\begin{tikzpicture}[remember picture, overlay]
		\node [thumb,text centered,anchor=north west,] at ($%
		(current page.north west)-%
		(0,\thumbtopmargin+\value{letternum}*\thumbheight)%
		$) {#1};
	\end{tikzpicture}
}
% create a new command to set a new lettergroup
\newcommand{\lettergroup}[1]{%
	\fancyhead[LO]{\oddthumb{#1}}%
	\fancyhead[RE]{\eventhumb{#1}}%
	\fancypagestyle{chapterstart}{%
		\renewcommand{\headrulewidth}{0pt}
		\renewcommand{\footrulewidth}{0pt}
		\fancyhf{}
		\chead{\oddthumb{#1}}% chapters start only on odd pages
		\fancyfoot[RO,LE]{\textsf{\thepage}} %\fancyfoot[RO,LE]{\textsf{Page \thepage}}
	}
	\thispagestyle{chapterstart}
	\stepcounter{letternum}%
}


%%%%%create own labels for intro and discussion
\newcommand{\oddthumbID}[1]{%
	\begin{tikzpicture}[remember picture, overlay]
		\node [thumb,text centered,anchor=north east,] at ($%
		(current page.north east)-%
		(0,\thumbtopmargin+\value{letternum}*\thumbheight)%
		$) {#1};
	\end{tikzpicture}
}
\newcommand{\eventhumbID}[1]{%
	\begin{tikzpicture}[remember picture, overlay]
		\node [thumb,text centered,anchor=north west,] at ($%
		(current page.north west)-%
		(0,\thumbtopmargin+\value{letternum}*\thumbheight)%
		$) {#1};
	\end{tikzpicture}
}
\newcommand{\lettergroupID}[1]{%
	\fancyhead[LO]{\oddthumbID{#1}}%
	\fancyhead[RE]{\eventhumbID{#1}}%
	\fancypagestyle{chapterstart}{%
		\renewcommand{\headrulewidth}{0pt}
		\renewcommand{\footrulewidth}{0pt}
		\fancyhf{}
		\chead{\oddthumbID{#1}}% chapters start only on odd pages
		\fancyfoot[RO,LE]{\textsf{\thepage}} %\fancyfoot[RO,LE]{\textsf{Page \thepage}}
	}
	\thispagestyle{chapterstart}
	\stepcounter{letternum}%
}


















%%%% CREATE BOX CHAPTER

\usepackage[english]{babel}
\usepackage[newparttoc,explicit, clearempty]{titlesec}%
\usepackage[titles]{tocloft}
\renewcommand{\cftpartpresnum}{BOOK\enspace}
\renewcommand{\cftchapaftersnum}{.}
\renewcommand\cftchapdotsep{\cftdotsep}

\titleformat{\part}[display]{\bfseries\filcenter \def\partname{Book}}{\Huge\MakeUppercase{\partname}\enspace\thepart}{10pt}{\Huge #1}[\thispagestyle{empty}]%

\titleformat{\chapter}[display]{\filcenter\bfseries}{\LARGE\MakeUppercase{\chaptername}~\thechapter}%
{1\baselineskip}
{\huge#1}%
\titleformat{name=\chapter, numberless}[block]{\filcenter\bfseries}{}%
{0pt}{\huge#1\ifstrequal{#1}{\contentsname}{}{\addcontentsline{toc}{chapter}{#1}}}%


\usepackage[titletoc]{appendix} %
\AtBeginEnvironment{appendices}{\def\chaptername\appendixname}
\AtEndEnvironment{appendices}{\def\chaptername\oldchaptername}
\newenvironment{newchapterbox}{%
	\def\chaptername{BOX}\def\appendixname{BOX}\appendices}%
{\endappendices}



%\makeatletter\renewcommand\tableofcontents{%
	%\chapter*{\contentsname}%
	%\@starttoc{toc}%
	%}
%\makeatother
%%% END


\usepackage[nottoc]{tocbibind}	%numbib


%\usepackage[final]{pdfpages}

\usepackage[]{natbib} %numbers,sort&compress
\setlength{\bibsep}{0.2pt plus 0.3ex}

\usepackage[Sonny]{fncychap} %Sonny





\setlength{\parindent}{2em} 
\renewcommand{\contentsname}{Table of Contents}
\renewcommand{\listfigurename}{List of Figures}
\renewcommand{\listtablename}{List of Tables}
\renewcommand{\appendixname}{}

%numbering of sections
%\numberwithin{section}{chapter}
%\numberwithin{subsection}{section}

\makeatletter
\newenvironment{chapquote}[2][2em]
{\setlength{\@tempdima}{#1}%
	\def\chapquote@author{#2}%
	\parshape 1 \@tempdima \dimexpr\textwidth-2\@tempdima\relax%
	\itshape}
{\par\normalfont\hfill--\ \chapquote@author\hspace*{\@tempdima}\par\bigskip}
\makeatother

\makeatletter
\def\mainmatter{%
	\cleardoublepage
	\@mainmattertrue
	\pagenumbering{arabic}
	\def\mainmatter{\cleardoublepage\@mainmattertrue}
}
\makeatother



\definecolor{light-gray}{gray}{0.70} %not always visible on a dell screen!!!
\definecolor{mygreen}{HTML}{23A48B}
\definecolor{myyellow}{HTML}{F49F1F}
\definecolor{myred}{HTML}{C24133}
\definecolor{mygray}{gray}{0.90}
\setlength{\parindent}{2em}

\usepackage{pdflscape}
\usepackage{afterpage}

\usepackage[colorlinks,linkcolor=black,urlcolor=black,citecolor=black,hypertexnames=false]{hyperref} %load hyperref after fncychap
\hypersetup{%
	pdftitle = {},
	pdfsubject = {PhD thesis},
	pdfkeywords = {},
	pdfauthor = {Cesare Pacioni},
	pdfcreator = {\LaTeX\ with package \flqq hyperref\frqq},
}
%\usepackage[]{cleveref} % load cleveref after hyperref
\usepackage{bookmark}

%\renewcommand{\arraystretch}{1.7}




%%%%%%%%%%%%%%%% BEGIN DOCUMENT %%%%%%%%%%%%%%%%%%%%%%%
\begin{document}
%\includepdf[pages=-]{../cover/Cover.pdf}
%\cleardoublepage
\thispagestyle{mainmatter} % empty
	\frontmatter
	\pagestyle{frontmatter}
	\lstset{language=Perl}

	%%%%%%%%%%%%%%%%  BEGIN TITLEPAGE  %%%%%%%%%%%%%%%%%%
\begin{titlepage}
    
    \begin{center}    
    
        \thispagestyle{empty}
        
        \vspace*{3.00cm}
        
        {\centering \huge \textbf{From the tropics \\ to the temperate zone} \par}
        \vspace{0.5cm}
        {\Large \textbf{A comprehensive exploration of thermoregulation in small passerines} \par}
        
        \vspace{7.0 cm}
        
    \end{center}
    
\end{titlepage}

\newpage
		
	\color{black}
	\newpage 
	\thispagestyle{empty}

	\vspace*{\fill}

	\begin{small}

	\noindent \textcopyright 2024 Cesare Pacioni

	\vspace{0.5cm}	

	\noindent Pacioni C. (2024). From the tropics to the temperate zone. A comprehensive exploration of thermoregulation in small passerines. Ph.D. thesis, Ghent University, Ghent, Belgium.

	\vspace{0.5cm}	

\noindent Printed by: University Press, Wachtebeke, Belgium\\
Cover: Camilla Pacioni\\

	\vspace*{0.5cm}
	
	\noindent The research presented in this study was financially supported by Research Foundation – Flanders – under bilateral research cooperation with the Russian Science Foundation
	
	\vspace{1cm}
\end{small}	

	
	\newpage{\thispagestyle{empty}\cleardoublepage}
	\color{black}
	\newpage 
	\thispagestyle{empty}
\begin{center}
			\thispagestyle{empty}
			
			\begin{figure}[h!]
				\centering
				\includegraphics[width=0.18\textwidth]{figures/UGent_logo.png}\hfill
				\includegraphics[width=0.18\textwidth]{figures/FacultySciences_logo.png}\hfill
				\includegraphics[width=0.18\textwidth]{figures/FWO_logo.jpg}\hfill
				\includegraphics[width=0.18\textwidth]{figures/lomonosov_moscow_state_university_logo.jpg}
				\end{figure}
			
			\vspace*{2.50cm}
			
			  {\centering \huge \textbf{From the tropics \\ to the temperate zone} \par}
        \vspace{0.5cm}
        {\Large \textbf{A comprehensive exploration of thermoregulation in small passerines} \par}

			
			\vspace{2.5 cm}
			
			{\normalsize Cesare Pacioni} 
			
			\vspace{1.0 cm}
			
			{\normalsize 2024}	
			
			\vspace{1.0 cm}
			
			{\footnotesize Ghent University, Faculty of Sciences, Department of Biology, Terrestrial Ecology Unit}
			
			\vspace{0.5cm}
			
			{\footnotesize Thesis submitted in fulfillment of the requirements for the degree of\\
 			Doctor (Ph.D.) in Science: Biology}

\end{center}
\newpage
		
	\color{black}
	\newpage 
	\thispagestyle{empty}

		
	{\small \noindent \textbf{Supervisors:} \\
			\hspace{10mm}Prof. Dr. Luc Lens\\
			\hspace{10mm}Dr. Diederik Strubbe}\\

		{\small \noindent \textbf{In collaboration with the Department of Vertebrate Zoology (Faculty of Biology; M.V. Lomonosov Moscow State University; Moscow, Russia):} \\
			\hspace{10mm}Prof. Dr. Anvar Kerimov\\
			\hspace{10mm}Dr. Andrey Bushuev}\\

	{\small \noindent \textbf{Examination committee:}\\
		\hspace{10mm}Prof. Dr. Dries Bonte (chair) \\
		\hspace{10mm}Dr. Beate Apfelbeck\\
		\hspace{10mm}Dr. Luis Reino\\
		\hspace{10mm}Prof. Dr. Geert Janssens\\
		\hspace{10mm}Prof. Dr. Matthew Shawkey} \\
	



		%%%%%%%%%%%%%%%%  BEGIN LISTS   %%%%%%%%%%%%%%%%%%%%%%%
	\newpage{\thispagestyle{empty}\cleardoublepage}
	{\setstretch{0.98}\tableofcontents}
	%\tableofcontents
	



%\cleardoublepage
%\thispagestyle{empty} % empty 
%\hbox{}
\clearpage



\newpage
\subsection*{Glossary}

\begin{table}[!ht]
    \centering
    \resizebox{\textwidth}{!}{%
        \begin{tabular}{|>{\centering\arraybackslash}m{0.2\textwidth}|>{\centering\arraybackslash}m{0.8\textwidth}|}
            \hline
            \textbf{Term} & \textbf{Definition} \\ 
            \hline
            Basal metabolic rate (BMR) & Basal metabolic rate (BMR) is the rate of metabolism of a resting, postabsorptive, non-reproductive, adult animal, measured during the inactive circadian phase at a thermoneutral temperature (White and Kearney, 2012) \\ 
            \hline
            Cold tolerance & The minimum ambient temperature that the animal can tolerate during summit metabolic rate measurements (Swanson and Liknes, 2006) \\ 
            \hline
            Ecophysiology & The study of how the environment interacts with the physiology of an organism (McNab, 2008) \\ 
            \hline
            Ectotherms & Animals that derive their body heat primarily from the environment (e.g., most invertebrates, fish, amphibians, and reptiles; Labocha and Hayes, 2019) \\ 
            \hline
            Endotherms & Animals that produce sufficient internal heat to elevate body temperature above environmental temperature (e.g., most birds and mammals; Labocha and Hayes, 2019) \\ 
            \hline
            Field metabolic rate (FMR) & Field metabolic rate (FMR) is the total energy cost a wild animal pays during the course of a day. FMR includes the costs of basal metabolism (BMR), thermoregulation, locomotion, feeding, predator avoidance, alertness, posture, digestion and food detoxification, reproduction and growth, and other expenses that ultimately appear as heat, as well as any savings resulting from hypothermia (Nagy, 1987) \\ 
            \hline
            Normothermia & A state when body temperature is regulated within standard limits (Trevisanuto et al., 2018) \\ 
            \hline
            Phenotypic flexibility & Reversible adjustments in an individual's phenotype (Piersma and Drent, 2003) \\ 
            \hline
            Rest-phase hypothermia & An actively regulated reduction in body temperature below the normothermic temperatures (Nilsson et al., 2020) \\ 
            \hline
            Summit metabolic rate (M$_{\text{sum}}$) & Maximum metabolic rate achieved during cold exposure (Cortés et al., 2015) \\ 
            \hline
            Metabolic expansibility (ME) & The ratio of M$_{\text{sum}}$ to BMR (Minnaar et al., 2014) \\ 
            \hline
            Thermal conductance & The ease with which heat leaves or enters the body of an organism (McNab, 1980) \\ 
            \hline
            Thermoneutral zone (TNZ) & The range of ambient temperatures at which normothermic body temperature is maintained without regulatory changes in metabolic heat production (Ruuskanen et al., 2021) \\ 
            \hline
        \end{tabular}%
    }
\end{table}

%%%%%%%%%%%%%%%%%%%%%%%%%%%%%%%%%%%%%General introduction %%%%%%%%%%%%%%%%%%%%%%%%%%%%%%%%%%%%%%%%%

\mainmatter
\pagestyle{mainmatter}

%%%%other type of boxes for intro
\setlength{\thumbwidth}{0.8cm}
\setlength{\thumbheight}{1cm}
\tikzset{
	thumb/.style={
		%draw=black,
		fill=gray,%white
		text=gray,
		minimum height=\thumbheight, %0cm, %\thumbheight,
		text width=\thumbwidth, %0cm,
		outer sep=0pt,%   outer sep=10pt,
		font=\sffamily\Large,
	}
}
\chapter*{General introduction} \label{Introduction}
\chaptermark{General introduction}
\addcontentsline{toc}{chapter}{General introduction}
\lettergroupID{\thechapter}

\begin{flushright} \color{black}Cesare Pacioni\color{black}\end{flushright}

\color{black}

\newpage
\renewcommand\thesection{I.\arabic{section}}

\noindent In this dissertation, I examine how passerine birds adapt physiologically to changing environmental conditions, with a particular focus on intraspecific and seasonal variation in thermoregulation. In this introduction, I begin by emphasizing the importance of accurately assessing the effects of temperature change on species' ecophysiological traits, especially for endothermic organisms such as birds. Second, I highlight and explain the primary physiological mechanisms used by birds to cope with environmental change and briefly explore the evolutionary origins of endothermy. Third, I address the major gaps in avian thermoregulation, highlighting in particular the lack of intraspecific and tropical studies. Finally, in the concluding section, I outline the main objectives and the main chapters of this dissertation.\\

	Human-induced climate change poses a significant threat to global biodiversity, with studies suggesting that up to one in six species may be at risk of extinction (Bellard et al., 2012; Kannan and James, 2009; Pörtner et al., 2022; Thomas et al., 2004; Urban, 2015). Understanding how organisms will respond to these changes has been identified as a major challenge in ecology and conservation biology (Schwenk et al., 2009). One of the most evident consequences of climate change is that the rate of human-induced global temperature variation is exceeding the limits of natural climate variability (IPCC 2022). Because environmental temperatures determine animal performance and energy use, and ultimately abundance and distribution (Bozinovic et al., 2011), increased temperature extremes, variability, and unpredictable weather (heat waves and cold spells) have been shown to cause distributional range shifts (Freeman et al., 2018a; Freeman et al., 2018b; La Sorte et al., 2010; Neate-Clegg et al., 2020; Poloczanska et al., 2011; Wethey and Woodin, 2008), as well as increase mortality events and affect the overall persistence of many species (Cunningham et al., 2021; Jones et al., 2010; Marbà and Duarte, 2010; McKechnie and Dunn, 2019; Pearce‐Higgins et al., 2015). Additionally, land cover changes, such as deforestation and agricultural expansion, further exacerbate these impacts by fragmenting habitats, reducing connectivity, and altering microclimatic conditions (Oliver et al., 2015; Opdam and Wascher, 2004; Travis, 2003; Urban et al., 2024). For example, remnant natural forest fragments are typically surrounded by a matrix of modified land cover with low structural complexity. Such simplified land cover can, among other things, allow more solar radiation to reach the ground, leading to local temperature changes around and within the natural forest patch (Saunders et al., 1991; Hofmeister et al., 2019). In addition, urbanization leads to higher temperatures through the heat island effect, which, in combination with climate change, leads to more frequent and intense heat waves in cities (Urban et al., 2024), threatening species that are unable to adapt quickly to changing conditions.\\

	Accurate assessments of the effects of temperature change on species are therefore critical for effective biodiversity conservation policy. However, current approaches mostly rely on estimating species' environmental tolerances using phenomenological techniques that correlate observed species' distribution patterns with macroclimatic variables, and pay little attention to the actual mechanistic links between organisms and their (thermal) environments (Helmuth et al., 2005; Santini et al., 2021). There is a growing recognition of the need to incorporate mechanisms of species responses to environmental change to improve assessments and achieve better management outcomes (Helmuth et al., 2005; Keith et al., 2008; Urban et al., 2016). In this context, directing attention towards the mechanistic link between a species' physiological responses and its environment (i.e., ecophysiology) has been proposed as a more robust alternative (Fuller et al., 2010; Teal et al., 2018; Tourinho and Vale, 2023). However, the integration of ecophysiology into ecological assessment has been limited, with most studies focusing on ectotherms (Kearney et al., 2009; Tingley et al., 2014). Ectotherms are animals that derive their body heat primarily from the environment (e.g., most invertebrates, fish, amphibians, and reptiles), whereas endotherms are animals that derive their body heat primarily from metabolism (e.g., most birds and mammals). Despite the extensive knowledge of how ectotherms respond to environmental change, our understanding of endothermic responses has lagged because of the more complex relationship between endothermic thermal physiology and environmental temperatures (Carter et al., 2023; Buckley et al., 2012; Kingsolver et al., 2013; Ragland and Kingsolver, 2008). \\

	Among endotherms, birds may be particularly vulnerable to climate change due to their relatively small size and (largely) diurnal activity. For example, Khaliq et al. (2014) showed that about 15\% of bird species currently experience maximum ambient temperatures above their heat tolerance, and this rises to over 35\% under climate change scenarios. Milne et al. (2015) studied 12 bird species in South Africa and identified potential links between climate change and population declines, attributing these declines to the birds' limited tolerance to higher temperatures. Skwarska et al. (2021) found that unpredictable cold spells prevented birds from actively foraging and properly feeding nestlings. In this context, the thermoregulatory mechanism of birds, characterized by significant energy expenditure to maintain a constant body temperature, suggests that climate change and its temperature fluctuations may have profound consequences specific to birds compared to other animals. Khaliq et al. (2014) assessed global variation in thermal tolerances of endotherms to climate change and found that distributions of birds were more strongly governed by their thermal tolerances than mammalian distributions are, and thus that birds' thermal physiology is more directly linked to their ambient climatic conditions than that of mammals. Understanding how birds respond physiologically to temperature change is therefore crucial for predicting future species distributions and population dynamics, and for implementing effective conservation strategies (Chown and Gaston, 2008; Wikelski and Cooke, 2006; Williams et al., 2008). \\

Moreover, changes in global temperature and precipitation regimes are expected to increase the impact of invasive species (Dukes and Mooney, 1999; Walther et al., 2009), for example, by allowing populations of introduced species that are not currently invasive to become invasive (Hellmann et al., 2008; Mainka and Howard, 2010). An invasive species is defined by IPBES as a species introduced outside its natural past or present distribution whose introduction and/or spread threaten biological diversity. Biological invasions have profound impacts on biodiversity and ecosystem functioning and services, resulting in significant economic and health costs (IPBES 2023). For a species to become established and invasive in a new environment, individuals must first overcome environmental barriers, which include both biotic and abiotic components (Crowl et al., 2008). Among the abiotic components, temperature is often considered to be one of the most important in determining invasive distributions due to its effects on many biological processes (Johnston and Bennett, 2008; Pörtner, 2002; Somero, 2005). In this context, the study of invasive species can provide further insight into physiological adaptations to temperature change, as these species often establish invasive populations in comparatively colder climates compared to their native range (Bates and Bertelsmeier, 2021; Liu et al., 2020; 2022). For example, ring-necked parakeets (\textit{Psittacula krameri}) have managed to establish several populations in many European cities, which are considerably colder than their native range (Strubbe et al., 2015). The same is true for several members of the Estrildidae family, which are known to establish numerous invasive bird populations in cold areas (Stiels et al., 2015). However, ecophysiology has only recently been applied to invasion biology, marking it as an emerging field with the potential to provide valuable insights into addressing the challenges of species invasions and devising effective conservation and management strategies (Kelley, 2014; Boardman et al., 2022).\\

\section{Avian thermoregulation}
\subsection{Thermoneutral zone and basal metabolic rate}

\noindent Overall, birds can thermoregulate through behavioral and morphological adjustments (Krishnan et al., 2023; Mortola, 2021; Pavlovic et al., 2018; Ritchison, 2023). Birds can use behaviors such as wing flapping, thermal panting, and sun avoidance to dissipate heat (Ritchison, 2023). Heat loss is also proportional to a bird's effective surface area and is therefore influenced by posture (Pavlovic et al., 2018). Conversely, birds can conserve heat by crouching, fluffing feathers, and huddling for warmth (Krishnan et al., 2023). In regions with cold winter climates, resident birds reduce thermal conductance (i.e., the ease with which heat leaves or enters the body of an organism; McNab, 2012). Thermal conductance is significantly influenced by several factors, including the surface-to-volume ratio of the body (with higher ratios indicating higher conductance), the presence of subcutaneous fat, the amount of blood flow to the skin and subcutaneous tissues, and various plumage characteristics such as feather color, structure, and optical properties (Cooper, 2002; Novoa et al., 1994).\\

	However, thermoregulation in birds is primarily related to metabolic rate, which is defined as the rate at which an organism produces body heat (Price and Dzialowski, 2018). Birds actively maintain a constant body temperature by adjusting their rates of heat production and loss, and this ability is largely a function of an individual's ability to generate, retain, and dissipate metabolic body heat (Fristoe et al., 2015; XuanYuan et al., 2023). For the body temperature of endotherms to remain constant and independent of environmental temperature, the rate of internal heat production must balance the rate of heat loss. Because the rate of heat loss is proportional to the temperature difference between the body temperature and the ambient temperature (ΔT = \( \lvert T_{b} - T_{a} \rvert \)), the rate of metabolism must be proportional to ΔT for body temperature to remain constant (McNab, 2012).\\

As a result of compensatory changes in behavior and morphology, the metabolic rate of a bird remains constant over a range of ambient temperatures. This range of temperatures is known as the "thermoneutral zone" (TNZ) and has been used as an indicator of a species' long-term ability to tolerate thermal variation (Khaliq et al., 2014; Milne et al., 2015). The TNZ (Figure I.1) of a species is characterized by two critical temperatures: the lower critical temperature (LCT) and the upper critical temperature (UCT). Both mark the point at which metabolic rate begins to increase to facilitate thermoregulation. For this reason, thermoregulation in birds is generally non-linear and is commonly assumed to follow the classic Scholander-Irving model (Scholander et al., 1950).\\

	\renewcommand{\thefigure}{I.\arabic{figure}}
	\begin{figure}[h!]
		\begin{center}
			\includegraphics[width=\textwidth]{figures/TNZ_intro.png}
		\end{center}
		\begin{footnotesize}
			\caption{Thermoneutral zone (TNZ) curve illustrating the relationship between temperature and metabolic rate, with basal metabolic rate (BMR) representing the minimum metabolic rate, summit metabolic rate (M$_{\text{sum}}$) representing the maximum metabolic rate achieved during cold exposure, and including lower critical temperature (LCT) and upper critical temperature (UCT) as the two critical temperatures that mark the onset of metabolic rate adjustments for thermoregulatory purposes. Note: Figure adapted in part from Pollock et al. (2019). \label{figI.1}}
		\end{footnotesize}
	\end{figure}

	The metabolic rate measured within the TNZ of a species when it is inactive during an inactive period (at night for most birds) and postabsorptive (i.e., not digesting a meal) is called basal metabolic rate (BMR). BMR is primarily a function of the central organs (Swanson, 2010). It represents the sum of maintenance energy requirements and a large proportion of a bird's total energy expenditure (Ricklefs et al., 1996; Tang et al., 2022). BMR serves as a standardized, baseline parameter. Its long history of empirical measurement and comparative analysis makes BMR an important trait for comparing thermoregulatory performance between different avian taxa (Tieleman et al., 2003a;b; White et al., 2007; Wikelski et al., 2003). \\

\subsection{Outside the thermoneutral zone}

\noindent At temperatures above the TNZ, excessive heat can endanger survival or reduce performance (Khaliq et al., 2014). Maintaining normothermia (a state in which body temperature is regulated within normal limits) under these conditions requires a trade-off between evaporative heat loss to avoid hyperthermia (an increase in body temperature above normothermia) and body water retention to avoid dehydration (Ritchison, 2023). To counteract high temperatures, birds rely primarily on evaporative heat loss. This is a critical thermoregulatory mechanism in environments where ambient temperatures routinely exceed normothermic body temperature, and is achieved through processes such as panting and wing flapping (Ritchison, 2023).\\

	At ambient temperatures below the TNZ, the metabolic rate increases until a limit on ΔT is reached, which usually defines the minimum ambient temperature that the animal can tolerate (i.e., cold tolerance). The maximum resting metabolic rate elicited by cold exposure is called summit metabolic rate (M$_{\text{sum}}$) and commonly interpreted as an indicator of a bird’s ability to endure cold (Swanson and Garland, 2009; Petit et al., 2017). Heat production in resting birds is primarily achieved through shivering (Hohtola, 2004; Ritchison, 2023), but recent research, however, suggests potential contributions from non-shivering thermogenesis to thermogenic capacity (Pani and Bal, 2022). M$_{\text{sum}}$ is then primarily a function of muscle activity during shivering and is typically measured in an atmosphere consisting of 21\% oxygen and 79\% helium (heliox). Because of its high thermal conductivity, heat in a heliox atmosphere leaves an animals' body up to three times as fast as in normal air, allowing maximum heat production to be achieved at subfreezing temperatures and avoiding damage to peripheral body parts such as e.g. toes (Rosenmann and Morrison, 1974; Janský, 1966). Together, M$_{\text{sum}}$ and BMR describe the lower and upper limits of metabolic output in resting endotherms (i.e., metabolic scope; Swanson, 2010), and the ratio of M$_{\text{sum}}$ to BMR (i.e., metabolic expansibility, ME) is often interpreted as the maximum ability of an organism to increase its heat production (Cooper and Swanson, 1994).\\

	When confronted with colder temperatures, at least some avian taxa reduce energy expenditure by entering a state of hypothermia, rather than increasing metabolic rate to compensate for heat loss (McKechnie and Lovegrove 2002). Rest-phase hypothermia, an active lowering of body temperature below the normothermic temperatures, shares similarities with other heterothermic responses such as daily torpor and hibernation. However, it differs in that birds exhibiting rest-phase hypothermia remain responsive to external stimuli (Reinertsen 1996). This allows birds to reduce energy and water expenditure during periods of environmental stress, such as cold temperatures (McKechnie and Lovegrove, 2002).\\

\subsection{Functional link of metabolic rates and the aerobic capacity model for the evolution of endothermy}

\noindent The emergence of endothermy represents a significant evolutionary advantage for birds and mammals, giving them the ability to maintain high levels of activity over a wide range of ambient temperatures. Although energetically costly, this adaptation has proven advantageous, as evidenced by the ecological success of these classes, which are characterized by large geographic ranges and adaptability to diverse habitats (McNab, 2012). However, the evolutionary origins and drivers of endothermy are debated. The prevailing theory is that the primary advantage that led to the evolution of endothermy was the ability to perform increased aerobic activities. Known as the aerobic capacity model (Bennett and Ruben, 1979), this theory suggests that increased activity levels, coupled with increases in energy assimilation capacity and increases in body temperature resulting from metabolic heat production, enhanced the overall fitness of endothermic animals. Hence, natural selection favored the evolution of skeletal muscles capable of sustaining prolonged activity and high metabolism (i.e., M$_{\text{sum}}$), with correlated increases in BMR reflecting the energetic cost of maintaining the physiological machinery required to support high aerobic capacity. The aerobic capacity model therefore posits a positive phenotypic correlation between BMR and M$_{\text{sum}}$. While interspecific studies generally support this correlation in birds (Auer et al., 2017; Rezende et al., 2002), suggesting a functional link, at the intraspecific level, variation in BMR and M$_{\text{sum}}$ suggests potential independent physiological control without a clear functional link. For example, O'Connor (1995) found that BMR was seasonally stable in house finches in Michigan, USA, while M$_{\text{sum}}$ was higher in winter. Similarly, Le Pogam et al. (2020) showed that snow buntings (\textit{Plectrophenax nivalis}) increased their M$_{\text{sum}}$ by about 25\% during cold Canadian winters without a concomitant increase in BMR.\\

If the aerobic capacity model assumption of a mechanistic link between minimum and maximum metabolic performance is valid, then such a link should be demonstrable for both inter- and intraspecific comparisons. However, our understanding is complicated by the limited amount of intraspecific research compared to interspecific research, and by the fact that there is much less variation in metabolic values within species than between species. Interspecific comparisons include birds from a wider range of body sizes and phylogenetic affinities, and therefore have a much wider range of metabolic values than intraspecific comparisons. This greater variation in metabolic values may provide a higher level of resolution for detecting phenotypic correlations between metabolic values (Swanson et al., 2012). In addition, few studies have tested the assumptions of the aerobic capacity model in tropical birds, and the paucity of M$_{\text{sum}}$ measurements in these species makes it difficult to draw definitive conclusions. Assessing correlations between BMR and M$_{\text{sum}}$ between individuals is a first step in a more direct test of the aerobic model (Glazier, 2007). This should be complemented by analyses of genetic variances and covariances between BMR and M$_{\text{sum}}$ to predict how they may respond to selection and how these covariances may affect their evolutionary trajectories, providing a more comprehensive understanding of the validity of the aerobic model (Nespolo et al., 2011; Wone et al., 2009; 2015).\\

\section{Variation in avian ecophysiological responses}

\noindent Phenotypic flexibility, a category of phenotypic plasticity (i.e., the ability of an organism to change its phenotype in response to environmental changes; Suzuki et al., 2020), involves short-term, reversible adjustments in an individual's phenotype (Piersma and Drent, 2003; Piersma and Van Gils, 2011). It serves as an important mechanism for responding to environmental variability, allowing organisms to cope with rapid changes (Bonamour et al., 2019; Li et al., 2024; Piersma and Van Gils, 2011). Ignoring the phenotypic flexibility of thermoregulatory traits could potentially lead to under- or overestimation of the true thermal risks bird species are exposed to. Therefore, understanding how key thermoregulatory traits of birds vary over time and space is paramount for accurately assessing and predicting their responses to environmental change. However, despite extensive studies of thermoregulatory flexibility in temperate birds (Bech and Mariussen, 2022; Milbergue et al., 2018; Nord et al., 2021; Le Pogam et al., 2020; Oswald et al., 2021; Petit et al., 2014; Petit et al., 2017; Swanson, 2010), our understanding of these adjustments at the intraspecific level and in species inhabiting tropical environments remains limited.\\

\subsection{Seasonal variation in metabolic rates: temperate vs tropical}

\noindent Birds in temperate regions experience marked seasonal changes in temperature and they often exhibit seasonal variation in key thermoregulatory traits (Swanson, 2010). The most common winter phenotype in birds wintering in cold climates is characterized by increased cold tolerance through increases in M$_{\text{sum}}$ and BMR, and decreases in thermal conductance. Winter increases in M$_{\text{sum}}$ are associated with changes in skeletal muscle, which is the primary thermogenic organ in birds (Marsh, 1981; Marsh and Dawson, 1989), and this is usually correlated with increased cold tolerance in birds (Swanson and Olmstead, 1999; Swanson and Liknes, 2006). In this case, the increase in BMR may represent a cost of maintaining the metabolic machinery required for increased thermogenic capacity (Barceló et al., 2017; Dutenhoffer and Swanson, 1996; McWilliams and Karasov, 2014; Vézina et al., 2011). In addition, cold-wintering passerines typically exhibit increases in body mass, fat reserves, and plumage density, as well as decreases in body temperature, which effectively reduce thermal conductance and consequently limit their energy requirements (Cooper, 2002; Cooper and Swanson, 1994; González-Medina et al., 2023; Swanson, 1990; Zheng et al., 2008).\\

Compared to temperate regions, the thermoregulatory responses of birds in warmer tropical and (sub)tropical climates are less well understood. Tropical birds are known for their slow pace of life (Wiersma et al., 2007a), with low annual reproductive output, high annual survival rate, slow growth of nestlings, and long post-fledgling dependency on parents (Cheng and Martin, 2012; Williams et al., 2010; Ricklefs, 2000). Most research shows that resident tropical birds experiencing low temperature variability are likely to have less flexible ecophysiological traits to cope with a changing environment, probably due to less strong evolutionary selection for high thermogenic capacity (Wiersma et al., 2007b). For example, Londoño et al. (2015) found no difference in BMR of 253 bird species across a 2.6 km altitudinal gradient in Peru, suggesting that BMR of neotropical birds is independent of ambient temperature. This suggests that many tropical forest bird species that currently experience limited temperature variation may be highly vulnerable to the effects of climate change. However, recent studies have shown that tropical birds may also exhibit phenotypic flexibility. For example, Pollock et al. (2019) examined seasonal variation in thermoregulatory traits in four temperate and 41 tropical birds. Contrary to expectations, their results showed that while tropical species have a narrower TNZ (more sensitive to temperature changes), they also exhibit considerable thermoregulatory flexibility across seasons. There is therefore a gap in our understanding of the precise physiological mechanisms used by tropical birds to cope with changing environments. Furthermore, globally, the majority of invasive bird species are (sub)tropical birds (Reino et al., 2017). As mentioned before, they often successfully colonize climates in their introduced distributional range that differ from their native distributional range (Bates and Bertelsmeier, 2021; Liu et al., 2020; 2022). This highlights the need for direct ecophysiological experiments on tropical birds to accurately understand their thermal tolerances and, ultimately, to explain their success in their invasive distributional range. \\

Research aimed at elucidating these physiological adjustments in both temperate and tropical species has primarily relied on interspecific comparisons, treating a given species as a single physiological entity and assuming comparable responses between conspecific populations (Burton et al., 2011; Konarzewski and Książek, 2013; Reed et al., 2011; Thomas et al., 2004). However, especially for wide-ranging animals, different populations within a species are expected to have different adaptations and thermal tolerances depending on local environmental conditions, making the assumption that a single phenotype is universally suitable for all conditions unrealistic (Cavieres and Sabat, 2008; Furness, 2003; Wikelski et al., 2003; Root, 1988). Palacio and Clark (2023) recently demonstrated that integrating morphological, physiological and genetic variation within a species provide a comprehensive framework for better understanding biodiversity responses to global environmental change. Cabello-Vergel et al. (2022) found differences in thermoregulatory patterns among populations of great tits (\textit{Parus major}), highlighting the importance of considering intraspecific variation in thermoregulatory traits when predicting the status and effects of environmental change on birds. Cruz-Neto and Bozinovic (2004) concluded that by comparing individuals within a species, intraspecific studies can validate predictions from interspecific comparisons and identify additional factors beyond the scope of interspecific research. Therefore, recognizing and incorporating intraspecific variation in physiological studies is crucial to improving our understanding of how organisms respond to environmental challenges.\\

\subsection{Annual variation in energy expenditure}

\noindent The efficient allocation of energy during the annual cycle is crucial for life history, and the adaptation of metabolic rates to varying energy demands can significantly affect survival and fitness (Grunst et al., 2023; Hayes and O'Connor, 1999; Petit et al., 2017; Piersma and Van Gils, 2011). These annual cycles are influenced by a range of ecological processes that vary over time. For example, temperate birds experience seasonal shifts in physiological states and environmental conditions, such as breeding and wintering. To ensure survival and activity, they must adjust their physiology to maintain a balance between energy intake and expenditure (Karasov, 1986). For example, during breeding, temperate birds balance investment in their offspring with investment in their own subsequent survival (Welcker et al., 2015). When parents are unable to maintain a balance between these investments and the energy resources available to them, individuals face challenges in meeting external demands (Wikelski and Cooke, 2006). Moreover, wintering in temperate regions presents unique challenges for small passerines. Because of their inability to store significant internal energy reserves, they must meet their daily energy needs under harsh environmental conditions when foraging opportunities decrease and food becomes scarce (Blem, 2000). In this context, field metabolic rates (FMR), which represent the total energy cost a wild animal pays during the course of a day (Nagy 1978), provides researchers with a means of quantifying energy expenditure and allocation patterns in free-living animals (McKechnie and Swanson, 2010). The method known as doubly labeled water (DLW) is widely used to determine FMR. It is based on the analysis of variations in the rates at which heavy isotopes of hydrogen (such as deuterium ${}^{2}$H) and oxygen (${}^{18}$O) are eliminated from the body, which in turn allows the calculation of carbon dioxide (CO$_{\text{2}}$) production (Lifson and McClintock 1966; Nagy, 1983; Speakman, 1997; Figure I.2).\\

	\renewcommand{\thefigure}{I.\arabic{figure}}
	\begin{figure}[h!]
		\begin{center}
			\includegraphics[width=\textwidth]{figures/FMR_INTRO.png}
		\end{center}
		\begin{footnotesize}
			\caption{Doubly labelled water (DLW) method illustration showcasing the isotope-elimination technique. This method operates on the principle that oxygen labeled on body water is eliminated due to water flow and (CO$_{\text{2}}$) production, while hydrogen labeled on body water is eliminated solely via water flow, enabling the measurement of (CO$_{\text{2}}$) production from the difference in elimination rates over time of the two labels. DLW serves as a method to measure the average daily metabolic rate of an organism over a specified period. Note: Figure adapted in part from Raman and Schoeller, 2005. \label{figI.2}}
		\end{footnotesize}
	\end{figure}

Many hypotheses have emerged to explain how populations adjust their energy allocation and metabolism to environmental variation. One hypothesis (Figure I.3a) is that energy expenditure peaks in winter due to the challenges posed by harsh climates with colder ambient temperatures, forcing animals to invest more in thermoregulation. This hypothesis is mainly supported by birds living at high latitudes, where avian reproductive investment is high and birds may face severe thermoregulatory demands, especially during the winter season (Doherty et al., 2001; Sgueo et al., 2012; Cooper, 2002). Two other important hypotheses, the 'reallocation hypothesis' and the 'increased demand hypothesis' (Masman et al., 1988), have emerged in our understanding of annual energy allocation. The reallocation hypothesis (Figure I.3b) proposes a shift in energy expenditure from winter thermoregulation to reproductive activities, resulting in no net difference in seasonal energy requirements (Bryant and Tatner, 1988; Weathers and Sullivan, 1993). In contrast, the increased demand hypothesis (Figure I.3c) predicts that energy expenditure peaks during breeding and exceeds all other seasons (Gales and Green, 1990; Masman et al., 1988).\\

\renewcommand{\thefigure}{I.\arabic{figure}}
	\begin{figure}[h!]
		\begin{center}
			\includegraphics[width=\textwidth]{figures/Increased demand_reallocation.jpg}
		\end{center}
		\begin{footnotesize}
			\caption{Scheme illustrating the main hypotheses regarding winter and breeding energy allocation. a) proposes that energy expenditure peaks in winter due to the challenges posed by harsh climates, particularly colder temperatures. b) the reallocation hypothesis proposes a shift in energy expenditure from winter thermoregulation to reproductive activities, resulting in no net difference in seasonal energy requirements. c) the increased demand hypothesis predicts that energy expenditure peaks during breeding and exceeds all other seasons. \label{figI.3}}
		\end{footnotesize}
	\end{figure}

The reallocation hypothesis is mainly supported by studies in desert and temperate climates. In deserts, aridity selects for lower parental investment during breeding, as limited food and water availability favors smaller clutch sizes, lower energy and water requirements of parents and offspring, and reduced growth rates (Tieleman et al., 2004). In temperate regions, birds may incur substantial, but not extreme, thermoregulatory costs in winter and raise large broods in spring, consistent with roughly equal seasonal energy expenditure (Mugaas and King, 1981; Weathers and Sullivan, 1993). However, the empirical evidence supporting this hypothesis is ambiguous, possibly due to increased spatial and interannual climate variability, particularly in temperate regions, resulting in geographical and temporal variation in the balance of energy requirements for winter survival and spring breeding. Conversely, the increased demand hypothesis is supported primarily by research on tropical species, consistent with predictions from life history theory (Williams et al., 2010; Wells and Schaeffer, 2012; Jones et al., 2020). The constant thermoregulatory costs in the warm and stable climate of the tropics suggest that reproductive requirements are likely to be the main drivers of seasonal variation in energy expenditure. \\

Taken together, these hypotheses underscore the complexity of energy allocation patterns and the intricate adaptive strategies that populations employ to cope with environmental challenges. However, many of these studies comparing winter and breeding energy allocation have relied predominantly on FMR as the main predictor of daily energy expenditure, without direct assessment of maximal aerobic capacity and maintenance metabolism (Masman et al., 1988; Weathers and Sullivan, 1993; Williams, 2001; Bryant and Tatner, 1988; Gales and Green, 1990), leaving this particular aspect poorly understood. Indeed, birds are expected to respond to such seasonal energetic changes in required 'work' or 'activity' by altering their basal and maximal aerobic capacity accordingly. The ability to sustain flight (e.g., parental care activity) and shivering thermogenesis (e.g., thermoregulation) are both functions of skeletal muscle mass (Guglielmo, 2010). Such investments in muscle mass and concomitant changes in the gut and digestive organs that allow for higher daily food consumption (Nilsson, 2002) underlie correlated changes in maximal aerobic capacity (i.e., M$_{\text{sum}}$) and maintenance metabolism (i.e., BMR) as predicted by the aerobic capacity model for the evolution of endothermy (Bennett and Ruben, 1979). As a result, a positive phenotypic correlation between BMR and M$_{\text{sum}}$ is expected during these demanding periods, supporting the assumption of the aerobic capacity model. In light of these considerations, it becomes clear that focusing solely on FMR may overlook crucial aspects of metabolic adaptation in response to seasonal challenges, highlighting the importance of investigating other metabolic rates. This represents a notable knowledge gap in our understanding of avian energy allocation strategies during different seasons and underscores the need for further research in this area. \\

\section{Objectives and dissertation outline}

\noindent The overall aim of this dissertation is to investigate intraspecific and seasonal variation in the physiological adjustments that small birds employ in response to changing environmental conditions, particularly in the context of thermoregulation. First, I will investigate the poorly understood thermoregulatory capacity of tropical birds by assessing seasonal variation in metabolic rates and whether the assumptions of the aerobic capacity model for the evolution of endothermy hold true. This will be done using three closely related tropical species that have been successfully introduced into Europe (\textbf{Chapter 1}), which serve as valuable models due to the colder and more variable climate they encounter compared to their native range. \textbf{Chapters 2} and \textbf{3} will focus on addressing intraspecific variation in metabolic rates. I will examine intraspecific metabolic responses by first (\textbf{Chapter 2}) using two different populations of a common temperate species experiencing different climates, and second (\textbf{Chapter 3}) by examining the ecophysiology of a native population of a tropical species studied in \textbf{Chapter 1}. Finally, I will combine measurements of maximal aerobic capacity and maintenance metabolism with field metabolic rate measurements to assess seasonal variations in metabolic rates of a temperate bird species (\textbf{Chapter 4}), providing more detail on the annual energy allocation of birds and allowing assessment of whether this allocation is consistent with the assumptions of the aerobic capacity model for the evolution of endothermy. \\

Specifically, in \textbf{chapter 1}, I investigate the magnitude and direction of seasonal variation in thermoregulatory traits in three closely related Afrotropical passerines introduced to Europe (common waxbill \textit{Estrilda astrild}, orange-cheeked waxbill \textit{E. melpoda}, and black-rumped waxbill \textit{E. troglodytes}). Because of their popularity as caged pet birds, these small birds have established several invasive populations, including in regions where the climate is colder than in their native range. I kept a captive-bred population of each species in a ‘common garden’ aviary in a temperate area in Belgium to test the prediction that phenotypic flexibility allows these species to cope with the colder and more variable Belgian climate compared to their native distribution range.\\

In \textbf{chapter 2}, I investigate intraspecific geographic variation in ecophysiological traits, using the great tit (\textit{Parus major}) as a case study. The great tit is a valuable species for this purpose, breeding from approximately 10°S to 71°N and remaining resident even at the northernmost limit of its breeding range. The chapter focuses on two populations, one living in a maritime temperate climate characterized by mild winters (Belgium, Gontrode, Melle) and the other living in a continental climate characterized by long and cold winters (Russia, Zvenigorod Biological Station, Moscow Oblast), to test the prediction that individuals from the cold, continental population will be characterized by higher maximal thermogenic capacity (i.e., M$_{\text{sum}}$) but lower maintenance costs (i.e., BMR) compared to those from the maritime temperate population.\\

In \textbf{chapter 3}, I investigate the ecophysiological characteristics of a wild-caught population of the common waxbill in part of its native range (South Africa) by assessing how this species adjusts its metabolic rate over a range of temperatures to identify its TNZ as an indicator of a species' thermal tolerance. In this way, I aim to assess the intraspecific variation in thermoregulatory traits between a captive population (Chapter 1) and wild-caught individuals.\\

In \textbf{chapter 4}, I examine the metabolic signatures of energy requirements in great tits. Together with the common proxy for energy expenditure (FMR), I measure BMR and M$_{\text{sum}}$ to understand how birds allocate energy annually to cope with changes in environmental conditions (e.g., winter) and physiological states (e.g., breeding). I also assess whether metabolic rates during these challenging periods are consistent with the predictions of the aerobic capacity model for the evolution of endothermy.\\

In the \textbf{general discussion}, I synthesize and link the results of the research described in the previous chapters.











%%%%%%%%%%%%%%%%%%%%%%%%%%%%%%%%%%%%%%
%%%%%%%%%%%%%%%%%%%%%%%%%%%%%%%%%%%%%%
%%%%%%%%%%%%%%%%%%%%%%%%%%%%%%%%%%%%%%
%%%%%%%%%%%%%%%%%%%%%%%%%%%%%%%%%%%%%%
%%%%%%%%%%%%%%%%%%%%%%%%%%%%%%%%%%%%%%
%%%%%%%%%%%%%%%%%%%%%%%%%%%%%%%%%%%%%%
%%%%%%%%%%%%%%%%%%%%%%%%%%%%%%%%%%%%%%
%%%%%%%%%%%%%%%%%%%%%%%%%%%%%%%%%%%%%%
%%%%%%%%%%%%%%%%%%%%%%%%%%%%%%%%%%%%%%
%%%%%%%%%%%%%%%%%%%%%%%%%%%%%%%%%%%%%%
%%%%%%%%%%%%%%%%%%%%%%%%%%%%%%%%%%%%%%
%%%%%%%%%%%%%%%%%%%%%%%%%%%%%%%%%%%%%%
%%%%%%%%%%%%%%%%%%%%%%%%%%%%%%%%%%%%%%
%%%%%%%%%%%%%%%%%%%%%%%%%%%%%%%%%%%%%%
%%%%%%%%%%%%%%%%%%%%%%%%%%%%%%%%%%%%%%
%%%%%%%%%%%%%%%%%%%%%%%%%%%%%%%%%%%%%%
%%%%%%%%%%%%%%%%%%%%%%%%%%%%%%%%%%%%%%
%%%%%%%%%%%%%%%%%%%%%%%%%%%%%%%%%%%%%%

%%%%%%%%%%%%%%%%%%%%%%%%%%%%%%%% CHAPTER TWO  %%%%%%%%%%%%%%%%%%%%%%%%%%%%%%%%%%%

\setlength{\thumbwidth}{0.8cm}
\setlength{\thumbheight}{1cm}
\tikzset{
	thumb/.style={
		%   draw=black,
		fill=light-gray,
		text=black,
		minimum height=\thumbheight, %\thumbheight,
		text width=\thumbwidth,
		outer sep=0pt,%   outer sep=10pt,
		font=\sffamily\Large,
	}
}
\pagestyle{mainmatter}
\renewcommand\thesection{\arabic{chapter}.\arabic{section}}
\renewcommand{\thefigure}{\arabic{chapter}.\arabic{figure}}
\chapter{Seasonal variation in thermoregulatory capacity of three closely related Afrotropical Estrildid finches introduced to Europe}\label{chapter1}
\chaptermark{Chapter 1}
\lettergroup{\thechapter}	

\begin{flushright}
        \textcolor{black}{Cesare Pacioni}\\
\textcolor{black}{Marina Sentís}\\
    \textcolor{black}{Anvar Kerimov}\\
    \textcolor{black}{Andrey Bushuev}\\
    \textcolor{black}{Luc Lens}\\
    \textcolor{black}{Diederik Strubbe}


\vspace*{2cm}
  \textcolor{black}{Adapted from: Pacioni et al. (2023). \textit{J. Therm. Biol.}, 113, 103534.} 

\end{flushright}
\clearpage

\section{Abstract}
A species' potential geographical range is largely determined by how the species responds physiologically to its changing environment. It is therefore crucial to study the physiological mechanisms that species use to maintain their homeothermy in order to address biodiversity conservation challenges, such as the success of invasions of introduced species. The common waxbill \textit{Estrilda astrild}, the orange-cheeked waxbill \textit{E. melpoda}, and the black-rumped waxbill \textit{E. troglodytes} are small Afrotropical passerines that have established invasive populations in regions where the climate is colder than in their native ranges. As a result, they are highly suitable species for studying potential mechanisms for coping with a colder and more variable climate. Here, we investigated the magnitude and direction of seasonal variation in their thermoregulatory traits, such as basal (BMR), summit (M$_{\text{sum}}$) metabolic rates and thermal conductance. We found that, from summer to autumn, their ability to resist colder temperatures increased. This was not related to larger body masses or higher BMR and M$_{\text{sum}}$, but instead, species downregulated BMR and M$_{\text{sum}}$ toward the colder season, suggesting energy conservation mechanisms to increase winter survival. BMR and M$_{\text{sum}}$ were most strongly correlated with temperature variation in the week preceding the measurements. Common waxbill and black-rumped waxbill, whose native ranges encompass the highest degree of seasonality, showed the most flexibility in metabolic rates (i.e., stronger downregulation toward colder seasons). This ability to adjust thermoregulatory traits, combined with increased cold tolerance, may facilitate their establishment in areas characterized by colder winters and less predictable climates.



\vspace*{\fill}
\noindent \textbf{Keywords:} Invasive species; Waxbills; Climate variability hypothesis; Basal metabolic rate; Summit metabolic rate; Thermal conductance
	
\clearpage

\section{Introduction}


Understanding and forecasting how species respond to changing environmental conditions is important for tackling global change challenges, such as climate change and biological invasions. Globally, one in six species may be at risk of extinction because of climate change (Urban, 2015), while at the same time invasions by introduced non-native species have become one of the main drivers' of biodiversity loss worldwide (Bongaarts, 2019). Policy responses and management strategies for mitigating these threats are, however, complicated by substantial uncertainties inherent to future scenarios of biodiversity redistribution under global change (Thuiller et al., 2019). For example, widely-used phenomenological models that correlate contemporary species' occurrence data with spatial environmental predictors to characterize species' climatic tolerances frequently overestimate or underestimate the degree to which species' distribution ranges are likely to change over time (Santini et al., 2021). Moreover, forecasts of areas at risk of invasion are further complicated by the fact that invasive species often colonize climates in their introduced range that are different from their native range (Bates and Bertelsmeier, 2021; Liu et al., 2020, 2022). Ultimately, species' potential geographical ranges are determined by their physiological response to their environment (Bozinovic et al., 2011; Bozinovic and Naya, 2014), and understanding the mechanistic links between climate and distribution is increasingly recognized as a prerequisite for predicting future species distributions and implementing successful conservation strategies (Pilowsky et al., 2022). The uptake of ecophysiology into ecological forecasting has, however, remained limited, at least partly because of real and perceived concerns regarding data availability and uncertainties over the exact physiological mechanisms animals can employ to cope with changing environments (Boult and Evans, 2021; Buckley et al., 2018).\\

Animals can respond to changing environmental conditions by displaying a set of behavioral, physiological, and morphological changes that allow them to keep their body temperature within acceptable limits (Bush et al., 2008; McKechnie, 2008). This capacity for homeothermy is largely a function of an individual's ability to generate, retain, and dissipate metabolic body heat (Fristoe et al., 2015). Heat generation and retention capacity are governed by an animal's basal metabolic rate (BMR), summit metabolic rate (M$_{\text{sum}}$), and thermal conductance. BMR reflects the minimum rate of energy expenditure (measured within the thermoneutral zone of a species) that a resting, post-absorptive individual requires to maintain normothermic body temperatures (McNab, 2012). M$_{\text{sum}}$ is defined as the maximum aerobic resting metabolic rate elicited by cold exposure in endotherms (McKechnie and Swanson, 2010). These two parameters, which have been extensively studied in birds (McNab, 2009; 2012; Swanson and Garland, 2009; Wiersma et al., 2007b), describe the lower and upper limits of metabolic output in resting endotherms (Swanson, 2010). The ratio of M$_{\text{sum}}$ over BMR (referred to as metabolic expansibility, ME) is often interpreted as the maximal ability of an organism to increase its heat production (Cooper and Swanson, 1994). Thermal conductance, which is the ease with which heat leaves or enters an organism's body (McNab, 2012), depends strongly on body surface/body volume ratios (higher ratios imply higher conductance), the quantity of subcutaneous fat, the amount of blood circulation in the skin and subcutaneous tissues, and for birds also on plumage characteristics (e.g., feather color, structure, and optical properties) (Cooper, 2002; Novoa et al., 1994). For example, many cold-temperate wintering passerines exhibit increased body mass, fat stores, plumage density, and lower body temperatures, reducing their thermal conductivity and thus limiting their energy demand (Cooper, 2002; Cooper and Swanson, 1994; Swanson, 1990; Zheng et al., 2008), and/or show an upregulated (summit) metabolism, in order to be able to endure cold spells (Petit et al., 2017; Swanson, 2001).\\

The aerobic capacity model for the evolution of endothermy (Bennett and Ruben, 1979) postulates that natural selection favors individuals with a high aerobic capacity (i.e., M$_{\text{sum}}$), whereby correlated increases in BMR reflect the energetic costs required to sustain the physiological machinery necessary to support a high aerobic capacity. This positive phenotypic correlation can, for example, emerge when individuals invest in more flight muscle mass for increased shivering thermogenesis, whereby maintenance costs associated with higher muscle mass raise the BMR (Mckechnie and Swanson, 2010). Alternatively, M$_{\text{sum}}$ and BMR may covary because muscle mass investment is accompanied by increases in the gut and digestive organs to process enough food to power muscle thermogenesis (Chappell et al., 2011). Empirical support for the aerobic capacity model is, however, mixed. For example, whereas two studies on temperate-zone birds (Dutenhoffer and Swanson, 1996 (n = 10 species); Rezende et al., 2002 (n = 42)) found a strong and positive correlation between BMR and M$_{\text{sum}}$, Le Pogam et al. (2020) found that during harsh Canadian winters, snow buntings (\textit{Plectrophenax nivalis}) were able to upregulate their M$_{\text{sum}}$ with more than 25\% without an obvious concurrent increase in maintenance costs (i.e., BMR), pointing to alternative non-shivering thermogenesis mechanisms underlying M$_{\text{sum}}$. Additionally, Nord et al. (2021) found that during Scottish winters, resident tit species upregulate red blood cell mitochondrial volume and respiration rate. This, however, did not result in a higher production of ATP-molecules, but rather in an uncoupling of electron transport from ATP-production, an exothermic reaction that resulted in an increase in heat-producing capacity at the subcellular level. Currently, little is known about how these relationships and processes operate in non-temperate birds.\\

Two main hypotheses for explaining variation in thermoregulatory traits are the ‘cold tolerance hypothesis’ (Liknes and Swanson, 1996) which predicts higher metabolic rates in birds from colder environments, and the ‘climate variability hypothesis’ (Sun et al., 2022) which predicts higher flexibility in metabolic traits in more variable climates. Both hypotheses have been invoked to explain why temperate species, in comparison to tropical ones, show considerable variation in thermoregulatory traits (Pollock et al., 2019), enabling them to tolerate a wide range of temperatures year round (Bozinovic et al., 2011; Swanson, 2010). For example, increasing BMR and M$_{\text{sum}}$ (Arens and Cooper, 2005; Liknes et al., 2002; McKechnie et al., 2015) and/or decreasing thermal conductance (Novoa et al., 1994) in winter, are considered changes that indicate a greater thermoregulatory capacity and an improved cold tolerance (Swanson, 1990). The climate variability hypothesis also corroborates the often lower metabolic rates and narrower thermal tolerances observed for tropical species compared to temperate species (Khaliq et al., 2014; Pollock et al., 2019). Within temperate birds, variation in metabolic rates is generally associated with their winter distribution limits (Bozinovic and Naya, 2014; Canterbury, 2002), but the extent to which non-temperate species respond to the climate and its variation across their range remains unclear. Londoño et al. (2015), for example, found no difference in BMR of 253 bird species across a 2.6 km altitudinal gradient in Peru, suggesting that BMR in Neotropical birds is independent of environmental temperature. In contrast, over a period of nine years, Bushuev et al. (2021) showed substantial variation in BMR among 134 bird species captured in Vietnam, which was strongly related to variation in ambient temperature in the week preceding capture and measurement.\\

In this study, we investigate the magnitude and direction of seasonal variation in thermoregulatory traits in three closely related Afrotropical passerine birds (common waxbill \textit{Estrilda astrild}, orange-cheeked waxbill \textit{E. melpoda}, and black-rumped waxbill \textit{E. troglodytes}) introduced to Europe. Because of their popularity as pet cage birds (Reino et al., 2017), these small (body mass range $\sim$6–11 g) granivorous birds typical of grassy habitats, marshes, and swamps (Del Hoyo et al., 1992) have established multiple invasive populations, including in regions where the climate is colder than in their native range (e.g., parts of south-western Europe; Keller et al., 2020). This makes them highly appropriate model species for analyzing potential mechanisms underlying the observed variation in thermoregulatory strategies. From March to October 2021, we kept a captive-bred population of each species in a ‘common garden’ aviary in temperate-area Belgium to test the prediction that phenotypic plasticity will allow these species to cope with the colder and more variable climate in Belgium compared to their native range. First, we predict that from summer to autumn, waxbills will upregulate their BMR and M$_{\text{sum}}$, increase their body mass, and decrease their thermal conductance. Second, we expect to find a positive relationship between BMR and M$_{\text{sum}}$, as well as a positive relationship between M$_{\text{sum}}$ and cold tolerance. Lastly, we predict that seasonal changes in thermoregulatory traits are more strongly related to the degree of variation in temperature than to differences in mean temperature, and that common and black-rumped waxbills show more phenotypic variation than orange-cheeked waxbills, given that both former species have more climatically variable native ranges (see below).\\

\section{Material and methods}
\renewcommand\thesection{\arabic{chapter}.\arabic{section}}
\renewcommand{\thefigure}{\arabic{chapter}.\arabic{figure}}


\subsection{Study species}

Our three study species belong to the genus Estrilda (family Estrildidae), a group of small, largely granivorous passerines native to Africa, Asia, and Australasia that are commonly traded as pet birds (Cardoso and Reino, 2018; Reino et al., 2017). These three species are similarly sized, with the common waxbill (CW) as the largest bird (7–11 g), followed by the orange-cheeked (OC, 7–9 g) and the black-rumped waxbill (BR, 7–8 g) (Table 1.1). CW has the most extensive natural distribution, occurring over most of sub-Saharan Africa. BR is distributed across much of the Sahel, ranging from Senegal to western Ethiopia, while OC is restricted to parts of western and central Africa (Del Hoyo et al., 1992). These native distributions correspond to a varying degree of native climate variation, whereby OC inhabits the most seasonally stable environments (as measured by temperature seasonality), followed by BR and CW (Figure 1.1).

\clearpage


\begin{sidewaystable}[!ht]
\begin{center}
\begin{footnotesize}
\footnotesize
\caption*{\textbf{Table 1.1}: Mean, standard deviation (SD), minimum (Min), and maximum (Max) values for body mass, basal metabolic rates (BMR), summit metabolic rate (M$_{\text{sum}}$), metabolic expansibility (ME), thermal conductance (for BMR and M$_{\text{sum}}$), and T$_{\text{a}}$ (temperature at which hypothermia was reached for the (M$_{\text{sum}}$) per species (CW: common waxbill; OC: orange-cheeked waxbill; BR: black-rumped waxbill) and per season.}
\centering
    \begin{tabular}{cccccc}
    \hline
        \textbf{} & \textbf{} & \multicolumn{2}{c}{\textbf{Summer}} & \multicolumn{2}{c}{\textbf{Autumn}} \\ \hline
        ~ & ~ & \multicolumn{1}{c}{Mean ± SD} & \multicolumn{1}{c|}{Min - Max} & \multicolumn{1}{c}{Mean ± SD} & \multicolumn{1}{c}{Min - Max} \\ \hline
        \multirow{3}{*}{Body mass (g)} & CW & 8.6 ± 1.3 & 6.6–10.5 & 8.3 ± 1.1 & 6.6–10.5 \\ \cline{2-6}
        ~ & OC & 7.8 ± 0.5 & 7.2–8.8 & 7.6 ± 0.4 & 7.2–8.4 \\ \cline{2-6}
        ~ & BR & 7.0 ± 0.6 & 6.4–8.2 & 7.0 ± 0.5 & 6.5–8.2 \\ \hline
        \multirow{3}{*}{BMR (ml O$_{\text{2}}$/min)} & CW & 0.64 ± 0.17 & 0.39–0.94 & 0.52 ± 0.08 & 0.45–0.76 \\ \cline{2-6}
        ~ & OC & 0.52 ± 0.05 & 0.45–0.61 & 0.54 ± 0.07 & 0.41–0.61 \\ \cline{2-6}
        ~ & BR & 0.56 ± 0.08 & 0.40–0.65 & 0.48 ± 0.06 & 0.37–0.55 \\ \hline
        \multirow{3}{*}{M$_{\text{sum}}$ (ml O$_{\text{2}}$/min)} & CW & 2.67 ± 0.52 & 1.94–3.85 & 2.46 ± 0.49 & 1.75–3.47 \\ \cline{2-6}
        ~ & OC & 2.50 ± 0.44 & 1.72–3.04 & 2.12 ± 0.20 & 1.87–2.52 \\ \cline{2-6}
        ~ & BR & 2.54 ± 0.59 & 1.60–3.35 & 2.02 ± 0.29 & 1.67–2.54 \\ \hline
        \multirow{3}{*}{ME [M$_{\text{sum}}$/BMR]} & CW & 4.59 ± 2.04 & 2.86–9.52 & 4.76 ± 0.74 & 3.70–6.22 \\ \cline{2-6}
        ~ & OC & 4.83 ± 0.96 & 3.07–6.12 & 4.00 ± 0.46 & 3.48–4.96 \\ \cline{2-6}
        ~ & BR & 4.66 ± 1.60 & 3.00–8.46 & 4.23 ± 0.79 & 3.51–5.87 \\ \hline
        \multirow{3}{*}{Thermal conductance for BMR (ml O$_{\text{2}}$/h*°C)} & CW & 2.83 ± 0.74 & 1.67–4.36 & 2.30 ± 0.38 & 1.02–3.35 \\ \cline{2-6}
        ~ & OC & 2.34 ± 0.30 & 2.04–2.80 & 2.36 ± 0.32 & 1.80–2.69 \\ \cline{2-6}
        ~ & BR & 2.46 ± 0.37 & 1.70–3.02 & 2.14 ± 0.31 & 1.62–2.45 \\ \hline
        \multirow{3}{*}{Thermal conductance for M$_{\text{sum}}$ (ml O$_{\text{2}}$/h*°C)} & CW & 3.68 ± 0.85 & 2.99–6.01 & 3.40 ± 0.57 & 2.53–4.67 \\ \cline{2-6}
        ~ & OC & 3.80 ± 0.62 & 2.80–4.64 & 2.93 ± 0.25 & 2.58–3.31 \\ \cline{2-6}
        ~ & BR & 3.70 ± 0.90 & 2.39–5.18 & 2.82 ± 0.34 & 2.37–3.39 \\ \hline
        \multirow{3}{*}{T$_{\text{a}}$ (°C)} & CW & -2.3 ± 2.6 & -5.6–4.8 & -2.6 ± 1.3 & -4.6–0.8 \\ \cline{2-6}
        ~ & OC & 2.0 ± 2.5 & -2.3–5.1 & -2.0 ± 1.4 & -4.0–0.5 \\ \cline{2-6}
        ~ & BR & 1.2 ± 2.6 & -2.1–4.6 & -3.0 ± 1.6 & -5.1–0.8 \\ \hline
    \end{tabular}
\end{footnotesize}
		\end{center}
\end{sidewaystable}
\clearpage

\renewcommand{\thefigure}{1.\arabic{figure}}
	\begin{figure}[h!]
\small
		\begin{center}
			\includegraphics[width=0.4\linewidth]{figures/invasive-native.png}
		\end{center}
		\begin{footnotesize}
			\caption{Density plots visualizing how native and invasive range occurrences of our three study species (orange-cheeked, common waxbill, and black-rumped waxbill) are distributed across a gradient of thermal variability (measured as temperature seasonality, x-axis). Temperature seasonality is defined as the standard deviation of the mean monthly temperature, multiplied by 100. The black line in the invasive range graph represents the aviary site WorldClim temperature seasonality data. \label{fig1.1}}
		\end{footnotesize}
	\end{figure}
\clearpage

As a corollary to the pet trade, CW, OC, and BR have been introduced to numerous localities worldwide and have established invasive populations mainly in other tropical regions, such as Brazil and the Hawaiian Islands, but also in areas colder than their native range, such as Portugal and Spain (Downs and Hart, 2020; Dyer et al., 2017). These three waxbill species were chosen for this study because of (i) the fact that a European climate likely represents a considerable thermoregulatory challenge for these small-bodied species of (sub)tropical origin - indeed Keller et al. (2020) report that in Spain, waxbill populations are sensitive to harsh winters; (ii) their generally similar ecology but different climate variation across their native ranges; (iii) the fact all three species established (partly overlapping) invasive populations in Iberia, and (iv) their ample availability for purchase from (pet) bird breeders.\\

All birds (CW n = 19, OC n = 22 and BR n = 12) used in this study were obtained from a commercial supplier and kept in an outdoor aviary planted with bamboo from late March 2021 to late October 2021. Birds were sheltered from direct rain and wind, and had access to an indoor resting chamber where the temperature was kept above 10 °C at all times (though birds rarely made use of it). Food and water were provided ad libitum, which may potentially affect the energetic metabolism of our study birds compared to wild, free-ranging conspecifics (see Discussion; McKechnie et al., 2006). Ambient air temperature (T\textsubscript{a}) was monitored in both the outdoor and indoor aviary parts using temperature loggers (Inkbird IBS-TH1). All birds were ringed for individual identification with three colored plastic rings. The birds were taken to the lab the day before the metabolic measurements and returned to the aviary afterwards.\\

\subsection{Ecophysiological trait measurements}

	BMR was estimated at night by flow-through respirometry (Lighton, 2018) in July (summer) and October (autumn) by measuring O$_{\text{2}}$ consumption rates (VO$_{\text{2}}$; ml/min). Prior to these measurements, individuals were weighed to the nearest 0.1 g and placed in airtight plastic chambers with a volume of 1.1 l. All birds were food-deprived for 2 h before respirometry. All chambers were maintained inside a custom-made darkened climate control unit (Combisteel R600) set at T\textsubscript{a} of 28 °C, which is within their thermoneutral zone (Supplementary Text 1; Figure 1.5). Ambient air was supplied by two pumps and divided into eight separate streams that were directed to a mass-flow meter (FB-8, Sable Systems) with needle valves adjusted to provide a flow of 650 ml/min. Then, the airstreams were directed to the eight metabolic chambers (seven for each bird and one empty reference chamber as a baseline). The excurrent airstreams were connected to a multiplexer (RM-8, Sable Systems) which allowed one chamber airstream to be sampled independently from the others. Excurrent air from the bird and the baseline channels were alternately subsampled and pulled through a Field Metabolic System (FMS-3, Sable Systems). Birds were measured alternately in cycles, together with several baselines. The time of measurement for each bird within a cycle, and the length of each cycle, depended on the number of birds within a session (usually around 30 min per bird with three cycles during the night). Measurements lasted, on average, $\sim$9 h. After the respirometry measurement, the birds were weighed again to the nearest 0.1 g and were fed with water and food ad libitum before being returned to the aviaries. For the BMR measurements, the FMS-3 was calibrated regularly. Specifically, prior to each experiment, we spanned the O$_{\text{2}}$ sensor at 20.94\% by using the fixed-span mode with ambient air flowing through a Drierite® column (Lighton, 2018). We set the CO$_{\text{2}}$ and the water-vapor sensor to zero with pure nitrogen (N$_{\text{2}}$) every 7–14 days. We spanned the CO$_{\text{2}}$ sensor by using a reference gas with a known CO$_{\text{2}}$ content (1\%) every 7–14 days.\\

M$_{\text{sum}}$ was measured individually during daytime in summer (July) and autumn (October) as the maximum cold-induced VO$_{\text{2}}$ in a heliox atmosphere (a mixture of 21\% O$_{\text{2}}$ and 79\% He; Rosenmann and Morrison, 1974), using the sliding cold exposure technique (Swanson et al., 1996). Prior to measurements, each bird was weighed to the nearest 0.1 g and placed in a 0.9 l metal chamber. The chamber (with the bird) was then placed in the climate control unit, which was set at a T$_{\text{a}}$ of 20 °C and supplied with flowing heliox gas several minutes before starting the trial to allow the bird to acclimatize (Noakes et al., 2017). Heliox was pumped using a flow rate of $\sim$510 ml/min set on the FB-8. Incurrent heliox was split into baseline and experimental channels. Excurrent gas stream was passed through a column of Drierite® before the FMS-3 to remove water vapor. All M$_{\text{sum}}$ trials started with a 7 min baseline heliox measurement. This period allowed for the air in the metabolic chambers to be completely replaced by heliox before recording any data from the chamber. After the baseline, the setup switched to the experimental channel, and the climate control unit was set at -10 °C, causing the temperature in the metabolic chamber to decrease at a rate of $\sim$0.8 °C per min. The body temperature of the bird was measured before, during, and after the trial with a T-type 36 gauge thermocouple (5SC-TT-TI-36-2 M, Omega) attached to the skin over the pectoral muscles. M$_{\text{sum}}$ measurements were stopped whenever individuals became hypothermic, which was inferred from a drop in body temperature (for those individuals where the thermocouple remained attached) and/or from a steady decline in VO$_{\text{2}}$ during several min (Swanson et al., 1996). After removal from the chamber, baseline values were again recorded for at least 5 min. Each bird was weighed to the nearest 0.1 g both before and after metabolic tests and was placed in a cage located in a warm room with water and food ad libitum when the test was finished, before being returned to the aviaries. For the M$_{\text{sum}}$ measurements, the FMS-3 was calibrated every day by spanning the O$_{\text{2}}$ sensor at the concentration of O$_{\text{2}}$ in the heliox cylinder.\\

\subsection{Respirometry and thermal conductance data analysis}

The software ExpeData (Sable Systems) was used to record trials and extract BMR (ml O$_{\text{2}}$/min) and M$_{\text{sum}}$ (ml O$_{\text{2}}$/min) using equation 9.7 from Lighton (2018). The lowest stable part of the curve (average of 11 min, following Bushuev et al. (2018)) was selected to estimate BMR over the entire night. M$_{\text{sum}}$ was considered as the highest 5-min average VO$_{\text{2}}$ over the test period, and data were corrected for drift in O$_{\text{2}}$, CO$_{\text{2}}$, and H$_{\text{2}}$O baselines using the Expedata Data Drift Correction function. Thermal conductance, expressed in ml O$_{\text{2}}$/h*°C, was estimated by using equation C = M/(Tb - Ta) (McNab, 2012, p. 20), where M is the VO$_{\text{2}}$ (either from BMR or M$_{\text{sum}}$), Tb is the body temperature and T$_{\text{a}}$ is the ambient temperature (28 °C for the BMR and the Ta at which hypothermia was reached for the M$_{\text{sum}}$). Because most individuals removed or lost the thermocouple during the trials, only a small subsample of birds during the summer (n = 9 CW, n = 12 OC, and n = 8 BR) had their body temperatures recorded. For individuals whose body temperature was unknown, species-level average body temperatures were used to calculate thermal conductance.\\

\subsection{Native range climate variability}

To quantify the thermal variability across the native range of each of our three study species, we extracted available occurrence data for each species from the Global Biodiversity Information Facility (GBIF) repository and assessed whether there is a difference in ambient temperature seasonality (the ‘bio4’ variable of the WorldClim v2 dataset; Fick and Hijmans, 2017). We retained occurrences that were within the species' natural distribution range, as given by BirdLife's extent of occurrence digital maps. We first buffered the original range maps with a distance of 0.5° to reduce potential errors associated with georeferencing and digitalization procedures and excluded those areas occupied only during the non-breeding season or during migration (Strubbe et al., 2013).\\

\subsection{Statistical data analyses}

Phenotypic variation in thermoregulatory capacity is mostly assessed using (generalized) linear models ((G)LM, e.g., Chamane and Downs, 2009; Petit et al., 2014; Swanson et al., 2014; Swanson et al., 2020) but sometimes also using the repeatability of measurements (e.g., McKechnie et al., 2007; Swanson and King, 2013). Here, we applied (G)LM for data analysis. Mass-independent metabolic rates were assessed by adding body mass as a covariate in the models. We ran a set of linear mixed models with BMR, M$_{\text{sum}}$, ME, thermal conductance, and cold tolerance (heliox temperature at which M$_{\text{sum}}$ was reached) as dependent variables. In all models, we included species and its two-way interactions with other fixed effects while specifying individual bird ID as a random effect (‘lme4’ R-package; Bates et al., 2015). For assessing to what extent changes in thermal conductance are driven by changes in metabolic rates and body mass (Nogueira de Sá and Chaui-Berlinck, 2022), a series of linear models with species, Δ(summer - autumn) thermal conductance, Δ BMR/M$_{\text{sum}}$, and Δ body mass (plus interactions) as dependent variables was run, using the ‘domir’ package to determine relative variable importance. When testing for relationships between BMR and M$_{\text{sum}}$, we first calculated the residuals from the regression of (log) BMR and (log) M$_{\text{sum}}$ on (log) body mass, and then used the residual values of M$_{\text{sum}}$ as the dependent variable and the residual value of BMR as the explanatory variable (Downs et al., 2013), to avoid collinearity issues. We used a backward stepwise procedure to eliminate non-significant interactions and variables. Post-hoc comparisons between species and seasons were performed with the emmeans function in the ‘emmeans’ package (Lenth et al., 2019). We used interquartile ranges as a criterion to identify outliers by using the ‘quantile’ function. Three outliers were present in the data (two for the CW regarding M$_{\text{sum}}$ measurements and one for the BR regarding a BMR measurement). Then, we use the ‘subset’ function to eliminate outliers. After removing them, the analysis was repeated. The excluded outliers did not change the results or the interpretation. For all models, the normality of residuals was tested and verified (i.e., Shapiro–Wilk W > 0.9), and the significance level was set at $p \leq 0.05$.\\

To test the hypothesis that seasonal changes in thermoregulatory traits are more strongly related to temperature variation than to changes in mean temperature, following Stager et al. (2016) and Bushuev et al. (2021), we used the AIC criterion to rank a set of competing linear mixed models wherein we tested how well body mass, BMR and M$_{\text{sum}}$ were explained by a set of eight temperature variables (mean temperature and its standard error (SE), absolute maximum and minimum temperatures, mean maximum and minimum temperatures, absolute temperature range and average daily temperature difference). These variables were calculated based on outdoor aviary temperatures in the week before the metabolic measurements (following Bushuev et al., 2021). The best fitting model was selected based on AIC values. Analyses were conducted jointly for all species (including a bird ID nested within species as a random effect) and for each species separately (including bird ID as a random effect). Body mass, BMR, and M$_{\text{sum}}$ were log transformed before all analyses. Details of the statistical analyses are available in the online version of the paper.\\

In the main analysis, we only included individuals that were measured twice (once in summer and once in autumn: CW n = 14, OC n = 8 and BR n = 9). However, Supplementary file 1 (RMarkdown HTML) presents the same analyses conducted on all birds (n = 19 CW in summer and n = 17 CW in autumn, n = 22 OC in summer and n = 18 OC in autumn, n = 12 BR in summer and n = 12 BR in autumn), showing qualitatively similar results.\\

\section{Results}
\renewcommand\thesection{\arabic{chapter}.\arabic{section}}
\renewcommand{\thefigure}{\arabic{chapter}.\arabic{figure}}

For all three species, (log) body mass was positively correlated with (log) BMR and (log) M$_{\text{sum}}$, with all p < 0.0005 (Figure 1.3 and 1.4). Average body temperature (only measured for summer birds) for CW was 41.6 ± 0.7 °C (n = 9), for OC was 41.6 ± 0.7 °C (n = 12) and for BR was 41.5 ± 0.8 °C (n = 8).

Waxbills had a lower thermal conductance (Table 1.1) in autumn compared to summer (thermal conductance as VO$_{\text{2}}$ [BMR]: p < 0.05, as VO$_{\text{2}}$ [M$_{\text{sum}}$]: p < 0.05). BMR and M$_{\text{sum}}$ explained $\sim$25\% and $\sim$30\% of the seasonal change in thermal conductance. However, waxbill body mass (Table 1.1) did not differ between seasons (all p > 0.1), and seasonal changes in metabolic rates varied by species in a way that largely contradicted the prediction that basal and summit metabolic rates would be upregulated from summer to autumn. More specifically, BMR was significantly higher in summer than in autumn for CW (p = 0.0027) and BR (p = 0.0135), but not for OC (p = 0.4489), while M$_{\text{sum}}$ was significantly higher in summer than in autumn for BR (p = 0.0226), but not for CW (p = 0.5489) and OC (p = 0.1658). While this is the first study determining the metabolic rates of common waxbills, previous estimates for black-rumped (Cade et al., 1965; Lasiewski et al., 1964; Weathers and Nagy, 1984) and orange-cheeked waxbills (Marschall and Prinzinger, 1991; Stephens et al., 2001) align well with our results (see Discussion). Figure 1.2 shows violin plots visualizing all the measured thermoregulatory traits.\\

Contrary to predictions based on the aerobic model of endothermy, we found no evidence for a positive correlation between BMR and M$_{\text{sum}}$ (p > 0.1), while metabolic expansibility (the ratio of M$_{\text{sum}}$ over BMR) did not differ between species or seasons (all p > 0.1), with an average value of 4.51 ± 1.20. Although all three species were able to tolerate colder temperatures (Table 1.1) during autumn compared to summer (all p < 0.05), M$_{\text{sum}}$ was not predictive of cold tolerance for any of the species or seasons (all p > 0.1), which was contrary to expectations. Cold tolerance depended only on body mass, with heavier individuals able to tolerate colder temperatures (p < 0.0001).\\

We found that seasonal changes in thermoregulatory characteristics were most strongly related to the level of variation in temperature between seasons. More specifically, we found that both BMR and M$_{\text{sum}}$ were most strongly, positively associated with the SE of the mean ambient temperature (T$_{\text{a}}$) in the week before the measurements (Δ AIC with second best ranked model: 7.67 for BMR and 5.24 for M$_{\text{sum}}$). The SE of the Ta also had the highest impact on the BMR and M$_{\text{sum}}$ of each species separately. Mean ± SD summer temperature was 21.87 ± 1.82 °C whereas mean autumn temperature was 13.80 ± 0.59 °C. The absolute temperature range in summer was 20.84 ± 0.68 °C, whereas in autumn it was 17.99 ± 0.91 °C (Table 1.3). Details of the results of all statistical analyses are available in Supplementary file 1 (RMarkdown HTML).\\

Concerning the native range climate variability, OC inhabits the most seasonally stable environments (Figure 1.1), followed by BR and CW. Details of the results of all statistical analyses are available in Supplementary file 1 (RMarkdown HTML).\\

\clearpage

\renewcommand{\thefigure}{1.\arabic{figure}}
	\begin{figure}[h!]
		\begin{center}
\small
			\includegraphics[width=0.7\textwidth]{figures/Figure_1600.png}
		\end{center}
		\begin{footnotesize}
			\caption{\footnotesize Violin plots visualizing ecophysiological traits of individuals that were measured twice, measured in summer versus autumn, for the three waxbills of the study. Violin plots illustrate kernel probability densities, i.e. the width of the area represents the proportion of the data located there. 1) basal metabolic rates BMR; 2) summit metabolic rates M$_{\text{sum}}$; 3) metabolic expansibility (ratio of summit over basal metabolic rate; ME); 4) thermal conductance TC measured at BMR; 5) thermal conductance TC measured at M$_{\text{sum}}$; 6) minimum ambient heliox temperature reached during M$_{\text{sum}}$. Lines and stars indicate significant differences between seasons. ***p < 0.001; **p < 0.01; *0.05 < p < 0.01; ns: not significant.\label{fig1.2}}
		\end{footnotesize}
	\end{figure}
\clearpage

\section{Discussion}
\renewcommand\thesection{\arabic{chapter}.\arabic{section}}
\renewcommand{\thefigure}{\arabic{chapter}.\arabic{figure}}

We show that three species of Afrotropical estrildid finches housed in outdoor aviaries in Belgium exhibit an increased capacity to withstand colder temperatures from summer to autumn. This higher cold tolerance was not associated with higher body masses, or (mass-independent) BMR or M$_{\text{sum}}$, as measurements either remained unchanged or decreased towards autumn. We also found no evidence for the mechanistic link between BMR and M$_{\text{sum}}$ expected by the aerobic model of endothermy, as they were unrelated. Heavier birds were able to tolerate colder temperatures, but summit metabolism was not a predictor of cold tolerance. We found that BMR and M$_{\text{sum}}$ were most strongly related to temperature variation, measured as the standard error of the mean ambient temperature in the week preceding the metabolic measurements. While summer's mean temperature was about 8 °C higher than autumn's, temperature fluctuations were smaller in the latter season. Overall, our results suggest that captive-bred Estrildid finches respond to seasonal environmental changes by reducing their maintenance energy costs (i.e., lowering their metabolic rates) and achieving higher autumnal cold tolerance by reducing the ease with which heat leaves their bodies (i.e., their thermal conductance). Orange-cheeked waxbills, which inhabit the least seasonal environment in their native range, indeed showed the least phenotypic plasticity in metabolic rates. Black-rumped waxbills showed variation in both basal and summit metabolic rates, whereas the more widespread common waxbill only showed variation in basal rates, which is partly consistent with the hypothesis that climatic variation in the native range predicts variation in thermoregulation.\\

To our knowledge, this is the first study to measure BMR of the common waxbill and M$_{\text{sum}}$ on all three study species. Although there have been studies addressing BMR of the orange-cheeked waxbill and the black-rumped waxbill, most of these focused on measuring the active-phase/resting metabolic rate (RMR) during the day, and this, combined with differences in respirometry methodology, complicates comparisons with our data. Boyce et al. (2020) state that RMR values are about 15\% higher than BMR values in 147 species of birds across northern, southern, and tropical latitudes. By using this correction factor to convert RMR to BMR, we obtained estimates of BMR that were generally higher than previously reported (see Table 1.2). However, Lasiewski et al. (1964) found BMR and RMR to be close to each other for black-rumped waxbill (n = 1), and argued that values obtained from small birds during the day can be compared to those at night. Accepting Lasiewski et al.’s (1964) conjecture that BMR and RMR are equal for small birds, results in our BMR values being very close to previously reported values (literature average weighted value for OC: 4.14 ml O\textsubscript{2} g/h versus 4.54 in this study (10\% higher); BR: 4.31 ml O\textsubscript{2} g/h versus 4.46 here (3\% higher).

\clearpage

\begin{sidewaystable}[!ht]
    \centering
\caption*{\textbf{Table 1.2}: Comparison of basal metabolic rates (BMR; ml O$_{\text{2}}$ (g*h)\textsuperscript{-1}) between this study and studies on orange-cheeked waxbill and black-rumped waxbill with weighted means. Values with * were converted from rest metabolic rate (RMR) to BMR and standardized to ml O$_{\text{2}}$ (g*h)\textsuperscript{-1}, following Boyce et al. (2020)}
    \begin{tabular}{cccc}

    \hline
\textbf{Reference} & \textbf{Species} & \textbf{BMR ml O$_{\text{2}}$ (g*h)\textsuperscript{-1}} & \textbf{Sample size n} \\
    \hline
        Stephens et al. (2001) & Orange-cheeked waxbill & 4.97* & 7 \\ 
    \hline
        Marschall and Prinzinger (1991) & Orange-cheeked waxbill & 3.17* & 6 \\ 
            \hline
~ & ~ & ~ & ~ \\ 
        ~ & Weighted mean & 3.79 & ~ \\ 
\hline
        This study & Orange-cheeked waxbill & 4.54 & 8 \\
\hline
\end{tabular}

\bigskip
    \begin{tabular}{cccc}
    \hline
\textbf{Reference} & \textbf{Species} & \textbf{BMR ml O$_{\text{2}}$ (g*h)\textsuperscript{-1}} & \textbf{Sample size n} \\
    \hline
        Weathers and Nagy (1984) & Black-rumped waxbill & 4.77* & 9 \\ 
\hline
        Cade et al. (1965) & Black-rumped waxbill & 3.74* & 6 \\ 
\hline
        Lasiewski et al., 1964 & Black-rumped waxbill & 3.60 & 1 \\ 
\hline
        ~ & ~ & ~ & ~ \\ 
        ~ & Weighted mean & 3.78 & ~ \\ 
        ~ & ~ & ~ & ~ \\ 
\hline
        This study & Black-rumped waxbill & 4.46 & 9 \\
\hline
    \end{tabular}
\end{sidewaystable}

\clearpage

\subsection{The aerobic model of endothermy}

Our findings fit the emerging view that the relationship between metabolic rates and avian thermoregulation is more complex than expected based on the aerobic capacity model alone (Bennett and Ruben, 1979). As we did not find any phenotypic correlation between BMR and M$_{\text{sum}}$, our results differ from the conclusion reached by Auer et al. (2017), who found predictions were generally well supported across vertebrate taxa and especially in birds. While a host of studies on species living in seasonal environments have indeed documented a winter acclimatization whereby BMR and M$_{\text{sum}}$ are elevated in tandem (Cooper and Swanson, 1994; Welman et al., 2022), the mechanisms underlying this covariation remain debated. For example, Petit et al. (2013) monitored seasonal changes in Canadian black-capped chickadees (\textit{Poecile atricapillus}) and found a mismatch between upregulation of BMR and M$_{\text{sum}}$. A potential explanation for this observation is that BMR and M$_{\text{sum}}$ respond to different constraints and vary independently, with chickadee flight muscles changing early in response to the appearance of the first cold temperatures while BMR changes are linked to changes in diet, such as an increased consumption of seeds during winter (Petit et al., 2013).\\

Studies assessing within-species geographical and seasonal variation in BMR and M$_{\text{sum}}$, found mixed support for the aerobic model relationships between ambient temperature, BMR, and M$_{\text{sum}}$ (Noakes et al., 2017; van de Ven et al., 2013). The only study on tropical birds so far has found no correlation between BMR and M$_{\text{sum}}$, possibly because the comparatively warm climate imposes weak selection for thermogenic capacity (Wiersma et al., 2007a). The absence of a relationship between BMR and M$_{\text{sum}}$ found in our estrildid finches, together with the lack of seasonal changes in metabolic expansibility, suggests that, at least for these species, BMR and M$_{\text{sum}}$ are not functionally linked. An alternative explanation is that the lack of correlation found here is a result of studying a captive population, as a reduced need for (foraging) flights can reduce pectoralis muscle strength (Peng et al., 2013).\\

Also contrary to the aerobic capacity model is the finding that, for none of the three waxbill species studied, thermogenic capacity was related to cold tolerance. Since heat production in birds is assumed to be primarily accomplished via shivering thermogenesis of their pectoralis muscles (Swanson et al., 2017) and M$_{\text{sum}}$ is often considered to be mainly a function of muscular activity, we expected to find such a positive correlation (Swanson, 2001). Indeed, a comparative study on 21 North-American small birds confirmed a phenotypic link between cold tolerance and summit metabolism (Swanson and Liknes, 2006), although the amount of explained variance tended to be small and relatively large changes in cold tolerance could occur with only minor changes in M$_{\text{sum}}$ (Swanson, 2001). In contrast, studies on several South-African birds found that M$_{\text{sum}}$ was not related to cold tolerance, suggesting that M$_{\text{sum}}$ may respond to other factors than thermoregulation alone (Noakes et al., 2017, 2020; van de Ven et al., 2013). Milbergue et al. (2018) showed that large muscles are not a requirement for improving thermogenic capacity in birds and that instead, for birds, non-shivering thermogenesis can considerably contribute to thermogenic capacity as well (Barceló et al., 2017; Milbergue et al., 2018; Noakes et al., 2020; Vézina et al., 2011). So far, cellular metabolic intensity and adjustments are strongly underappreciated as underlying drivers of avian cold tolerance (Pani and Bal, 2022). However, as we did not collect blood or tissue samples, the extent to which molecular mechanisms contribute to thermoregulation and cold tolerance in Estrildid finches remains to be tested.\\

\subsection{Cold tolerance}

All waxbill species increased their cold endurance from summer to autumn, but in contrast with our predictions, this was not associated with higher body masses or increased metabolic rates. Heavier-bodied individuals can withstand cold exposure for longer periods than lighter individuals (e.g., see Piersma, 1984; Stuebe and Ketterson, 1982) and many bird species increase body mass to cope with cold ambient temperatures (e.g., Zungu et al., 2013; Krams et al., 2010). However, fattening responses are not universal, and many other studies also failed to find evidence of seasonal mass differences in small birds (Nzama et al., 2010 and references therein). As even a slight increase in body mass can reduce flight maneuverability (van den Hout et al., 2010), birds may have to trade off the risks of hypothermia and predation (Carr and Lima, 2013). While our captive birds were visually exposed to overflying raptors such as Eurasian sparrowhawks (\textit{Accipiter nisus}), experimental simulation of risk-taking behavior (sensu e.g., Seress et al., 2011) is required to test whether (perceived) predation risk might explain the lack of seasonal mass change observed here.\\

The finding that none of the three waxbill species in this study upregulated their metabolic rates from summer to autumn contrasts with the increased metabolic rates during colder seasons commonly observed in species inhabiting temperate regions (Swanson, 2010; McKechnie et al., 2015). Instead, the downregulation of metabolic rates we observed fits with Lovegrove's (2000, 2003) suggestion that, because of the generally less severe winters compared to the northern hemisphere, energy conservation would be the primary seasonal metabolic response in Afrotropical species, as lower maintenance energy requirements may increase winter survival (Maddocks and Geiser, 2000; Sharbaugh, 2001). Lower winter metabolic rates have been observed in multiple studies (e.g., Lindsay et al., 2009; Oswald et al., 2021; Smit and McKechnie, 2010). Similar findings have been published for captive-bred birds, such as rock kestrels (\textit{Falco rupicolus}, Bush et al., 2008) and ring-necked parakeets (\textit{Psittacula krameri}, Thabethe et al., 2013). The seasonal changes in waxbill metabolic rate (despite access to ad libitum food) that we observed suggest that even when confronted with a climate different from their native range, these small birds follow an energy saving strategy during colder seasons. Such energy savings may help these birds to successfully establish and invade regions characterized by colder winters and less predictable weather, such as the south-western parts of Europe, where they are currently expanding their invasive distribution ranges (Franch et al., 2021).\\

Autumn thermal conductance measured at BMR was about 90\% of summer values and $\sim$80\% at M$_{\text{sum}}$ (Table 1.1). Higher summer compared to winter thermal conductance in birds has been linked to seasonal changes in plumage characteristics (Cooper, 2002; Cooper and Swanson, 1994; Piersma et al., 1995; Swanson, 1990), as increased insulation capacity of the feather layer allows birds to tolerate colder temperatures and/or reduce energy expenditure (Novoa et al., 1994). Indeed, downregulation of metabolic rates only partially explained the lower autumn thermal conductance observed here, suggesting that changes in plumage density or feather morphology may underlie waxbills' higher autumn cold tolerance. In line with this, Barve et al. (2021) found that across species, the proportion of a feather's plumulaceous (downy) section increased with elevation, leading to the conclusion that increases in feather insulation capacity at colder temperatures are a widespread evolutionarily driven response across both temperate and tropical taxa. Moulting patterns in estrildid finches are poorly known, and Vowles and Vowles (1987) report that the timing of adult moult of common waxbill in southern Portugal differs strongly from that in their native range. According to Vowles and Vowles (1987), all adults go through a complete postnuptial moult between July and October, with occasional partial moults of body and wing coverts in January–February. During our study period, we did not observe moult of primary wing feathers, but as we did not collect detailed data on contour or down feathers we cannot rule out (subtle) changes in these insulating feathers from summer to autumn. Future studies aimed at identifying the mechanisms underlying (waxbill) thermoregulatory capacities may thus need to take into account the potential role of (changes in) plumage structure and insulation, for example by assessing whether plumage structure differs between native and invasive populations.\\

\subsection{Temperature variability and metabolic rates}

The climate variability hypothesis is mainly invoked to explain macroecological differences in the temperature dependence of metabolic rates between tropical and temperate zone species. However, how individual birds respond to the degree of thermal variability they experience is less well known. Recently, using dark-eyed junco's (\textit{Junco hyemalis}), Stager et al. (2021) showed that both under field and lab-based conditions, juncos exhibited intra-specific variation in thermogenic capacity that correlated with the heterogeneity of their native thermal environment. Range-wide variation in allele frequencies was also correlated with the annual temperature range, providing evidence that thermal heterogeneity may be an important determinant of population genetic structure across the \textit{J. hyemalis} radiation (Stager et al., 2021). Our results suggest that such a role for thermal heterogeneity may also hold true for Afrotropical waxbill species, as metabolic rates were most strongly correlated with mean temperature variability in the week preceding the measurements. Orange-cheeked waxbills showed the least seasonal variability in metabolism, as was expected based on their climatically stable West to Central African distribution range. Whereas we expected the widespread common waxbill, which can be found across most of sub-Saharan Africa, to show the most variability, it is surprising that only the black-rumped waxbills showed seasonal variability in both basal and summit metabolic rates. A potential explanation may be that metabolic capacity can be influenced by other factors than ambient temperatures alone. For example, several studies have linked metabolic characteristics to (precipitation-driven) patterns of primary productivity (e.g., see Cooper and Geiser, 2008; Maldonado et al., 2012; White and Kearney, 2013). The black-rumped waxbills' native distribution roughly corresponds to the Sahel zone, which is characterized by a generally dry climate and a short rainy season with irregular precipitation. While this species' native-range breeding phenology is poorly studied, especially in hot and arid regions, birds may need to rapidly respond to temporary favorable conditions after rainfall (Duursma et al., 2018). All available studies assessing waxbill metabolic traits have so far been conducted on captive birds, as was the case here. Habituation to the stable and ad libitum food conditions and the availability of an indoor shelter area where ambient temperatures were kept above 10 °C may have led to less metabolic variability than expected for common waxbills. Our results thus suggest that research on free-living birds across their native ranges is needed to elucidate the ultimate drivers of their metabolic variability.\\

\subsection{Caveats}

The use of captive-raised individuals can be considered a limitation of this study. McKechnie et al. (2006) showed that because of phenotypic adjustments to often more homogenous artificial environments, captive-raised birds tend to have different metabolic rates than their wild-caught counterparts. Changes in both internal organs (e.g., gut, liver, and kidney due to the availability and/or quality of food (Moore and Battley, 2006)) and muscles (e.g., reduced flight muscles because of limited need and opportunity for exercise in captivity) likely underlie metabolic differences between wild and captive individuals (McKechnie et al., 2006). Captivity effects are not universal, as some studies found only a small or even no effect of long-term captivity on BMR (Weathers et al., 1983) and M$_{\text{sum}}$ (Swanson and King, 2013), yet caution should be used when extrapolating our findings to free-living native and invasive wild birds.\\

A second caveat is that for the estimation of thermal conductance, we assumed that the body temperature of our waxbills remained stable from summer to autumn, as because of difficulties with thermocouple attachment, body temperatures were only measured in summer. Studies show mixed evidence about seasonal variations in body temperature in passerine birds. For example, Zheng et al. (2008) showed no significant seasonal variation in body temperature for Chinese bulbuls (\textit{Pycnonotus sinensis}, $\sim$29 g). Cooper and Swanson (1994) found the same result for the black-capped chickadees ($\sim$13 g), while the body temperature of free-living Chinese hwamei (\textit{Garrulax canorus}, $\sim$58g) varied significantly from winter to summer (Wu et al., 2015). Piersma et al. (1995) studied the effect of body temperature on thermal conductance for a long-distance migrating shorebird, the knot (\textit{Calidris canutus}, $\sim$110 g), and found little evidence for an effect of body temperature changes of up to 2 °C on thermal conductance. Yet, we cannot exclude the possibility that the decline in thermal conductance and associated higher cold tolerance we observed is also influenced by changes in core body temperature from summer to autumn, in addition to the possible role of plumage characteristics mentioned above.\\

\section{Conclusions}

Our results suggest that an energy saving strategy represents the principal seasonal metabolic response in our study set of captive-bred, small, Afrotropical waxbills, as lower maintenance energy requirements can improve winter survival. This, combined with increased cold tolerance, may facilitate their establishment in new areas characterized by colder winters and less predictable weather. The mechanisms underlying these changes, however, remain unclear, as neither the relationship between thermogenic capacity and cold tolerance nor the link between BMR and M$_{\text{sum}}$ predicted by the aerobic capacity model of endothermy were supported by our study. We recommend that future studies include the potential role of non-shivering thermogenesis and of changes in the structure and insulation of the plumage when assessing variation in bird thermoregulatory capacity. Although our findings partially support the climate variability hypothesis, research on free-living waxbills is needed to test to what extent the heterogeneity of the native thermal environment influences (variations in) thermoregulatory capacity and strategy. Aspects of phenotypic variation in thermoregulatory capacity in birds, as well as the underlying mechanisms, have received extensive research, but much less so for tropical species. More research in this direction will help solidify our understanding of the ecophysiological responses (tropical) birds use to cope with changing environments.


	\subsection*{Acknowledgements}

We thank Michel Strubbe and Claire Lannoy for their support. We are grateful to Emma Bossuyt and Fleur Petersen for the assistance provided during data collection. We also acknowledge the support of Sheldon J. Cooper, François Vézina, Ryan O’ Connor, Götz Eichhorn, Tanja M.F.N. Van de Ven and Matthew J. Noakes in setting up the experiments.

	\subsection*{Data availability}

The data used in this manuscript can be found at \newline
(https://data.mendeley.com/datasets/3mgmwbvsht/1) while the statistical scripts used can be consulted via Supplementary file 1 (RMarkdown HTML) in the online version of the manuscript. 




\clearpage
\renewcommand\thesection{\arabic{chapter}.\arabic{section}}
\section{Supplementary material}
\renewcommand\thesection{\arabic{chapter}.\arabic{section}}
\renewcommand{\thefigure}{\arabic{chapter}.\arabic{figure}}

\renewcommand\thesection{\arabic{chapter}.\arabic{section}}
\renewcommand{\thefigure}{1.\arabic{figure}}
	\begin{figure}[h!]
		\begin{center}
\small
			\includegraphics[width=0.7\textwidth]{figures/Picture2.png}
		\end{center}
		\begin{footnotesize}
			\caption{\footnotesize Relationship between (log)Body mass [g] and (log)BMR [ml/min] for each species (CW: common waxbill; OC: orange-cheeked waxbill; BR: black-rumped waxbill).\label{fig1.2}}
		\end{footnotesize}
	\end{figure}
\clearpage

\renewcommand{\thefigure}{1.\arabic{figure}}
	\begin{figure}[h!]
		\begin{center}
\small
			\includegraphics[width=0.7\textwidth]{figures/Picture3.png}
		\end{center}
		\begin{footnotesize}
			\caption{\footnotesize Relationship between (log)Body mass [g] and (log)M$_{\text{sum}}$ [ml/min] for each species (CW: common waxbill; OC: orange-cheeked waxbill; BR: black-rumped waxbill).\label{fig1.2}}
		\end{footnotesize}
	\end{figure}
\clearpage


\begin{sidewaystable}[!ht]
    \centering
\small
\caption*{\textbf{Table 1.3}: Mean ± standard deviation (SD), minimum (Min) and maximum (Max) for each Ta variable in each season. AbsMax: the highest of the daily maximum temperatures; AbsMin: the lowest of the daily minimum temperatures; AbsTrange: the difference between absolute maximum and minimum temperatures; AverageDailyDelta: the mean difference between the daily minimum and maximum temperatures; MeanMax: the average of the daily maximum temperatures; MeanMin: the average of the daily minimum temperatures; MeanT: the average of the daily mean temperatures; SeMeanT: the standard error of the mean temperature.}
\begin{adjustbox}{max width=\textwidth}    
\begin{tabular}{cccccccccc}

    \hline
        ~ & ~ & AbsMax °C & AbsMin °C & AbsTrange °C & MeanMax °C & MeanMin °C & AverageDailyDelta °C & MeanT °C & SeMeanT °C \\ \hline
        Summer & Mean ± SD & 35.31 ± 0.90 & 14.47 ± 0.25 & 20.84 ± 0.68 & 29.82 ± 1.95 & 16.69 ± 1.05 & 13.18 ± 1.38 & 21.87 ± 1.82 & 0.14 ± 0.01 \\ 
        ~ & Min - Max & 34.65 - 36.60 & 14.20 - 14.79 & 20.26 - 21.81 & 27.54 - 31.72 & 15.95 - 18.20 & 11.71 - 15.07 & 20.01 - 24.25 & 0.13 - 0.16 \\ 
        Autumn & Mean ± SD & 25.42 ± 1.81 & 7.43 ± 1.26 & 17.99 ± 0.91 & 20.74 ± 0.73 & 9.90 ± 0.55 & 10.91 ± 0.91 & 13.80 ± 0.59 & 0.11 ± 0.02 \\ 
        ~ & Min - Max & 23.40 - 27.55 & 6.35 - 9.06 & 16.83 - 18.91 & 19.80 - 21.37 & 9.18 - 10.55 & 9.89 - 12.19 & 13.23 - 14.54 & 0.09 - 0.13 \\ \hline
    \end{tabular}
\end{adjustbox}
\end{sidewaystable}

\clearpage

	\subsection*{The thermoneutral zone (TNZ)}
The thermoneutral zone (TNZ) was determined for the common waxbill and the orange-cheeked waxbill in each season (summer and autumn), by monitoring O$_{\text{2}}$ consumption (VO$_{\text{2}}$) in three post-absorptive individuals per species during the rest phase at a series of air temperatures (from 15 to 40 °C), executed in random order (following van de Ven et al., 2013). Lasiewski et al. (1964) exposed black-rumped waxbills to 40°C without harm, yet unexpectedly, four common waxbills suddenly died at 35°C. Black-rumped waxbill TNZ measurements were halted and TNZ data used here were taken from Lasiewski et al. (1964). The lowest stable part of the curve (11 min) was selected to estimate the lowest VO$_{\text{2}}$ over the entire night. The software ExpeData (Sable Systems) was used to record trials and extract VO$_{\text{2}}$ using equation 9.7 from Lighton (2008). Figure 2.5 shows the VO$_{\text{2}}$ expressed in ml O$_{\text{2}}$/min at different ambient temperatures T (°C) for all the species. Although the limits of the zone of thermal neutrality are not well defined by the curves, most individuals tend to consume less oxygen at ambient temperatures within 27-35 °C. Raw data is available in Supplementary File 2 (TNZ.xlsx).\\

\renewcommand{\thefigure}{1.\arabic{figure}}
	\begin{figure}[h!]
		\begin{center}
\small
			\includegraphics[width=0.7\textwidth]{figures/Picture4.png}
		\end{center}
		\begin{footnotesize}
			\caption{\footnotesize O$_{\text{2}}$ consumption (VO$_{\text{2}}$) expressed in ml/min at different ambient temperatures (T °C) for all the species in both seasons. The numbers at the top of the boxplots indicate the number of measurements taken at each temperature.\label{fig1.2}}
		\end{footnotesize}
	\end{figure}



%%%%%%%%%%%%%%%%%%%%%%%%%%%%%%%% CHAPTER THREE  %%%%%%%%%%%%%%%%%%%%%%%%%%%%%%%%%%%

\setlength{\thumbwidth}{0.8cm}
\setlength{\thumbheight}{1cm}
\tikzset{
	thumb/.style={
		%   draw=black,
		fill=light-gray,
		text=black,
		minimum height=\thumbheight, %\thumbheight,
		text width=\thumbwidth,
		outer sep=0pt,%   outer sep=10pt,
		font=\sffamily\Large,
	}
}
\pagestyle{mainmatter}
\renewcommand\thesection{\arabic{chapter}.\arabic{section}}
\renewcommand{\thefigure}{\arabic{chapter}.\arabic{figure}}
\chapter{Metabolic adjustments to winter severity in two geographically separated great tit (\textit{Parus major}) populations}\label{chapter2}
\chaptermark{Chapter 2}
\lettergroup{\thechapter}	

\begin{flushright}
        \textcolor{black}{Cesare Pacioni}\\
    \textcolor{black}{Andrey Bushuev}\\
\textcolor{black}{Marina Sentís}\\
    \textcolor{black}{Anvar Kerimov}\\
    \textcolor{black}{Elena Ivankina}\\
    \textcolor{black}{Luc Lens}\\
    \textcolor{black}{Diederik Strubbe}


\vspace*{2cm}
  \textcolor{black}{Adapted from: Pacioni et al. (2024) \textit{J. Exp. Zool. Part A}, 341(4), 410-420}

\end{flushright}
\clearpage


\section{Abstract}
Understanding the potential limits placed on organisms by their ecophysiology is crucial for predicting their responses to varying environmental conditions. A main hypothesis for explaining avian thermoregulatory mechanisms is the aerobic capacity model, which posits a positive correlation between basal (BMR) and summit (M$_{\text{sum}}$) metabolism. Most evidence for this hypothesis, however, comes from interspecific comparisons, and the ecophysiological underpinnings of avian thermoregulatory capacities hence remain controversial. Indeed, studies have traditionally relied on between-species comparisons, although, recently, there has been a growing recognition of the importance of intraspecific variation in ecophysiological responses. Therefore, here, we focused on great tits (\textit{Parus major}), measuring BMR and M$_{\text{sum}}$ during winter in two populations from two different climates: maritime-temperate (Gontrode, Belgium) and continental (Zvenigorod, Russia). We tested for the presence of intraspecific geographical variation in metabolic rates and assessed the predictions following the aerobic capacity model. We found that birds from the maritime-temperate climate (Gontrode) showed higher BMR, whereas conversely, great tits from Zvenigorod showed higher levels of M$_{\text{sum}}$. Within each population, our data did not fully support the aerobic capacity model's predictions. We argued that the decoupling of BMR and M$_{\text{sum}}$ observed may be caused by different selective forces acting on these metabolic rates, with birds from the continental-climate Zvenigorod population facing the need to conserve energy for surviving long winter nights (by keeping their BMR at low levels) while simultaneously being able to generate more heat (i.e., a high M$_{\text{sum}}$) to withstand cold spells.


\vspace*{\fill}
\noindent \textbf{Keywords:} Metabolic rates; Great tit; Aerobic capacity model; Basal metabolic rate; Summit metabolic rate.

\section{Introduction}

A key question in ecology is how species adjust their physiology to cope with different environmental conditions, as a better understanding of the underlying processes may allow ecologists to better predict how organisms will respond to changes in their environment (Bozinovic and Pörtner, 2015; Herrando-Pérez et al., 2023). This is particularly important for endotherms, as they need to maintain their core body temperature within a relatively narrow range. While endothermy allows animals to be active over a wide range of ambient temperatures, it comes at a potentially high energetic cost and places additional demands on their physiology (Boyles et al., 2011; Kronfeld-Schor and Dayan, 2013). Studies attempting to understand these physiological adjustments have relied heavily on interspecific comparisons, considering a given species as a homogenous physiological unit and assuming that conspecific populations have comparable responses (Reed et al., 2011; Thomas et al., 2004). However, different populations of a single species are likely to exhibit varying adjustments and different thermal tolerances to local environmental conditions (Cavieres and Sabat, 2008; Furness, 2003; Wikelski et al., 2003), as it is unlikely that a single phenotype will be the best fit for all conditions. This may be particularly true for populations that experience contrasting climatic conditions across their distribution range (Cavieres and Sabat, 2008; Root, 1988). Indeed, previous studies have shown that within-species variability in ecophysiological traits (e.g., metabolic rates) can be high, and by comparing individuals within a species, intraspecific studies can be used to test predictions derived from between-species comparisons and to identify factors beyond those revealed by interspecific studies (Cruz‐Neto and Bozinovic, 2004).\\

Widely distributed resident bird species with geographical distributions spanning a range of climatic zones therefore provide an interesting study system for assessing intraspecific geographical variation in ecophysiological traits related to thermoregulation and providing functional explanations for it. The wide distribution suggests a role for local adaptation and/or phenotypic flexibility for population persistence in different climates (Kendeigh and Blem, 1974; Piersma and Drent, 2003). For instance, while interspecific studies have shown that differences in metabolic rates between species are largely because of differences in body mass, reflecting either local adaptation or phenotypic plasticity, functional explanations for this scaling remain controversial (Giancarli et al., 2023; White et al., 2019). In line with the ‘food-habitat’ hypothesis (Thompson, 2019), interspecific studies often document strong correlations between diet and metabolism, such as the finding that species feeding on a combination of insects with seeds and fruits typically have higher metabolic rates (McNab, 2009). However, intraspecific comparisons typically show mixed support for this hypothesis (McKechnie and Swanson, 2010), suggesting that individual-level variability in factors such as enzymatic plasticity and the use of energy-saving mechanisms such as facultative torpor must be taken into account to understand the functional significance of such correlations (Cruz‐Neto and Bozinovic, 2004).\\

Failure to identify, quantify, and explain intraspecific variation in thermoregulatory capacities can also lead to erroneous forecasts of range shifts because of climate change (Bozinovic et al., 2011; Pearman et al., 2010). To gain a comprehensive understanding of species' responses to environmental changes, it is crucial to consider their thermoregulatory capacities (Boyles et al., 2011). Considerable uncertainty however remains regarding the extent to which local populations of a given species respond to weather and climate conditions across their range by adjusting their physiological characteristics. Birds from highly seasonal environments are known to be able to increase their cold tolerance in winter, which they achieve through physiological adjustments in, for example, body mass and metabolic rate (Swanson, 1990, 2010; Swanson and Olmstead, 1999; Swanson and Vézina, 2015). Several studies have documented that the magnitude of such thermogenic adjustments correlates with climate severity (Dawson et al., 1983; O’Connor, 1996). However, Swanson (1993) compared M$_{\text{sum}}$ in winter-acclimatized dark-eyed juncos (\textit{Junco hyemalis}) from the cold winter climate of South Dakota (USA) and the milder winter climate of western Oregon (USA), but found that M$_{\text{sum}}$ was similar between the two populations. More recently, Stager et al. (2021) studied dark-eyed juncos across North America and found that while thermogenic capacity was higher in colder areas, there was substantial variation among populations in the extent to which they adjusted their M$_{\text{sum}}$ in response to climate. \\

Similar debates exist about the mechanisms underlying intraspecific variation in M$_{\text{sum}}$. An important explanation for this comes from the aerobic capacity model for the evolution of endothermy (Bennett and Ruben, 1979), which assumes a positive correlation between basal and summit metabolism, e.g., because of energetic maintenance costs associated with increased muscle mass for shivering thermogenesis and/or increased investment in the gut and digestive organs to process enough food to fuel muscle thermogenesis (McKechnie and Swanson, 2010). Between-species comparisons generally support the aerobic capacity model (Auer et al., 2017; Dutenhoffer and Swanson, 1996; Rezende et al., 2002). Intraspecific studies, in contrast, provide inconclusive support for functional correlations between BMR and M$_{\text{sum}}$ (Swanson et al., 2012). For example, Liknes and Swanson (1996) found that BMR and M$_{\text{sum}}$ were positively correlated in winter and late summer for white-breasted nuthatches (\textit{Sitta carolinensis}) and downy woodpeckers (\textit{Picoides pubescens}) in South Dakota. However, at higher latitudes, winters are not only colder but are also characterized by shorter day lengths, limiting the time available for foraging. High maintenance costs (i.e., high BMR) during long nights may cause birds to exhaust their energy reserves, leading to selection against high winter BMR (Bozinovic and Sabat, 2010; Broggi et al., 2005). Indeed, O’Connor (1995) found that BMR was seasonally stable in house finches (\textit{Haemorhous mexicanus}) in Michigan, USA, whereas M$_{\text{sum}}$ was higher in winter. Therefore, intraspecific studies examining variations in both metabolic rates have yielded inconsistent results, leaving uncertainties regarding the extent of variation within species. \\

Here, we studied intraspecific geographic variation in ecophysiological traits hypothesized to underpin avian thermoregulation, using great tits (\textit{Parus major}) as a case study. The great tit is one of the best studied bird species, breeding from approximately 10°S to 71°N, and remaining resident even at the northernmost limit of its breeding range (Cramp et al., 1993; Silverin, 1995). To this end, we measured the basal (BMR) and summit (M$_{\text{sum}}$) metabolic rates in two populations of great tits, one living in a maritime-temperate climate characterized by mild winters (Belgium, Gontrode, Melle), and the other living in a continental climate characterized by long and cold winters (Russia, Zvenigorod Biological Station, Moscow Oblast). We predicted (i) that individuals from the cold, continental population will be characterized by higher maximal thermogenic capacity (i.e., M$_{\text{sum}}$) but lower maintenance costs (i.e., BMR) compared with those from the maritime temperate population, and that (ii) BMR and M$_{\text{sum}}$ will be correlated within each population, following the aerobic capacity model. \\
\section{Material and methods}
 
\subsection{Study areas, trapping, and maintenance}
The research in Belgium (Gontrode, Melle) took place in the Aelmoeseneie forest (50.975°N, 3.802°E), covering an area of 28.5 ha. The forest is a mixed deciduous forest surrounded by residential areas and agricultural fields. The fieldwork was carried out during the late winter (February 01 - March 11, 2022), with an average daylight of 9.5 h (obtained from R package "suncalc"; Thieurmel et al., 2019). Since autumn 2015, the forest has been equipped with 84 standard nest boxes specifically designed for great tits. The climate in this region is maritime-temperate, characterized by mild winters and constant rainfall throughout the year (corresponds to the Cfb subtype in the K{\"o}ppen climate classification). To monitor the ambient temperature (T$_{\text{a}}$) in the forest, 20 TMS-4 data loggers were deployed (with a DS7505U+ digital thermometer manufactured by Maxim Integrated), positioned approximately 15 cm above the ground (Wild et al., 2019).\\


The research in Russia (Zvenigorod) took place at the Zvenigorod Biological Station (55.701°N, 36.723°E), affiliated with Lomonosov Moscow State University. The station territory encompasses two small settlements within the mixed forest of the Moskva River valley, as well as a predominantly spruce forest in the watershed area. The fieldwork was carried out during the late winter (January 22 - February 02, 2021; January 27 - February 23, 2022; March 04-17, 2023), with an average daylight of 8.5 h (obtained from R package ‘suncalc’; Thieurmel et al., 2019). Although the reserve spans a total territory of 715 ha and comprises 540 nest boxes, the winter bird catching was carried out only near the feeder located at the center of one of the settlements. Additionally, a nighttime check was conducted only on the nearest nest boxes within a distance of 250 m from the feeder. The climate of Zvenigorod region is temperate continental (corresponding to the Dfb subtype in the K{\"o}ppen climate classification). To monitor the T$_{\text{a}}$ in the forest, a DS1921G Thermochron iButton logger (Dallas Semiconductor) was used, positioned approximately 1.5 m above the ground near the feeder.\\

In both populations, bird capture consisted of nightly nest box checks and daily mist netting. The captured birds were transported to a nearby laboratory, where they were individually ringed for identification purposes. They were also assessed for age (1st winter or adult), sex (based on plumage characteristics) and weighed to the nearest 0.1g before being placed in metabolic chambers to obtain their BMR during the remainder of the night. At sunrise, birds were removed from the metabolic chambers and placed in individual cages with ad libitum access to food (mealworms and sunflower seeds) and water. During the afternoon, birds were returned to a metabolic chamber to determine their M$_{\text{sum}}$. After M$_{\text{sum}}$, birds were returned to their individual cages in a warm room with water and food ad libitum until they were later (30 min) released at the site of capture. The study protocol in Belgium was approved by the Ethics Committee on Animal Experiments VIB/Faculty of Science of Ghent University (EC2020-063), and in Russia by the Bioethics Committee of Lomonosov Moscow State University (applications 120-a and 120-a-2 for the experimental procedures, and 10.2-hous. and 10.3-hous. for the short-term housing of birds).\\

In Gontrode (Table 2.1), we measured BMR in 40 individuals (19 males and 21 females, including 35 adults and 5 1st-winter). Of these, we also measured the M$_{\text{sum}}$ of 36 individuals (16 males and 20 females, including 31 adults and 5 1st-winter). In Zvenigorod (Table 2.1) we measured BMR in 128 individuals (77 males and 51 females, including 51 adults and 77 1st-winter), and M$_{\text{sum}}$ was measured in 20 individuals using flow-through respirometry (11 males and 9 females, including 6 adults and 14 1st-winter) and 35 individuals using closed-circuit respirometry (21 males and 13 females, including 14 adults and 20 1st-winter). Table 2.2 provides a summary of the T$_{\text{a}}$ in both locations.\\

\subsection{BMR measurements}

The assessment of winter BMR in Gontrode was conducted at night using open flow-through respirometry (Lighton, 2018). To measure BMR, the oxygen consumption (VO$_{\text{2}}$) of 40 individuals was monitored, following the methods described in Pacioni et al. (2023a). To maximize the likelihood of birds being post-absorptive during measurements, the first two hours of measurement were discarded (Secor, 2009). Before and after the respirometry measurement, the body masses of the birds were taken to the nearest 0.1g. Each individual was then placed in a 1.1-liter plastic chamber within a darkened climate control unit (Combisteel R600). Ambient air was delivered by two pumps at a flow rate of 400 ml/min. The chambers were maintained at 25°C, which falls within the birds' thermoneutral zone (Bech and Mariussen, 2022; Pacioni et al., 2023b), determined according to the procedures outlined by van de Ven et al. (2013). The birds were measured in cycles, along with several baselines, with the timing and duration of measurements varying depending on the number of birds present during each session. On average, each bird underwent measurements for approximately 30 minutes per cycle, with three cycles conducted throughout the night. Following the metabolic measurements, the birds were returned to their cages and provided with water and food ad libitum. Additional details regarding calibration and the respirometry setup can be found in Pacioni et al. (2023a). \\

The assessment of winter BMR in Zvenigorod was conducted similarly to BMR measurements in Gontrode. VO$_{\text{2}}$ measurements were carried out throughout the night (from 14 h in late January to 11 h in mid-March) using open flow-through respirometry. An 8-channel system enabled the measurement of VO$_{\text{2}}$ in up to 7 birds (on average, 3.3 individuals). Outdoor air was pushed through columns containing silica gel. Subsequently, the dehumidified air was directed at an average flow rate of 430 ml/min into 1.25-liter polypropylene chambers housing the birds, which were maintained at a T$_{\text{a}}$ of 26.5°C in thermostats. The air from the chambers was dried using a small chamber with 10-20 mesh Drierite® (W.A. Hammond Drierite Co. Ltd) and directed into the flowmeter of the FoxBox respirometer (Sable Systems). A subsample of the airflow, at a rate of 100 ml/min, was then directed to the O$_{\text{2}}$ and CO$_{\text{2}}$ analyzers in two FoxBoxes (the second one was used for control), which recorded gas concentrations and flow rates every 6 sec. The measurements of gas concentrations in the airflow from the chambers with birds and the reference chamber were alternated, with durations of 20-25 min for each bird and 5-10 min for the baseline measurement. The minimum VO$_{\text{2}}$ (BMR) typically occurred around 3:30 am. Additional details regarding calibration and leakage testing can be found in Bushuev et al. (2021).\\

\subsection{M$_{\text{sum}}$ measurements}

During the day, individual measurements of winter M$_{\text{sum}}$ were conducted in Gontrode on 36 individuals. The maximum cold-induced oxygen consumption (VO$_{\text{2}}$) in a heliox atmosphere (79\% helium, 21\% oxygen) was used as the indicator for M$_{\text{sum}}$, following the methodology outlined in Pacioni et al. (2023a). The sliding cold exposure method of Swanson et al. (1996) was used, using open flow-through respirometry. Prior to and after the trials, the body mass of each bird was recorded with an accuracy of 0.1g. Subsequently, the birds were placed in a 0.9-liter metal chamber. The chamber, along with the bird inside, was positioned within the same climate control unit used for BMR measurements, with an initial temperature setting of 10°C. The climate control unit was supplied with flowing heliox gas a few minutes prior to the trial, allowing the bird to acclimate. The heliox gas was pumped into the chamber at a flow rate of approximately 812 ml/min. Each M$_{\text{sum}}$ trial began with a 7-minute baseline measurement using heliox, ensuring complete replacement of the air in the metabolic chambers before data recording commenced. After the baseline period, the experimental channel was activated. Following the removal of the bird from the chamber, baseline values were once again recorded for a minimum duration of 5 minutes. The trial was stopped when a steady decline in VO$_{\text{2}}$ was observed for several minutes. M$_{\text{sum}}$ was considered reached when the body temperature of a bird after a trial was 38°C (Cooper and Gessaman, 2005), measured by inserting a thermocouple (5SC-TT-TI-36-2M; Omega) coated with vaseline into the cloaca. The bird was then placed in a warm room with access to water and food. Additional details regarding calibration and the respirometry setup can be found in Pacioni et al. (2023a).\\

In Zvenigorod, M$_{\text{sum}}$ was estimated during the day using both open flow-through respirometry (utilizing the same FoxBox system as for BMR measurements; n = 20) and closed-circuit respirometry (n = 35), but the static cold exposure method described by Swanson et al. (1996) was employed. The design of the open flow-through respirometry setup was generally similar to that used in Gontrode. Before the start of the experiment, the birds were placed inside a 1.3-liter metal chamber with its interior painted black. Subsequently, the chamber was flushed with heliox until the volume of gas passing through the chamber reached six times its volume. The flow through the chamber was then reduced to 1.5 l/min, and a 5-minute reference measurement of concentrations of O$_{\text{2}}$ and CO$_{\text{2}}$ (baselining) was initiated. During this period, the chamber was carefully placed inside an Alpicool C40 refrigerator (avg. T$_{\text{a}}$ = -10°C) filled with a mixture of propylene glycol, propanol, and water. The heliox flow from the gas cylinder passed through the FoxBox mass flow meter and then entered the chamber with the bird through a 20-meter-long tube, which was also located inside the refrigerator to allow the gas mixture to cool down. The experimental channel was activated every 20-25 min, alternating with 5-min baseline periods. After the chamber, the flow was subsampled at a rate of 200 ml/min. The subsampled gas was then passed through a 20 ml column containing Drierite® 10-20 mesh absorbent to remove water. Subsequently, the air passed through two FoxBoxes (the second one was used for control), which recorded O$_{\text{2}}$ and CO$_{\text{2}}$ concentrations every second.\\

To measure M$_{\text{sum}}$ in a closed system, a respirometer constructed by D.V. Petrovski (Institute of Cytology and Genetics, Russian Academy of Sciences) was used. The bird, housed in a metal mesh cage, along with a permeable container containing CO$_{\text{2}}$ and water absorbent (KOH granules), was placed inside a cylindrical steel chamber. This 1.2-liter chamber was located in WAECO TropiCool TC-35FL-AC thermoelectric refrigerator filled with antifreeze liquid (see above) at T$_{\text{a}}$ = -1°C. Subsequently, the chamber was purged with heliox six times its volume. During the VO$_{\text{2}}$ measurement, the heliox inside the airtight chamber with the bird was mixed using an internal membrane pump for a duration of 50 seconds. Following this, the decreased pressure within the chamber was balanced with atmospheric pressure by briefly opening a valve connecting the chamber to an oxygen pillow with pure O$_{\text{2}}$ for 5 seconds. Another 5 seconds were dedicated to gas mixture mixing, so the cycle repeated every minute. The decrease in partial pressure of O$_{\text{2}}$ in the chamber with the bird during each cycle was measured using an internal pressure gauge and converted to VO$_{\text{2}}$/min. Further details can be found in Moshkin et al. (2002) and Vasilieva et al. (2020). Similar to Gontrode, a notable decrease in VO$_{\text{2}}$ within a few minutes served as an indication of hypothermia. To measure cloacal temperature, we used a K-type thermocouple coated with vaseline and connected it to a calibrated Testo 175 T3 thermologger.\\

\subsection{Respirometry and data analyses}

For the Gontrode data, BMR (ml O$_{\text{2}}$/min) and M$_{\text{sum}}$ (ml O$_{\text{2}}$/min) were extracted using the ExpeData software provided by Sable Systems. The calculations for BMR, TNZ, and M$_{\text{sum}}$ were performed using equation 9.7 from Lighton (2008). The lowest stable section of the curve, averaged over 5 minutes, was used to estimate BMR and TNZ throughout the entire night, while the highest 5-minute average VO$_{\text{2}}$ during the test period was used to estimate M$_{\text{sum}}$. To ensure accuracy, all data were adjusted for drift in O$_{\text{2}}$, CO$_{\text{2}}$, and H$_{\text{2}}$O baselines utilizing the Drift Correction function available in ExpeData.\\

For the Zvenigorod data on BMR measurements, the calculation of VO$_{\text{2}}$ from fractional concentrations of O$_{\text{2}}$ and CO$_{\text{2}}$ was performed using the equation from Bushuev et al. (2018). However, for the M$_{\text{sum}}$ trials in the open flow-through respirometer, the flow rate was estimated prior to the metabolic chamber. Therefore, the calculation of VO$_{\text{2}}$ for these trials was conducted using a different equation, specifically equation 9.7 from Lighton (2008). In this equation, \%H$_{\text{2}}$O was set to zero, as water was eliminated using a chemical dryer. To estimate BMR and M$_{\text{sum}}$, the minimum and maximum running average VO$_{\text{2}}$ values over a 5-minute period were used, respectively. To account for drift in the O$_{\text{2}}$ and CO$_{\text{2}}$ baselines, linear correction was applied using two adjacent baselines: one before and one after the VO$_{\text{2}}$ measurement. In closed-circuit respirometry, the VO$_{\text{2}}$ readings were adjusted to standard conditions (STP) and then M$_{\text{sum}}$ was calculated using the 5-minute maximum running average.\\

\subsection{Statistical analyses}

Linear regression models with a Gaussian error distribution were used to test whether body mass, BMR, M$_{\text{sum}}$, and metabolic expansibility (i.e., the ratio between M$_{\text{sum}}$ and BMR; ME) differed between the two populations, specifying body mass, BMR, M$_{\text{sum}}$, and, ME as dependent variables while adding location, sex, and age as covariates, with 2-way and 3-way interactions. Here, we followed Swanson et al. (1996), and considered M$_{\text{sum}}$ values obtained from the sliding cold exposure method and the static cold exposure method to be comparable. Models were first run using whole-body metabolic rates, and then also using mass-independent metabolic rates. Mass-independent metabolic rates were considered as the residuals of regressions of (log) BMR, and (log) M$_{\text{sum}}$ on (log) body mass (after the measurements). Prior to investigating potential intraspecific geographic variation in BMR, we assessed whether BMR and M$_{\text{sum}}$ from Zvenigorod differed among the three years (2021, 2022, and 2023). Our analysis did not find any significant differences (p>0.1), hence we decided to consider BMR and M$_{\text{sum}}$ measurements from three years as a single dataset for further analysis. As Zvenigorod M$_{\text{sum}}$ was measured by two different methods (i.e., open flow (n=20) and closed circuit (n=35) respirometry, see above), we did not lump all Zvenigorod data together but instead performed separate analyses for the open flow and closed circuit data. Closed-circuit M$_{\text{sum}}$ values were significantly higher compared to the open-flow data (see Supplementary File 1), and therefore, in the remainder of the manuscript, we focus on the more conservative and comparable open-flow Zvenigorod M$_{\text{sum}}$ measurements only. Regarding the assumption of the aerobic capacity model, similar linear models were used to test for positive correlations between body mass and metabolic rates, and between BMR and M$_{\text{sum}}$. To avoid collinearity issues when assessing the relationship between mass-independent BMR and mass-independent M$_{\text{sum}}$ we first calculated the residuals from the regression of (log) BMR and (log) M$_{\text{sum}}$ on (log) body mass and then used the residual values of M$_{\text{sum}}$ as the dependent variable and the residual value of BMR as the explanatory variable (Downs et al., 2013).\\

For all models, we used a backward stepwise procedure to eliminate non-significant interactions and variables. Post-hoc comparisons between species and seasons were performed with the emmeans function in the ‘emmeans’ package (Lenth, 2022). We used interquartile ranges as a criterion to identify outliers by using the quantile function. Then, we use the subset function to eliminate outliers. For all models, the normality of residuals was tested and verified (i.e., Shapiro-Wilk W > 0.9), and the significance level was set at $p \leq 0.05$. Body mass, BMR, M$_{\text{sum}}$, and ME were log-transformed before all analyses. Statistical analysis was performed using R v. 4.2.2 software (R Core Team, 2022). Data and script of this study are available in Medeley Data repository (DOI: 10.17632/t84ntxxcbn.1).\\

\section{Results}
\subsection{Variation in body mass and metabolic rate}

We found a positive correlation between body mass and whole-body BMR, and between body mass and whole-body M$_{\text{sum}}$ in both populations (BMR: Gontrode: R=0.48, p<0.01; Zvenigorod: R=0.58, p<0.0001; M$_{\text{sum}}$: Gontrode: R=0.49, p<0.001; Zvenigorod: R=0.45, p<0.05). Individuals from Zvenigorod were significantly heavier (either considering the body mass before and after the metabolic measurements; p<0.05) than those from Gontrode (Gontrode: 16.4 ± 1.5g; Zvenigorod: 17.6 ± 1.0g; Table 2.1). Males were significantly (p<0.0001) heavier than females between and within populations, while adults and subadults did not differ in body mass between and within populations (p>0.05). Individuals from Gontrode had a significantly higher BMR than those from Zvenigorod, regardless of whether the analysis was conducted on whole-body (about two-fold higher, p<0.0001) or mass-independent (about four-fold higher, p<0.0001) metabolic rates. Males had a significantly higher whole-body BMR than females (p<0.05) between and within locations, because of their larger body mass, as mass-independent BMR did not differ between the sexes (p>0.1). Birds from Zvenigorod measured using the same open-flow respirometry set-up as used in Gontrode had a significantly higher M$_{\text{sum}}$ (an increase of more than 30\% (p<0.0001) for whole-body and almost two times higher (p<0.05) for mass-independent) than those from Gontrode. Great tits from Gontrode had a ME of 3.92 ± 0.53, whereas great tits from Zvenigorod had a whole-body ME of 5.83 ± 1.52. Whole-body ME from Zvenigorod was significantly higher (about 50\%, p<0.0001) compared to birds from Gontrode (Figure 2.1). \\
\subsection{Aerobic capacity model of endothermy predictions}

Whole-body BMR and M$_{\text{sum}}$ were also positively correlated in individuals from Gontrode (R=0.41, p<0.05), while mass-independent metabolic rates were not (p>0.1). In Zvenigorod, no significant correlations were found between BMR (whole-body or mass-independent) and M$_{\text{sum}}$ (Figure 2.2). For mass-independent values, the relationship was almost significant. However this result depended on one data point. When the bird with the lowest mass-independent M$_{\text{sum}}$ was omitted, the results change to r=-0.64 and p-value=0.14. Details of the results of all statistical analyses are available in Supplementary file 1.\\

\clearpage
\begin{sidewaystable}[!ht]
    \centering
\small
\caption*{\textbf{Table 2.1}: Summary of traits by location, sex, and age in great tits (Parus major). Note: Mean ± standard deviation of body mass (g), basal metabolic rate (BMR; ml O$_{\text{2}}$/min), summit metabolic rate (M$_{\text{sum}}$; ml O$_{\text{2}}$/min) and metabolic expansibility (i.e., the ratio between M$_{\text{sum}}$ and BMR; ME) in great tits (Parus major) from two different locations: Gontrode (Belgium) and Zvenigorod (Russia). The values for each location are further grouped by sex (female, F; males; M) and age (adult, ad; subadult; sad). The sample size n for each group is provided within brackets.}
\begin{adjustbox}{max width=\textwidth}    
    \begin{tabular}{cccccccc}
    \hline
        Location & sex & age & Body mass (g) & BMR (ml O$_{\text{2}}$/min) & M$_{\text{sum}}$ (ml O$_{\text{2}}$/min) & ME (M$_{\text{sum}}$/BMR) \\ \hline
        Gontrode & F & ad & 15.6 ± 1.1 (20) & 1.13 ± 0.12 (20) & 4.30 ± 0.37 (19) & 3.85 ± 0.48 (18) \\ 
        ~ & ~ & sad & 17.0 (1) & 1.18 (1) & 5.33 (1) & 4.52 (1) \\ 
        ~ & M & ad & 16.9 ± 1.6 (15) & 1.16 ± 0.16 (15) & 4.68 ± 0.21 (12)  & 3.93 ± 0.47 (10) \\ 
        ~ & ~ & sad & 18.1 ± 1.1 (4) & 1.35 ± 0.37 (4) & 5.19 ± 0.37 (4)  & 4.03 ± 0.92 (4) \\ 
        ~ & ~ & ~ & ~ & ~ & ~  & ~ \\ 
        ~ & mean & ~ & 16.4 ± 1.5 & 1.17 ± 0.17 & 4.56 ± 0.45 & 3.92 ± 0.53 \\ 
        ~ & ~ & ~ & ~ & ~ & ~ &  ~ \\ 
        Zvenigorod & F & ad & 17.1 ± 1.2 (21) & 1.05 ± 0.09 (21) & 6.80 ± 0.45 (3) & 6.47 ± 1.14 (3) \\ 
        ~ & ~ & sad & 16.9 ± 0.9 (30) & 1.07 ± 0.06 (30) & 5.91 ± 0.51 (6)  & 5.29 ± 0.88 (6) \\ 
        ~ & M & ad & 18.1 ± 0.7 (30) & 1.10 ± 0.06 (30) & 6.41 ± 0.38 (3)  & 6.03 0.77 (3) \\ 
        ~ & ~ & sad & 17.8 ± 0.8 (47) & 1.08 ± 0.07 (47) & 6.80 ± 0.62 (8)  & 6.41 0.40 (8) \\ 
        ~ & ~ & ~ & ~ & ~ & ~ &  ~ \\ 
        ~ & mean & ~ & 17.6 ± 1.0 & 1.08 ± 0.07 & 6.48 ± 0.64  & 6.11 ± 0.81 \\ \hline
    \end{tabular}
\end{adjustbox}
\end{sidewaystable}
\clearpage


\renewcommand{\thefigure}{2.\arabic{figure}}
	\begin{figure}[h!]
		\begin{center}
\small
			\includegraphics[width=1\textwidth]{figures/CH3.jpg}
		\end{center}
		\begin{footnotesize}
			\caption{\footnotesize Violin plots of A) body mass (g), B) whole-body basal metabolic rate (BMR; ml O$_{\text{2}}$/min), C) whole-body summit metabolic rate (M$_{\text{sum}}$; ml O$_{\text{2}}$/min), D) whole-body ME (M$_{\text{sum}}$/BMR), E) mass-independent (M-I) BMR and F) mass-independent (M-I) M$_{\text{sum}}$ in great tits (Parus major) from two different locations: Gontrode (Belgium, red) and Zvenigorod (Russia, blue). The p-value indicating the statistical significance of the differences is displayed between the two corresponding box plots.\label{fig3.1}}
		\end{footnotesize}
	\end{figure}
\clearpage

\renewcommand{\thefigure}{2.\arabic{figure}}
	\begin{figure}[h!]
		\begin{center}
\small
			\includegraphics[width=0.9\textwidth]{figures/FigureACM.jpg}
		\end{center}
		\begin{footnotesize}
			\caption{\footnotesize . Relationships between (log) BMR and (log) M$_{\text{sum}}$ in Gontrode, Belgium (A) and Zvenigorod, Russia (B) and between mass-independent BMR and mass-independent M$_{\text{sum}}$ in Gontrode (C) and Zvenigorod (D). After removing the bird with the lowest mass-dependent M$_{\text{sum}}$ in (D), the statistic changed to r=-0.64 and p-value=0.14.\label{fig3.2}}
		\end{footnotesize}
	\end{figure}
\clearpage

\begin{table}[!ht]
    \centering
\small
\caption*{\textbf{Table 2.2}: Ambient temperature in Gontrode and Zvenigorod across the study periods. Note: Mean ± standard deviation of ambient temperatures (T$_{\text{a}}$) in both locations during the study periods.}
\begin{adjustbox}{max width=\textwidth}    
    \begin{tabular}{ccccc}
    \hline
        ~ & ~ & January & February & March \\ \hline
        Gontrode & 2022 & - & 6.5 ± 1.8 °C & 9.3 ± 2.9 °C \\ 
        Zvenigorod & 2021 & -6.6 ± 7.1 °C & -11.6 ± 7.0 °C & -2.5 ± 6.8 °C \\ 
        ~ & 2022 & -6.1 ± 3.7 °C & -1.9 ± 3.4 °C & -2.9 ± 5.5 °C \\ 
        ~ & 2023 & -5.5 ± 7.6 °C & -5.0 ± 4.4 °C & -0.1 ± 5.0 °C \\ \hline
    \end{tabular}
\end{adjustbox}
\end{table}
\clearpage

\section{Discussion}
Here, we test for intraspecific variation in ecophysiological traits related to thermoregulation using great tits living in two geographically and climatically separate locations that experience different winter conditions. Observed differences between populations suggest that avian basal and summit metabolic rates may vary independently in response to environmental influences, as great tits from the colder site (Zvenigorod, Russia) had significantly higher thermogenic capacity (i.e., M$_{\text{sum}}$) than those from the warmer site (Gontrode, Belgium), but a lower basal metabolic rate (BMR). Contrary to the prediction of the aerobic capacity model, we find only weak support for a functional relationship between these metabolic rates at the individual level as (mass-independent) BMR and M$_{\text{sum}}$ were uncorrelated. \\

During winter, continental high-latitude areas are characterized by low temperatures, limited food resources, shorter foraging periods, and extended fasting periods during the long nights. Physiological adaptations have long been hypothesized to be key to enduring such challenging environments (Swanson and Olmstead, 1999). Indeed, more recently, Petit et al. (2017), for example, found that winter survival of black-capped chickadees (\textit{Poecile atricapillus}) in Quebec, Canada, was positively associated with M$_{\text{sum}}$. However, the relationship between winter survival and M$_{\text{sum}}$ was not linear, but exhibited a threshold function, and many individuals had M$_{\text{sum}}$ values significantly higher than required for survival. Petit et al. (2017) suggested that this increase in M$_{\text{sum}}$ may be because of selection for increased muscle mass to power the sustained foraging flights required to gather sufficient food in the generally resource-poor winter habitat, with the thermogenic effects providing a secondary benefit. Such an explanation for high M$_{\text{sum}}$ is less likely here, as higher muscle mass should correspond to higher overall energetic maintenance costs, which was not confirmed by our data. Moreover, birds from the colder Zvenigorod site were characterized by lower whole-body BMR than birds from Gontrode, despite having higher body mass. This suggests that the differences in mass between the populations are not because of a higher muscle mass, but rather that the Zvenigorod birds may have a greater amount of metabolically (relatively) inactive fat mass (Scott and Evans, 1992). Indeed, the accumulation of fat reserves during winter, known as winter fattening, is a well-known mechanism observed in small passerines to survive colder climates and longer nights (King and Farner, 1966; Pravosudov and Grubb, 1997).\\ 

The Zvenigorod population thus has a higher metabolic expansibility (whole-body M$_{\text{sum}}$/BMR; ME: ~6 times the BMR than ~4 times for the Gontrode population), which is commonly used as an indicator of the organism's ability to produce heat for a given level of metabolic maintenance cost (Arens and Cooper, 2005; Cooper and Swanson, 1994). As we found no support for the aerobic capacity model of endothermy, which postulates a positive correlation between minimum (BMR) and maximum (M$_{\text{sum}}$) aerobic metabolic rates (Bennett and Ruben, 1979), it remains unclear what mechanisms underlie the summit metabolic capacity of our study birds. The aerobic model has been supported primarily by interspecific studies (Dutenhoffer and Swanson, 1996; Rezende et al., 2002), but these relationships often disappear at the individual level, at least after accounting for variation in body mass (Swanson et al., 2012; Vézina et al., 2006). The reasons for the differences between inter- and intraspecific studies of these phenotypic correlations are not yet fully understood. On the one hand, this lack of correlations may represent a statistical artefact associated with much lower levels of variation in body mass and metabolic rates in intraspecific studies (Swanson et al., 2017). On the other hand, Swanson et al. (2023) carried out a literature review investigating flexibility in BMR, M$_{\text{sum}}$ and metabolic expansibility, and found that, in fact, for none of the six species for which data were available, higher levels of flexibility in M$_{\text{sum}}$ or metabolic expansibility did not typically result in increased maintenance costs (i.e., BMR). This suggests that non-shivering thermogenesis mechanisms may also contribute significantly to thermoregulation in birds (Pani and Bal, 2022). Indeed, multiple adaptations at the cellular and biochemical level have been shown to affect an organism's thermogenic capacity when exposed to changing temperatures (Milbergue et al., 2018; Swanson, 2010). For example, Nord et al. (2021) recently showed that resident great tits in western Scotland were able to upregulate mitochondrial respiration rate and mitochondrial volume in winter, thereby increasing thermogenic capacity at the subcellular level. Further intraspecific studies should consider measuring and accounting for the body composition of individual birds, in particular the relative contributions of fat and muscle mass to the observed body mass differences, as well as blood sampling to investigate the possible role of non-shivering thermogenesis. \\

Alternatively, the decoupling of BMR and M$_{\text{sum}}$ often observed in intraspecific studies may be caused by different selective forces acting on these metabolic rates (Petit et al., 2013), with birds from the Zvenigorod population facing the need to conserve energy (by keeping their BMR at comparatively low levels) while simultaneously increasing their thermogenic capacity (by increasing their M$_{\text{sum}}$). Swanson et al. (2017) argues that, from an energetic point of view, natural selection should generally favor reducing BMR to the lowest possible level under prevailing environmental or ecological demands, allowing energy to be allocated to other functions. Bozinovic and Sabat (2010) similarly suggested that in resource-poor habitats, organisms that can reduce their BMR will reduce their daily energy expenditure and hence food requirements, thereby increasing fitness through increased survival. Broggi et al. (2005) used a common garden experiment to investigate the physiological basis of interpopulation differences in BMR in Scandinavian great tits, and found that selection pressure for low BMR was particularly strong in more northerly populations, where the energetic costs of thermoregulation and activity are highest. More recently, a literature review by Stager et al. (2016) suggested that latitudinal trends in metabolic rate are primarily driven by a necessary balance between increased thermogenic capacity to cope with cold temperatures and pressure to reduce excess maintenance costs in cold and low food availability environments. \\

Our results are therefore consistent with the expectation that birds in resource-poor, cold environments will be characterized by low BMR but high M$_{\text{sum}}$. Gontrode, on the other hand, has a maritime climate that rarely reaches freezing temperatures, and the birds probably had access to abundant and stable food sources during the winter, as the forest is surrounded by a residential area with several gardens equipped with bird feeders (Dekeukeleire, 2021). However, great tits in Gontrode had a lower body mass than their counterparts in Zvenigorod. We argue that this difference is probably mainly due to the different energy reserve strategies used by the birds in response to their respective environments. It is possible that birds in Gontrode perceive food availability as abundant and stable, making them less inclined to store more energy. Zvenigorod birds may have taken the opposite approach, building up larger fat reserves as a mechanism to withstand the harsher winter conditions. Ambient air temperatures at Gontrode in February and March averaged 6.5 ± 1.8 °C and 9.3 ± 2.9 °C respectively (Table 2.2), compare with a normal average temperature of about 4.2 °C for February, and about 7.1 °C for March (Pacioni et al., 2023b). Due to these comparatively warm temperatures, relatively few great tits were found roosting in nest boxes, further suggesting that thermoregulatory demands and starvation risks were less severe for Gontrode birds, weakening selection for lower BMR in this population.\\

\section{Conclusions}
In conclusion, great tits from the colder site (Zvenigorod, Russia) had significantly higher thermogenic capacity (i.e., M$_{\text{sum}}$) than those from the warmer site (Gontrode, Belgium), but a lower basal metabolic rate (BMR). This contradicts the aerobic capacity model, showing that avian basal and summit metabolic rates may vary independently in response to environmental influences. The coupling or uncoupling of minimum and maximum metabolic rates at the intraspecific level may then be influenced by different selective pressures that shape local adaptations in response to different degrees of seasonality. Therefore, future intraspecific studies should consider the potential impact of local adaptations and selective pressures on different conspecific populations when testing the applicability of the aerobic capacity model. These findings would contribute to a better understanding of the adaptive strategies used by birds in different environments and highlight the importance of considering multiple factors when studying and comparing avian metabolic rates.
\clearpage
\subsection*{Acknowledgements}
We express our gratitude to ForNaLab (Gontrode) for giving us access to their facilities. We extend special thanks to Luc Willems for his invaluable support and to Dries Landuyt for providing us with the temperature data from the dataloggers (Gontrode). We are grateful to the administration of the Zvenigorod biological station for providing the necessary conditions for the study. Furthermore, we would like to acknowledge the valuable assistance provided by bachelor student Hanne Danneels during the data collection process. This study acknowledges funding by FWO-Vlaanderen (project G0E4320N), by Methusalem Project 01M00221 (Ghent University) awarded to Frederick Verbruggen, Luc Lens, and An Martel, and by the Russian Science Foundation (project 20-44-01005). Marina Sentís acknowledges the support of FWO-Vlaanderen (project 11E1623N).
\subsection*{Data availability}
Script and data are made available on Mendeley Data (DOI: 10.17632/t84ntxxcbn.1)













%%%%%%%%%%%%%%%%%%%%%%%%%%%%%%%% CHAPTER FOUR  %%%%%%%%%%%%%%%%%%%%%%%%%%%%%%%%%%%

\setlength{\thumbwidth}{0.8cm}
\setlength{\thumbheight}{1cm}
\tikzset{
	thumb/.style={
		%   draw=black,
		fill=light-gray,
		text=black,
		minimum height=\thumbheight, %\thumbheight,
		text width=\thumbwidth,
		outer sep=0pt,%   outer sep=10pt,
		font=\sffamily\Large,
	}
}
\pagestyle{mainmatter}
\renewcommand\thesection{\arabic{chapter}.\arabic{section}}
\renewcommand{\thefigure}{\arabic{chapter}.\arabic{figure}}
\chapter{Metabolic responses to cold: thermal physiology of native common waxbills (\textit{Estrilda astrild})}\label{chapter3}
\chaptermark{Chapter 3}
\lettergroup{\thechapter}	

\begin{flushright}
        \textcolor{black}{Cesare Pacioni}\\
\textcolor{black}{Marina Sentís}\\
    \textcolor{black}{Anvar Kerimov}\\
    \textcolor{black}{Andrey Bushuev}\\
    \textcolor{black}{Colleen T. Downs}\\
    \textcolor{black}{Luc Lens}\\
    \textcolor{black}{Diederik Strubbe}


\vspace*{2cm}
  \textcolor{black}{This chapter is available as a preprint on bioRxiv: https://doi.org/10.1101/2024.01.18.576192}

\end{flushright}
\clearpage

\section{Abstract}

Ecophysiological studies of invasive species tend to focus on captive individuals and their invasive range. However, the importance of gaining a thorough understanding of their physiology in their native range, where autecological knowledge is limited, has played a crucial role in assessing species ecophysiological responses to the often novel environmental conditions encountered in their introduced range. Here, we investigated the ecophysiological characteristics of a population of the wild-caught common waxbill (\textit{Estrilda astrild}), a successful global invader, in part of its native range (South Africa). We investigated how this species adjusts its resting metabolic rate over a range of temperatures to identify its thermoneutral zone (TNZ) as an indicator of a species' thermal tolerance. The observed TNZ curve predominantly followed the classic Scholander-Irving model, with metabolic rates increasing linearly at temperatures outside the TNZ. However, we found an inflection point at moderately cold temperatures (16°C) where the common waxbill began to decrease its metabolic rate. This finding highlights the potential use of an energy-saving strategy as an adaptive response to cold, such as facultative hypothermic responses through a reduction in body temperature, which may explain their success as an invasive species. We argue that although metabolic infection points have been repeatedly identified in studies of TNZ, the specific mechanisms behind metabolic down-regulation at low temperatures remain underexplored in the literature. We therefore suggest that future research should focus on investigating body temperature variation, with particular emphasis on its potential contribution to metabolic adaptation in colder environments.



\vspace*{\fill}
\noindent \textbf{Keywords:} Ecophysiology; Thermoneutral zone; Common waxbill
\clearpage

\section{Introduction}
Global climate change, characterized by rising temperatures and more frequent extreme weather events, is putting up to one out of six species at risk of extinction as they face unprecedented challenges in responding to rapid environmental change (Urban, 2015). Furthermore, the introduction of alien species poses a significant additional threat to global biodiversity, ecosystem services and economic resources (Bongaarts, 2019; Castro‐Díez et al., 2019; Diagne et al., 2021; Pyšek et al., 2020; Shirley and Kark, 2009). According to the latest IPBES Invasive Alien Species Assessment (2023), at least 37.000 established alien species have been introduced by human activities, of which 3.500 are also invasive. Changes in global temperature and precipitation regimes are likely to amplify the impacts of invasive species (Dukes and Mooney, 1999; Walther et al., 2009), for example, by allowing populations of introduced species that are not presently invasive to become invasive (i.e., spread and cause impacts; Hellmann et al., 2008; Mainka and Howard, 2010). For both the effective conservation of native species and the management of invasive alien species, it is crucial to accurately assess how species will respond to new environmental conditions. In this context, knowledge of the physiological mechanisms used by species to adapt to changing climatic conditions can greatly improve ecological predictions of species range dynamics (Huey et al., 2021).\\

Understanding how animals thermoregulate to buffer unfavorable thermal fluctuations in their environment, and what their thermoregulatory limits are, have long been central questions in animal ecophysiology (Bozinovic et al., 2011). Thermoregulation in endotherms is generally non-linear and is thought to follow the classic Scholander-Irving model of endothermic homeothermy (Scholander et al., 1950), which includes a thermoneutral zone (TNZ). The latter defines the range of ambient temperatures within which an endotherm can maintain its body temperature without increasing its metabolic rate above the basal metabolic rate (BMR) required to maintain basic life-sustaining functions (McNab, 2012). The TNZ of a species is characterized by two critical temperatures: the lower critical temperature (LCT) and the upper critical temperature (UCT). Both mark the point at which the metabolic rate begins to rise above the BMR to maintain normothermia. Several studies have used the concept of TNZ as an indicator of a species' long-term ability to tolerate thermal variation. For example, Khaliq et al. (2014) showed that around 15\% of bird species presently experience maximum ambient temperatures above their UCT, and this rises to over 35\% under climate change scenarios. This suggests that birds will increasingly face thermoregulatory constraints on fitness-related traits, such as activity levels, reproduction, and survival. For example, Milne et al. (2015) studied 12 bird species in South Africa and identified potential links between climate warming and population declines in several passerine species associated with fynbos habitat, such as the Cape rockjumper (\textit{Chaetops frenatus}), and attributed these declines to the birds' limited tolerance to higher temperatures.\\

A better understanding of thermoregulatory strategies and capacities can also contribute to more accurate risk assessment of invasive species, as they often expand their niche into novel climates (Liu et al., 2022). For example, ring-necked parakeets (\textit{Psittacula krameri}) have successfully colonized many European cities, which are considerably colder than their native range (Strubbe et al., 2015). The same applies to several members of the family Estrildidae, which is notable for establishing numerous invasive bird populations (Stiels et al., 2015). This group of small, finch-like birds comprises 138 species (Winkler et al., 2020) and occupies diverse habitats across Africa, southern Asia, and Australasia, with the highest species concentrations occurring in the tropics (Goodwin, 1982). While several estrildid finches are classified as regional agricultural ‘pests’ (Gleditsch and Brooks, 2020), they are widely traded and have been described as the ‘single most important avicultural family’ (Ribeiro et al., 2020). With 81 introduced species worldwide (Dyer et al., 2017), it is considered the most successful non-native family of birds among tropical passerines (Lever, 2005), with most of the species traded as pet birds (Cardoso and Reino, 2018; Reino et al., 2017). \\

Due to their widespread popularity as pet cage birds (Reino et al., 2017), most studies on estrildid physiology have focused on captive individuals, while relatively few have examined the ecophysiology of free-ranging individuals (Allen and Hume, 2001; Cooper et al., 2019, 2020; Gerson et al., 2019; Sheldon and Griffith, 2018). Pacioni et al. (2023a) and Sentís et al. (2023) studied the energetic metabolism of captive common waxbills (\textit{Estrilda astrild}, 7-9 g), but no further information is available on the free-ranging individuals. Two closely related Aftrotropical estrildid finches, which also have established non-native populations (Ascensão et al., 2021), have also only been studied in captivity. In a broader study, Marschall and Prinzinger (1991) studied the thermal physiology of five estrildid species, including the orange-cheeked waxbill (\textit{E. melpoda}), and showed that each of these birds had different physiological adaptations to their specific habitats in different tropical climates. Cade et al. (1965) and Lasiewski et al. (1964) studied the physiology of captive black-rumped waxbills (\textit{E. troglodytes}), but found contrasting results regarding the species' TNZ. Moreover, Lasiewski et al. (1964) also claimed that when properly fasted, waxbill metabolic rates measured during the day (in darkened metabolic chambers) and during the night were similar. Stephens et al. (2001) came to the same conclusion for captive orange-cheeked waxbills. These findings contrast with review studies (e.g., Aschoff and Pohl, 1970) which suggest that daytime avian metabolic rates are on average 20-25\% higher than nighttime rates, highlighting the need for further investigation of diurnal variation in metabolic rates in estrildid species. \\

Studying invasive species solely in captivity or in their invaded range might overlook crucial information about their ability to adjust to specific environmental conditions (Boardman et al., 2022). For example, it has been shown that a broader understanding of species in their native distribution range is essential for assessing how species may respond to the often novel environmental conditions they encounter in their introduced distribution range (Stuart et al., 2023). However, in contrast to the increasing knowledge of invasive species in their non-native range, autecological knowledge of many invasive species in their native range is still minimal (Ros et al., 2016). The same applies for the common waxbill, for which the distribution and dispersal of well-studied invasive populations in Iberia have been found to be strongly influenced by climate and habitat gradients (Sullivan and Franco, 2018), but accurate prediction of invasion risk is difficult (Stiels et al., 2015). In an attempt to improve accuracy, Strubbe et al. (2023) used thermal physiological approaches to model the species' potential European distribution range expansion, but faced challenges in inferring key functional traits because of a lack of empirical data from wild individuals, forcing them to rely on allometric predictions. Moreover, while experiments on captive birds have value, the non-natural biotic and abiotic conditions to which animals are exposed can lead to important ecophysiological changes (Beaulieu, 2016). Prioritizing research on wild animals, as well as on their native distribution range, is therefore crucial for gaining ecologically relevant insights into species' potential for distributional expansion. Therefore, here we investigated the ecophysiological characteristics of wild-caught common waxbills in a part of their native range (South Africa). Specifically, we assessed how this species adjusts its ρ- (resting) phase metabolic rate over a range of temperatures to identify its thermoneutral zone. We also investigated α- (active) phase metabolic rate patterns, predicting that (i) fasted birds would have a lower resting metabolic rate than non-fasted birds, and (ii) the α-phase metabolic rates of fasted birds would be higher than the ρ-phase metabolic rates. By filling this gap in the understanding of how wild-caught common waxbills in a part of their native distribution range physiologically adjust to temperature variation, our research will provide crucial data that could potentially be incorporated into models aimed at predicting invasion dynamics and associated risks.\\

\section{Material and methods}

\subsection{Study area and bird capture and maintenance}

The study was conducted in South Africa (Pietermaritzburg, KwaZulu-Natal) near the Darvill Wastewater Treatment Site (N -29.60, E 30.43). Between September 30 and November 4 (2022), free-ranging common waxbills were attracted by song playback and captured using mist nets. Birds with a brood patch were released immediately. In total, 42 adult waxbills were withheld for respirometry analyses. Upon capture, these birds were tagged with colored plastic rings for individual identification, aged, sexed, and then brought to the Animal House of the University of KwaZulu-Natal, Pietermaritzburg, where they were housed in outdoor aviaries (1 x 3 x 2 m) with shelter from sun and rain and with perches provided. Food (finch mix and millet) and water were provided ad libitum. Measurements of metabolic rate in the nocturnal ρ-phase started on the same day as the birds were captured. Birds were kept in the aviary for an average of 10 days and released at the original capture site. This work was carried out under ethical permit 'VIB EC2022-090', SAFRING permit No. '0163' and Ezemvelo KZN Wildlife permit 'OP 475'. \\

\subsection{ρ- (resting) phase metabolic rate}
At sunset (19h00), after being weighed to the nearest 0.1 g, birds were placed in a 1.1 L airtight plastic chamber within a temperature-controlled environmental chamber (CMP2244, Conviron, Winnipeg, Canada) for ρ-phase metabolic rate measurement. The six temperatures tested were 12, 16, 20, 24, 28, and 31.5°C. Each plastic chamber contained a chicken-wire mesh to ensure a normal sleeping posture. Metabolic rate was estimated using flow‐through respirometry (Lighton, 2018) by measuring O$_{\text{2}}$ consumption (VO$_{\text{2}}$; ml/min). Up to seven birds were measured during the same night. Ambient air was supplied by two pumps and divided into separate streams directed to a mass-flow meter (FB-8, Sable Systems) to provide a constant flow of ~650 ml/min. Excurrent air from the bird and the baseline channels was alternately subsampled and passed through a Field Metabolic System (FMS-3, Sable Systems). Birds were measured alternately in cycles along with multiple baselines. The time of measurement for each bird within a cycle (and the length of each cycle) depended on the number of birds within a session. The average measurement time was 11 h. The first 2 h were discarded to ensure that birds were post-absorptive. After the respirometry measurement (06h00), the birds were weighed again to the nearest 0.1g and returned to the aviary. Some individuals were measured more than once at certain temperatures, resulting in a dataset with repeated measurements. Further details of the respirometry setup can be found in Pacioni et al. (2023a).\\

\subsection{α- (active) phase metabolic rate}
To determine α-phase metabolic rates, birds were first weighed to the nearest 0.1g, and then placed in a 1.1 L airtight plastic chamber within the temperature-controlled environmental chamber (see above). The temperature in this chamber was maintained at 28°C (which is within the TNZ of the South African waxbill, see below). Each day, four birds were measured using the same respirometry set up described above, in 5-min cycles, for a total of 10 cycles per bird. The birds were kept in the metabolic chambers from 14h00 to 18h00. After the respirometry measurements, the birds were weighed again to the nearest 0.1g and returned to the aviary. The same four birds were not selected for the ρ-phase metabolic rates on that night to reduce potential stress. \\

\subsection{Respirometry and data analyses}
We used ExpeData software (Sable Systems) to record each experimental trial and extract metabolic rate values (ml O$_{\text{2}}$/min). To estimate both the ρ-phase and the α-phase metabolic rates, the lowest stable section of the curve (averaged over 5 min) was selected using equation 9.7 from Lighton (2018).\\

The effect of ambient temperature (12, 16, 20, 24, 28, and 31.5°C) on ρ-phase metabolic rates was analyzed using a generalized additive mixed model (GAMM), with individual bird ID specified as a random effect ('gamm' function from ‘mgcv’ R-package; Wood, 2023). A segmented regression model was applied to identify points at which the relationship between ρ-phase metabolic rates and temperature may change ('segmented' R-package; Muggeo, 2008).\\

Changes in α-phase metabolic rates (28°C) were tested by quantifying the variation in metabolic rate over time using measurements taken at 5-min intervals over 4 h. A GAMM was used to identify temporal trends in α-phase metabolic rates at 28°C. The inflection point of the GAMM curve was identified to delineate the period distinguishing 'fasted' and 'non-fasted' stage. Consequently, linear regression models (‘lm’ function) with a Gaussian error distribution were used to investigate whether the metabolic rates during the ρ-phase (28°C) differed significantly from those during the fasted α-phase (28°C). When repeated measurements per individual were available, the lowest metabolic rate per individual was selected. The GAMM ('gamm' function) incorporated a basis dimension (k) of 3 as a value for the smooth terms in the GAMM fits to capture relationships no more complex than unimodal, and included individual ID as a random effect (Wood, 2023). \\

Sex was included as an independent factor in all models. Models were first run using whole-body metabolic rates and then using mass-independent metabolic rates, determined as the residuals derived from regressions of (log-transformed) ρ-phase and (log-transformed) α-phase metabolic rates on (log-transformed) body mass. Three different measures of body mass were considered in the analysis: body mass before and after the metabolic rate measurements, and the mean of these two measures. As consistent results were obtained for all three measures, the body mass measured after the metabolic measurements was used in all subsequent analyses. Variables and their interaction that were not statistically significant were removed using a backward stepwise procedure. Given that birds were caged for several days, number of days since capture was included as a fixed covariate to account for the potential effect of captivity. Interquartile ranges were used as a criterion to identify outliers (14 outliers were identified in the α-phase metabolic rates dataset), using the 'quantile' function in R. Shapiro-Wilk tests were performed to check the normality of the residuals (Shapiro-Wilk W > 0.9). The significance level was set at $p \leq 0.05$. Body mass, ρ-phase, and α-phase metabolic rates were log transformed before analyses. All statistical analyses were performed using R software v. 4.2.2 (R Core Team 2022), details of which are available in the RMarkdown HTML file (https://doi.org/10.6084/m9.figshare.25020026.v1).\\

\section{Results}

\subsection{ρ-phase metabolic rates}
The values for ρ-phase metabolic rates are summarized in Table 3.1. Birds exhibited their lowest whole-body and mass-independent rates when exposed to an ambient temperature of 28°C, and showed an increase in metabolic rate below this temperature (whole-body, Figure 3.1a; mass-independent, Supplementary information Figure 3.3a). As the temperature decreased to ~16°C (inflection point), metabolic rates also began to decrease (Figure 3.1b and Supplementary information Figure 3.3b).\\

\subsection{α-phase metabolic rates}
GAMM analyses supported a decline in metabolic rate over time, for both whole-body (p < 0.001; Figure 3.2) and mass-independent (p < 0.001; Figure 3.4) measurements, visually plateauing off at ~120 min (Figure 3.2 and Supplementary information Figure 3.4). Whole-body α-phase metabolic rates (fasted birds only) were on average 48\% higher than whole-body ρ-phase metabolic rates (28°C), and this difference was also significant for both whole-body (p > 0.05, Figure 3.1a) and mass-independent (p > 0.05, Supplementary information Figure 3.3A) measurements. There was no evidence of a positive correlation between ρ-phase metabolic rate (28°C) and α-phase metabolic rate (fasted birds only). Neither sex nor number of days since capture were significant (all p > 0.10).\\

\clearpage

\begin{table}[!ht]
    \centering
\small
\caption*{\textbf{Table 3.1}: \textbf{ρ- (resting) phase metabolic rate}. Mean ± SD (standard deviation), minimum-maximum values, and sample size (n) of the ρ- (resting) phase whole-body metabolic rates (ml O$_{\text{2}}$/min) for various temperatures.}
\begin{adjustbox}{max width=\textwidth}    
    \begin{tabular}{cccc}
    \hline
        Temperature (°C) & Mean ± SD & Min-Max & n \\ \hline
        ~ & (ml O$_{\text{2}}$/min) & (ml O$_{\text{2}}$/min) & ~ \\ 
        12 & 1.08 ± 0.21 & 0.68 - 1.55 & 27 \\ 
        16 & 1.13 ± 0.19 & 0.83 - 1.63 & 27 \\ 
        20 & 1.15 ± 0.28 & 0.69 - 1.76 & 29 \\ 
        24 & 1.04 ± 0.36 & 0.36 - 1.76 & 28 \\ 
        28 & 0.92 ± 0.31 & 0.50 - 1.53 & 49 \\ 
        31.5 & 0.99 ± 0.40 & 0.42 - 1.60 & 19 \\ \hline
    \end{tabular}
\end{adjustbox}
\end{table}

\clearpage


\renewcommand{\thefigure}{3.\arabic{figure}}
	\begin{figure}[h!]
		\begin{center}
\small
			\includegraphics[width=1\textwidth]{figures/CombinedPlot.jpg}
		\end{center}
		\begin{footnotesize}
			\caption{\footnotesize A) Boxplots of whole-body ρ- (resting) phase metabolic rate (ml O$_{\text{2}}$/min) of wild common waxbill (\textit{Estrilda astrild}) at various ambient temperatures. (Note: The blue filled circle represents the mean value of the α- (active) phase metabolic rates measured at 28°C. Boxplot whiskers extend to the minimum or maximum value within 1.5 times the interquartile range). B) The whole-body ρ- (resting) phase metabolic rate (MR ml O$_{\text{2}}$/min) at various temperatures. (Note: The vertical red dashed line indicates the inflection point at 16.6°C. The shaded band indicates the 95\% confidence interval around the segmented regression line, highlighting the region where there is 95\% confidence in the true regression line).\label{fig4.1}}
		\end{footnotesize}
	\end{figure}

\clearpage

\renewcommand{\thefigure}{3.\arabic{figure}}
	\begin{figure}[h!]
		\begin{center}
\small
			\includegraphics[width=1\textwidth]{figures/Figure 4.jpg}
		\end{center}
		\begin{footnotesize}
			\caption{\footnotesize Whole-body α- (active) phase metabolic rates (ml O$_{\text{2}}$/min) of common waxbill (\textit{Estrilda astrild}) over the 4 h fasting period. (Note: The shaded ribbon represents the 95\% confidence interval).\label{fig4.1}}
		\end{footnotesize}
	\end{figure}

\clearpage

\section{Discussion}
The observed thermoneutral zone (TNZ) curve predominantly follows the classic Scholander-Irving model, where endotherms maintain a stable basal metabolic rate (BMR) within a range of ambient temperatures, but must produce additional metabolic energy to keep their body temperature constant when the ambient temperature falls below or rises above it. Common waxbill metabolic rates were lowest at an ambient temperature of 28°C. Below this temperature, ρ-phase (resting) metabolic rates increased linearly, until an inflection point at around 16°C where birds began to decrease their ρ-phase metabolic rates. We also found that α- (active) phase metabolic rates (fasted) were 48\% higher than ρ-phase metabolic rates (28°C), and that whole-body and mass-independent α-phase metabolic rates showed an unimodal decrease over time, plateauing off at ca 120 min.\\

Although the specific limits of the TNZ could not be precisely defined because of a temperature difference of 4°C between the ambient temperatures tested, it is reasonable to conclude that the lower value of the TNZ (i.e., the lower critical temperature) for common waxbills fell approximately around 28°C. This was consistent with findings from captive individuals of this species (Pacioni et al., 2023a). As temperatures fell below this point, we observed an increase in both whole-body and mass-independent ρ-phase metabolic rates. However, the curve showed an inflection point at around 16°C, where both whole-body and mass-independent ρ-phase metabolic rates began to decrease. This may suggest adaptive mechanisms, such as facultative hypothermic responses, to conserve energy in colder conditions (Geiser, 2021; McKechnie and Lovegrove, 2002). Facultative hypothermia, such as resting phase hypothermia, refers to the adaptive ability to reduce body temperature in response to environmental conditions (e.g., cold temperatures), with a 30-40\% reduction in metabolic rate (Reinertsen, 1983). This phenomenon has been observed in species spanning 16 different orders and 31 families of birds (Ritchison, 2023), some of which have a small body mass (Ruf and Geiser, 2015). For example, small passerines (i.e., blue tit \textit{Cyanistes caeruleus}, willow tit \textit{Poecile montanus}, and black-capped chickadee \textit{Poecile atricapillus}) have demonstrated the ability to reduce their body temperature by several degrees after prolonged exposure to cold conditions as an energy-saving strategy to increase winter survival (Brodin et al., 2017). However, a study by Andreasson et al. (2019) found that juvenile great tits (\textit{Parus major}) and juvenile blue tits were less likely to adopt hypothermia as an energy-saving strategy at very low ambient temperatures ($\sim$-8°C) when faced with an increased perceived predation risk. Similarly, Cooper and Gessaman (2005) showed that mountain chickadees (Poecile gambeli) and juniper titmice (\textit{Baeolophus ridgwayi}) used more energy to reduce their body temperature when the ambient temperature was exceptionally low ($\sim$-10°C). As the temperatures examined in our study were not extremely low, a potential hypothermic mechanism is likely to be energetically advantageous for our birds.\\

Numerous studies have shown a breakdown of the Scholander-Irving model at lower temperatures, with metabolic rates deviating from linearity at lower temperatures (Noakes et al., 2013; Nzama et al., 2010; Steiger et al., 2009; Thabethe et al., 2013; van de Ven et al., 2013). However, these studies did not discuss the precise mechanism underlying this metabolic downregulation at lower temperatures. As we did not monitor body temperature during our measurements, it is important to acknowledge this limitation of our study. Without direct measurements of body temperature, our understanding of the observed inflection point in our TNZ curve is limited, leading to speculation and highlighting a potential avenue for future research into the relationship between body temperature adaptations and responses to colder temperatures in our study species.\\

As the physiological ability to adapt to low temperatures has been shown to influence the geographic distribution of endothermic organisms (Khaliq et al., 2017; Hayes et al., 2018), range distribution expansion may be limited by physiological constraints, such as those associated with the ability to sustain energetically elevated rates of thermogenesis (heat production) for long periods (Buckley et al., 2018). Indeed, although cold temperatures outside the TNZ are not directly lethal to the bird itself, the prolonged maintenance of high metabolic rates, and hence high energy requirements, may be constrained by environmental factors (e.g., food availability). This is particularly relevant for invasive species such as the common waxbill, which tends to occupy colder climates in its invasive distribution range compared with its native distribution range (Stiels et al., 2011). Lasiewski et al. (1964) and Stephens et al. (2001) conducted comparable studies on black-rumped waxbill and orange-cheeked waxbill, both of which have invasive populations in Iberia along with the common waxbill. These studies examined the TNZ of these species at ambient temperatures similar to or lower than those in our study. Their results showed a linear up-regulation of metabolic rate when the ambient temperatures were below the TNZ for both species. This evidence suggests that these two species may lack the ability to limit energy expenditure at colder temperatures and may explain why the common waxbill is a more successful invasive species in Iberia than its counterparts (Ascensão et al., 2021).\\

In addition to TNZ, we also examined the patterns of α-phase (active) metabolic rates and found significant differences between the (whole-body and mass-independent) ρ-phase (28°C) and the α-phase (28°C, fasted birds only) metabolic rates. Specifically, during the α-phase, the whole-body metabolic rates of the fasted birds were significantly higher by 48\% than during the ρ-phase. These results do not support the conclusion of Lasiewski et al. (1964) and Stephens et al. (2001) that metabolic rates during ρ-phase and α-phase are generally not different in small birds. Furthermore, a recent study by Ellis and Gabrielsen (2019) argued that both α-phase and ρ-phase metabolic rates represent BMR. Other studies that have found evidence of a circadian rhythm in metabolic rates have typically reported smaller differences. For example, McKechnie and Lovegrove (1999) found that the ρ-phase BMR was approximately 20\% lower than the α-phase BMR in fed black-shouldered kites (\textit{Elanus caeruleus}). Daan et al. (1989) found that, in kestrels (\textit{Falco tinnunculus}), the metabolic rate during the α-phase was 22-27\% higher than the metabolic rate during the ρ-phase. However, these studies involved metabolic measurements lasting for 24 h. The longer measurement period may have resulted in a lower percentage increase in metabolic rate during the α-phase compared with the 4 h measurements we performed. This difference could be attributed to potential acclimatization effects, where birds might adjust to the laboratory conditions over time, leading to a lower metabolic response during the α-phase. The duration of exposure may then influence the degree to which birds modulate their metabolic rates, highlighting the importance of considering the duration of measurements when understanding circadian rhythm-related variations in avian energetics. \\

Finally, we observed an effect of fasting on the α-phase metabolic rates, with a statistically significant decrease in metabolic rates from a non-fasted to a fasted state. This reduction in metabolism is likely to be a consequence of the energy expenditure associated with digesting and assimilating an ingested meal (Brody and Lardy, 1946; Rubner, 1902). Similarly, Cade et al. (1965) observed a decrease in CO$_{\text{2}}$ production in the black-rumped waxbill during the transition from fed to fasting conditions, with α-phase metabolic rate values at 3 h post-feeding approximately 30\% lower than initial values. Furthermore, in line with these results, we found that common waxbills reached a post-absorptive state at $\sim$120 min, as shown in Figure 3.2, where metabolic rate consumption started to plateau, with final metabolic rate values about 40\% lower than the initial values. These values suggest a rapid digestion process, consistent with other estrildid species (Cade et al., 1965).\\

\section{Conclusions}
Our results highlight the potential use of an energy-saving strategy by common waxbills as an adaptive response to cold, which may explain their success as an invasive species. Although metabolic inflection points have been repeatedly identified in studies of TNZ, the specific mechanisms underlying this metabolic down-regulation at low temperatures have not been investigated by many authors. Therefore, we suggest that future research should prioritize the study of body temperature variation, focusing on elucidating its potential contribution to metabolic adaptation to colder environments. Finally, we found that metabolic rates in the α-phase were significantly higher than those in the ρ-phase. Therefore, we do not support the conclusion of Lasiewski et al. (1964) and Stephens et al. (2001) that metabolic rates during α-phase and ρ-phase are generally not different in small birds.\\

\subsection*{Acknowledgements}
We thank Preshnee Singh and Ebrahim Ally (Centre for Functional Biodiversity, School of Life Sciences, University of KwaZulu-Natal) for their invaluable assistance during the fieldwork. This study was supported by the Research Foundation - Flanders (grant G0E4320 N) through a bilateral research collaboration with the Russian Science Foundation (grant 20-44-01005). In addition, the Methusalem Project 01M00221 (Ghent University), awarded to Frederick Verbruggen, Luc Lens and An Martel, contributed to this research. Marina Sentís acknowledges the support of FWO-Vlaanderen (project 11E1623N). In addition, Cesare Pacioni and Marina Sentís acknowledge the support of two FWO travel grants (V441722N and V441822N), which made this study possible. Colleen T. Downs was supported by the National Research Foundation (ZA, grant 98404) and the University of KwaZulu-Natal.\\

\subsection*{Data availability}
All datasets and R scripts (RMarkdown) not included in the manuscript, but used for analysis are archived on the public repository Figshare\newline (https://doi.org/10.6084/m9.figshare.25020026.v1)
 \newpage
\section{Supplementary material}

\renewcommand{\thefigure}{3.\arabic{figure}}
	\begin{figure}[h!]
		\begin{center}
\small
			\includegraphics[width=0.7\textwidth]{figures/CombinedPlot_MI.jpg}
		\end{center}
		\begin{footnotesize}
			\caption{\footnotesize A) Boxplots of mass-independent ρ- (resting) phase metabolic rate of the common waxbill (\textit{Estrilda astrild}) at various ambient temperatures. (Note: The blue filled circle represents the mean value of the α- (active) phase metabolic rates measured at 28°C. Boxplot whiskers extend to the minimum or maximum value within 1.5 times the interquartile range). B) The mass-independent ρ- (resting) phase metabolic rate at various temperatures. (Note: The vertical red dashed line indicates the inflection point at 16.6°C. The shaded band indicates the 95\% confidence interval around the segmented regression line, highlighting the region where there is 95\% confidence in the true regression line).\label{fig4.1}}
		\end{footnotesize}
	\end{figure}
\clearpage

\clearpage
\renewcommand{\thefigure}{3.\arabic{figure}}
	\begin{figure}[h!]
		\begin{center}
\small
			\includegraphics[width=1\textwidth]{figures/Figure 4_Mass-independent.jpg}
		\end{center}
		\begin{footnotesize}
			\caption{\footnotesize Mass-independent α- (active) phase metabolic rates of common waxbill (\textit{Estrilda astrild}) over the four hour fasting period. (Note: The shaded ribbon represents the 95\% confidence interval).\label{fig4.1}}
		\end{footnotesize}
	\end{figure}










%%%%%%%%%%%%%%%%%%%%%%%%%%%%%%%% CHAPTER FOUR  %%%%%%%%%%%%%%%%%%%%%%%%%%%%%%%%%%%

\setlength{\thumbwidth}{0.8cm}
\setlength{\thumbheight}{1cm}
\tikzset{
	thumb/.style={
		%   draw=black,
		fill=light-gray,
		text=black,
		minimum height=\thumbheight, %\thumbheight,
		text width=\thumbwidth,
		outer sep=0pt,%   outer sep=10pt,
		font=\sffamily\Large,
	}
}
\pagestyle{mainmatter}
\renewcommand\thesection{\arabic{chapter}.\arabic{section}}
\renewcommand{\thefigure}{\arabic{chapter}.\arabic{figure}}
\chapter{Great tits (\textit{Parus major}) in a west European temperate forest show little seasonal variation in metabolic energy requirements}\label{chapter4}
\chaptermark{Chapter 4}
\lettergroup{\thechapter}	

\begin{flushright}
        \textcolor{black}{Cesare Pacioni}\\
\textcolor{black}{Marina Sentís}\\
\textcolor{black}{Catherine Hambly}\\
\textcolor{black}{John R. Speakman}\\
    \textcolor{black}{Anvar Kerimov}\\
    \textcolor{black}{Andrey Bushuev}\\
    \textcolor{black}{Luc Lens}\\
    \textcolor{black}{Diederik Strubbe}


\vspace*{2cm}
  \textcolor{black}{Adapted from: Pacioni et al. (2023). \textit{J. Therm. Biol.}, 118, 103748.}

\end{flushright}
\clearpage


\section{Abstract}

Understanding how birds annually allocate energy to cope with changing environmental conditions and physiological states is a crucial question in avian ecology. There are several hypotheses to explain species' energy allocation. One prominent hypothesis suggests higher energy expenditure in winter due to increased thermoregulatory costs. The "reallocation" hypothesis suggests no net difference in seasonal energy requirements, while the "increased demand" hypothesis predicts higher energy requirements during the breeding season. Birds are expected to adjust their mass and/or metabolic intensity in ways that are consistent with their energy requirements. Here, we look for metabolic signatures of seasonal variation in energy requirements of a resident passerine of a temperate-zone (great tit, \textit{Parus major}). To do so, we measured whole-body and mass-independent basal (BMR), summit (M$_{\text{sum}}$), and field (FMR) metabolic rates during late winter and during breeding in Belgian great tits. During the breeding season, birds had on average 10\% higher whole-body BMR and FMR compared to winter, while their M$_{\text{sum}}$ decreased by 7\% from winter to breeding. Mass-independent metabolic rates showed a 10\% increase in BMR and a 7\% decrease in M$_{\text{sum}}$ from winter to breeding. Whole-body BMR was correlated with M$_{\text{sum}}$, but this relationship did not hold for mass-independent metabolic rates. The modest seasonal change we observed suggests that great tits in our temperature study area maintain a largely stable energy budget throughout the year, which appears mostly consistent with the reallocation hypothesis.



\vspace*{\fill}
\noindent \textbf{Keywords:} Great tit; Energy expenditure; Metabolic rates
\clearpage

\section{Introduction}

The way that birds allocate energy in order to cope with changes in environmental conditions (e.g., fluctuations in ambient temperature) and physiological states (e.g., reproductive seasonality) is a fundamental question in avian ecology. Energy expenditure is an important aspect of a species' ecology and can vary considerably during the annual cycle. Differential energy allocation during the annual life cycle, such as during breeding and wintering, is mostly a population-specific trait that depends largely on particular environmental and ecological conditions (McNab, 2012). Multiple, non-mutually exclusive hypotheses have been proposed to explain how and why populations adjust their energy allocation and metabolism to environmental variations. A prominent hypothesis predicts that energy expenditure should be highest in winter as harsher climates with colder ambient temperatures force animals to invest more in thermoregulation. Two other main hypotheses are the “reallocation hypothesis” and the “increased demand hypothesis” (Masman et al., 1988). The first of these hypotheses predicts a shift in energy expenditure from winter thermoregulation to reproductive activities, resulting in no net difference in seasonal energy requirements (Bryant and Tatner, 1988; Weathers and Sullivan, 1993). The increased demand hypothesis then predicts that energy demand should be highest during breeding, exceeding all other seasons (Gales and Green, 1990; Masman et al., 1988).\\

Birds are expected to respond to such seasonal energetic changes in required “work” or “activity” by changing their mass and/or metabolic intensity accordingly. For example, when faced with cold stress, birds can increase their pectoral muscle size and/or metabolic intensity for improved shivering thermogenic capacity (Swanson and Vézina, 2015) and improved cold tolerance, defined as the ability to withstand cold stress for prolonged periods (Cooper, 2002; Schweizer et al., 2022; Swanson, 2001). Such investments in muscle size and activity and concomitant changes in the gut and digestive organs that allow for higher daily food consumption (Nilsson, 2002) underlie correlated changes in maximal aerobic capacity (i.e., summit metabolic rate; M$_{\text{sum}}$) and maintenance metabolism (i.e., basal metabolic rate; BMR), as predicted by the aerobic capacity model (Bennett and Ruben, 1979).\\

Research on tropical birds tends to support the increased demand hypothesis, which fits expectations from life-history theory (Williams et al., 2010). Tropical birds are characterized by a slow ‘pace-of-life’, with comparatively long life spans and low annual reproductive investment (i.e., clutch size; Murray, 1985). Because of the warm, stable climate, thermoregulatory costs are constant, and breeding requirements are likely to drive seasonal variation in energy expenditure. Wells and Schaeffer (2012), for example, found that M$_{\text{sum}}$ was highest during the breeding season for seven resident bird species in Panama, while Jones et al. (2020) found the same for the black-headed nightingale-thrush (\textit{Catharus mexicanus}) in Honduras. Outside the tropics, avian reproductive investment is higher, but stronger climate seasonality in non-tropical regions leads to more variable thermoregulatory costs, and birds may face severe thermoregulatory demands, especially during the winter season (Swanson, 2010). Indeed, at higher latitudes, studies typically find that winter energy expenditure exceeds that of other seasons. For example, Carolina chickadees (\textit{Poecile carolinensis}) and northern cardinals (\textit{Cardinalis cardinalis}) in Ohio, where winters are severe, had significantly higher field metabolic rates during winter (Doherty et al., 2001; Sgueo et al., 2012). Cooper (2002) used BMR to reach similar conclusions for populations of mountain chickadee (\textit{Poecile gambeli}) and juniper titmouse (\textit{Baeolophus griseus}) in northern Utah.\\

The reallocation hypothesis is mainly supported by studies of desert and temperate climates. In deserts, aridity selects for a lower parental investment during breeding, as the limited food and water availability favors smaller clutch sizes, lower energy and water requirements of parents and offspring, and reduced growth rates (Tieleman et al., 2004). During winter, desert animals can frequently experience cold temperatures that impose substantial winter thermoregulatory costs, and together with limited breeding investment, this results in little variation in annual expenditure, as has been found for a range of species, including dune larks (\textit{Calendulauda erythrochlamys}) in the central Namib Desert (Williams, 2001), verdins (\textit{Auriparus flaviceps}) in the Colorado Desert (Webster and Weathers, 2000), and Arabian babblers (\textit{Argya squamiceps}) in the Negev Desert (Anava et al., 2000; 2002). In temperate climates, on the other hand, birds incur substantial, but not extreme, thermoregulatory costs in winter and raise large broods in spring, consistent with the roughly equal seasonal energy expenditure predicted by the reallocation hypothesis. Indeed, several studies from temperate regions support the reallocation hypothesis: for example, black-billed magpies (\textit{Pica hudsonia}) in Washington State (Mugaas and King, 1981) and yellow- (\textit{Junco phaeonotus}) and dark-eyed juncos (\textit{Junco hyemalis}) in California (Weathers and Sullivan, 1993) had winter field metabolic rates that were not significantly different from average breeding season values. However, the empirical record is far from clear, possibly because, especially in temperate areas, higher spatial and interannual climate variability may cause the balance between energy requirements for winter survival and spring breeding to vary geographically and temporally (Karger et al., 2023). In the Netherlands, for example, both long-eared owls (\textit{Asio otus}) and common kestrels (\textit{Falco tinnunculus}) had higher field metabolic rates during breeding (Masman et al., 1988; Wijnandts, 1984), supporting the increased demand rather than the reallocation hypothesis. In California, however, the energy expenditure of white-crowned sparrows (\textit{Zonotrichia leucophrys nuttalli}) was 17\% higher in winter than during the breeding season, despite comparatively mild winters (Weathers et al., 1999).\\

The aerobic capacity model is most strongly supported by a range of studies on birds living in cold areas. For instance, Dutenhoffer and Swanson (1996) found that in 10 bird species from southeast South Dakota, whole-body and mass-independent BMR and M$_{\text{sum}}$ were positively correlated, and the same was true for black-capped chickadees (\textit{Poecile atricapillus}) in Canada (Lewden et al., 2012). Findings from tropical birds, in contrast, found a lack of support for the aerobic capacity model, which may be explained by the fact that tropical species presumably experience less strong selection for high thermogenic capacity (Pacioni et al., 2023a; Swanson and Garland, 2009; Wiersma et al., 2007a; 2007b) and display more variable patterns than their temperate-zone counterparts (Noakes and McKechnie, 2020). For example, Wiersma et al. (2007b) found that mass-independent M$_{\text{sum}}$ and mass-independent BMR were not correlated in 19 tropical lowland forest birds. Moreover, M$_{\text{sum}}$ is often considered a measure of maximal (shivering) heat production and an indicator of the level of sustainable thermogenic capacity, as many studies document its positive association with cold tolerance (Swanson, 2001, 2010; Swanson and Liknes, 2006). However, several studies have reported that winter increases in cold tolerance can occur without corresponding changes in M$_{\text{sum}}$ (Dawson et al., 1983; Swanson, 1993). Thus, cold tolerance and M$_{\text{sum}}$ do not always change in tandem, and the extent of their phenotypic correlation is still controversial (Swanson et al., 2012).\\

Here, we study the metabolic signatures of seasonal variation in the energy requirements of a small resident passerine of temperate zone deciduous forests (the great tit, \textit{Parus major}) and assess whether its metabolic rates conform to the predictions of the aerobic capacity model of endothermy. The great tit is a small ($\sim$18g, Dunning, 2007) insectivorous passerine bird that is native to large parts of Eurasia. Great tits predominantly inhabit open deciduous and mixed forests, as well as forest edges. They are also found in various human-altered landscapes, including plantations, hedgerows, orchards, parks, gardens, and the fringes of cultivated areas (Gosler et al., 2020). The species readily accepts nest boxes for breeding. The species has therefore been used as a model for a range of eco-evolutionary research, including studies focusing on energy expenditure (Broggi and Nilsson, 2023; Tinbergen and Dietz, 1994; Ulgezen et al., 2019; Welbers et al., 2017). From an energetic point of view, smaller birds such as great tits radiate more body heat per unit of mass due to their higher surface-to-volume ratio and thus face comparatively high thermoregulatory costs at colder temperatures (Bergmann, 1848). In our study, we therefore measured the basal (BMR), summit (M$_{\text{sum}}$), and field (FMR) metabolic rates during late winter and during the chick-rearing period. Our study area in northern Belgium consists of a mixed deciduous forest embedded in a largely residential area and is characterized by a maritime temperate climate with comparatively mild winters (KMI, 2022). Given this mild climate, we predict that (i) energy expenditure will be higher during breeding compared to pre-breeding late winter (i.e., conforming to the increased demand hypothesis). Based on the assumptions of the aerobic capacity model, we further expect to find that (ii) BMR and M$_{\text{sum}}$ are positively correlated and (iii) that M$_{\text{sum}}$ correlates with increased cold tolerance.\\

\section{Material and methods}
\subsection{Study area} 
The study was carried out in the Aelmoeseneie forest, a 28.5 ha mixed deciduous forest situated in Gontrode (Melle, Belgium), where residential areas and agricultural fields surround the field site. It is one of the few remaining old forest fragments in the region (i.e., it has been forested since before 1775), and is dominated by oak (\textit{Quercus robur}) and beech \textit{(Fagus sylvatica}) trees, with ash (\textit{Fraxinus excelsior}) and maple (\textit{Acer pseudoplatanus}) in the wetter parts of the area (Dekeukeleire, 2021). The forest has been equipped with 84 standard nest boxes for great tits since autumn 2015 (height 1.5m; dimensions 23x9x12cm; entrance 32 mm) (Dekeukeleire, 2021). The climate of this region is maritime-temperate, characterized by mild winters and significant precipitation in all seasons. During the study period, the average temperatures in winter were 6.5 ± 1.8 °C in February and 9.3 ± 2.9 °C. The ambient temperature in the forest was recorded using 20 TMS-4 dataloggers placed -15 cm from the ground (Wild et al., 2019). The fieldwork was performed in February/mid-March 2022 (late winter) and in April/May 2022 (breeding). During the breeding season, only two individuals were captured and measured that were also captured and measured during the winter season.
\subsection{Experimental protocol during winter}
Birds were captured during nightly nest box checks, with roosting birds removed from nest boxes between 17h and 19h. Birds were then taken to a nearby laboratory, where they were ringed for individual identification, aged (1st winter or adult), sexed (based on plumage characteristics), and weighed to the nearest 0.1g. Immediately after, birds were placed in metabolic chambers to obtain their BMR during the remainder of the night. At sunrise, birds were removed from the metabolic chambers and placed in individual cages with ad libitum access to food (mealworms and sunflower seeds) and water. During the afternoon (ca. 13h30-16h30), birds were returned to the metabolic chamber to determine their M$_{\text{sum}}$. After M$_{\text{sum}}$, birds were returned to their individual cages, where they remained for the rest of the day and also during the night. The next evening (ca. 19h–20h), birds were injected with doubly labeled water (DLW) to determine their FMR and kept in a dark cotton bag for ca. 1h to allow the isotopes to mix with the body fluids, and then a blood sample was taken. Birds were then returned to the same nest box in which they had been captured ca. 48h earlier. During winter, BMR of 40 individuals was measured (19 males and 21 females, including 35 adults and 5 1st winter). For all but four of these individuals, winter M$_{\text{sum}}$ was also measured (16 males and 20 females, including 31 adults and 5 1st winter). These four individuals showed unusual behavior the day after the BMR measurements (e.g., reduced movement activity), so no M$_{\text{sum}}$ measurements could be obtained. Due to the mild temperatures during our study, nest box occupancy for nocturnal roosting was low, which contributed to difficulties in recapturing individuals within 24–48 h for FMR measurements. For only two individuals, FMR measurements were obtained during winter.
\subsection{Experimental protocol during spring}
In contrast to winter sampling, BMR, M$_{\text{sum}}$ and FMR were not measured on the same set of individuals during the breeding season, as this would result in breeding birds being away from their brood for too long (ca. 48h). Therefore, BMR was measured on breeding females that were taken from their nest boxes, while M$_{\text{sum}}$ and FMR were measured on different individuals caught during the day using mist nets. For BMR, all nest boxes were checked at least twice a week from April onwards to determine whether they were occupied by great tits and to monitor breeding stages (laying dates, clutch size, and hatching dates). On the day when the chicks were 12 days old and capable of self-thermoregulation (Radford et al., 2001), females were removed from their nest boxes after dark (ca. 20h30). As in winter, the birds were taken to the laboratory to measure their BMR, except that at sunrise, the birds were immediately removed from the metabolic chamber and returned to their original nest boxes, where they all resumed rearing their offspring. For M$_{\text{sum}}$ and FMR, mist nets were placed at various locations in the forest in the afternoon (ca. 14h-19h). Captured birds were transported to the laboratory, where they were placed in metabolic chambers to determine their M$_{\text{sum}}$, after which they were put in cages. After some period of rest ($\sim$30 min), birds were injected with DLW following the same procedure as above, and birds were released back into the forest near their capture site. During the breeding season, BMR from ten adult females and M$_{\text{sum}}$ from six adult males and one adult female were measured. FMR measurements were only obtained from three individuals during spring due to unforeseen circumstances, namely a number of non-field days caused by the illness of team members. This complication affected the capturing, injection, and recapturing of a larger number of individuals.

\subsection{Basal metabolic rates}
BMR (open flow-through respirometry; Lighton, 2018) was measured by monitoring the oxygen consumption (VO$_{\text{2}}$) of individual birds, following the methods described in Pacioni et al. (2023a). Briefly, two pumps were used to provide ambient air, delivering a flow rate of 400 ml/min. All chambers were kept at 25 °C, which is within the thermoneutral zone (TNZ), which was determined following the methods of van de Ven et al. (2013). Although the curves (Figure 4.3) do not clearly identify the TNZ of our great tits, most individuals tend to have a low VO$_{\text{2}}$ between 22.5 and 30 °C, in accordance with other estimates (e.g., Bech and Mariussen, 2022). Each individual bird was placed in sealed plastic metabolic chambers with a volume of 1.1L. All chambers were placed in a custom-designed, darkened climate control unit. BMR measurements started at ca. 20h (winter) and ca. 21h (summer) and continued until dawn (winter: ca. 08h, summer: ca. 06h30). The number of birds that were measured during the same night varied from 3 to 7 in winter (4 on average) and from 1 to 4 in spring (2 on average). Additional information regarding the BMR can be found in the supplementary material.

\subsection{Summit metabolic rates}
M$_{\text{sum}}$ was measured as the maximum cold-induced VO$_{\text{2}}$ in a heliox (79\% Helium - 21\% Oxygen) atmosphere (Rosenmann and Morrison, 1974), following the methods described in Pacioni et al. (2023a). Briefly, birds were placed in a 0.9L metal chamber, and subsequently, the chamber with the bird inside was then placed in the climate control unit, which was set to an initial temperature of 10 °C. The metabolic chamber was supplied with flowing heliox gas several minutes prior to the start of the trial, allowing the bird to acclimatize. The heliox gas was pumped at a flow rate of approximately 812 ml/min. After the acclimatization period, the temperature in the metabolic chamber was gradually reduced (cooling rate of $\sim$0.8 °C per min) and the oxygen consumption of the individual in the metabolic chamber was monitored in real time. Trials were stopped after a steady decline in VO$_{\text{2}}$ was observed for several minutes. When taking out the bird from the metabolic chamber, a thermocouple (5SC-TT-TI-36-2M; Omega) was inserted into the cloaca to verify whether the birds were hypothermic (i.e., a body temperature $\sim$ 38°C, Cooper and Gessaman, 2005). All birds in this study were hypothermic after the M$_{\text{sum}}$ measurements (average body temperature: 31.7 ± 1.2 °C). The bird was then placed back in the cage, which was located in a heated room with water and food ad libitum. Additional information regarding the M$_{\text{sum}}$ can be found in the supplementary material.\newline
The study protocol of this study was approved by the Ethics Committee Animal Experiments VIB/Faculty of Science of Ghent University (EC2020-063).
\subsection{Respirometry data analysis}
The software ExpeData from Sable Systems was used to record trials and extract BMR (ml O$_{\text{2}}$/min) and M$_{\text{sum}}$ (ml O$_{\text{2}}$/min). BMR, TNZ, and M$_{\text{sum}}$ were calculated using equation 9.7 from Lighton (2008) by considering the lowest stable part of the curve (average of 5 min) to estimate BMR and TNZ over the entire night and the highest 5-min average VO$_{\text{2}}$ over the test period to estimate M$_{\text{sum}}$. All data were corrected for drift in O$_{\text{2}}$, CO$_{\text{2}}$, and H2O baselines using the Drift Correction function in ExpeData.
\subsection{Field metabolic rates}
FMR (kJ/day) was measured using the DLW technique, following Speakman (1997). Blood samples were collected from one unlabeled individual to estimate the background isotope enrichments of \textsuperscript{2}H and \textsuperscript{18}O. Different individuals (winter: n = 10, spring: n = 6) were then weighed to the nearest 0.1g before being injected into the pectoral muscle with 0.1 ml of DLW (661699 ppm \textsuperscript{18}O, 345686 ppm \textsuperscript{2}H). The syringes (Micro Fine + Insulin Syringe with Needle; Camlab) were weighed before and after administration (±0.0001g) in order to calculate the exact dose the bird received, and the time of injection was recorded. Individuals were then kept in a bag for ca. 1h to allow the isotopes to mix with the body fluids. Before releasing the bird at the original capture site, an initial blood sample (100 μl) was taken from the ulnar vein. When recaptured by nighttime nest box controls or daytime mist-netting 24–48 h after release, another blood sample was taken from the opposite ulnar vein. The bird was weighed again to the nearest 0.1g and the time of the blood sampling was recorded. All blood samples were kept in heparinized capillary tubes (Micro-Pipette 100 μl; Camlab), and flame-sealed using a butane torch. Analysis of the isotopic enrichment of blood was performed blind using a Liquid Isotope Water Analyser (Los Gatos Research, USA) (Berman et al., 2012). Initially, the blood encapsulated in capillaries was vacuum distilled (Nagy, 1983), and the resulting distillate was used for analysis. Samples were run alongside five lab standards for each isotope and three international standards to correct for day-to-day machine variation and convert delta values to ppm. A single-pool model was used to calculate rates of CO$_{\text{2}}$ production as recommended for use in animals less than 5 kg in body mass (Speakman, 1993). To convert CO$_{\text{2}}$ into energy expenditure, an RQ = 0.75 was assumed and used 27.89 kJ/L CO$_{\text{2}}$, following Speakman (1997). To calculate body composition of the birds, the dilution space (Nd) was estimated from the enrichment of deuterium in the blood sample collected 1h after injection. This reflects the total body water (TBW) content, which is then converted to fat free mass (FFM) following Speakman (1997).
\subsection{Statistical analysis}
Linear regression models with a Gaussian error distribution were used to test for differences in energy expenditure and cold tolerance (considered here as the heliox temperature at which M$_{\text{sum}}$ was reached) between the winter and breeding seasons. The models used (log-transformed) FMR, BMR, M$_{\text{sum}}$, and cold tolerance as dependent variables, and season, sex, age, laying date, and clutch size as independent factors. Models using whole-body metabolic rates were first run, then they were run using mass-independent metabolic rates (by adding log-transformed body mass as a covariate) to test for changes in metabolic intensity. Similar linear models were used to test for positive correlations between body mass, (whole-body) metabolic rates, and cold tolerance. For the FMR, where data were available for both total body mass and fat-free body mass (FFM), analyses were repeated using FFM instead of body mass.\\

A statistical inference was made based on full models, with all covariates included as fixed effects. Post-hoc tests (Table 4.2) were conducted using the emmeans function from the 'emmeans' package (Lenth et al., 2023). Outliers were identified based on interquartile ranges using the quantile function and removed using the subset function. Within the dataset, two M$_{\text{sum}}$ values were identified as outliers and excluded. The normality of residuals was tested and confirmed (Shapiro-Wilk W > 0.9) for all models, and the significance level was set at $p \leq 0.05$. Statistical analysis was performed using R software v. 4.2.2 (R Core Team, 2022). A detailed description of the statistical analysis can be found in Supplementary File 1 (RMarkdown HTML).
\clearpage

\section{Results}

Great tits showed only modest changes in proxies of energy expenditure between winter and breeding (Table 4.1; Figure 4.1; Table 4.2). Whole-body BMR and FMR were $\sim$10\% higher during breeding compared to late winter (p = 0.096 and p = 0.822, respectively), while M$_{\text{sum}}$ decreased by $\sim$7\% from the winter to the breeding period (p = 0.056). Mass-independent metabolic rates showed a $\sim$10 increase in BMR (p = 0.050) and a $\sim$7\% decrease in M$_{\text{sum}}$ (p = 0.045) from winter to breeding. For FMR, probably due to the small sample size (n = 5), no meaningful relationship with body mass was found (i.e., a non-significant correlation between FMR and body mass, or FFM, see below), precluding the calculation of mass-independent FMR. Great tits were able to significantly tolerate colder temperatures during the winter compared to the breeding season (p = 0.004), with a winter average of -18.1 ± 1.7 °C and a breeding average of -16.5 ± 1.5 °C (heliox temperatures at which M$_{\text{sum}}$ was reached). 1st winter individuals had a significantly higher body mass than adults (p = 0.027), which was associated with a higher whole-body and mass-independent M$_{\text{sum}}$ (p = 0.001). The average metabolic expansibility (ME, the ratio of M$_{\text{sum}}$ over BMR) during the winter season was 3.94, while during the breeding season the average was 3.19 (population level, see above, as we did not gather BMR and M$_{\text{sum}}$ on the same individuals during the breeding season). BMR (both whole-body and mass-independent) was not significantly correlated with clutch size (which varied from 4 to 9 eggs, 6.2 ± 1.4) nor with the laying date (which varied from the 22nd of April to the 1st of May; all p > 0.1).\\

Body mass did not differ between late winter and the breeding period (p = 0.251), and was significantly and positively correlated to whole-body BMR and M$_{\text{sum}}$ (p = 0.021 and p = 0.004, respectively, Figure 4.4). FMR was negatively but non-significantly correlated to body mass and to FFM (p = 0.20 and 0.14, respectively). Whole-body BMR and M$_{\text{sum}}$ were positively correlated (p < 0.05). Whole-body M$_{\text{sum}}$ was higher in individuals who could tolerate lower temperatures (i.e., had a higher cold tolerance; Figure 4.2) (p = 0.047). Mass-independent BMR and mass-independent M$_{\text{sum}}$ were, however, not correlated (p = 0.24; Figure 4.2).

\clearpage



\begin{sidewaystable}[!ht]
    \centering
\caption*{\textbf{Table 4.1}: Mean ± standard deviation (SD), minimum (Min) - maximum (Max) values for FMRw (late winter), FMRb (breeding), BMRw (late winter), BMRb (breeding), M$_{\text{sum}}$w (late winter), M$_{\text{sum}}$b (breeding), cold tolerance (late winter and breeding) and body mass (after BMR measurements; late winter and breeding) per sex and age. Values in brackets refer to the sample size.}
\begin{adjustbox}{max width=\textwidth}    
    \begin{tabular}{cccccccccc}
    \hline
        ~ & ~ & Male & ~ & Female & ~ & Adult & ~ & 1st winter & ~ \\ \hline
        ~ & ~ & Mean ± SD & Min - Max & Mean ± SD & Min - Max & Mean ± SD & Min - Max & Mean ± SD & Min - Max \\ 
        Whole-body & FMRw [kJ/day] & 67.4 (1) & ~ & 65.2 (1) & ~ & 66.3 ± 1.5 (2) & 65.2-67.4 & - & - \\ 
        ~ & FMRb [kJ/day] & - & - & 72.7 ± 21.1 (3) & 56.3-96.5 & 72.7 ± 21.1 (3) & 56.3-96.5 & - & - \\ 
        ~ & BMRw [ml O$_{\text{2}}$/min] & 1.23 ± 0.23 (19) & 0.86-1.89 & 1.20 ± 0.18 (21) & 0.81-1.60 & 1.22 ± 0.20 (35) & 0.81-1.89 & 1.32 ± 0.14 (5) & 1.22-1.56 \\ 
        ~ & BMRb [ml O$_{\text{2}}$/min] & - & - & 1.35 ± 0.25 (10) & 0.89-1.58 & 1.35 ± 0.25 (10) & 0.89-1.58 & - & - \\ 
        ~ & M$_{\text{sum}}$w [ml O$_{\text{2}}$/min] & 4.94 ± 0.58 (16) & 3.92-6.75 & 4.27 ± 0.58 (20) & 2.61-5.33 & 4.57 ± 0.66 (31) & 2.61-6.75 & 5.22 ± 0.33 (4) & 4.83-5.70 \\ 
        ~ & M$_{\text{sum}}$b [ml O$_{\text{2}}$/min] & 4.38 ± 0.52 (5) & 3.92-5.21 & 3.84 (1) & ~ & 4.29 ± 0.52 (6) & 3.84-5.21 & - & - \\ 
        Mass-independent [residuals] & FMRw & 0.06 (1) & ~ & -0.11(1) & ~ & -0.02 ± 0.12 (2) & -0.11-0.06 & - & - \\ 
        ~ & FMRb & - & - & 0.02 ± 0.17 (3) & -0.15-0.20 & 0.02 ± 0.17 (3) & -0.15-0.20 & - & - \\ 
        ~ & BMRw & -0.02 ± 0.13 (19) & -0.29-0.33 & -0.01 ± 0.10 (21) & -0.21-0.20 & -0.02 ± 0.17 (35) & -0.15-0.19 & 0.06 ± 0.23 (5) & -0.16-0.33 \\ 
        ~ & BMRb & - & - & 0.07 ± 0.17 (10) & -0.21-0.24 & 0.07 ± 0.17 (10) & -0.21-0.24 & - & - \\ 
        ~ & M$_{\text{sum}}$w & 0.05 ± 0.12 (16) & -0.07-0.38 & -0.01 ± 0.13 (20) & -0.42-0.18 & 0.04 ± 0.14 (31) & -0.07-0.38 & 0.08 ± 0.07 (4) & -0.00-0.15 \\ 
        ~ & M$_{\text{sum}}$b & -0.09 ± 0.11 (5) & -0.23-0.07 & -0.16(1) & ~ & -0.01 ± 0.11 (6) & -0.23-0.07 & - & - \\ 
        Cold tolerance\textsubscript{w} [°C] & ~ & -18.5 ± 1.6 (11) & -22.0--15.0 & -17.9 ± 1.8 (18) & -20.0--14.0 & -18.1 ± 1.7 (27) & -22.0--14.0 & -18.0 ± 1.1 (2) & -19.2--17.0 \\ 
        Cold tolerance\textsubscript{b} [°C] & ~ & -16.6 ± 1.7 (5) & -19.0--15.0 & -16.0(1) & ~ & -16.5 ± 1.5 (6) & -19.0--15.0 & - & - \\ 
        Body mass\textsubscript{w} [g] & ~ & 18.0 ± 1.4 (19) & 14.9-20.4 & 16.3 ± 1.1 (21) & 14.5-18.3 & 17.1 ± 1.5 (35) & 14.5-20.4 & 18.4 ± 1.2 (5) & 17.0-20.1 \\ 
        Body mass\textsubscript{b} [g] & ~ & 17.6 ± 0.9 (5) & 16.6-18.8 & 16.8 ± 0.9 (10) & 15.8-18.4 & 17.1 ± 0.9 (10) & 15.8-18.8 & - & - \\ \hline
    \end{tabular}

\end{adjustbox}
\end{sidewaystable}
\clearpage

\renewcommand{\thefigure}{4.\arabic{figure}}
	\begin{figure}[h!]
		\begin{center}
			\includegraphics[width=1\textwidth]{figures/Figure1(1).jpg}
		\end{center}
		\begin{footnotesize}
			\caption{\footnotesize The violin plots display the comparison of BMR and M$_{\text{sum}}$ between the winter and breeding seasons. A) whole-body BMR. B) mass-independent BMR. C) whole-body M$_{\text{sum}}$. D) mass-independent M$_{\text{sum}}$. The p-value representing the difference between the two corresponding violin plots is displayed between the two corresponding violin plots. Mass-independent metabolic rates were considered as the residuals from the regression of (log) BMR and (log) M$_{\text{sum}}$ on (log) body mass.\label{fig4.1}}
		\end{footnotesize}
	\end{figure}
\clearpage

\renewcommand{\thefigure}{4.\arabic{figure}}
	\begin{figure}[h!]
		\begin{center}
\small
			\includegraphics[width=0.7\textwidth]{figures/Figure2.jpg}
		\end{center}
		\begin{footnotesize}
			\caption{\footnotesize A) The relationship between (log) BMR and (log) M$_{\text{sum}}$. B) The relationship between mass-independent M$_{\text{sum}}$ and mass-independent BMR. C) The relationship between cold tolerance and log(M$_{\text{sum}}$). Blue dots = winter. Red dots = breeding season. The r values represent the Pearson coefficient correlations.\label{fig4.1}}
		\end{footnotesize}
	\end{figure}
\clearpage

\section{Discussion}
Our results show that there is only little seasonal variation in great tit metabolic rates, with changes not exceeding 10\%. Consistent with the increased demand hypothesis, which predicts that energy demand should be highest during breeding, BMR and FMR were about $\sim$10\% higher during the spring period. Conversely, in line with the hypothesis that the winter period is energetically most demanding, we found some evidence for a minor decline in M$_{\text{sum}}$ and a corresponding reduction in cold tolerance from late winter to spring. However, compared to other studies documenting seasonal variation in metabolic energy expenditure (see below), the changes found here are small, and overall, the results may most closely corroborate the reallocation hypothesis, which predicts no net difference in energy requirements because of a shift in energy expenditure from winter thermoregulation to reproductive activities in spring. Our results do not fully support the predictions of the aerobic capacity model, as there was no significant relationship between the mass-independent levels (metabolic intensity) of BMR and M$_{\text{sum}}$, suggesting that the correlations we found between whole-body BMR and M$_{\text{sum}}$ can be driven by variation in body mass alone.
\subsection{Seasonal variation in metabolic energy requirements}
Studies reporting significant differences in avian energetic metabolism between seasons find that birds can strongly up- or downregulate their metabolism. BMR has been found to exhibit a range of changes, from a decrease of approximately 65\% to an increase of about 35\% when transitioning from winter to summer. Similarly, the M$_{\text{sum}}$ can vary from a decrease of 55\% to an increase of 35\% (Table 4.5). Despite our expectation of a more demanding breeding season for great tits in Belgium, the seasonal changes found here are much smaller (i.e., an increase in BMR of $\sim$10\% and a M$_{\text{sum}}$ decrease $\sim$7\%), suggesting that our results are most consistent with the reallocation hypothesis. This finding is in line with two other studies that assessed great tit BMR in winter compared to spring. Hissa and Palokangas (1970) found that despite a difference of more than 20 °C in mean winter and spring temperatures in Turku (Finland), they recorded a seasonal difference in (mass-specific) BMR of only $\sim$5\% higher. Similarly, Gavrilov (2014) reported $\sim$8\% higher winter BMR values for great tits kept in aviaries in the Curonian Spit (Russia). In contrast, Nilsson and Råberg (2001) found evidence for the increased demand hypothesis, as the breeding BMR of their study population in Lund (Sweden) was more than 20\% higher than in winter. Given the exceptionally mild weather during the winter in which we conducted our study (i.e., among the top five highest minimum temperatures in the last 30 years; KMI, 2022), it is unlikely that the higher than expected winter expenditure we document here is solely related to thermoregulatory costs. One possible explanation may be the widespread presence of bird feeders in and around our study area (Dekeukeleire, 2021), which has been shown to influence winter energy expenditure in birds. For example, Broggi et al. (2021) found that populations of willow (\textit{Poecile montanus}) and blue tit (\textit{Cyanistes caeruleus}) with access to feeders were larger and heavier and therefore spent more energy in maintenance. We expected, in accordance with the increased demand hypothesis, a larger difference between winter and breeding, with higher metabolism during breeding. The lack of substantial differences in metabolic rates between winter and chick rearing could be because winter feeding may increase winter metabolism, and thus at least partly mask differences in energy expenditure between winter and breeding in our study area.\\

When comparing the great tit body masses and metabolic rates found in this study with the existing literature, we found that our results largely fall within the range of values reported (Supplementary Material Tables 4.3 and 4.4). The mean body mass in our study was 16.2 ± 0.4 g, compared to the literature mean of 17.7 ± 1.5 g. Our whole-body BMR was 1.2 ± 0.2 ml O$_{\text{2}}$/min, closely matching the literature mean of 1.2 ± 0.2 ml O$_{\text{2}}$/min, while our whole-body FMR was 70.2 ± 3.1 kJ/day, at the lower end of the literature range (literature mean: 93.3 ± 9.8 kJ/day). Our mass-specific BMR values (0.074 ± 0.01 ml O$_{\text{2}}$ min\textsuperscript{-1}g\textsuperscript{-1}) were at the higher end of the literature range (0.066 ± 0.01 ml O$_{\text{2}}$ min\textsuperscript{-1}g\textsuperscript{-1}), while our mass-specific FMR values (4.3 ± 0.3 ml O$_{\text{2}}$ min\textsuperscript{-1}g\textsuperscript{-1}) fell within the range of literature values (5.2 ± 0.6 ml O$_{\text{2}}$ min\textsuperscript{-1}g\textsuperscript{-1}). More fine-grained comparisons between our study and others are hampered because of our smaller sample sizes and a number of methodological differences (e.g., the use of open versus closed respirometry for BMR, intramuscular injection of doubly labeled water in the pectoralis major versus intraperitoneal injection). More research, including meta-analyses summarizing longitudinal studies on the winter and the breeding energetics of individuals and populations, is needed to fully unravel the factors that influence seasonal energy allocation in wild birds and how these may vary across latitude and between habitats.\\

To our knowledge, our study is the first to measure the seasonal variation of M$_{\text{sum}}$ in great tits. A winter increase in mass-independent M$_{\text{sum}}$ is a common finding in studies conducted in highly seasonal areas (McKechnie and Swanson, 2010), together with an increase in body mass (Piersma, 1984; Stuebe and Ketterson, 1982) to enhance winter survival (Broggi et al., 2007; 2019). The rather limited increase in M$_{\text{sum}}$ documented here is consistent with the relatively mild winter temperatures recorded during our study period (i.e., 6.5 ± 1.8 °C in February and 9.3 ± 2.9 °C in March). Birds can achieve higher metabolic heat production and thus higher cold tolerance either by increasing their shivering thermogenic capacity by investing in large (mainly pectoral) muscle mass (see below) or, alternatively, by elevating the metabolic intensity of their tissues (Petit et al., 2014; 2017). Mass-independent M$_{\text{sum}}$ changes represent changes in the heat production per unit tissue mass (McKechnie, 2008; Swanson, 2010) and such changes may allow birds to rapidly respond to changes in weather conditions (e.g., cold spells) without requiring the synthesis of new shivering tissues (Vézina et al., 2011). However, more research is needed to test the relative importance of shivering thermogenesis in birds exposed to different climates and weather regimens.

\subsection{Aerobic capacity model}
We found that birds characterized by a higher whole-body BMR also had higher whole-body M$_{\text{sum}}$ values, but that mass-independent BMR was not significantly correlated to mass-independent M$_{\text{sum}}$. Our results, therefore, do not fully support the aerobic capacity model. Interspecific comparisons often document correlations between both whole-body and mass-independent BMR and M$_{\text{sum}}$, at least for non-tropical birds (Dutenhoffer and Swanson, 1996; Rezende et al., 2002), though intraspecific studies often report results similar to ours. For example, Swanson et al. (2012) showed a significant positive correlation between whole-body BMR and M$_{\text{sum}}$ in black-capped chickadees and house sparrows (\textit{Passer domesticus}) from South Dakota (USA), but not in mass-independent rates. The absence of a mass-independent correlation may be due to the fact that body mass differences between individuals can reflect differences in body composition, such as in (metabolically relatively inert) fat mass, while scaling metabolic rates by body mass to obtain mass-independent rates assumes a constant contribution of mass to metabolic rates (Daan et al., 1990; Hayes and Shonkwiler, 1996). Therefore, future studies investigating the variation and correlation of intraspecific metabolic rates should include assessments of body composition, such as the amount of body fat, the size and mass of body muscles, and the cardiopulmonary and digestive organs.\\

As expected, birds with a higher M$_{\text{sum}}$ were able to tolerate colder temperatures both in late winter and during chick rearing, suggesting that the ability of small birds to tolerate cold exposure is enhanced by pectoral muscle-driven heat production. In birds, the capacity for sustaining flight (i.e., parental care activity) and for shivering thermogenesis (i.e., temperature regulation) are both functions of skeletal muscle mass (Guglielmo, 2010). During the late winter period, high M$_{\text{sum}}$ enabled great tits to tolerate colder temperatures, which conforms to several inter- and intraspecific studies on avian cold tolerance. For example, Cooper (2000) found that higher winter pectoral muscle masses resulted in higher M$_{\text{sum}}$ and an increased ability to sustain cold temperatures in the mountain chickadee and the juniper titmouse (\textit{Baeolophus ridgwayi}) in North America. Likewise, Liknes and Swanson (2011) showed that the pectoral muscle mass acts as a significant contributor to M$_{\text{sum}}$ values, in turn increasing thermogenic capacity and cold tolerance for several South Dakota (USA) passerines. During the breeding period, there is little need for shivering thermogenesis, but sustained muscle activity is required for powering parental care activities, such as foraging flights (Koteja, 2000).\\

In addition, metabolic expansibility (ME), the ratio of M$_{\text{sum}}$ to BMR, often considered to reflect the ability of an organism to increase heat production (Arens and Cooper, 2005), was relatively low compared to that typically observed in temperate species (Swanson, 1993). Specifically, during the winter season, the average ME was 3.94, while during the breeding season, the average ME was 3.19 (population level, see above, as we did not measure BMR and M$_{\text{sum}}$ on the same individuals during the breeding season). This could be attributed to the overall lower metabolic scope of the great tits in our study, likely influenced by warmer winter temperatures compared to previous studies on seasonal energy expenditure in small passerines. However, it is important to note that our results fall within the range of ME values (2.9-4.7) reported for five other similarly sized European passerines (Saarela et al., 1995). Our results corroborate the literature supporting the view that M$_{\text{sum}}$ and BMR are flexible traits that vary in ways that are consistent with expected cold- and/or activity-induced costs (such as thermoregulation and parental care).

\section{Conclusions}
In conclusion, the slightly (but not significantly) higher basal and field metabolic rates during breeding are consistent with the predictions of the increased demand hypothesis. However, the changes observed in our study population are small compared to those reported in other studies. Therefore, our results may be more consistent with the reallocation hypothesis.

\subsection*{Acknowledgements}
We thank ForNaLab for providing us with access to their facilities. Especially we would like to thank Luc Willems for his support and Dries Landuyt for providing us with temperature data from the dataloggers. In addition, we are thankful for the assistance provided by bachelor student Hanne Danneels during the data collection process. The authors would like to express sincere gratitude to An Martel for her valuable instruction and guidance in teaching the techniques of injection and blood collection. This study also acknowledges funding by Methusalem Project 01M00221 (Ghent University) awarded to Frederick Verbruggen, Luc Lens and An Martel.

\subsection*{Data availability}
The data used in this manuscript can be found at\newline
(https://data.mendeley.com/datasets/fk83s6j4x2/2) while the statistical scripts used can be consulted via Supplementary file 1 (RMarkdown HTML) in the online version of the manuscript. 

\section{Supplementary material}

\subsection*{Respirometry set up}

\subsection*{BMR}

Nighttime BMR was assessed using flow-through respirometry. The measurement of O$_{\text{2}}$ consumption rates (VO$_{\text{2}}$; ml/min) was conducted. Before the measurements, weight measurements to the nearest 0.1g were taken for each individual. Then, each individual was placed in sealed plastic chambers with a volume of 1.1L. All chambers were positioned within a specially designed, darkened climate control unit (Combisteel R600). The ambient air was pushed by two pumps and distributed into eight distinct streams, which were directed to a mass-flow meter (FB-8, Sable Systems). The needle valves on the meter were adjusted to ensure a consistent flow rate of 400ml/min. The airflow was subsequently directed towards a variable number of metabolic chambers, ranging from 3 to 7 in winter and from 1 to 4 in spring, with one chamber left empty to serve as a baseline. The outgoing airstreams from the chambers were connected to a multiplexer (RM-8, Sable Systems), enabling independent sampling of each chamber's airstream. The excurrent air from both the bird and baseline channels was alternately subsampled and drawn through a Field Metabolic System (FMS-3, Sable Systems). The birds were measured in cycles alongside several baselines. The timing and duration of measurements for each bird within a cycle depended on the number of birds present during a session (typically averaging around 30 minutes per bird with three cycles during the night). The number of birds that were measured during the same night varied from 3 to 7 in winter (4 on average) and from 1 to 4 in spring (2 on average). On average, the entire measurement process lasted approximately 9 hours. The first two hours were discarded to ensure that birds were postabsorptive. Following the respirometry measurement, the birds were re-weighed with precision to the nearest 0.1g and put back in the cage with unrestricted access to both water and food (winter) or put back in their nest box (breeding). In order to ensure accurate BMR measurements, regular calibration of the FMS-3 was conducted. Specifically, before each experiment, the O$_{\text{2}}$ sensor (fuel cell) was calibrated at 20.94\% using the fixed-span mode, employing ambient air passing through a Drierite® column (Lighton, 2018). The CO$_{\text{2}}$ and water-vapor sensors were zeroed with pure nitrogen (N\textsubscript{2}) every 7–14 days. Additionally, the CO$_{\text{2}}$ infrared sensor was spanned using a reference gas containing a known CO$_{\text{2}}$ content (1\%) every 7–14 days.

\subsection*{M$_{\text{sum}}$}

Individual M$_{\text{sum}}$ measurements were conducted during the day within a heliox atmosphere consisting of 21\% O$_{\text{2}}$ and 79\% He. During the measurements, the birds were placed in a 0.9 l metal chamber. Subsequently, the chamber with the bird inside was positioned within the climate control unit, which was adjusted to a starting temperature of 10°C. The unit was supplied with flowing heliox gas several minutes prior to the start of the trial, allowing the bird to acclimatize. The heliox gas was pumped at a flow rate of approximately 812 ml/min, as set on the FB-8. The incoming heliox gas was divided into baseline and the experimental channel. The outgoing gas stream passed through a Drierite® column before reaching the FMS-3, effectively removing water vapor from the sample. Each M$_{\text{sum}}$ trial commenced with a 7-minute baseline heliox measurement, allowing complete replacement of the air in the metabolic chambers with heliox before any data was recorded. Following the baseline period, the setup switched to the experimental channel, initiating a gradual decline in temperatures within the climate control unit. Subsequent to removal from the chamber, baseline values were once again recorded for a minimum duration of 5 minutes. Before and after the metabolic tests, each bird's weight was measured to the nearest 0.1g. After the test concluded, the birds were placed in a warm room with unlimited access to water and food. To ensure accurate M$_{\text{sum}}$ measurements, the FMS-3 was calibrated daily by spanning the O$_{\text{2}}$ sensor to match the concentration of O$_{\text{2}}$ in the heliox cylinder.

\clearpage

\renewcommand{\thefigure}{4.\arabic{figure}}
	\begin{figure}[h!]
		\begin{center}
			\includegraphics[width=1\textwidth]{figures/TNZ.png}
		\end{center}
		\begin{footnotesize}
			\caption{\footnotesize O$_{\text{2}}$ consumption (VO$_{\text{2}}$) expressed in ml/min at different ambient temperatures (T, °C) for great tits (\textit{Parus major}). The sample size for each temperature is n=3. The same individuals were used for all the measurements.\label{fig4.1}}
		\end{footnotesize}
	\end{figure}
\clearpage

\renewcommand{\thefigure}{4.\arabic{figure}}
	\begin{figure}[h!]
		\begin{center}
			\includegraphics[width=1\textwidth]{figures/FigureS2.jpg}
		\end{center}
		\begin{footnotesize}
			\caption{\footnotesize A) The relationship between (log) BMR and (log) body mass. B) The relationship between (log) M$_{\text{sum}}$ and (log) body mass. Blue dots = winter. Red dots = breeding season. The r values represent the Pearson coefficient correlations.\label{fig4.1}}
		\end{footnotesize}
	\end{figure}
\clearpage


\begin{sidewaystable}[!ht]
    \centering
\caption*{\textbf{Table 4.2}: Seasonal differences in basal metabolic rates (BMR), summit metabolic rates (M$_{\text{sum}}$) and field metabolic rates (FMR) in great tits. Values are presented as the mean ± standard error. Lower CL: Lower confidence limit. Upper CL: Upper confidence limit. Degrees of Freedom (df). p-value represents the significance level of the comparison between seasons, as determined by the `emmeans` post hoc analysis. "b" refers to the breeding season, "w" refers to the winter season. "b - w" indicates the difference between the breeding and winter seasons. For the FMR, probably given the small sample size (n=5), no meaningful relationship with body mass was found, precluding the calculation of mass-independent FMR.}
\begin{adjustbox}{max width=\textwidth}    
    \begin{tabular}{ccccccccc}
    \hline
        ~ & ~ & Season & Mean & Standard Error & Degrees of Freedom & Lower CL & Upper CL & p-value \\ \hline
        Mass-independent & BMR & b & 0.29 & 0.06 & 46 & 0.18 & 0.41 & ~ \\ 
        ~ & ~ & w & 0.19 & 0.03 & 46 & 0.12 & 0.25 & ~ \\ 
        ~ & ~ & b - w & 0.11 & 0.05 & 46 & - & - & 0.0503 \\ 
        ~ & M$_{\text{sum}}$ & b & 1.48 & 0.04 & 35 & 1.39 & 1.57 & ~ \\ 
        ~ & ~ & w & 1.56 & 0.02 & 35 & 1.52 & 1.60 & ~ \\ 
        ~ & ~ & b - w & -0.08 & 0.04 & 35 & - & - & 0.0454 \\ 
        Whole-body & BMR & b & 0.26 & 0.06 & 45 & 0.14 & 0.38 & ~ \\ 
        ~ & ~ & w & 0.17 & 0.03 & 45 & 0.10 & 0.24 & ~ \\ 
        ~ & ~ & b - w & 0.09 & 0.05 & 45 & - & - & 0.0955 \\ 
        ~ & M$_{\text{sum}}$ & b & 1.49 & 0.04 & 36 & 1.41 & 1.57 & ~ \\ 
        ~ & ~ & w & 1.56 & 0.02 & 36 & 1.52 & 1.60 & ~ \\ 
        ~ & ~ & b - w & -0.07 & 0.04 & 36 & - & - & 0.0561 \\ 
        ~ & FMR & b & 4.28 & 0.25 & 2 & 3.18 & 5.37 & ~ \\ 
        ~ & ~ & w & 4.19 & 0.20 & 2 & 3.35 & 5.04 & ~ \\ 
        ~ & ~ & b - w & 0.08 & 0.32 & 2 & - & - & 0.8224 \\ \hline
    \end{tabular}
\end{adjustbox}
\end{sidewaystable}

\clearpage

\begin{table}[!ht]
    \centering
\caption*{\textbf{Table 4.3}: Field metabolic rates (FMR), body mass, and mass-specific FMR (FMR divided by the body mass) of studies on great tits (\textit{Parus major}). Weighted mean averages for comparison with the FMR values of this study. *The use of different CO$_{\text{2}}$ coefficients can significantly affect FMR values, making comparisons a bit difficult. ** While all our own metabolic rates refer to "mass-independent" values (i.e., body mass was included as a covariate), for the purpose of comparison with other research papers, we calculated "mass-specific" values (i.e., metabolic rate divided by body mass) because raw metabolic rate data were not available from all papers.}
\begin{adjustbox}{max width=\textwidth}    
    \begin{tabular}{cccccc}
    \hline
        Author(s) & FMR* (kJ/day) & Body mass (g) & Mass-specific** FMR & Period & Sample size n \\ \hline
        Sanz, 1998 & 103.20 & 17.5 & 5.90 & Feeding chicks & 10 \\ 
        Tinbergen and Dietz, 1994 & 95.10 & 17.7 & 5.37 & Feeding chicks & 32 \\ 
        Sanz et al., 2000 & 72.00 & 17.4 & 4.14 & Feeding chicks & 7 \\ 
        ~ & 97.90 & 17.7 & 5.53 & ~ & 27 \\ 
        ~ & 103.20 & 17.8 & 5.80 & ~ & 10 \\ 
        Tinbergen and Verhulst, 2000 & 88.70 & 17.4 & 5.10 & Feeding chicks & 13 \\ 
        ~ & 102.10 & 18.0 & 5.67 & ~ & 14 \\ 
        ~ & 105.50 & 17.5 & 6.03 & ~ & 11 \\ 
        De Heij et al., 2008 & 79.30 & 20.1 & 3.95 & Incubation & 14 \\ 
        Bryan and Bryant, 1999 & 111.20 & 21.5 & 5.17 & Incubation & 8 \\ 
        Tinbergen and Wiersma, 2003 & 77.50 & 17.5 & 4.43 & Feeding chicks & 10 \\ 
        ~ & 84.20 & 18.0 & 4.68 & ~ & 10 \\ 
        Verhulst and Tinbergen, 1997 & 92.40 & 17.8 & 5.19 & Feeding chicks & 4 \\ 
        ~ & 90.00 & 17.8 & 5.06 & ~ & 5 \\ 
        ~ & 82.40 & 17.8 & 4.63 & ~ & 9 \\ 
        ~ & 72.60 & 17.7 & 4.10 & ~ & 6 \\ 
        ~ & 92.70 & 17.6 & 5.27 & ~ & 6 \\ 
        ~ & 85.80 & 17.8 & 4.82 & ~ & 5 \\ 
        ~ & 105.60 & 17.5 & 6.03 & ~ & 4 \\ 
        ~ & 86.30 & 17.4 & 4.96 & ~ & 5 \\ 
        ~ & 97.50 & 17.4 & 5.60 & ~ & 7 \\ 
        ~ & 103.10 & 17.6 & 5.86 & ~ & 5 \\ 
        ~ & 88.60 & 17.2 & 5.15 & ~ & 4 \\ 
        ~ & 100.20 & 17.5 & 5.73 & ~ & 6 \\ 
        Nagy et al., 1999 & 96.20 & 18.1 & 5.31 & Winter & 13 \\ 
        ~ & ~ & ~ & ~ & ~ & ~ \\ 
        Weighted means & 93.28 & 17.9 & 5.18 & ~ & ~ \\ 
        ~ & ~ & ~ & ~ & ~ & ~ \\ 
        Observed in this study & 66.32 & 16.8 & 3.96 & Winter & 2 \\ 
        ~ & 72.71 & 16.3 & 4.48 & Feeding chicks & 3 \\ 
        ~ & ~ & ~ & ~ & ~ & ~ \\ 
        Weighted means & 70.15 & 16.5 & 4.27 & ~ & ~ \\ \hline
    \end{tabular}
\end{adjustbox}
\end{table}

\clearpage



\begin{table}[!ht]
    \centering
\caption*{\textbf{Table 4.4}: Basal metabolic rates (BMR), body mass, and mass-specific BMR (BMR divided by the body mass) of studies on great tits (\textit{Parus major}). Weighted mean averages for comparison with the BMR values of this study. * While all our own metabolic rates refer to "mass-independent" values (i.e., body mass was included as a covariate), for the purpose of comparison with other research papers, we calculated "mass-specific" values (i.e., metabolic rate divided by body mass) because raw metabolic rate data were not available from all papers}
\begin{adjustbox}{max width=\textwidth}    
    \begin{tabular}{cccccc}
    \hline
        Author(s) & BMR (ml O$_{\text{2}}$/min) & Body mass (g) & Mass-specific* BMR & Period & Sample size n \\ \hline
        Bouwhuis et al., 2011 & 1.36 & 18.2 & 0.075 & Winter & 694 \\ 
        Broggi and Nilsson, 2023 & 0.89 & 17.6 & 0.051 & Winter & 42 \\ 
        Broggi et al., 2004 & 1.15 & 18.5 & 0.062 & Winter & 17 \\ 
        ~ & 1.00 & 18.8 & 0.053 & ~ & 24 \\ 
        Broggi et al., 2007 & 1.15 & 18.9 & 0.061 & Winter & 324 \\ 
        Broggi et al., 2019 & 1.10 & 19.7 & 0.056 & Winter & 159 \\ 
        Broggi et al., 2022 & 1.28 & 19.2 & 0.067 & Winter & 160 \\ 
        Daan et al., 1990 & 0.90 & 16.0 & 0.056 & - & 22 \\ 
        Gavrilov, 2014 & 0.98 & 16.4 & 0.06 & Summer & 20 \\ 
        ~ & 1.11 & 17.1 & 0.065 & Winter & 20 \\ 
        Hissa and Palokangas, 1970 & 1.14 & 18.4 & 0.062 & Winter & 5 \\ 
        ~ & 1.12 & 19.0 & 0.059 & Summer & 5 \\ 
        Kerimov and Ivankina, 1999 & 1.23 & 19.2 & 0.064 & Autumn & 363 \\ 
        ~ & ~ & ~ & ~ & Winter & ~ \\ 
        Lindström and Kvist, 1995 & 0.77 & 14.9 & 0.052 & Autumn & 1 \\ 
        Mathot et al., 2015 & 1.11 & 18.1 & 0.061 & Winter & 142 \\ 
        Mathot et al., 2016 & 0.96 & 18.0 & 0.053 & Winter & 111 \\ 
        Nilsson and Råberg, 2001 & 0.93 & 15.9 & 0.059 & Winter & 11 \\ 
        ~ & 1.05 & 16.5 & 0.064 & Nest building & 9 \\ 
        ~ & 1.19 & 18.2 & 0.065 & Egg laying & 12 \\ 
        ~ & 1.12 & 16.1 & 0.07 & Feeding chicks & 14 \\ 
        Ots et al., 2001 & 1.25 & 18.7 & 0.067 & Winter & 42 \\ 
        Playà-Montmany et al., 2021 & 0.96 & 16.6 & 0.058 & - & 24 \\ 
        Reinertsen and Haftorn, 1986 & 1.04 & 16.5 & 0.064 & Winter & 2 \\ 
        Steen, 1958 & 1.21 & 18.6 & 0.065 & Winter & 2 \\ 
        ~ & ~ & ~ & ~ & ~ & ~ \\ 
        Weighted means & 1.21 & 18.4 & 0.066 & ~ & ~ \\ 
        ~ & ~ & ~ & ~ & ~ & ~ \\ 
        Observed in this study & 1.17 & 16.4 & 0.071 & Winter & 40 \\ 
        ~ & 1.35 & 15.9 & 0.085 & Feeding chicks & 10 \\ 
        ~ & ~ & ~ & ~ & ~ & ~ \\ 
              Weighted means & 1.20 &            16.3 &            0.074 & ~ & ~ \\ \hline
    \end{tabular}
\end{adjustbox}
\end{table}

\clearpage


\begin{table}[!ht]
    \centering
\caption*{\textbf{Table 4.5}: Percentage variation in mass-specific metabolic rates for studies investigating seasonal acclimatization (summer - winter) in wild-caught birds. The asterisk (*) denotes significant values.}
\begin{adjustbox}{max width=\textwidth}    
    \begin{tabular}{cccc}
    \hline
        BMR & ~ & ~ & ~ \\ \hline
        Reference & Species & \% variation (summer - winter) & Climate \\ 
        Arens and Cooper, 2005 & House sparrow (\textit{Passer domesticus}) & 64.3* & Temperate \\ 
        Cooper and Swanson, 1994 & Black-capped chickadee (\textit{Poecile atricapillus}) & 17.8* & Temperate \\ 
        Dawson and Carey, 1976 & American goldfinch (\textit{Carduelis tristis}) & 24.1 & Temperate \\ 
        Dawson et al., 1985 & House finch (\textit{Carpodacus mexicanus}) & 1.4 & Temperate \\ 
        Hissa and Palokangas, 1970 & Great tit (\textit{Parus major}) & 2.3 & Temperate \\ 
        Liknes and Swanson, 1996 & White-breasted nuthatch (\textit{Sitta carolinensis}) & 48.6* & Temperate \\ 
        ~ & Downy woodpecker (\textit{Dryobates pubescens}) & 48.8* & ~ \\ 
        Nzama et al., 2010 & House sparrow (\textit{Passer domesticus}) & -24.4* & Temperate \\ 
        O'Connor, 1995 & House finch (\textit{Carpodacus mexicanus}) & 5.9 & Temperate \\ 
        Smit and McKechnie, 2010 & African scops-owl (\textit{Otus senegalensis}) & -23.4* & Subtropical \\ 
        ~ & Pearl-spotted owlet (\textit{Glaucidium perlatum}) & -30.3* & ~ \\ 
        ~ & Fork-tailed drongo (\textit{Dicrurus adsimilis}) & -34.6* & ~ \\ 
        ~ & Crimson-breasted shrike (\textit{Laniarius atrococcineus}) & -29.2* & ~ \\ 
        ~ & White-browed sparrow-weaver (\textit{Plocepasser mahali}) & -17.2* & ~ \\ 
        Swanson, 1991 & Dark-eyed junco (\textit{Junco hyemalis}) & 5.6* & Temperate \\ 
        Weathers and Caccamise, 1975 & Monk parakeet (\textit{Myiopsitta monachus}) & -21.8* & Temperate \\ 
        Wu et al., 2015 & Chinese Hwamei (\textit{Garrulax canorus}) & 19.9* & Temperate \\ 
        Zheng et al., 2008 & Eurasian tree sparrow (\textit{Passer montanus}) & 41.5* & Temperate \\ 
        M$_{\text{sum}}$ & ~ & ~ & ~ \\ 
        Reference & Species & \% variation (summer - winter) & Climate \\ 
        Arens and Cooper, 2005 & House sparrow (\textit{Passer domesticus}) & 29.8* & Temperate \\ 
        Cooper, 2002 & Mountain chickadee (\textit{Poecile gambeli}) & 26.1* & Temperate \\ 
        ~ & Juniper titmouse (\textit{Baeolophus ridgwayi}) & 9.6* & ~ \\ 
        Cooper and Swanson, 1994 & Black-capped chickadee (\textit{Poecile atricapillus}) & 35.8* & Temperate \\ 
        Dawson and Smith, 1986 & American goldfinch (\textit{Carduelis tristis}) & 15.0* & Temperate \\ 
        Dawson et al., 1983 & House finch (\textit{Carpodacus mexicanus}) & 9.2 & Temperate \\ 
        ~ & House finch (\textit{Carpodacus mexicanus}) & 2.9 & ~ \\ 
        Hart, 1962 & House sparrow (\textit{Passer domesticus}) & 42.9* & Temperate \\ 
        ~ & Evening grosbeak (\textit{Hesperiphona vespertina}) & 17.6* & ~ \\ 
        ~ & European starling (\textit{Sturnus vulgaris}) & 8.7* & ~ \\ 
        Liknes and Swanson, 1996 & White-breasted nuthatch (\textit{Sitta carolinensis}) & 55.2* & Temperate \\ 
        ~ & Downy woodpecker (\textit{Dryobates pubescens}) & 51.7* & ~ \\ 
        Liknes et al., 2002 & American goldfinch (\textit{Carduelis tristis}) & 31.4* & Temperate \\ 
        O’Connor, 1995 & House finch (\textit{Carpodacus mexicanus}) & 30.4* & Temperate \\ 
        Saarela et al., 1995 & Eurasian siskin (\textit{Spinus spinus}) & 8.4* & Temperate \\ 
        ~ & Eurasian greenfinch (\textit{Chloris chloris}) & -16.2 & ~ \\ 
        Swanson, 1990 & Dark-eyed junco (\textit{Junco hyemalis}) & 27.9* & Temperate \\ 
        Swanson and Liknes, 2006 & Black-capped chickadee (\textit{Poecile atricapillus}) & 36.7* & Temperate \\ 
        ~ & White-breasted nuthatch (\textit{Sitta carolinensis}) & 27.5* & ~ \\ 
        ~ & Downy woodpecker (\textit{Dryobates pubescens}) & -7.0 & ~ \\ 
        ~ & Northern cardinal (\textit{Cardinalis cardinalis}) & 34.7* & ~ \\ 
        ~ & House finch (\textit{Carpodacus mexicanus}) & 8.5 & ~ \\ 
        Swanson and Weinacht, 1997 & Northern bobwhite (\textit{Colinus virginianus}) & 9.1 & Temperate \\ 
        Swanson et al., 2009 & House sparrow (\textit{Passer domesticus}) & 10.7* & Temperate \\ 
        Wells and Schaeffer, 2012 & Rufous-tailed hummingbird (\textit{Amazilia tzacatl}) & -31.8* & Tropical \\ 
        ~ & Blue-gray tanager (\textit{Thraupis episcopus}) & -12.7* & ~ \\ 
        ~ & Variable seedeater (\textit{Sporophila corvina}) & -27.0* & ~ \\ 
        ~ & Bananaquit (\textit{Coereba flaveola}) & -14.2 & ~ \\ 
        ~ & Dusky antbird (\textit{Cercomacroides tyrannina}) & -22.7* & ~ \\ 
        ~ & Chestnut-blacked antbird (\textit{Poliocrania exsul}) & -21.6 & ~ \\ 
        ~ & Slaty antshrike (\textit{Thamnophilus atrinucha}) & -35.2* & ~ \\ 
        ~ & Tufted titmouse (\textit{Baeolophus bicolor}) & 44.9* & ~ \\ \hline
    \end{tabular}
\end{adjustbox}
\end{table}



%%%%%%%%%%%%%%%%%%%%%%%%%%%%%%%%%%%%%General discussion %%%%%%%%%%%%%%%%%%%%%%%%%%%%%%%%%%%%%%%%%


 \csname @openrightfalse\endcsname	
 \clearpage
 \thispagestyle{plain} % empty 
 %\CenterWallPaper{1.1}{CH3.jpg}
 \backmatter
% \newpage{\thispagestyle{empty}\cleardoublepage}
 %\ClearWallPaper

\chapter{General discussion}
\label{discussion}
\chaptermark{General discussion}
\lettergroupID{\thechapter}

\begin{flushright} \color{black}Cesare Pacioni
	\color{black}\end{flushright}


\color{black}
\newpage
\clearpage


\setcounter{section}{0}
\setcounter{figure}{0}
\renewcommand*{\thesection}{D.\arabic{section}}
\renewcommand*{\thefigure}{D.\arabic{figure}}

In my PhD, I investigated how small birds adapt physiologically to changing environmental conditions, with a focus on thermoregulation. We started by quantifying the thermoregulatory capacity of tropical birds by analyzing seasonal variations in metabolic rates and testing the assumptions of the aerobic capacity model for the evolution of endothermy. Three tropical species introduced to Europe were used as models because of their exposure to colder climates. We then examined intraspecific variation in metabolic rates by comparing different populations of a temperate species and by examining a native population of one of the previously studied tropical species. Finally, we combined measurements of aerobic capacity and metabolism with field measurements of metabolic rates in a temperate bird species to understand seasonal variation in metabolic rates and energy allocation and to assess their consistency with the aerobic capacity model of endothermic evolution. Here integrate these research findings in the light of available and lacking information and suggest priorities for future research.\\

\section{Thermoregulatory capacity of tropical birds}
A presumed lack of phenotypic flexibility in tropical birds has traditionally been attributed to their often comparatively small distribution ranges, implying limited thermal tolerances and increased thermal sensitivity, but the empirical evidence supporting this hypothesis is surprisingly limited (Colwell et al., 2008; Laurance et al., 2011). Only recently has research focused on how individual tropical birds respond to the magnitude of thermal variability they experience, and the results discussed above (chapters 1 and 3) contribute to the emerging view that tropical birds appear to be able to adjust their thermoregulatory traits and exhibit greater phenotypic flexibility than previously thought (McKechnie et al., 2015; Pollock et al., 2019). However, because the tropics are warming more and experiencing more extreme heat events than temperate regions in the context of climate change (Zeng et al., 2021), most of the ecophysiological studies in the literature have focused on the upper part of the TNZ (Czenze et al., 2020; Noakes et al., 2016; Wojciechowski et al., 2021). In their recent review, Monge et al. (2023) showed that many tropical birds living in hot and arid environments actually have the physiological capacity to cope with fluctuating temperatures and are well prepared to withstand elevated levels of heat. Pollock et al. (2021) measured thermoregulatory responses to acute heat stress in 81 bird species (23 temperate, 58 tropical) by assessing the heat tolerance limit (the air temperature at which an endotherm loses the ability to regulate its body temperature). They found that overall, from a physiological perspective, tropical birds may not be systematically more vulnerable to climate warming than temperate birds.\\

In this dissertation, however, we focused on investigating the thermoregulatory capacity of (sub)tropical species in the lower part of the TNZ. The choice of species that are also successful invaders as a model was ideal because they experience different climates in their invasive range compared to their native range. This choice also allowed us to investigate whether their thermoregulatory capacity might contribute to their success as invasive species. We showed (chapter 1) that the common waxbill, orange-cheeked waxbill, and black-rumped waxbill were able to tolerate colder temperatures in autumn compared to the summer, and this was associated with a decrease in both BMR and M$_{\text{sum}}$. Similarly, we showed (chapter 3) that free-living common waxbills decreased their BMR when experiencing colder temperatures.\\

Several other species in the Southern Hemisphere have also shown a decrease in BMR in response to cold temperatures as a means of reducing thermoregulatory costs (Bush et al., 2008; Thabethe et al., 2013). For example, research by Smit and McKechnie (2010) showed a significant reduction in winter BMR compared to summer BMR in five South African bird species, suggesting that winter BMR in (sub)tropical environments is primarily driven by the need for energy conservation rather than cold tolerance. Similarly, Maddocks and Geiser (2000) observed a significant decrease in winter BMR compared to summer in Australian silver eyes (\textit{Zosterops lateralis}) to cope with long winter nights. Swanson et al. (2017) and Bozinovic and Sabat (2010) proposed that in resource-limited environments and periods (e.g., winter), organisms may enhance survival by decreasing BMR, thereby reducing daily energy expenditure and food requirements and allowing energy to be allocated more efficiently to other essential functions. Consequently, our findings are consistent with the existing literature indicating that (sub)tropical birds are likely to adopt an energy conservation strategy, rather than increasing BMR in response to colder temperatures. Facultative hypothermia may be the mechanism used by native common waxbills (chapter 3) to achieve such energy conservation. Indeed, a reduction in metabolic rate has been shown to be a strategy commonly used by many avian species during cold periods (Doucette et al., 2011; Körtner et al., 2000; 2001; Ruf and Geiser, 2015). However, the lack of studies on (sub)tropical species highlights the need for additional research to determine whether reduced thermoregulatory costs through facultative hypothermia are common strategies used by invasive (sub)tropical species living in cold climates.\\

Seasonal patterns of M$_{\text{sum}}$ variation in (sub)tropical birds are much more variable than BMR (McKechnie et al., 2015), with increases, decreases, or no seasonal change in winter (Noakes and McKechnie, 2020; Noakes et al., 2017; Pollock et al., 2019; Smit and McKechnie, 2010; Van de Ven et al., 2013; Wells and Schaeffer, 2012). Common waxbills held in aviaries in Belgium tolerated colder temperatures (chapter 1) as the cold season approached, but contrary to our expectations, this tolerance was not associated with increased M$_{\text{sum}}$ (i.e., increased shivering thermogenesis). This was particularly surprising, as it suggests that M$_{\text{sum}}$ may be influenced by factors beyond thermoregulation alone, and challenges the assumption that the main response to cold temperatures is an increase in M$_{\text{sum}}$, as previously thought. Increased cold tolerance may alternatively result from an increase in body mass and/or insulation. Increased fat reserves are commonly observed in overwintering (temperate) birds (Cooper, 2007; Krishnan et al., 2023; Le Pogam et al., 2020; Swanson, 1991). These fat stores serve several purposes, including providing energy for thermoregulation, acting as a high-energy reserve during times of food shortage (Dawson and Marsh, 1986), and increasing insulation, especially when stored subcutaneously. However, waxbills showed no significant seasonal variation in body mass (chapter 1). This is consistent with numerous other studies of small (sub)tropical species (Gosler, 2002; Polo et al., 2007), suggesting that adjustments in fat stores during seasonal acclimation are of limited importance for small birds, due to the constraints imposed by their small body size and locomotor requirements (Swanson and Vézina, 2015).\\

The lack of a clear association between M$_{\text{sum}}$ and body mass with increased cold tolerance raises questions about additional factors that are likely to drive the ability of (sub)tropical species to tolerate colder temperatures. For example, recent studies suggest that non-shivering thermogenesis may contribute to increased thermogenic capacity in birds (Barceló et al., 2017; Milbergue et al., 2018; Noakes et al., 2020; Vézina et al., 2011). It has been shown that cellular and biochemical adjustments, such as changes in mitochondrial number, aerobic enzyme activities, substrate transport pathways, and possibly mechanisms that promote muscular non-shivering thermogenesis, can contribute to increased cold tolerance by altering the thermogenic capacity of the organism (Milbergue et al., 2018; Swanson, 2010). For example, Milbergue et al. (2022) investigated the role of mitochondrial function in thermoregulatory responses to cold in black-capped chickadees (\textit{Poecile atricapillus}) and found that thermogenic capacity is nevertheless associated with significant adjustments in muscle or liver mitochondrial machinery. However, while non-shivering thermogenesis has been extensively documented in mammals (Janský, 1973) and juvenile birds (Dumonteil et al., 1995; Teulier et al., 2014), limited research has examined non-shivering thermogenesis in adult birds, particularly from tropical regions, resulting in a gap in our understanding. Therefore, studying metabolic rate variations and non-shivering thermogenesis simultaneously may provide a more comprehensive understanding of cold tolerance in (sub)tropical birds.\\

Another mechanism for increasing cold tolerance would be to reduce thermal conductance by increasing plumage insulation (2.5. Discussion; Cooper, 2002; Cooper and Swanson, 1994; González-Medina et al., 2023; Piersma et al., 1995; Swanson, 1990). Feathers play a role in reducing the loss of body heat to the environment and are therefore an integral part of avian thermoregulation (Dove et al., 2007; Stettenheim, 2000). For example, the structure of avian contour feathers, consisting of both proximal plumulaceous (downy) and distal pennaceous sections, has been shown to play a critical role in thermal insulation (Dove et al., 2007; Stettenheim, 2000). Within species, these structures are variable and controlled by environmental conditions (Broggi et al., 2005; Vagasi et al., 2012). Therefore, it would be interesting to conduct a comprehensive investigation of the role of plumage insulation between native and invasive bird populations, with the aim of determining whether variation in plumage characteristics serves as one of the primary drivers of cold adaptability in invasive (sub)tropical bird species. Moreover, behavioral mechanisms, such as e.g. sunbathing. also influence the range of ambient temperatures birds are able to withstand (Krishnan et al., 2023). For example, research has shown that birds can decrease their metabolic rate by up to 50\% when moving from a shaded, windy location to a sunny, wind-protected area (Wolf and Walsberg, 1996). Similarly, fluffing their feathers to trap air (i.e., ptiloerection) and thereby increasing insulation can reduce heat loss (Hill et al., 1980), though the magnitude of thermoregulatory benefits offered by ptiloerection in birds remains difficult to quantify. Moreover, some birds has been shown to crowd and/or flock together to share body heat and minimize heat loss (Pani and Bal, 2022). The lack of detailed investigation of these potential behavioral mechanisms in this study highlights a potential gap in our understanding of how these birds cope with cold temperatures.  Therefore, our findings prompt further investigation into the thermoregulatory strategies used by (sub)tropical bird species to cope with cold temperatures. The observed decrease in metabolic rate in response to cold temperatures suggests that energy conservation may be the primary metabolic response in these birds, contributing to their widespread invasion success in environments characterized by cold winters and less predictable weather. Furthermore, the potential role of facultative hypothermia, non-shivering thermogenesis, the influence of thermal conductance, and behavioral mechanisms on cold tolerance are potential factors that warrant further investigation in order to provide a more comprehensive understanding of the factors influencing cold tolerance and thermoregulatory adaptations in these species.\\

\section{Variation and functional link of metabolic rates}

\subsection{Annual variation in energy expenditure}
Energy is critical for the basic functioning and maintenance of organisms (Bonn et al., 2004), and understanding how individuals allocate it annually to cope with changing environmental conditions and physiological states is a critical question in avian ecology. Studies have mainly used FMR as the primary predictor of annual energy expenditure (Bryant and Tatner, 1988; Gales and Green, 1990; Masman et al., 1988; Weathers and Sullivan, 1993; Williams, 2001). Our study (chapter 4) highlights the importance of including both BMR and M$_{\text{sum}}$ in assessments of energy allocation, particularly in the context of resident temperate species. These species undergo demanding life history events, such as reproduction and wintering, that require adjustments in metabolic rates (XuanYuan et al., 2023). This flexibility allows for upregulation during periods of increased metabolic demand and downregulation to prevent excessive energy expenditure during less demanding periods. Relying solely on FMR to assess energy expenditure may overlook important aspects of energy allocation, including thermoregulation and maintenance costs (Broggi and Nilsson, 2023). Furthermore, given the unclear relationship between BMR, M$_{\text{sum}}$, and FMR, FMR cannot be considered a reliable predictor of BMR and M$_{\text{sum}}$. While some studies have examined the relationship between BMR and FMR and found significant correlations in avian species (Daan et al., 1990; Daan et al., 1991; Koteja, 1991), others have found significant correlations in mammals but not in birds (Ricklefs et al., 1996). To my knowledge, the relationship between M$_{\text{sum}}$ and FMR has not been studied. This gap in knowledge underscores the importance of further research to elucidate the interplay between these metabolic parameters and their implications for energy allocation strategies in avian species. \\

The relatively small seasonal variation in metabolic rates between winter and breeding (chapter 4) suggests that great tits maintain a relatively constant energy budget between these two periods, which is broadly consistent with the reallocation hypothesis. The reallocation hypothesis, as proposed by West (1968), posits that birds experience a decrease in energy costs for foraging and thermoregulation during periods of maximum food availability and moderate ambient temperatures, which typically coincide with breeding seasons. These energy savings can then be redirected to activities associated with breeding. In studies from cold Arctic environments or highly seasonal environments, it is easy to detect differences in energy expenditure between periods. However, for resident temperate birds with less pronounced seasonality (chapter 4), where winters are not extremely cold, detecting differences becomes more challenging and requires a closer examination of energy allocation. For example, in the literature, the seasonal allocation of energy expenditure shows the greatest variability in studies conducted in milder winter climates (Table 4.5). These studies show both increases and decreases in energy expenditure during winter, as well as cases where no clear seasonal patterns emerge (Noakes et al., 2017; Pollock et al., 2019; van de Ven et al., 2013; Wells and Schaeffer, 2012), suggesting that energy expenditure in mild winter climates is not solely related to thermoregulation, but may be influenced by other factors such as changes in food availability. This may have been our case (chapter 4), as the great tits experienced mild winter conditions and were surrounded by bird feeders, which may have influenced the dynamics of energy allocation and expenditure. The presence of bird feeders likely provided a continuous and readily accessible food source, potentially mitigating the typical seasonal fluctuations in energy expenditure associated with foraging. For example, studies have shown that species that rely on difficult-to-capture prey tend to have higher energy expenditure (Zhang et al., 2021), especially during the breeding season (Weathers and Sullivan, 1993). In general, when more food is consumed, greater processing capacity is required, which can lead to increased size of digestive and cardiopulmonary organs (Mueller and Diamond, 2001, Nilsson, 2002, Petit et al., 2014), which in turn affects BMR. \\

Therefore, in addition to an increased focus on BMR and M$_{\text{sum}}$, several approaches could be used. For example, conducting extensive multi-year studies across seasons with varying degrees of winter severity could provide valuable data on how temperature variations affect annual energy expenditure patterns. Examining the diet composition of birds during the winter and breeding seasons could shed light on how changes in food availability affect energy allocation strategies. Furthermore, given the demonstrated impact of urban environments on avian energy allocation (Ellis and Bowman, 2020), studying the same bird species in less urbanized areas could provide valuable insights into how habitat types influence energy allocation strategies, as understanding how birds adapt their energy allocation in different habitats can tell us more about their physiological flexibility. Finally, as climate change results in more frequent and unpredictable weather events (e.g., heat waves and cold spells), incorporating the temperatures experienced by birds in the previous week or day (chapter 2; Bushuev et al., 2021) could provide specific and timely information on the immediate impacts of climate variability on avian energy allocation. This could help to determine whether these events have a greater impact on total energy expenditure during certain seasons. Incorporating these ecological considerations into our interpretation highlights the complexity of energy dynamics in avian populations and the need for further investigation of the interplay between environmental factors and energy allocation strategies in birds.\\

\subsection{Intraspecific variation in metabolic rates}

Metabolic rates have been used extensively in comparative physiology to address important ecological and evolutionary questions (Norin and Metcalfe, 2019). Traditionally, with species treated as units, these studies have relied heavily on interspecific comparisons (McKechnie and Swanson, 2010; McNab, 2002; Reed et al., 2011; Swanson, 2010; Thomas et al., 2004). Despite being similar in size and shape, organisms can exhibit significant variation in metabolic rates, sometimes differing by several orders of magnitude (Gillooly et al., 2001; Kleiber, 1961; Mueller and Diamond, 2001). Recently, however, there has been a shift towards examining intraspecific variation as well (Swanson et al., 2020). While interspecific analyses inherently focus on determining the ultimate factors contributing to the variation in metabolic rates (Cruz-Neto and Jones, 2004), intraspecific analyses can provide insights into the specific factors contributing to metabolic variability, along with their underlying mechanisms and functional significance (Li et al., 2024). We focused on testing for intraspecific variation in ecophysiological traits related to thermoregulation using great tits living in two geographically and climatically separate locations with different winter conditions (chapter 2). We showed that BMR and M$_{\text{sum}}$ can vary independently in response to environmental factors. Great tits from the colder site (Zvenigorod, Russia) had a significantly higher thermogenic capacity (M$_{\text{sum}}$) compared to those from the warmer site (Gontrode, Belgium), but they had a lower BMR.\\

The higher M$_{\text{sum}}$ observed in the colder-climate great tit population (chapter 2) is broadly consistent with studies indicating that this metabolic rate is highly correlated with northwards geographic distribution limits in birds (Swanson and Garland, 2009). High M$_{\text{sum}}$ is typically found in birds inhabiting regions with cold winter climates, suggesting its importance as a determinant of the ability to overwinter in such environments (Swanson et al., 2009). For example, Swanson and Garland (2007) analyzed M$_{\text{sum}}$ data from 44 bird species and found that climate plays a significant role in determining M$_{\text{sum}}$ variation, with birds in colder climates having higher M$_{\text{sum}}$ than those in warmer climates. A notable difference from these findings, however, is that an increase in M$_{\text{sum}}$ is usually accompanied by a corresponding increase in BMR (Auer et al., 2017; Cooper and Swanson, 1994; Rezende et al., 2002; Welman et al., 2022). The finding, in fact, that great tits (chapter 2) inhabiting colder climates had lower BMR than those in warmer climates contrasts with previous studies suggesting a positive relationship between BMR and winter severity in widely distributed species (Petit et al., 2017). For example, Broggi et al. (2007) found that great tits from a northern population experiencing cold seasons had higher BMR than those from a southern population. Similarly, Wikelski et al. (2003) studied stonechat (\textit{Saxicola rubicola}) populations in different geographic locations and found that the population living in the coldest climate had the highest resting metabolic rate. However, recent research has suggested that latitudinal trends in metabolic rate may also be influenced by a complex interplay between increased thermogenic capacity and the need to reduce excessive maintenance costs in cold environments with limited food availability. For example, Le Pogam et al. (2020) showed that snow buntings (\textit{Plectrophenax nivalis}) increased their M$_{\text{sum}}$ by about 25\% during cold Canadian winters without a concomitant increase in BMR. Liu et al. (2023) studied the thermoregulatory traits of the northwardly colonizing White-browed Laughingthrush (\textit{Pterorhinus sannio}) in comparison to coexisting species, revealing a lower BMR similar to tropical species despite colonizing a temperate zone, suggesting an energy-saving strategy as a thermoregulatory adaptation allowing its expansion to colder latitudes. In the same direction, from an energetic perspective, Swanson et al. (2017) showed that natural selection should tend to favor minimizing BMR to conserve energy for other functions, and this should be particularly evident in northern populations where thermoregulation and activity costs are highest, such as the great tit population in Russia (chapter 2). \\

To test for intraspecific variation, we also studied common waxbills in a part of their native range (South Africa) for comparison with the captive population held in Belgium (chapter 1). Both chapters (1 and 3) highlight comparable findings regarding the response of waxbills to cold temperatures, characterized by a downregulation of BMR and M$_{\text{sum}}$ toward colder seasons (chapter 1) and a downregulation of basal metabolic rate at colder temperatures (chapter 3), suggesting a common energy conservation strategy among populations regardless of their origin or environmental conditions. Overall, these findings are consistent with Lovegrove's (2000, 2003) suggestion that due to the generally milder winters in the Afrotropical region compared to the Northern Hemisphere, energy conservation becomes the primary seasonal metabolic response in Afrotropical species, which may be used as a strategy to cope with cold winters in their invasion range. However, studies have shown that invasive populations of waxbills in Spain are sensitive to severe winter conditions (Keller et al., 2020). Therefore, it would be of particular interest to investigate the thermoregulatory characteristics of wild, free-living common waxbill within its invasive range. \\

Also in the context of biological invasions, intraspecific physiological analysis has recently been identified as one of the most important tools for understanding and predicting invasion success (Boardman, 2022). Clusters of hypotheses and frameworks in invasion biology that incorporate physiology focus primarily on traits associated with invasiveness (Enders et al., 2020), and individuals of an invasive species are often assumed to be genetically uniform or to respond uniformly to all environmental conditions. In this context, Boardman et al. (2022) recently argued that studying intraspecific variation in invasive species from a physiological perspective would provide a comprehensive understanding of the physiological mechanisms that allow these organisms to survive in novel conditions. While there is an encouraging trend toward increased use of physiology in invasion-related studies, the proportion of invasion papers that include physiology remains low, suggesting that the potential of physiology in the study of invasive species is still underutilized (Boardman et al., 2022). \\

\subsection{Functional link of metabolic rates}

The aerobic capacity model for the evolution of endothermy proposes a positive correlation between BMR and M$_{\text{sum}}$ (Bennett and Ruben, 1979). This relationship has been supported by many studies at the interspecific level (Dutenhoffer and Swanson, 1996; Rezende et al., 2002), and a recent meta-analysis of phenotypic correlations between these two metabolic rates in different vertebrate species confirmed this positive association (Auer et al., 2017). In reviewing these studies, it should be noted that this relationship was found where data were largely obtained from temperate birds, whereas no relationship was found for tropical species, which are likely to face little selective pressure for high levels of regulatory thermogenesis (Wiersma et al., 2007a). Consistent with this, we found no evidence to support the aerobic capacity model among the three waxbill species studied at the intraspecific level (chapter 1).\\

Studies at the intraspecific level have more recently challenged the assumption of the aerobic capacity model (Noakes et al., 2017; van de Ven et al., 2013). We tested (chapter 2 and 4) the intraspecific validity of the aerobic capacity model, and results suggest that at least for the species studied here, no meaningful relationships were present between basal and summit metabolism, revealing that the functional relationship between metabolic rates is more complex than previously thought. One possible explanation is that BMR and M$_{\text{sum}}$ respond to different constraints and vary independently. BMR is primarily a function of internal organ activity, whereas M$_{\text{sum}}$ is primarily a function of skeletal muscle function (Swanson, 2010). We discussed (chapter 2) that a decoupling of BMR and M$_{\text{sum}}$ could be caused by different selective pressures affecting these metabolic rates (Petit et al., 2013). In particular, great tits living in the coldest climates faced the need to conserve energy by maintaining a relatively low BMR while simultaneously increasing their thermogenic capacity. Although studies in species living in seasonal environments have documented a winter response with BMR and M$_{\text{sum}}$ elevated in tandem, the mechanisms underlying this covariation remain controversial. For example, Barceló et al. (2016) found that food consumption and digestive organs were the main drivers of variation in BMR in white-throated sparrows (\textit{Zonotrichia albicollis}), while M$_{\text{sum}}$ varied positively with skeletal muscle, suggesting that BMR and M$_{\text{sum}}$ may respond in tandem but reflect different physiological systems. \\

Moreover, we found (chapter 4) that while birds with higher whole-body BMR also had higher whole-body M$_{\text{sum}}$, there was no significant correlation between mass-independent BMR and mass-independent M$_{\text{sum}}$, consistent with other intraspecific studies (Swanson et al., 2012). The lack of a significant mass-independent correlation may be due to variation in body composition among individuals, where differences in body mass among birds may reflect differences in body composition, particularly in metabolically inert tissues such as fat mass (Scott and Evans, 1992), and is particularly important in small birds where body mass varies little among individuals. Scaling metabolic rates by body mass alone to obtain mass-independent rates assumes a constant contribution of body mass to metabolic rates, which may not be true when differences in body composition are considered. When the analysis is adjusted for differences in body mass, the relationship between BMR and M$_{\text{sum}}$ becomes less clear due to possible variations in the proportion of metabolically active tissues relative to body mass. Therefore, incorporating data on the relative size of different body parts and organs may be critical to accurately determine whether the assumptions of the aerobic capacity model hold true at the intraspecific level.\\

\section{Future directions in avian ecophysiology research}

Having demonstrated the complexity of avian physiology, several avenues of future research are necessary to further our understanding of avian thermoregulation, annual energy allocation, and the validity of the aerobic capacity model for endothermy. Below, I present potential research directions in each of these areas.\\

First and foremost, this dissertation was primarily concerned with the effects of temperature on physiological responses. However, it's important to recognize that in natural environments, several factors beyond temperature can significantly affect thermal physiology. Factors such as wind, solar radiation (Wolf and Walsberg, 1996), and humidity (Gerson et al., 2014) can also influence the thermal environment experienced by animals, and may need to be explicitly considered to fully understand an organism's response to its microclimate. For example, studying the interplay between humidity and ambient temperature, especially when they impede effective heat dissipation, could provide valuable insights into species-specific vulnerabilities under climate change (Song and Beissinger, 2020). Moreover, as is generally the case in ecology and conservation biology, research efforts in avian physiology have focused primarily on temperate zones, and the few studies on tropical birds have primarily examined their physiological adaptations to high temperatures. However, with the onset of global warming and increasing climatic unpredictability, it is crucial to investigate how these birds cope with extreme weather events, including cold spells. Therefore research on tropical birds should focus on how these birds can respond to both cold and heat extremes. This is even more important given that many invasive species in temperate zones originate from the tropics. Therefore, a better understanding of the mechanisms they use to tolerate cold temperatures is essential to accurately assess and predict their invasion success.\\

My research suggests that the common waxbill may have the physiological capacity to further expand its invasive range towards colder areas, e.g. further north in Europe. Indeed, introduced waxbills have been found to have a very long breeding season, with breeding birds found from early March until at least November, even in northern Portugal (Beltrão et al., 2021), where cold spells are not uncommon. This behavior highlights the species' flexible response to harsh environmental conditions, possibly through the energy-saving strategy outlined above. This potential for expansion into colder regions is supported by several factors. In addition to the energy conservation strategy mentioned above, Msum measurements indicate that the species has the thermogenic capacity to withstand cold spells, as shown in Table 1.1, where the average cold tolerance is recorded as -2°C in heliox. Given the higher thermal conductivity of heliox, this is equivalent to about -10°C in normal air. In addition, observations of captive birds kept in open aviaries during the winter (without shelter) show their ability to withstand mild freezing temperatures. It is therefore reasonable to assume that large parts of Europe with temperate oceanic climates may also be at risk of invasion due to the common waxbills’ ability to cope with (somewhat) cold(er) environments.\\

However, as temperatures continue to rise under climate change, future scenarios may present new challenges for the common waxbill, particularly in southern Portugal, where a warmer and drier climate is expected to manifest in the coming decades.  Although my studies focused more on cold challenges rather than on heat stress, waxbills experience hot temperatures in parts of their native range, and da Silva et al. (2018) showed that in Brazil they have been able to invade climates that are on average warmer than their native range, suggesting that the species may have a considerable tolerance to higher temperatures as well. However, it is important to consider the wider ecological circumstances, and the Brazilian context likely is different from southern Portugal. For example, potential desertification and loss of habitats suitable for waxbills, such as reedbeds and other wetlands, in southern Portugal under climate change could have a negative impact on the invasive distribution of the species. Shifts in distribution and abundance may therefore be due not only to the direct effect of temperature on metabolism and energy expenditure,  but also to changes in habitat structure and availability. Nevertheless, it remains crucial to understand the physiological responses of the common waxbill at the upper limits of its TNZ. This avenue of research would provide valuable insights into the species' ability to adapt to changing temperature conditions and habitat dynamics.\\

In general, and in line with several recent studies (Boardman et al., 2022), there is a growing consensus on the need for further research into the physiology of (invasive avian) species. Such research is essential to refine models for assessing and predicting invasion success (Boardman et al., 2022). These data have the potential to improve our understanding of how invasive species adapt to novel environments and will become an important tool for predicting the future occurrence of invasive species under climate change. However, the transition from physiological research to the prediction of invasion risk requires the consideration of additional ecological mechanisms and factors. Recent research (Briscoe et al., 2022) emphasises that accurate predictions of species responses to environmental change require the incorporation of explicit mechanisms into predictive models. These mechanisms include not only physiological responses to environmental stressors, but also diverse aspects such as life history, population dynamics, dispersal and biotic interactions. Moreover, it is also important to address that laboratory measurements, while informative, may not fully capture the complexity of real-world conditions. Factors such as wind patterns, solar radiation and habitat structure can significantly influence species responses and dispersal patterns. Incorporating different ecological mechanisms into predictive frameworks to understand and predict the risks of invasive bird species in a changing climate is therefore essential to better predict and mitigate their impacts on ecosystems and biodiversity.\\


With respect to the aerobic capacity model, further research is needed to examine the validity of the model at the intraspecific level. This includes exploring the physiological mechanisms and selective pressures that influence the decoupling of BMR and M$_{\text{sum}}$, as well as the role of body composition in shaping mass-independent metabolic rates. In addition, future research at the intraspecific level should shift its focus from a heavy reliance on FMR for annual energy allocation to a greater consideration of BMR and M$_{\text{sum}}$. Long-term studies analyzing the interactions between these metabolic parameters and their effects on energy allocation strategies are needed, and the study of diet composition could provide a deeper understanding of energy dynamics in avian populations, especially in birds that experience very little seasonal variation in environmental temperature. Furthermore, environmental temperature is highly variable in both space and time, and some aspects of environmental temperature are predictable (e.g., seasonal changes) while others are not (e.g., extreme events). Therefore, a temperature regime has multiple dimensions that can be described in both space and time, with the potential to shape biological patterns in different ways (Garcia et al., 2014). In the context of climate change leading to more frequent and unpredictable weather events, incorporating short-term temperature fluctuations experienced by birds could provide valuable insights into birds' energy allocation strategies, and this could help determine whether these events have a greater impact on overall energy expenditure during certain seasons. \\

\section{Conclusions}

In this dissertation, I investigated how passerines adapt physiologically to changing environmental conditions, with a focus on intraspecific and seasonal variation in thermoregulation. My results showed that an energy conservation strategy plays a critical role in the seasonal metabolic responses of invasive Afrotropical waxbills, potentially enhancing their winter survival and facilitating their establishment in new areas with colder winters and unpredictable weather. Moreover, by focusing on the intraspecific level, we showed that considering a species as a physiological unit can be misleading, as different metabolic traits may be influenced by different selective pressures that shape local adaptations, for example in response to different degrees of seasonality. Finally, the complexity of these responses challenges the assumptions of the aerobic capacity model for the evolution of endothermy and points to the need for further research to investigate why and when basal and summit metabolic rates are phenotypically and functionally linked.\\

\clearpage




%%%%%%%%%%%%%%%%%%%%%%%%%%%%%%%%%%%%%Acknowledgments%%%%%%%%%%%%%%%%%%%%%%%%%%%%%%%%%%%%%%%%%


%\thispagestyle{plain}
%\hbox{}
%\clearpage


%%%%%%%%%%%%%%%%%%%%%%%%%%%%%%%%%%%%%Acknowledgments%%%%%%%%%%%%%%%%%%%%%%%%%%%%%%%%%%%%%
	
\chapter*{Acknowledgments/Ringraziamenti}
\pagestyle{mainmatter}
\chaptermark{Acknowledgments/Ringraziamenti}
\addcontentsline{toc}{chapter}{Acknowledgments/Ringraziamenti}
\label{Acknowledgments/Ringraziamenti}
	


\begin{flushright} \color{black}Cesare Pacioni
	\color{black}\end{flushright}


\clearpage

		Writing acknowledgements is a big challenge for me, as there are so many memories and people to thank that it's almost impossible to put them into words, but I'll try. No, I'll do it. As Yoda famously said, "Do or do not. There is no try". First and foremost, my supervisors. \textbf{Luc}, I'd like to express my immense gratitude for everything you've done throughout my PhD. Your kindness and support have made all the difference. Your enthusiasm and positive energy are contagious. You believed in me from day one and I couldn't be happier or more honored to have had the chance to work with you. It's been an incredible experience that I will always treasure. \textbf{Diederik}, my limited English vocabulary may not allow me to fully express what you've meant to me over the past three years. The amount of help and support I've received from you can't be estimated. Even during the toughest times (remember those M$_{\text{sum}}$ measurements?), you were always there for me. Thank you for shaping me into a scientist, from our first email exchange when I was searching for a PhD until now. Your trust and encouragement were exactly what I needed. Every meeting with you started with funny stories and ended with solutions and great ideas (and ended up with the loss of one of my pens as well, but that's okay). I feel honored to have been your PhD student and to have shared this journey of ecophysiological measurements with you. Your humor, intelligence, and love for science have made my PhD unforgettable. You're the best promoter and mentor anyone could ask for. \textbf{Andrey} and \textbf{Anvar}, many thanks to both of you for your support over the past three years. I have learnt so much from both of you. You were always willing to help. I can't believe that less than three years ago I couldn't even turn on the machine to take measurements. I couldn't be happier to have had you both on the project and I can't wait for the day when we can finally meet, have a drink and talk about metabolic rates face to face. \textbf{Marina}, we started this PhD together and I couldn't have asked for a better partner in crime. COVID and moving to a foreign country made me feel isolated at first, but having you with me from the beginning made everything better. Thank you for always being such a good companion. Thank you for teaching me so much about birds and boosting my eBird profile. Thank you for the fun board game nights. I can proudly count on one hand the number of times I have beaten you at Wingspan. Thank you for letting me be your cat-sitter a few times. I'd do it again any time you’ll need it. I think it is very difficult to find two PhD students who have spent as much time together as we have. In the lab and during fieldwork, at every moment I felt privileged to have a friend with me rather than just a colleague. You have made this journey easier and more fun. Thanks for the kind words of support when I did not believe in myself. I will miss you in the office. Whether I move to the 9th, or I move to a different city, country or a galaxy, I know I can always count on your friendship. All the best for the rest of your PhD. You’ll do great, I’m sure. You bow to no one, my friend. \textbf{Aleix}, thank you for being such a good friend. Your enthusiasm is contagious and I always have fun when you are around. I still remember the first time we met in Citadelpark. After that, we met several times and when Marina told me that you were about to officially move to Belgium, I was very happy because I knew that we could hang out and play board games together. And now, I'm so happy about your new job. You totally deserve it. You are a funnier and smarter version of Bob The Builder. Bedankt en tot ziens, mi amigo. \textbf{Silvija}, your friendship and company have made my PhD more enjoyable and fun. Thank you for always stopping by my desk for a quick chat, always with your two mugs in hand. Thank you for the amazing cakes you always bake for everyone *Be Our Guest playing in the background*. You made me try whiskey sours, but that's OK. Thank you for the thousands of instagram reels you send me daily and I only watch them a week after you send them. Thank you for introducing me to the world of zumba. I still have to teach you how to drive: I haven't forgotten, don't worry. You've basically just started your own PhD: I know you'll crush it. Taxidriverug will hopefully be around for a while and wishes you all the best for your PhD. Clean the window and Go To The Distance, my friend. \textbf{Felipe}, you are one of the funniest and smartest people I know. It's impressive how easily you can switch from serious discussions to cracking jokes. Every time we meet, I have fun and learn something. You’ve been very important to me over the last three years, with your words of support and friendship. Thanks for always coming to my desk to have a small talk and a small laugh, and also to get some food from me as well (you are welcome). I know that if I need a chat over a beer, I can always count on you. Your ambition will take you far, I am sure of it. Academia needs your curiosity and your passion for knowledge. I wish you all the best for your PhD. Su navi per mari, amico mio. With \textbf{Marthe}, you make a great couple, and I thank you for the nice moments we shared together in my last year of my PhD. \textbf{Jens} and \textbf{Roki}. We've had some great times together. We traveled a lot and played so many games (though charades is definitely not our thing, Jens). Your company always made me laugh and it made my life as a PhD more fun. I'm really happy for the journey you have recently decided to start and happy that you, Roki, have moved to Belgium, so that we can see each other more often in the future. Best wishes with your PhD, Jens. Koolmees might be aggressive and difficult to handle, but they are cute and adorable companions. I wish you both all the best for the future. \textbf{Emma B}, Marina and I couldn't have asked for a better companion during our fieldwork in Portugal. Your kindness, humor and willingness to help made the whole experience so much easier and more enjoyable. I had so much fun and I really miss those times. We were so happy to have you and that you did not run away the second time we asked you to go to Portugal. I always keep those adventures and stories in mind and it brings a smile to my face. I am so glad that you were a big part of my PhD and now that you live in Ghent, we will hopefully see each other more often. I always look forward to playing Wingspan with you. The best part is waiting 5-6 minutes for you to finish your turn. Best wishes with your PhD, but remember: we were and always will be The Students. We know (wink, wink). \textbf{Emma VR} and \textbf{Margarita}, over the past few months you have both brought a lot of joy and energy to the office and to TEREC in general. It was a much needed boost for the final rush of my PhD. Thank you for always being there to roast me at any time of the day and for always bringing a smile to my face. Your kind words of support always made my day. Emma, thanks for always leaving random things on my desk. Margarita, thank you for your singing advice. Wherever the future takes us, I am sure we will still be hanging out together. I have been very privileged to have three amazing female scientists in the office. Laughter, science and moral support: I will miss all of that. \textbf{Femke}, you are one of the kindest and most helpful people I know. You have always been there for me over the past three years. Thank you for stopping by my desk often to ask how I was doing. Thank you for your advice. You have been one of the few people always around who is ready to help newcomers to TEREC. Thank you for squash. I'm really sorry that we were neighbors for too short a time. But I know that if I send you a message or come down/up to the tenth floor, you are always available for a chat and to meet up. Thank you from the bottom of my heart, and I am extremely happy that you stayed after your PhD. TEREC wouldn't be the same without you. \textbf{Daan}, thank you very much for the nice conversations and for the always nice advice about the great tits. \textbf{Ruben}, you and Femke were the ones I always looked up to. Thank you for always being there with your help and advice. It was really hard to see you leave TEREC, but I'm really happy that you've found a job outside of academia that you really enjoy. It's always great to see you at TEREC events from time to time and it's a reminder of the impact you've had on everyone here. I wish you all the best for the future, my friend. \textbf{Alizee}, even though our paths didn't cross often during my PhD, your constant availability and support via WhatsApp meant a lot. Whether it was for a quick chat or to seek advice, you were always there for me. I'm looking forward to the day when we can finally meet in person and enjoy some board games together. May the force be with you, my padawan. \textbf{Laurence}, thank you for your positivity and enthusiasm. Thank you for the advice you gave me for my postdoc application. Thank you for the nice conversations and for always being such a fun person at parties. We share a huge passion for music and I cannot wait to see you perform live. Je te souhaite tout le meilleur pour l'avenir, mon ami. \textbf{Charlotte T}, you bring so much joy and kindness to TEREC. Thank you for always being so kind to stop at my desk and have a small chat. Your support during these past few months, whether for my postdoc application or my PhD thesis, has meant a lot to me. I wish you all the best for your PhD. You'll do great, I am sure of it. \textbf{Bram C}, thank you for always giving me good advice about great tits. Thanks for always stopping by before you left the office and saying "Don't work too late, OK?”. Thanks for the thousands of kicker games (Belgium vs Southern Europe). You are about to finish your PhD and I wish you all the best. As I recently told you, I will not disappear if I have to move to another floor. Geen zorgen, mijn vriend. \textbf{Charlotte VD}, thank you for your kind words of support over the past few months. It's nice to see you more often in the office now that you're more or less done with the data collection. Don't stress too much about the PhD and the postdoc. You are doing great and will do great. Your ambition will take you far, I am sure. If I ever move, I'll put an extra chair in my new office and you can always stop by for a chat. Remember: chi dorme non piglia pesci. \textbf{Katrien} and \textbf{Garben}, you are two amazing enthusiastics scientists. Your passion for science and spiders is contagious. I wish you all the best for the future. \textbf{Frederik M}, your enthusiasm and intelligence are remarkable. Thank you for the nice conversations over the last three years. Especially about tiramisù. Yours was great and you should be proud. \textbf{Camille}, thank you for your kind words of support over the last few months. It’s always nice and fun to talk to you. I wish you all the best for the future. A big thank you to \textbf{Alix}, \textbf{Aurore}, and \textbf{Thomas C}: it is always nice to see you back at TEREC. Thank you, \textbf{Karen} and \textbf{Jonathan}, for all the nice conversations and I wish you all the best in Sweden. To the new faces at TEREC: \textbf{Bram VB}, \textbf{Diego}, \textbf{Ellen}, \textbf{Frederik VD}, \textbf{Godefroi}, \textbf{Heleen}, \textbf{Lucy}, \textbf{Maaike}, and \textbf{Maxime} (welcome back). I want to give a warm welcome to all of you. I wish you nothing but the best for your research. \textbf{Angelica}, thank you for always stopping by my desk while you go and get the coffee. Thank you always for the kind words of support. Thank you for all the help you have given me over the last three years. I really appreciate it. \textbf{Viki} and \textbf{Pieter}, thank you for always being available for help and advice. \textbf{Hans}, thank you for all the help and guidance you have given me over the last three years. In return, I sometimes let you win at kicker. You are welcome. TEREC is truly fortunate to have the four of you. \textbf{Nicky} and \textbf{Thomas P}, although our interactions were brief during my PhD, I would like to extend my sincere thanks to both of you. I wish you all the best in your future careers. \textbf{Dries}, thank you for being such a good team builder. Your presence, enthusiasm and humor are so important for TEREC. Thank also you for chairing my PhD defense.  \textbf{TEREC}, in general, thank you for being such an open and international research group. Thank you for the drinks at Maison W. Thank you for the parties and the fun and constructive lab meetings. I am so proud to have been a part of you. My jury members, \textbf{Beate Apfelbeck}, \textbf{Luis Reino}, \textbf{Geert Janssens} and \textbf{Matthew Shawkey}. You provided me with helpful comments that greatly improved this thesis. Thank you all. Cara \textbf{Prof.ssa Scocco} e caro \textbf{Prof Catorci}. Desidero esprimere la mia sincera gratitudine per il vostro costante sostegno e ispirazione durante il mio percorso universitario. Grazie a voi, ho scoperto la mia passione per le scienze naturali. I vostri insegnamenti e il vostro entusiasmo hanno alimentato il mio amore per la ricerca. Vi ringrazio di cuore per aver sempre creduto in me e per avermi spronato a dare il meglio di me stesso. Anche dopo il mio trasferimento in Belgio, avete continuato a collaborare con me e a darmi supporto, dimostrando un impegno straordinario. Sono profondamente grato per tutto ciò che avete fatto per me. Senza di voi, non sarei arrivato fin qui. Vi ringrazio di cuore. Alle amiche e agli amici di sempre. In ordine alfabetico. \textbf{Barbara}, essere mamma è sicuramente un lavoro a tempo pieno e perciò apprezzo profondamente ogni momento che riusciamo a trascorrere insieme. Un abbraccio a \textbf{Filippo}, \textbf{Maddalena} e a \textbf{Luca}. \textbf{Denny}, ancora mi fa strano vederti papà. Mi fa strano vederti in generale, ma va bene uguale. La tua semplicità e i tuoi modi british ti contraddistinguono. Sono certo che sarai un padre perfetto per quel pirulitto che hai. Con \textbf{Irene} fate una bella coppia e crescerete \textbf{Emily Jr} nel migliore dei modi, ne sono certo. Magari con l’aiuto di Laura, ma va bene uguale. \textbf{Emily}, mi mancano tantissimo le nostre serate trascorse a ridere e a giocare. La tua allegria e la tua risata contagiosa sono uniche, e ogni momento passato insieme è sempre un toccasana per il morale. Grazie per la tua spontaneità e per la gioia che porti nella mia vita. Non c'è nulla di più divertente di trascorrere del tempo con te. So che posso sempre messaggiarti e in un modo o nell’altro mi strapperai un sorriso. Ti abbraccio forte. \textbf{Alberto}, grazie per la tua semplicità e la tua generosità. Sei un ragazzo d’oro.  \textbf{Laura}, sei non solo una mamma devota e amorevole, ma anche un'amica straordinaria. I numerosi messaggi che invii non solo dimostrano il tuo affetto, ma sono anche un costante segno di sostegno e vicinanza. Mi mancano le nostre passeggiate insieme. Un abbraccio speciale a \textbf{Sophie} e a \textbf{Marco}. Vi auguro tutto il bene di questo mondo. \textbf{Matteo}, la tua ironia e il tuo sostegno non passano mai inosservati. Le foto di uccelletti che mi invii per l'identificazione sono sempre un tocco speciale nelle mie giornate. Anche se spesso non so minimamente di cosa si tratti, mi piacciono le conversazioni che creiamo attraverso queste condivisioni. Talvolta hai dubbi su te stesso, e non dovresti. Grazie per i meme, i riferimenti di AG e G e tutte le risate che mi regali. La tua capacità di farmi ridere è un dono prezioso. \textbf{Miriana}, volevo esprimerti la mia gratitudine per tutti i messaggi che mi invii, per il sostegno e per le infinite conversazioni che abbiamo condiviso. La tua presenza costante e il tuo sostegno incondizionato sono stati importantissimi per me durante il dottorato. La tua generosità e la tua prontezza ad aiutare gli altri sono davvero ammirabili. Spesso penso di non meritare la tua amicizia. Ti auguro il meglio per il tuo nuovo lavoro e per tutte le tue future avventure. \textbf{Sara}, grazie per la persona che sei. Il tuo sorriso contagioso illumina e ci fa sentire sempre benvenuti e apprezzati. Grazie per la tua generosità e il tuo affetto sincero. La tua ironia e il tuo senso dell'umorismo sono unici e irresistibili. Riesci sempre a farci ridere anche nei momenti più grigi. Grazie per essere sempre presente e per mettere gli amici al primo posto, anche prima del cibo, il che, conoscendoti, è davvero un gesto straordinario. In conclusione, grazie. Mi avete supportato e sopportato in tutti questi anni. Ogni volta che sono tornato, è stata per me una gioia vedervi. Cenare e giocare insieme con voi non ha prezzo. Grazie per farmi ridere così tanto, per commentare sempre i miei calzini e per ascoltarmi quando cerco di spiegare un gioco. Grazie per le battute ogni tre secondi. Ho fatto molti sacrifici per poter realizzare il mio sogno di diventare ricercatore, ma i momenti spensierati con voi, così come le vostre parole di conforto e supporto, hanno reso tutto più facile e divertente. Vi auguro tutto il bene di questo mondo. Grazie per i messaggi che mi inviate, i calorosi abbracci quando ci vediamo, le risate e il supporto. Non vorrei far parte di un club che accetterebbe me come membro (cit), quindi, grazie di cuore.  Cari \textbf{zitroni}, sono passati ormai 7 anni dal nostro primo incontro a Friburgo e non c'è giorno in cui non pensi alle avventure e alle risate condivise insieme. Cerchiamo di vederci almeno una volta all'anno, o ogni due anni, e sembra sempre che il tempo non sia mai trascorso. Vi ringrazio per il sostegno costante, per i meme, per le lunghe chat e per le videochiamate durante il lockdown. Grazie anche per essere venuti in Belgio. Non vedo l'ora di vedervi presto. In ordine alfabetico: \textbf{Annalisa}, \textbf{Fabio}, \textbf{Gianluca}, \textbf{Marika}, \textbf{Margherita}, \textbf{Simona}, \textbf{Valentino}, ve se vole bene. Ma s magn u picch’? \textbf{Hilde}, \textbf{Patrick}, \textbf{Nele} and \textbf{Matthias}. I would like to thank you all for your kindness and hospitality. From the lovely dinners to the engaging conversations and walks. Your warm welcome and supportive words over the past few months have meant the world to me. I appreciate your efforts to include me by speaking English. I will try to learn Dutch as soon as possible. Thank you for everything. \textbf{Camilla}, \textbf{Leonardo} e \textbf{Giacomo}. Desidero dedicare qualche istante per esprimere la mia  gratitudine per tutto il prezioso sostegno che mi avete offerto nel corso del mio percorso di studi. Le vostre parole di incoraggiamento, la vostra presenza costante e il vostro affetto hanno reso questo viaggio accademico molto più significativo e sopportabile di quanto avrei mai potuto immaginare. Nonostante le nostre vite ci abbiano tenuto lontani fisicamente, siete stati sempre presenti, illuminando i momenti bui e rendendo le sfide meno gravose da affrontare. Vi sono profondamente grato per essere sempre stati lì per me. La vostra presenza nella mia vita è un tesoro inestimabile, e mi sento incredibilmente fortunato ad avervi accanto. Non vedo l'ora di potervi ospitare qui a Ghent. Inoltre, non posso fare a meno di menzionare quanto sia grato per la meravigliosa copertina che mi hai fatto, Cami; è davvero uno splendido gesto che apprezzo enormemente. Infine, vorrei esprimere i miei migliori auguri alla piccola \textbf{Flora}, che ha portato ancora più gioia e felicità nella nostra famiglia. Spero che possa crescere felice e sana, e che possa essere circondata sempre dall'amore e dall'affetto. ZUCCA! Grazie ancora. Vi voglio bene. \textbf{Mamma} e \textbf{papà}, in questo momento così speciale, desidero rivolgervi un sentito ringraziamento per tutto l'amore, il sostegno e la guida che mi avete donato nel corso della mia vita e, in particolare, durante il mio percorso di studi. Siete stati la mia fonte inesauribile di forza e gentilezza. Con la vostra costante presenza e il vostro affetto infinito, avete curato ogni mia preoccupazione e risolto ogni dubbio. I vostri consigli saggi e il vostro sostegno (anche economico) mi hanno dato la fiducia necessaria per affrontare ogni sfida con coraggio e determinazione. Se oggi sono chi sono, è grazie a voi e alla vostra straordinaria dedizione. Le vostre parole di incoraggiamento e il vostro sostegno instancabile mi hanno ispirato a dare sempre il massimo e a non arrendermi mai di fronte alle difficoltà. Senza di voi, non avrei mai potuto raggiungere questo traguardo così importante nella mia vita. Vi ringrazio di cuore per tutto ciò che avete fatto per me, per il vostro amore infinito e per la vostra dedizione senza pari. Siete il mio più grande tesoro e la mia fonte di ispirazione costante. Vi voglio bene più di qualsiasi parola possa esprimere e sono eternamente grato per tutto ciò che avete fatto e fate per me. \textbf{Judith}, you've been such an important part of my PhD. Your energy and positivity are contagious, and they've kept me going even when things got tough. There were times when I was feeling down, but you were always there to lift me up. Your support means the world to me. Thanks for listening to me going crazy about my research and for giving me all the time and space I needed over the last few months to complete this PhD. It hasn't always been easy, but you have always supported and encouraged me. Thank you for the laughs and for the nice walks. Thank you for patiently waiting for me because I saw or heard a bird. Thank you for teaching me the Flemish dialect. Thank you for your background songs, from Eros to Il Coccodrillo come fa. There are nine million bicycles in Beijing, that's a fact. The moments of laughter made my days, with your smile and kindness. Living with you has been an absolute joy, and I feel incredibly fortunate to have you in my life. Thank you for everything you do. You just make everything better.
\clearpage

	%%%%%%%%%%%%%%%%%%%%%%%%%%%%%%%%%%%%%  Bibliography  %%%%%%%%%%%%%%%%%%%%%%%%%%%%%%%%%%%%%%%%%%%%%%%%
%	\thispagestyle{empty}

	%\chapter*{Bibliography}
\chapter*{Bibliography}
\pagestyle{mainmatter}
\chaptermark{Bibliography}
\addcontentsline{toc}{chapter}{Bibliography}
\label{Bibliography}
	

		\begin{footnotesize}
		\color{black}
		\setlength{\thumbwidth}{0cm}
		\setlength{\thumbheight}{0cm}
		\setlength{\bibsep}{0.0pt}

	\end{footnotesize}


		\bibliographystyle{plain}
		\begin{footnotesize}
			\bibliography{My thesis library}
		\color{black}
		\setlength{\thumbwidth}{0cm}
		\setlength{\thumbheight}{0cm}
		\setlength{\bibsep}{0.0pt}

\end{document}
pdflatex your_document.tex

